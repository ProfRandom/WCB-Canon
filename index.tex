% Options for packages loaded elsewhere
% Options for packages loaded elsewhere
\PassOptionsToPackage{unicode}{hyperref}
\PassOptionsToPackage{hyphens}{url}
%
\documentclass[
  letterpaper,
]{book}
\usepackage{xcolor}
\usepackage[margin=1in]{geometry}
\usepackage{amsmath,amssymb}
\setcounter{secnumdepth}{5}
\usepackage{iftex}
\ifPDFTeX
  \usepackage[T1]{fontenc}
  \usepackage[utf8]{inputenc}
  \usepackage{textcomp} % provide euro and other symbols
\else % if luatex or xetex
  \usepackage{unicode-math} % this also loads fontspec
  \defaultfontfeatures{Scale=MatchLowercase}
  \defaultfontfeatures[\rmfamily]{Ligatures=TeX,Scale=1}
\fi
\usepackage{lmodern}
\ifPDFTeX\else
  % xetex/luatex font selection
\fi
% Use upquote if available, for straight quotes in verbatim environments
\IfFileExists{upquote.sty}{\usepackage{upquote}}{}
\IfFileExists{microtype.sty}{% use microtype if available
  \usepackage[]{microtype}
  \UseMicrotypeSet[protrusion]{basicmath} % disable protrusion for tt fonts
}{}
\makeatletter
\@ifundefined{KOMAClassName}{% if non-KOMA class
  \IfFileExists{parskip.sty}{%
    \usepackage{parskip}
  }{% else
    \setlength{\parindent}{0pt}
    \setlength{\parskip}{6pt plus 2pt minus 1pt}}
}{% if KOMA class
  \KOMAoptions{parskip=half}}
\makeatother
% Make \paragraph and \subparagraph free-standing
\makeatletter
\ifx\paragraph\undefined\else
  \let\oldparagraph\paragraph
  \renewcommand{\paragraph}{
    \@ifstar
      \xxxParagraphStar
      \xxxParagraphNoStar
  }
  \newcommand{\xxxParagraphStar}[1]{\oldparagraph*{#1}\mbox{}}
  \newcommand{\xxxParagraphNoStar}[1]{\oldparagraph{#1}\mbox{}}
\fi
\ifx\subparagraph\undefined\else
  \let\oldsubparagraph\subparagraph
  \renewcommand{\subparagraph}{
    \@ifstar
      \xxxSubParagraphStar
      \xxxSubParagraphNoStar
  }
  \newcommand{\xxxSubParagraphStar}[1]{\oldsubparagraph*{#1}\mbox{}}
  \newcommand{\xxxSubParagraphNoStar}[1]{\oldsubparagraph{#1}\mbox{}}
\fi
\makeatother


\usepackage{longtable,booktabs,array}
\usepackage{calc} % for calculating minipage widths
% Correct order of tables after \paragraph or \subparagraph
\usepackage{etoolbox}
\makeatletter
\patchcmd\longtable{\par}{\if@noskipsec\mbox{}\fi\par}{}{}
\makeatother
% Allow footnotes in longtable head/foot
\IfFileExists{footnotehyper.sty}{\usepackage{footnotehyper}}{\usepackage{footnote}}
\makesavenoteenv{longtable}
\usepackage{graphicx}
\makeatletter
\newsavebox\pandoc@box
\newcommand*\pandocbounded[1]{% scales image to fit in text height/width
  \sbox\pandoc@box{#1}%
  \Gscale@div\@tempa{\textheight}{\dimexpr\ht\pandoc@box+\dp\pandoc@box\relax}%
  \Gscale@div\@tempb{\linewidth}{\wd\pandoc@box}%
  \ifdim\@tempb\p@<\@tempa\p@\let\@tempa\@tempb\fi% select the smaller of both
  \ifdim\@tempa\p@<\p@\scalebox{\@tempa}{\usebox\pandoc@box}%
  \else\usebox{\pandoc@box}%
  \fi%
}
% Set default figure placement to htbp
\def\fps@figure{htbp}
\makeatother





\setlength{\emergencystretch}{3em} % prevent overfull lines

\providecommand{\tightlist}{%
  \setlength{\itemsep}{0pt}\setlength{\parskip}{0pt}}



 


\makeatletter
\@ifpackageloaded{bookmark}{}{\usepackage{bookmark}}
\makeatother
\makeatletter
\@ifpackageloaded{caption}{}{\usepackage{caption}}
\AtBeginDocument{%
\ifdefined\contentsname
  \renewcommand*\contentsname{Table of contents}
\else
  \newcommand\contentsname{Table of contents}
\fi
\ifdefined\listfigurename
  \renewcommand*\listfigurename{List of Figures}
\else
  \newcommand\listfigurename{List of Figures}
\fi
\ifdefined\listtablename
  \renewcommand*\listtablename{List of Tables}
\else
  \newcommand\listtablename{List of Tables}
\fi
\ifdefined\figurename
  \renewcommand*\figurename{Figure}
\else
  \newcommand\figurename{Figure}
\fi
\ifdefined\tablename
  \renewcommand*\tablename{Table}
\else
  \newcommand\tablename{Table}
\fi
}
\@ifpackageloaded{float}{}{\usepackage{float}}
\floatstyle{ruled}
\@ifundefined{c@chapter}{\newfloat{codelisting}{h}{lop}}{\newfloat{codelisting}{h}{lop}[chapter]}
\floatname{codelisting}{Listing}
\newcommand*\listoflistings{\listof{codelisting}{List of Listings}}
\makeatother
\makeatletter
\makeatother
\makeatletter
\@ifpackageloaded{caption}{}{\usepackage{caption}}
\@ifpackageloaded{subcaption}{}{\usepackage{subcaption}}
\makeatother
\usepackage{bookmark}
\IfFileExists{xurl.sty}{\usepackage{xurl}}{} % add URL line breaks if available
\urlstyle{same}
\hypersetup{
  pdftitle={WorldCrafting Canon (Development Build)},
  pdfauthor={Gizmo Conrad},
  hidelinks,
  pdfcreator={LaTeX via pandoc}}


\title{WorldCrafting Canon (Development Build)}
\usepackage{etoolbox}
\makeatletter
\providecommand{\subtitle}[1]{% add subtitle to \maketitle
  \apptocmd{\@title}{\par {\large #1 \par}}{}{}
}
\makeatother
\subtitle{Foundational Framework and Technical Notes}
\author{Gizmo Conrad}
\date{2025-10-09}
\begin{document}
\frontmatter
\maketitle


\mainmatter
\bookmarksetup{startatroot}

\chapter{WorldCrafting Canon}\label{worldcrafting-canon}

Foundational Framework and Technical Notes

\hfill\break

\bookmarksetup{startatroot}

\chapter{Preface}\label{preface}

Welcome to the \textbf{WorldCrafting Canon}, the evolving reference
framework for the physical, cosmological, and mythosophical principles
underlying the Myndhre setting.

This development build includes preliminary and canonical notes
spanning: - Meta-frameworks and mathematical tools, - Stellar and
planetary structures, - Orbital dynamics, - Mononic classifications, -
Binary systems, - and supporting phonetic and terminological standards.

\begin{center}\rule{0.5\linewidth}{0.5pt}\end{center}

\emph{Note: This is a pre-release scaffold for testing Quarto build
integrity and cross-format rendering. Content order and completeness are
not final.}

\part{Meta Framework}

\chapter{Mononic and Morphological
Axes}\label{mononic-and-morphological-axes}

\section{Abstract}\label{abstract}

\textbf{Major Topics:}\\
- Establishes a hierarchical classification of \textbf{mononic
morphotypes}: monon categories defined by gravitational rounding, fusion
status, and structural physics.\\
- Categories:\\
- \textbf{Planemon {[}sci{]}:} gravitationally rounded body, no fusion;
includes dwarf planets and large moons.\\
- \textbf{Intermon {[}NEW{]}:} transitional-mass bodies between
planemons and stars; may fuse deuterium (brown dwarfs).\\
- \textbf{Stellamon {[}sci/neo{]}:} hydrogen-fusing stars; foundational
stellar monons.\\
- \textbf{Supermon {[}EXPANDED{]}:} high-mass stellar remnants (e.g.,
black holes, neutron stars) beyond stellamon scale but sub-galactic.\\
- \textbf{Ultramon {[}EXPANDED{]}:} supermassive objects (SMBHs) in
galactic centers, 10⁶--10⁹ ⨁.\\
- \textbf{Hypermon {[}EXPANDED{]}:} speculative hypermassive monons
(\textgreater10⁹ ⨁), e.g.~primordial collapses.\\
- Provides a summary grid of categories, prefixes, abbreviations, and
mass ranges (in Earth masses).

\textbf{Key Terms \& Symbols:}\\
- \textbf{Planemon {[}sci{]}.}\\
- \textbf{Intermon {[}NEW{]}.}\\
- \textbf{Stellamon {[}sci/neo{]}.}\\
- \textbf{Supermon {[}EXPANDED{]}.}\\
- \textbf{Ultramon {[}EXPANDED{]}.}\\
- \textbf{Hypermon {[}EXPANDED{]}.}

\textbf{Cross-Check Notes:}\\
- \textbf{Planemon, Stellamon} already appear in canon.\\
- \textbf{Supermon, Ultramon, Hypermon} existed in mass interval tables,
but this file adds full narrative definitions.\\
- \textbf{Intermon} is a new introduction.\\
- \textbf{Status:} {[}EXPANDED + NEW{]} --- consolidates the mononic
classification scheme; introduces Intermon

All physical bodies (\emph{monons}) in the WCB framework are described
along two orthogonal axes:\\
their \textbf{Mononic class}, which defines scale and self-coherence,\\
and their \textbf{Morphological character}, which defines material
composition and structure.\\
A third, contextual layer --- the \textbf{Categorical Envelope} ---
arises from the interaction of the two.

\begin{longtable}[]{@{}
  >{\raggedright\arraybackslash}p{(\linewidth - 6\tabcolsep) * \real{0.2299}}
  >{\raggedright\arraybackslash}p{(\linewidth - 6\tabcolsep) * \real{0.3678}}
  >{\raggedright\arraybackslash}p{(\linewidth - 6\tabcolsep) * \real{0.2989}}
  >{\raggedright\arraybackslash}p{(\linewidth - 6\tabcolsep) * \real{0.1034}}@{}}
\toprule\noalign{}
\begin{minipage}[b]{\linewidth}\raggedright
\textbf{Axis}
\end{minipage} & \begin{minipage}[b]{\linewidth}\raggedright
\textbf{Defines}
\end{minipage} & \begin{minipage}[b]{\linewidth}\raggedright
\textbf{Typical Examples}
\end{minipage} & \begin{minipage}[b]{\linewidth}\raggedright
\textbf{Field of Study}
\end{minipage} \\
\midrule\noalign{}
\endhead
\bottomrule\noalign{}
\endlastfoot
\textbf{Mononic (Vertical)} & Scale / mass domain --- the degree of
self-coherence under gravity & micromon → minimon → planemon → stellamon
→ ultramon & \textbf{Mononics} \\
\textbf{Morphological (Horizontal)} & Material / structural composition
--- what the body is made of & lithic, rheatic, hydruric, cryic,
metalluric & \textbf{Morphology} \\
\textbf{Categorical / Environmental Envelope} & Contextual synthesis of
Mononic class + dominant Morphotype & geotic, telluric, rheatic, xenotic
& --- \\
\end{longtable}

\subsection{Examples}\label{examples}

\begin{itemize}
\tightlist
\item
  \textbf{Earth:} lithic planemon → \emph{geotic} category.\\
\item
  \textbf{Jupiter:} rheatic planemon → \emph{rheatic} category.\\
\item
  \textbf{Ceres:} hydruric minimon → \emph{asteroidal} category.
\end{itemize}

\subsection{Definitions}\label{definitions}

\begin{itemize}
\tightlist
\item
  \textbf{Morphotype {[}neo{]}:} the primary compositional descriptor of
  a monon --- its dominant physical ``substance'' or structural
  character (e.g., \emph{lithic}, \emph{rheatic}, \emph{hydruric}).\\
\item
  \textbf{Category / Envelope {[}meta{]}:} an emergent classification
  expressing systemic context and behavior, derived from the combination
  of a monon's mass domain and its morphotype (e.g., \emph{geotic},
  \emph{telluric}, \emph{xenotic}).
\end{itemize}

\begin{quote}
\textbf{Summary:}\\
The \emph{Mononic axis} quantifies --- how massive and self-coherent a
body is.\\
The \emph{Morphological axis} qualifies --- what the body is made of.\\
The \emph{Categorical envelope} contextualizes --- how those traits
express within a stellar system.
\end{quote}

\chapter{WCB Canonical Mononic
Morphotypes}\label{wcb-canonical-mononic-morphotypes}

\emph{A classification of mass-based cosmic bodies by gravitational,
structural, and fusion characteristics.} \#\# 🔵 \textbf{Planemon}
\textgreater{} A gravitationally rounded object in the mass range
typical of planemons and planemon analogs. - \textbf{Mass Range}: -
167 dmt ⨁ ≤ m \textless{} 4.131 kt
(1.67~demiterrans~to~4.131~kiloterrans) - 0.000167 ⨁ ≤ m \textless{}
4.131 kt - \textbf{Fusion Status}: No fusion\\
- \textbf{Examples}: Earth, Mercury, Ganymede, Kepler-22b\\
- \textbf{Notes}: May include isolated or satellite-bound bodies;
includes dwarf planemons and major moons above hydrostatic threshold.
\#\# 🟠 \textbf{Intermon} \textgreater{} A transitional-mass object
between planemons and stars. - \textbf{Mass Range}:\\
- 4.131 kt \textless{} m ≤ 2.664 myt
(4.131~kiloterrans~to~2.664~myriaterrans) - 13♃ \textless{} m
\textless{} 80♃ - \textbf{Fusion Status}: Sub-stellar; may fuse
deuterium\\
- \textbf{Examples}: Brown dwarfs, isolated non-hydrogen fusors\\
- \textbf{Notes}: Symbolic ``liminal'' zone; fusion is partial or
temporary. \#\# 🔴 \textbf{Stellamon} \textgreater{} A self-luminous
object that sustains hydrogen fusion at its core. The foundational unit
of stellar systems. - \textbf{Mass Range}: - 2.664 myt \textless{} m ≤
1 hxt (2.664~myriaterrans~to~1~hexaterran)\\
- 80♃ \textless{} m \textless{} 31,466♃ - 0.08⊙ \textless{} m
\textless{} 300⊙ - \textbf{Fusion Status}: Core hydrogen fusion\\
- \textbf{Examples}: Proxima Centauri, the Sun, Sirius A\\
- \textbf{Notes}: All main-sequence stars fall here; upper bound set by
radiation pressure limits to approximately (M = 300⊙). \#\# 🔵
\textbf{Supermon} \textgreater{} A high-mass monon object that exceeds
the stellar mass range but is \textbf{not yet galactic} in scope.
Includes most known neutron stars and stellar black holes. -
\textbf{Mass Range}:\\
1  hxt \textless{} m ≤ 1 Mt (100,000 ⨁ -- 1,000,000 ⨁)\\
- \textbf{Common Forms}:\\
- Black Holes that have exceeded stellamon mass but not yet achieved
ultramon mass \#\# 🟣 \textbf{Ultramon} \textgreater{} A
\textbf{supermassive} object --- typically a black hole --- existing at
the centers of galaxies or as relics of early cosmic formation. -
\textbf{Mass Range}:\\
1  Mt \textless{} m ≤ 1 Gt (1 million -- 1 billion ⨁)\\
- \textbf{Common Forms}:\\
- SMBHs\\
\#\# 🔴 \textbf{Hypermon} \textgreater{} A speculative class of
\textbf{hypermassive monons}, potentially forming during early universe
conditions or beyond current observational limits. - \textbf{Mass
Range}:\\
m \textgreater1  Gt (\textgreater1 billion ⨁)\\
- \textbf{Speculative Examples}:\\
- Primordial hypercollapses\\
- Direct-collapse black holes from Population III stars\\
- Core seeds of hypermassive galaxies \#\#\# 🧭 Summary Grid (Expanded)

\begin{longtable}[]{@{}llll@{}}
\toprule\noalign{}
Name & Prefix & Abbr. & Mass Range (⨁) \\
\midrule\noalign{}
\endhead
\bottomrule\noalign{}
\endlastfoot
Stellamon & --- & --- & 26,641 -- 100,000 (0.08⊙ - 300⊙) \\
Supermon & super- & smt & 100,000 -- 1,000,000 \\
Ultramon & ultra- & umt & 1,000,000 -- 1,000,000,000 \\
Hypermon & hyper- & hmt & \textgreater1,000,000,000 \\
\end{longtable}

\chapter{The Anthropic Norm}\label{the-anthropic-norm}

\textbf{Major Topics:}\\
- Defines the \textbf{Anthropic Norm}: the universe and its functioning
as presently perceived, absent extreme or rare conditions.\\
- Closely tied to the \textbf{Mediocritic Principle of State}:\\
- The majority of matter and energy in the current epoch exists within a
\textbf{statistically dominant modal band} of states.\\
- Extreme states (e.g., black holes, neutronium, degenerate matter) may
be numerous but are \textbf{exceptions}, not norm-defining.\\
- Key insights for worldbuilding:\\
1. \textbf{Rarity ≠ Normativity}: unusual states, even if interesting,
should not define the baseline.\\
2. \textbf{Normativity = Modal, not Mean}: the ``normal'' universe is
not the statistical average, but the modal cluster where most
matter/energy exists most of the time.\\
- Examples:\\
- The universe is filled with stars, but is not ``mostly plasma'' under
current conditions.\\
- Neutron stars exist, but the cosmos is not composed of neutronium.\\
- Rocky planemons form from volatiles, but the universe remains
overwhelmingly hydrogen and helium gas.

\textbf{Key Terms \& Symbols:}\\
- \textbf{Anthropic Norm {[}NEW{]}.}\\
- \textbf{Mediocritic Principle of State {[}NEW{]}.}\\
- \textbf{Modal Cluster {[}sci{]}.}

\textbf{Cross-Check Notes:}\\
- No prior mention of these terms in canon; both are newly introduced
philosophical-structural concepts.\\
- Complements existing WCB philosophical principles (e.g., Protagorean
Principle).\\
- \textbf{Status:} {[}NEW{]} --- establishes a normative philosophical
baseline for worldbuilding.

The universe and its functioning as we currently perceive it, absent any
extreme conditions.

\subsection{The Anthropic Norm \& the Mediocritic Principle of
State}\label{the-anthropic-norm-the-mediocritic-principle-of-state}

\begin{quote}
The vast majority of matter and energy in the universe exists within a
narrow, statistically dominant band of physical states --- from which it
deviates only rarely and locally, within the current cosmological epoch.
\end{quote}

This leads to two foundational insights for thesiastic modeling: 1.
\textbf{Rarity is not normativity.}\\
Extreme states --- from black holes to degenerate matter --- may be
\emph{numerous}, but they are not \textbf{norm-defining}. They are
\textbf{local exceptions} to a global pattern.\\
2. \textbf{Normativity is modal, not mean.}\\
The Anthropic Norm does not lie at the statistical \emph{average}, but
at the \textbf{modal cluster}:\\
the zone in which \textbf{most matter and energy exist most of the time}
under current universal conditions. Thus: - The universe is filled with
stars, but is mostly \emph{not} plasma.\\
\emph{(It was \textbf{at one time} --- but that was a different
epoch.)}\\
- Neutron stars are real, but the universe is \emph{not} mostly
neutronium.\\
- Volatiles can form rocky planemons --- but the cosmos is
\emph{overwhelmingly} hydrogen and helium in gaseous form.

\chapter{🔑 The SANC Principle}\label{the-sanc-principle}

\section{Abstract}\label{abstract-1}

\textbf{Major Topics:}\\
- Definition of the \textbf{SANC Principle} (\emph{Science-Adjacent, No
Calculus}).\\
- Scope of mathematical tools for WCB: algebra, geometry,
trigonometry.\\
- Exclusion of higher-level tools: calculus, tensors, advanced
astrophysics.\\
- Guiding philosophy: ``Rigor without rigidity.''

\textbf{Key Terms \& Symbols:}\\
- SANC = Science-Adjacent, No Calculus.

\textbf{Cross-Check Notes:}\\
- Canon as of Glossary v1.212.\\
- Reinforces the methodological boundaries of WCB --- mathematical
accessibility for builders without requiring advanced formal training.\\
- Companion mottos: \emph{Simple, Approximate, Notationally Clear} and
\emph{Sufficient and Necessary Constructs}.

\textbf{SANC = Science-Adjacent, No Calculus.}

\begin{itemize}
\tightlist
\item
  ✅ We use \textbf{algebra, geometry, and trigonometry} --- the
  practical tools any builder can wield.\\
\item
  🚫 We skip \textbf{calculus, tensors, and full astrophysics} ---
  that's outside our scope.\\
\item
  🎯 Goal: \textbf{Rigor without rigidity.} Enough math to make worlds
  believable, never so much that it strangles imagination.
\end{itemize}

\emph{Think of SANC as your worldmaking passport: it gets you everywhere
you need to go,\\
but it won't weigh you down with equations better left to
astrophysicists.}

\chapter{Abstract}\label{abstract-2}

\textbf{Major Topics:}\\
- Comprehensive algebraic toolkit for ellipse geometry and orbital
mechanics.\\
- Full glossary of ellipse parameters: semi-major axis (a), semi-minor
axis (b), linear eccentricity (c), eccentricity (e), flattening (f),
axes (i, j), vertices, co-vertices, foci.\\
- Derived lengths: focus-maximus (d), focus-minimus (g), focal span (h),
semi-latus rectum (ℓ), latus rectum (q).\\
- Directrix system: center-to-directrix (m), focus-to-directrix (n),
vertex-to-directrix (s).\\
- Canonical equations for geometric relations, orbital radii, and
directrix properties.\\
- ``Given Any Two, Solve the Rest'' matrix for deriving all ellipse
properties from any two independent parameters.

\textbf{Key Terms \& Symbols:}\\
- a = semi-major axis.\\
- b = semi-minor axis.\\
- c = linear eccentricity.\\
- e = eccentricity (unitless).\\
- f = flattening.\\
- i, j = major/minor axes (2a, 2b).\\
- d, g, h = focus-maximus, focus-minimus, focal span.\\
- ℓ, q = semi-latus rectum, latus rectum.\\
- m, n, s = directrix distances.\\
- r(θ) = orbital radius equation.\\
- rₚ, rₐ = periapsis, apoapsis distances.

\textbf{Cross-Check Notes:}\\
- Eccentricity (e) canonically unitless and invariant under scale.\\
- Flattening (f) defined here as \(f = 1 - \dfrac{b}{a}\) --- ensure
consistency with other usage.\\
- Directrix is definitional: appears in geometry but not physical orbits
(noted explicitly).\\
- ``Solve the Rest'' matrix designed for symbolic algebra; reinforces
WCB's \textbf{SANC} approach.\\
- Overlaps with \textbf{Orbital Eccentricity and Seasonal Effects.md}
(use of rₚ, rₐ).

\# 🧭 Ellipse Geometry Solver --- WCB Reference

This reference provides a complete algebraic toolkit for solving any
ellipse --- geometric or orbital --- from any two independent
parameters. All terms match the diagram above.\\
\#\# 📘 Glossary of Ellipse Parameters All variables match the labeled
diagram above. \#\#\# 🔹 Core Axes \& Foci - \textbf{a} --- semi-major
axis ~ - \textbf{b} --- semi-minor axis ~ - \textbf{c} --- linear
eccentricity (center to focus) ~ - \textbf{e} --- eccentricity
(dimensionless) = \(\dfrac{c}{a}\)\\
- Describes how ``stretched'' the ellipse is.\\
- \(e = 0\) → perfect circle\\
- \(0 < e < 1\) → ellipse\\
- \(e = 1\) → parabola (degenerate case)\\
- \(e > 1\) → hyperbola (not an ellipse)\\
- Eccentricity is \textbf{unitless} and invariant under scale.\\
- It also defines the ellipse as a conic: \[
  \dfrac{\text{distance to focus}}{\text{distance to directrix}} = e
\] - \textbf{f} --- flattening \(b = 1 - \dfrac{b}{a}\) ~ - \textbf{i}
--- major axis \(=2a\) ~ - \textbf{j} --- minor axis \(=2b\) ~ -
\textbf{P, A} --- primary vertices - \textbf{X, Y} --- co-vertices (±b
along minor axis) ~ - \textbf{C} --- center of ellipse ~ - \textbf{f₁,
f₂} --- the two foci ~ \#\#\# 🔹 Derived Lengths\\
- \textbf{d} --- focus-maximus = vertex to opposite focus \(= a + c\) ~
- \textbf{g} --- focus-minimus = focus to nearest vertex ~\(= a - c\)
(e.g.~\(f_1P\)) ~ - \textbf{h} --- focal span \(=2c\) ~ - \textbf{ℓ} ---
semi-latus rectum \(=\dfrac{1}{2} q\) ~ - \textbf{q} --- latus rectum
\(= 2ℓ\) \#\#\# 🔹 Directrix System\\
- \textbf{m} --- center-to-directrix \(=\dfrac{a}{e}\) ~ - \textbf{n}
--- focus-to-directrix \(=m - c\) - \textbf{s} --- vertex-to-directrix
\(=m - a = \dfrac{c}{e} - a\) ~ \#\# 🧮 Canonical Equations \#\#\# 🔹
Geometric - \(c = ae\) ~ - \(b = a\sqrt{1 - e^2}\) -
\(a = \sqrt{b^2 + c^2}\) \#\#\# 🔹 Orbital Geometry -
\(f = a(1 - e^2) = \dfrac{b^2}{a}\) ~ -
\(r(\theta) = \dfrac{a(1 - e^2)}{1 + e \cos \theta}\) ~ -
\(r_p = a(1 - e)\) \#\#\# 🔹 Directrix Relationships -
\(m = \dfrac{a}{e}\) ~ - \(s = m - a = \dfrac{c}{e} - a\) ~ -
\(n = m - c = \dfrac{a}{e} - c\) \#\# 📐 What Is the Directrix? For an
ellipse, the \textbf{directrix} is:

\begin{quote}
A fixed vertical line such that, for any point PPP on the ellipse,\\
the ratio of the distance from PPP to a \textbf{focus} and the distance
from PPP to the \textbf{directrix} is equal to the \textbf{eccentricity}
eee.
\end{quote}

Formally:
\[\dfrac{\text{distance to focus}}{\text{distance to directrix}} = e\]
\emph{This defines the ellipse!} \#\# 📏 Where Is the Directrix?

There are \textbf{two directrices}, one on each side of the center, at a
distance: \[m = \dfrac{a}{e}\] \ldots{} from the center, where: -
\emph{a} is the semi-major axis\\
- \emph{e} is the eccentricity

So: - Right-side directrix: \(x = \dfrac{a}{e}\)\\
- Left-side directrix: \(x = -\dfrac{a}{e}\)\hspace{0pt} If \emph{e} →
0, the directrix moves off to infinity --- which makes sense, because a
circle (eccentricity 0) has no directrix-like behavior. \#\# 🎯 How the
Directrix Relates to the Ellipse You can think of the ellipse as a
\textbf{set of points} where: \[\dfrac{PF}{PD} = e\] Where: - PF is the
distance from a point P on the ellipse to a \textbf{focus}\\
- PD is the distance from that same point P to the \textbf{directrix}

This definition is symmetric and constructive: it's how conics are
\emph{defined} in classic geometry. \#\# 💡 So What Does It \emph{Do}?
The directrix is mostly a \textbf{definitional and constructional tool}
--- not something we see in physical orbits, but: - It gives us a clean
formula for an ellipse in Cartesian coordinates:
\[r(\theta) = \dfrac{p}{1 + e \cos \theta} \quad \text{where } p = \dfrac{b^2}{a}\]
- It shows up in \textbf{ray-tracing}, \textbf{parabolic reflectors},
\textbf{classical mechanics}, and \textbf{procedural shape generation}.
\#\# 🔍 Quick Facts

\begin{longtable}[]{@{}lc@{}}
\toprule\noalign{}
Concept & Value / Equation \\
\midrule\noalign{}
\endhead
\bottomrule\noalign{}
\endlastfoot
Directrix location & \(x= \pm \dfrac{a}{e}\)\hspace{0pt} \\
Distance center to directrix & \(m = \dfrac{a}{e}\)\hspace{0pt} \\
Distance vertex to directrix & \(s = m - a = \dfrac{c}{e}-a\) \\
Distance focus to directrix & \(n = m - c = \dfrac{a}{e} - c\) \\
Eccentricity via directrix & \(e = \dfrac{PF}{PD}\)\hspace{0pt} \\
\end{longtable}

\section{🧩 ``Given Any Two, Solve the Rest''
Matrix}\label{given-any-two-solve-the-rest-matrix}

Each row lists two known parameters and what you can solve from them
using canonical identities.

\begin{longtable}[]{@{}
  >{\centering\arraybackslash}p{(\linewidth - 4\tabcolsep) * \real{0.1429}}
  >{\centering\arraybackslash}p{(\linewidth - 4\tabcolsep) * \real{0.1429}}
  >{\centering\arraybackslash}p{(\linewidth - 4\tabcolsep) * \real{0.7143}}@{}}
\toprule\noalign{}
\begin{minipage}[b]{\linewidth}\centering
Known Pair
\end{minipage} & \begin{minipage}[b]{\linewidth}\centering
Solve For ~ ~ ~ ~
\end{minipage} & \begin{minipage}[b]{\linewidth}\centering
Method / Equation(s)
\end{minipage} \\
\midrule\noalign{}
\endhead
\bottomrule\noalign{}
\endlastfoot
a, e & b, c, f, m, s, n ~ &
\(b = a\sqrt{1 - e^2},\ c = ae,\ f = a(1 - e^2),\ m = \dfrac{a}{e},\ s = m - a,\ n = m - c\) \\
a, b ~ ~ ~ ~ ~ ~ ~ & c, e, f ~ ~ ~ ~ ~ &
\(c = \sqrt{a^2 - b^2},\ e = \sqrt{1 - \dfrac{b^2}{a^2}},\ f = \dfrac{b^2}{a}\) \\
a, c ~ ~ ~ ~ ~ ~ ~ & b, e, m, s, n ~ ~ &
\(e = \dfrac{c}{a},\ b = \sqrt{a^2 - c^2},\ m = \dfrac{a}{e},\ s = m - a,\ n = m - c\) \\
b, c ~ ~ ~ ~ ~ ~ ~ & a, e ~ ~ ~ ~ ~ ~ ~ &
\(a = \sqrt{b^2 + c^2},\ e = \dfrac{c}{a}\) \\
e, b ~ ~ ~ ~ ~ ~ ~ & a, c ~ ~ ~ ~ ~ ~ ~ &
\(a = \dfrac{b}{\sqrt{1 - e^2}},\ c = ae\) \\
rₚ, rₐ ~ ~ ~ ~ ~ ~ & a, c, e, b ~ ~ ~ ~ &
\(a = \dfrac{rₚ + rₐ}{2},\ c = \dfrac{rₐ - rₚ}{2},\ e = \dfrac{c}{a},\ b = a\sqrt{1 - e^2}\) \\
a, f ~ ~ ~ ~ ~ ~ ~ & b, e ~ ~ ~ ~ ~ ~ ~ &
\(b = \sqrt{af},\ e = \sqrt{1 - \dfrac{b^2}{a^2}}\) \\
a, m ~ ~ ~ ~ ~ ~ ~ & e, c ~ ~ ~ ~ ~ ~ ~ &
\(e = \dfrac{a}{m},\ c = ae\) \\
c, e ~ ~ ~ ~ ~ ~ ~ & a, b, m ~ ~ ~ ~ ~ &
\(a = \dfrac{c}{e},\ b = a\sqrt{1 - e^2},\ m = \dfrac{a}{e}\) \\
a, s ~ ~ ~ ~ ~ ~ ~ & m, e, c, b ~ ~ ~ ~ &
\(m = a + s,\ e = \dfrac{a}{m},\ c = ae,\ b = a\sqrt{1 - e^2}\) \\
a, n ~ ~ ~ ~ ~ ~ ~ & m, c, e, b ~ ~ ~ ~ &
\(m = n + c,\ c = m - n,\ e = \dfrac{a}{m},\ b = a\sqrt{1 - e^2}\) \\
\end{longtable}

\begin{quote}
🔍 \emph{Note: This matrix is designed for symbolic manipulation. Some
results require substitution into multiple chained equations.}\#\#
Abstract \textbf{Major Topics:}\\
- Reference sheet of exponent and logarithm rules for worldbuilding
math.\\
- General-use algebraic identities and transformations.\\
- Ratio, sum/difference, mean, product/quotient relationships between
bounds.\\
- Percentage calculations and range transformations.\\
- Integer sequences, summations, odd/even sequences.\\
- Special transforms (involutive, reciprocal percentages).\\
- Generalized metallic mean formula.\\
- Point--slope and slope--intercept line equations.\\
- Temperature scale conversions (Fahrenheit, Celsius, Kelvin).
\end{quote}

\textbf{Key Terms \& Symbols:}\\
- Exponents, roots, logs (standard algebra rules).\\
- Variables for bounds: u (upper), l (lower), s (sum), d (difference), r
(ratio), m (mean), p (product), q (quotient).\\
- Percentages: p, x, n, v.\\
- Σ = summation over integer ranges.\\
- Metallic mean: Nₓ.\\
- Line equations: slope (m), intercept (b), points (x₁, y₁), (x₂, y₂).\\
- Temperature variables: F (Fahrenheit), C (Celsius), K (Kelvin).

\textbf{Cross-Check Notes:}\\
- Functions as a \textbf{toolbox reference} rather than a conceptual
note.\\
- Reinforces WCB's \textbf{SANC} principle: simple, approximate,
notationally clear.\\
- Purely mathematical --- no worldbuilding-specific notation introduced
beyond standard algebra.\\
- Should remain a \textbf{utility reference} to support other canon
notes.

A reference sheet of exponent and logarithm rules useful in constructing
systems of thermal, gravitational, orbital, and energetic relationships
in scientifically-grounded worldbuilding.

\section{🔢 Rules of Exponents}\label{rules-of-exponents}

\subsection{➕ Product \& Quotient Rules}\label{product-quotient-rules}

\[
x^m \cdot x^n = x^{m+n}, \qquad \frac{x^m}{x^n} = x^{m-n}
\]

\subsection{📏 Index Rule}\label{index-rule}

\[
\sqrt[n]{x} = x^{\frac{1}{n}}, \qquad \sqrt[-n]{x} = \frac{1}{\sqrt[n]{x}} = x^{-\frac{1}{n}}
\]

\[
\sqrt[m]{\frac{x^n}{x^p}} = x^{\frac{n - p}{m}}, \qquad \sqrt[m]{x^n x^p} = x^{\frac{n + p}{m}}
\]

\subsection{🔁 Power Rule}\label{power-rule}

\[
(x^m)^n = x^{mn}, \qquad (x^{-m})^n = x^{-mn}
\]

\[
\sqrt[n]{\sqrt[m]{x}} = \sqrt[nm]{x}, \qquad \sqrt[n]{x^{\frac{1}{m}}} = x^{\frac{1}{nm}}
\]

\subsection{🧮 Power of a Fraction}\label{power-of-a-fraction}

\[
\left(\frac{x}{y}\right)^n = \frac{x^n}{y^n}
\]

\subsection{🎯 Fractional Exponents}\label{fractional-exponents}

\[
\sqrt[m]{x^n} = x^{\frac{n}{m}}, \qquad \sqrt[n]{x^m} = x^{\frac{m}{n}}, \qquad (\sqrt[m]{x})^n = x^{\frac{n}{m}}
\]

\subsection{🚫 Negative Exponent Rule}\label{negative-exponent-rule}

\[
x^{-n} = \frac{1}{x^n}
\]

\subsection{🕳️ Zero Exponent Rule}\label{zero-exponent-rule}

\[
x^0 = 1 \quad \text{(for } x \ne 0\text{)}
\]

\subsection{♾️ Infinity Exponent Rule}\label{infinity-exponent-rule}

\[
x^\infty = \infty, \qquad x^{-\infty} = 0
\]

\section{🧠 Additional Useful
Identities}\label{additional-useful-identities}

\subsection{🧩 Distributive Rule for Exponents over
Multiplication}\label{distributive-rule-for-exponents-over-multiplication}

\[
(xy)^n = x^n y^n, \qquad \left(\frac{x}{y}\right)^n = \frac{x^n}{y^n}
\]

\subsection{🔄 Logarithmic Inverses}\label{logarithmic-inverses}

\[
\log_b(b^x) = x, \qquad b^{\log_b(x)} = x
\]

\subsection{🪜 Logarithmic Expansion
Rules}\label{logarithmic-expansion-rules}

\[
\log(xy) = \log x + \log y, \qquad \log\left(\frac{x}{y}\right) = \log x - \log y, \qquad \log(x^n) = n \log x
\]

\subsection{🌀 Reciprocal Roots}\label{reciprocal-roots}

\[
\sqrt[n]{\frac{1}{x}} = \frac{1}{\sqrt[n]{x}} = x^{-\frac{1}{n}}
\]

\subsection{\texorpdfstring{📈 Arbitrary Exponentials in Terms of
\emph{e}}{📈 Arbitrary Exponentials in Terms of e}}\label{arbitrary-exponentials-in-terms-of-e}

\[
a^x = e^{x \ln a}
\]

\section{📊 Powers and Logs}\label{powers-and-logs}

\[
x^y = z \quad \Rightarrow \quad y = \frac{\log z}{\log x} = \log_x z
\]

\[
x = z^{\frac{1}{y}} = \sqrt[y]{z}
\]

\chapter{🧮 General-Use Equations}\label{general-use-equations}

\subsection{📘 Variable Definitions}\label{variable-definitions}

\begin{itemize}
\tightlist
\item
  \textbf{u} = upper bound\\
\item
  \textbf{l} = lower bound\\
\item
  \textbf{s} = sum of bounds (total)\\
\item
  \textbf{d} = difference between bounds (span)\\
\item
  \textbf{r} = ratio of lower to upper bound\\
\item
  \textbf{m} = mean (average)\\
\item
  \textbf{p} = product of bounds\\
\item
  \textbf{q} = quotient of bounds
\end{itemize}

\section{📏 Number Relationships}\label{number-relationships}

\subsection{➗ Ratio}\label{ratio}

\[
\text{ratio} = \frac{l}{u} = \frac{\text{sum} - \text{difference}}{\text{sum} + \text{difference}}
\]

\subsection{➕ Sum \& Difference
Identities}\label{sum-difference-identities}

\[
\text{sum} = u + l = d\left(1 + \frac{1}{r}\right), \qquad
\text{difference} = u - l = s\left(\frac{1 - r}{1 + r}\right)
\]

\subsection{🔀 Transformations Between
Bounds}\label{transformations-between-bounds}

\[
l = \frac{s - d}{2}, \qquad
u = \frac{s + d}{2}, \qquad
s = l + u, \qquad
d = u - l
\]

\subsection{🔁 Alternate Forms with
Ratio}\label{alternate-forms-with-ratio}

\[
l = \frac{s}{1 + r}, \qquad
u = \frac{rs}{1 + r}
\]

\[
l = s \cdot \left(\frac{1}{1 + r}\right), \qquad
u = s \cdot \left(\frac{r}{1 + r}\right)
\]

\[
l = \frac{s - d}{2}, \qquad
u = \frac{s + d}{2}, \qquad
l = \frac{(s - d) \cdot r}{1 + r}, \qquad
u = \frac{s \cdot r + d}{1 + r}
\]

\subsection{📊 Mean}\label{mean}

\[
m = \frac{u + l}{2}, \qquad
m = \frac{s}{2}, \qquad
m = u - \frac{d}{2}, \qquad
m = l + \frac{d}{2}
\]

\subsection{🧮 Inequality Notes}\label{inequality-notes}

\[
m \gg \frac{1}{2} u \quad \text{(if } u < 0 \text{ and } l > 0\text{)}
\]

\[
m = \frac{1}{2} u \quad \text{(if } l = 0\text{)}
\]

\[
m = u = l \quad \text{(if } u = l\text{)}
\]

\subsection{✖️ Product \& Quotient}\label{product-quotient}

\[
\text{Product } = xy = \frac{s^2 - d^2}{4}
\]

\[
\text{Quotient } = \frac{x + d}{x - d} = \frac{(x + y) + (x - y)}{(x + y) - (x - y)}
\]

\[
\frac{s}{d} = \frac{x + y}{x - y}, \qquad \frac{d}{s} = \frac{x - y}{x + y}
\]

\section{📉 Percentage from Portion}\label{percentage-from-portion}

\subsection{🔢 Percentage (p) from portion (x) of total
(n)}\label{percentage-p-from-portion-x-of-total-n}

\[
p = \frac{x}{n}, \qquad x = pn, \qquad n = \frac{p}{x}
\]

\subsection{📘 Definitions}\label{definitions-1}

\begin{itemize}
\tightlist
\item
  \textbf{p} = percentage\\
\item
  \textbf{x} = portion (part)\\
\item
  \textbf{n} = base number (whole) \#\# 🎯 Percentages in Ranges \#\#\#
  📐 Percentage (p) Represented by a Value (v) in Range (l \ldots{} u)
  \[
  p = \frac{v - l}{u - l}, \qquad v = p(u - l) + l, \qquad u = \frac{v - l}{p} + l, \qquad l = \frac{v - pu}{1 - p}
  \]
\end{itemize}

\subsubsection{🔤 Variable Definitions}\label{variable-definitions-1}

\begin{itemize}
\tightlist
\item
  \textbf{v} = value within the range\\
\item
  \textbf{l} = lower bound\\
\item
  \textbf{u} = upper bound\\
\item
  \textbf{p} = percentage of the ({[}l, u{]}) range represented by (v)
\end{itemize}

\subsection{🔁 Transfer a Percentage Between Two
Ranges}\label{transfer-a-percentage-between-two-ranges}

\subsubsection{🧮 General Equations}\label{general-equations}

Let: - ( R\_1 = {[}\text{min}, \text{num}, \text{max}{]} ) - ( R\_2 =
{[}z, x, y{]} ), where ( x \textless{} z \textless{} y )

Then: \[
\text{pct} = \frac{\text{num} - \text{min}}{\text{max} - \text{min}}, \qquad
z = \text{pct} \cdot (y - x) + x
\]

\subsubsection{✅ Example}\label{example}

Given: - ( R\_1 = {[}0, 6, 10{]} ) - ( R\_2 = {[}4, z, 6{]} )

Then: \[
\text{pct} = \frac{6 - 0}{10 - 0} = \frac{6}{10} = 0.60
\]

\[
z = 0.60 \cdot (6 - 4) + 4 = 0.60(2) + 4 = 5.2
\]

Check: \[
\frac{5.2 - 4}{6 - 4} = \frac{1.2}{2} = 0.60
\]

\section{🔢 Sums of Number Sequences}\label{sums-of-number-sequences}

\subsection{📘 Variable Definitions}\label{variable-definitions-2}

\begin{itemize}
\tightlist
\item
  \textbf{Σ} = sum of all integers in the range (l \ldots{} u)\\
\item
  \textbf{s} = sum of bounds: ( u + l )\\
\item
  \textbf{d} = difference: ( u - l )\\
\item
  \textbf{l} = lower bound of range\\
\item
  \textbf{u} = upper bound of range
\end{itemize}

\subsection{➕ Consecutive Integers from (l \ldots{}
u)}\label{consecutive-integers-from-l-u}

\[
s = u + l, \qquad d = u - l
\]

\subsubsection{🧮 Summation Formulas}\label{summation-formulas}

Basic: \[
\Sigma = \frac{u(u + 1)}{2} - \frac{l(l - 1)}{2}
\]

Symmetric: \[
\Sigma = \frac{l^2 - 1}{2} + \frac{u^2 + u}{2}, \qquad
\Sigma = \frac{(u + l)(u - l + 1)}{2}
\]

Sum in terms of ( d ) and ( s ): \[
\Sigma = \frac{d + 1}{2} s, \qquad
\Sigma = \frac{sd + s}{2}
\]

Expanded forms: \[
\Sigma = \frac{u^2 + u}{2} - \frac{l^2 - l}{2}
\]

\[
\Sigma = \frac{(u^2 + u) - (l^2 - l)}{2}, \qquad
\Sigma = \frac{u^2 - l^2 + u + l}{2}, \qquad
\Sigma = \frac{u^2 - l^2 + s}{2}
\]

\section{🔢 Integer Sequences and Special
Transformations}\label{integer-sequences-and-special-transformations}

\subsection{🟥 Consecutive Odd Integers (1 \ldots{}
u)}\label{consecutive-odd-integers-1-u}

\[
\Sigma_o = \left( \frac{u + 1}{2} \right)^2 = \frac{u^2 + 2u + 1}{4}
\]

\subsection{🟦 Consecutive Even Integers (2 \ldots{}
u)}\label{consecutive-even-integers-2-u}

\[
\Sigma_e = \left( \frac{u}{2} \right) \left( \frac{u}{2} + 1 \right) = \frac{u(u + 2)}{4} = \frac{u^2 + 2u}{4}
\]

\subsection{🧮 Count of Consecutive Integers in Range (1 \ldots{}
u)}\label{count-of-consecutive-integers-in-range-1-u}

\[
\left\lfloor \frac{u - l}{2} \right\rfloor + 1 \quad \text{for odd or even spacing}
\]

or for total count of integers: \[
\left\lfloor \frac{u - l}{2} \right\rfloor \cdot 2 + 1 = 2\left\lfloor \frac{u - l}{2} \right\rfloor + 1
\]

\subsection{📉 Percent Difference}\label{percent-difference}

\[
\%\Delta = 100 \cdot \frac{\text{new} - \text{old}}{\text{old}} = 100 \left( \frac{\text{new}}{\text{old}} - 1 \right)
\]

\subsection{🔁 Involutive
Transformation}\label{involutive-transformation}

Given: \[
y = \frac{1 - x}{1 + x}
\]

Then: \[
x = \frac{1 - y}{1 + y}
\]

This is its own inverse: ( f(f(x)) = x )

\subsection{🔄 Percentage Inversion (Reciprocal
Percentages)}\label{percentage-inversion-reciprocal-percentages}

If \(x\% \text{ of } y = y\% \text{ of } x\), then:

\[
\frac{x}{100} \cdot y = \frac{y}{100} \cdot x, \qquad
x \cdot \frac{y}{100} = y \cdot \frac{x}{100}
\]

\section{\texorpdfstring{🟨 Generalized Metallic Mean for Any Integer
\(x\)}{🟨 Generalized Metallic Mean for Any Integer x}}\label{generalized-metallic-mean-for-any-integer-x}

The traditional formula is:

\[
N_x = \frac{x + \sqrt{x^2 + 4}}{2}
\]

This generalizes the golden ratio ((x = 1)), silver ratio ((x = 2)),
bronze ratio ((x = 3)), etc.

Alternative equivalent form: \[
N_x = \sqrt{1 + \frac{x^2}{4}} + \frac{x}{2}
\]

This version is symmetric and may be more intuitive in nested radical
systems.

\section{📈 Point--Slope of a Line}\label{pointslope-of-a-line}

Given two points:

\[
\begin{gather}
P_1 = (x_1, y_1) \\[6pt]
P_2 = (x_2, y_2)
\end{gather}
\]

\subsection{🧮 Slope Formula}\label{slope-formula}

\[
m = \frac{y_1 - y_2}{x_1 - x_2}
\]

\subsection{🧾 Point--Slope Form}\label{pointslope-form}

\[
y - y_1 = m(x - x_1)
\]

Or rearranged: \[
y = m(x - x_1) + y_1
\]

\subsection{🎯 Slope--Intercept Form}\label{slopeintercept-form}

\[
y = mx + b
\]

\subsection{🔃 Converting from Point--Slope to
Slope--Intercept}\label{converting-from-pointslope-to-slopeintercept}

\[
\begin{align}
\text{Start with point–slope:} \quad & y - y_1 = m(x - x_1) \\[6pt]
\text{Distribute the slope:} \quad & y = mx - mx_1 + y_1 \\[6pt]
\text{Group constants:} \quad & y = mx + (y_1 - mx_1) \\[6pt]
\text{Therefore:} \quad & b = y_1 - mx_1
\end{align}
\]

\begin{quote}
🧠 \textbf{Note:} The subscripts vanish because their values get
absorbed into the constant ( b ).\\
The slope--intercept form still ``remembers'' your point --- just more
subtly.
\end{quote}

\section{🧪 Example 1}\label{example-1}

Given:\\
- ( P\_1 = (4.85, 0) )\\
- ( P\_2 = (1, 1) )

\subsection{Step 1: Find the slope}\label{step-1-find-the-slope}

\[
m = \frac{0 - 1}{4.85 - 1} = \frac{-1}{3.85}
\]

\subsection{Step 2: Use point--slope
form}\label{step-2-use-pointslope-form}

\[
y - 0 = \frac{-1}{3.85}(x - 4.85)
\]

\subsection{Step 3: Rearrange to slope--intercept
form}\label{step-3-rearrange-to-slopeintercept-form}

\[
y = \frac{-1}{3.85}x + \frac{4.85}{3.85} \approx -0.26x + 1.26
\]

\section{🧪 Example 2}\label{example-2}

Given:\\
- ( P\_1 = (0.5, 0) )\\
- ( P\_2 = (1, 1) )

\subsection{Step 1: Find the slope}\label{step-1-find-the-slope-1}

\[
m = \frac{0 - 1}{0.5 - 1} = \frac{-1}{-0.5} = 2
\]

\subsection{Step 2: Use point--slope
form}\label{step-2-use-pointslope-form-1}

\[
y - 0 = 2(x - 0.5)
\]

\subsection{Step 3: Rearrange to slope--intercept
form}\label{step-3-rearrange-to-slopeintercept-form-1}

\[
y = 2x - 1
\]

!{[}{[}Line Point-Slope Illustration.png\textbar500{]}{]}

\section{📉 Visualizing Point--Slope
Examples}\label{visualizing-pointslope-examples}

This plot shows the two lines derived in the examples above:

\begin{itemize}
\item
  \textbf{Blue Line}:\\
  From \(P_1 = (4.85, 0)\) and \(P_2 = (1, 1)\)\\
  \(y = -0.25974x + 1.25974\)
\item
  \textbf{Red Line}:\\
  From \(P_1 = (0.5, 0)\) and \(P_2 = (1, 1)\)\\
  \(y = 2x - 1\)
\end{itemize}

They intersect at the point (1, 1), which lies on both lines.

\section{🌡️ Temperature Scale
Equalities}\label{temperature-scale-equalities}

\subsection{🔁 Fahrenheit ↔ Kelvin}\label{fahrenheit-kelvin}

\textbf{Core conversion equations:} \[
C = K - 273.15, \qquad C = \frac{5}{9}(F - 32)
\]

So: \[
K - 273.15 = \frac{5}{9}(F - 32)
\]

Or rearranged: \[
K = \frac{5}{9}(F - 32) + 273.15
\]

And: \[
F = \frac{9}{5}(K - 273.15) + 32
\]

\subsection{🧪 Worked Example (Convert 255.372 K to
°F)}\label{worked-example-convert-255.372-k-to-f}

Start from: \[
K = 255.372
\]

Plug into conversion formula: \[
K - 273.15 = \frac{5}{9}(F - 32)
\]

\[
-17.7778 = \frac{5}{9}(F - 32)
\]

Multiply both sides by 9: \[
-160 = 5(F - 32)
\]

Divide by 5: \[
-32 = F - 32
\]

So: \[
F = 0^\circ\text{F}
\]

Alternatively, in reverse: \[
F = 0 \Rightarrow K = \frac{5}{9}(0 - 32) + 273.15 = -17.7778 + 273.15 = 255.372 \text{ K}
\]

\subsection{🌡️ Fahrenheit ↔ Celsius}\label{fahrenheit-celsius}

Standard conversion formulas: \[
F = \frac{9}{5}C + 32, \qquad C = \frac{5}{9}(F - 32)
\]

\subsubsection{🔁 Rearrangements:}\label{rearrangements}

From: \[
F = \frac{9}{5}C + 32
\]

Subtract 32: \[
F - 32 = \frac{9}{5}C
\]

Multiply both sides by 5: \[
5(F - 32) = 9C
\]

Divide by 9: \[
C = \frac{5}{9}(F - 32)
\]

\subsection{🧪 Worked Example (Convert -40°F to
°C)}\label{worked-example-convert--40f-to-c}

\[
C = \frac{5}{9}(-40 - 32) = \frac{5}{9}(-72) = -40
\]

So: \[
-40^\circ \text{F} = -40^\circ \text{C}
\] \#\#\# 🌡️ Kelvin ↔ Celsius The Kelvin and Celsius scales are offset
by a constant: \[
K = C + 273.15, \qquad C = K - 273.15
\]

\begin{quote}
📎 \textbf{Note:} The size of 1 degree is identical in both scales; only
the zero point differs. Water freezes at 0\,°C = 273.15\,K and boils at
100\,°C = 373.15\,K.
\end{quote}

\begin{quote}
📎 \textbf{Terminology Note:}\\
Temperatures on the Kelvin scale are written simply as \textbf{K},
without a degree symbol.\\
For example:\\
- Correct: \textbf{273.15 K} ✓ - Incorrect: \textbf{273.15\,°K} -- or --
\textbf{273.15 degrees Kelvin}
\end{quote}

\section{Abstract}\label{abstract-3}

\textbf{Major Topics:}\\
- Presents the \textbf{Euclidean Algorithm} as a systematic way to
compute the \textbf{greatest common divisor (GCD)} of two integers.\\
- Algorithm steps:\\
1. Start with integers a \textgreater{} b.\\
2. Divide a by b, record remainder r.\\
3. Replace a with b, b with r.\\
4. Repeat until r = 0. The last non-zero remainder is the GCD.\\
- Provides both modular notation (\(r = a \;mod\; b\)) and longhand
remainder calculation:\\
\[
  r = a - \left(b \times \left\lfloor \tfrac{a}{b} \right\rfloor \right)
  \]\\
- Includes a complete worked example (2436, 1172 → gcd = 4), with
step-by-step divisions.\\
- Notes applications in worldbuilding, especially for
\textbf{simplifying ratios} and finding integer relationships in
\textbf{synodic systems}.

\textbf{Key Terms \& Symbols:}\\
- \textbf{Euclidean Algorithm {[}NEW{]}.}\\
- \textbf{Greatest Common Divisor (GCD) {[}NEW{]}.}

\textbf{Cross-Check Notes:}\\
- Neither term appeared in canon previously.\\
- This file establishes both as new canonical entries.\\
- \textbf{Status:} {[}NEW{]} --- introduces Euclidean Algorithm and GCD
into WCB canon with worked examples.

\chapter{The Euclidean Algorithm}\label{the-euclidean-algorithm}

The Euclidean Algorithm provides an elegant and efficient (if somewhat
involved) way to compute the greatest common divisor (GCD) of two
integers. This is especially useful for simplifying ratios or finding
integer relationships when work with synodic systems

\section{The Algorithm}\label{the-algorithm}

The algorithm itself is deceptively short and simple: 1. Start with two
integers \(a\) and \(b\), such that \(a>b\). 2. Divide \(a\) by \(b\),
and note the \emph{remainder}, \(r\). - \(r = a\; MOD\; b\) 3. Replace
\(a\) with \(b\) and \(b\) with \(r\). 4. Repeat steps 2 - 3 until
\(f = 0\).

The last \emph{non-zero remainder} is the greatest common divisor
between \(a\) and \(b\).

\subsection{Example}\label{example-3}

\[
\begin{gather}
a = 2436 \qquad b = 1172 \\
r = a\;mod\;b = 92 \\[1em]
a = 1172 \qquad b = 92 \\
r = a\;mod\;b = 68 \\[1em]
a = 92 \qquad b = 68 \\
r = a\;mod\;b = 24 \\[1em]
a = 68 \qquad b = 24 \\
r = a\;mod\;b = 20 \\[1em]
a = 24 \qquad b = 20 \\
r = a\;mod\;b = 4\; ✓ \\[1em]
a = 20 \qquad b = 4 \\
⟶ r = a\;mod\;b = 0 \\
\end{gather}
\] If you don't have a tool that directly calculates modulos, \(r\) can
manually be calculated by: \[
r = a - \left( b \times \left\lfloor \dfrac{a}{b} \right\rfloor \right)
\] Here is an admittedly clunky longhand version for clarity \[
\begin{array}{r@{}l}
   2 \quad &\text{\scriptsize(Quotient)} \\[-0.2ex]
1172\,)\,\overline{2436}  \quad &\text{\scriptsize(Dividend)} \\[-0.4ex]
\underline{-2344}  \quad &\text{\scriptsize(Subtracted: $2\times1172$)} \\[-0.3ex]
\phantom{ }\,\,92 \quad &\leftarrow \text{\scriptsize(Remainder)}
\end{array}
\] \[
\begin{array}{r@{}l}
   12 \quad &\text{\scriptsize(Quotient)} \\[-0.2ex]
92\,)\,\overline{1172}  \quad &\text{\scriptsize(Dividend)} \\[-0.4ex]
\underline{-1104}  \quad &\text{\scriptsize(Subtracted: $12\times92$)} \\[-0.3ex]
\phantom{}\,\,68 \quad &\leftarrow \text{\scriptsize(Remainder)}
\end{array}
\] \[
\begin{array}{r@{}l}
   1 \quad &\text{\scriptsize(Quotient)} \\[-0.2ex]
68\,)\,\overline{92}  \quad &\text{\scriptsize(Dividend)} \\[-0.4ex]
\underline{-68}  \quad &\text{\scriptsize(Subtracted: $1\times68$)} \\[-0.3ex]
\phantom{}\,\,24 \quad &\leftarrow \text{\scriptsize(Remainder)}
\end{array}
\] \[
\begin{array}{r@{}l}
   2 \quad &\text{\scriptsize(Quotient)} \\[-0.2ex]
24\,)\,\overline{68}  \quad &\text{\scriptsize(Dividend)} \\[-0.4ex]
\underline{-48}  \quad &\text{\scriptsize(Subtracted: $2\times24$)} \\[-0.3ex]
\phantom{}\,\,20 \quad &\leftarrow \text{\scriptsize(Remainder)}
\end{array}
\] \[
\begin{array}{r@{}l}
   1 \quad &\text{\scriptsize(Quotient)} \\[-0.2ex]
20\,)\,\overline{24}  \quad &\text{\scriptsize(Dividend)} \\[-0.4ex]
\underline{-20}  \quad &\text{\scriptsize(Subtracted: $1\times20$)} \\[-0.3ex]
\phantom{}\,\,\mathbf{4} \; \boldsymbol{\checkmark} &\leftarrow \text{\scriptsize(Remainder)}
\end{array}
\] \[
\begin{array}{r@{}l}
   5 \quad &\text{\scriptsize(Quotient)} \\[-0.2ex]
4\,)\,\overline{20}  \quad &\text{\scriptsize(Dividend)} \\[-0.4ex]
\underline{-20}  \quad &\text{\scriptsize(Subtracted: $5\times4$)} \\[-0.3ex]
\phantom{}\,\,0 \quad &\leftarrow \text{\scriptsize(Remainder)}
\end{array}
\] \[
\text{Since this remainder is } 0 \text{, the previous remainder is the GCD}
\] \[
\begin{array}{c c c}
\therefore \; \gcd(2436,1172) = 4 
\end{array}
\]

\chapter{Abstract}\label{abstract-4}

\textbf{Major Topics:}\\
- Clarifies the relationship between informal \textbf{mononic terms}
(micromon, minimons, midimons) and the formal \textbf{symbolic terran
mass intervals} (demiterran, pentiterran, milliterran, centiterran,
etc.).\\
- Establishes boundary conditions:\\
- \textbf{micromons:} bodies below the demiterran scale.\\
- \textbf{minimons:} spans demiterran + pentiterran ranges.\\
- \textbf{midimons:} equivalent to milliterrans.\\
- \textbf{planemons:} strictly ≥ centiterran (planetary-mass).\\
- Provides worked mapping with Solar System bodies (e.g., Ceres, Pluto,
Titan, Earth).\\
- Emphasizes distinction between \textbf{monons} vs.~\textbf{planemons}:
micromons, minimons, and midimons are monons but \textbf{not}
planetary-mass objects.

\textbf{Key Terms \& Symbols:}\\
- \textbf{micromon {[}neo{]}.}\\
- \textbf{minimon {[}neo{]}.}\\
- \textbf{midimon {[}neo{]}.}\\
- \textbf{Morphotype--Terran Mapping {[}meta{]}.}

\textbf{Cross-Check Notes:}\\
- Complements \textbf{Meta 1 --- Principles.md} (morphotype definitions)
without revising them.\\
- Expands practical guidance for how informal mononic terms overlap with
symbolic terran intervals.\\
- No contradictions with existing canon.\\
- \textbf{Status:} {[}EXPANDED{]} clarification, not revision.

\section{📎 Sidebar: Revised Morphotype
Mapping}\label{sidebar-revised-morphotype-mapping}

Asteroidal and planetary bodies can be classified by \textbf{morphotype}
--- a size/mass tier that reflects their gravitational and geophysical
character.

\subsection{\texorpdfstring{🔹 \textbf{micromons} →
Microterrans}{🔹 micromons → Microterrans}}\label{micromons-microterrans}

\textbf{Mass:} \textless{} 0.0001 M⨁ - \emph{Example:} Mimas (0.0000063
M⨁)

\subsection{\texorpdfstring{🔹 \textbf{minimons} → Demi- and
Pentiterrans}{🔹 minimons → Demi- and Pentiterrans}}\label{minimons-demi--and-pentiterrans}

\textbf{Mass:} 0.0001--0.001 M⨁ (demi) to 0.001--0.01 M⨁ (penti) -
\emph{Examples:}\\
- Ceres (0.000157 M⨁, demiterran)\\
- Haumea (0.00067 M⨁, demiterran)

\subsection{\texorpdfstring{🔹 \textbf{midimons} →
Milliterrans}{🔹 midimons → Milliterrans}}\label{midimons-milliterrans}

\textbf{Mass:} 0.001--0.01 M⨁ - \emph{Examples:}\\
- Pluto (0.0022 M⨁)\\
- Eris (0.0028 M⨁)\\
- Triton (0.0036 M⨁)\\
- Europa (0.0080 M⨁)

\subsection{\texorpdfstring{🔹 \textbf{planemons} → Centiterrans and
Up}{🔹 planemons → Centiterrans and Up}}\label{planemons-centiterrans-and-up}

\textbf{Mass:} ≥ 0.01 M⨁ - \emph{Examples:}\\
- Luna (0.0123 M⨁)\\
- Io (0.015 M⨁)\\
- Callisto (0.018 M⨁)\\
- Titan (0.0225 M⨁)\\
- Ganymede (0.0248 M⨁)\\
- Mercury (0.055 M⨁)\\
- Mars (0.107 M⨁)\\
- Earth (1.0 M⨁)\\
- Rosetta-class worlds (fictional)

\subsection{\texorpdfstring{📖 \textbf{Worldbuilder
takeaway}}{📖 Worldbuilder takeaway}}\label{worldbuilder-takeaway}

\begin{itemize}
\tightlist
\item
  \textbf{micromons (microterrans):} moonlets, fragments.\\
\item
  \textbf{minimons:} sub-planet dwarfs (Ceres/Haumea-scale).\\
\item
  \textbf{midimons (milliterrans):} the iconic ``major dwarfs and big
  icy moons'' (Pluto, Europa, Triton).\\
\item
  \textbf{planemons:} planetary-scale bodies (Luna and up).
\end{itemize}

\subsection{\texorpdfstring{\textbf{Solar System Bodies Below 0.02
M⨁}}{Solar System Bodies Below 0.02 M⨁}}\label{solar-system-bodies-below-0.02-m}

(using WCB symbolic mass intervals) - \textbf{Centiterrans (0.01--0.1
M⨁)}\\
- Callisto → 0.018 M⨁\\
- Io → 0.015 M⨁\\
- Luna → 0.0123 M⨁ - \textbf{Milliterrans (0.001--0.01 M⨁)}\\
- Europa → 0.008 M⨁\\
- Triton → 0.0036 M⨁\\
- Eris → 0.0028 M⨁\\
- Pluto → 0.0022 M⨁ - \textbf{Demiterrans (0.0001--0.001 M⨁)}\\
- Haumea → 0.00067 M⨁\\
- Ceres → 0.000157 M⨁ - \textbf{Pentiterrans (0.00001--0.0001 M⨁)}\\
- Vesta → 0.000043 M⨁\\
- Enceladus → 0.000018 M⨁ - \textbf{Microterrans (0.000001--0.00001
M⨁)}\\
- Mimas → 0.0000063 M⨁

\chapter{Abstract}\label{abstract-5}

\textbf{Major Topics:}\\
- Explores challenges of interstellar travel across vast distances,
starting from sublight limitations (Voyager, relativistic travel).\\
- Surveys \textbf{FTL drive archetypes} across fiction and theory:\\
- \textbf{Warp drive (ST:TOS, ST:TNG, Voyager)} with formulas and
time-to-destination calculations.\\
- \textbf{Alcubierre metric {[}sci{]}:} spacetime bubble mechanics,
energy constraints, time dilation consequences.\\
- \textbf{Hyperdrives, jump drives, fold engines, wormholes {[}exo{]}:}
speculative traversal models with narrative implications.\\
- Worked worldbuilding examples:\\
- \textbf{Vault drive (Serelem Union):} chained jumps with recharge
mechanics.\\
- \textbf{Kaltai Entente:} civilizational reach limited by FTL lag.\\
- Detailed treatment of \textbf{relativistic effects}: Lorentz term,
time dilation, length contraction, relativistic mass.\\
- Implications for \textbf{culture, politics, and storytelling}: bioage
vs.~scope vs.~span, asynchronous signals vs.~returning ships.\\
- Hazards of relativistic travel: \textbf{microparticle impacts},
red/blue shift of starlight, radiation exposure.

\textbf{Key Terms \& Symbols:}\\
- \textbf{γ (gamma):} Lorentz factor {[}sci{]}.\\
- \textbf{β (beta):} velocity ratio v/c {[}sci{]}.\\
- \textbf{Vault drive {[}neo{]}:} hypothetical chained-jump FTL with
recharge intervals.\\
- \textbf{Bioage, scope, span {[}neo{]}:} narrative distinctions between
biological, proper, and chronological age under relativistic effects.

\textbf{Cross-Check Notes:}\\
- No collisions with existing canon.\\
- Expands \textbf{Meta 2 --- Math Tools} by applying relativistic
equations to travel/narrative contexts.\\
- Introduces new {[}neo{]} and {[}exo{]} terms, but all clearly
marked.\\
- Serves as a \textbf{bridge between real physics (sci) and fictional
FTL (exo/neo)} in WCB.

\chapter{Interstellar Distances}\label{interstellar-distances}

The universe is a very big place --- very, very big; okay, very, very,
\emph{very} big.

Our Milky Way galaxy comprises a very large volume (from our
perspective), most of which is empty space. The stellar density in the
Solar neighborhood is about 0.004 stars per cubic lightyear. That is to
say, on average, every star has 250 cubic light-years of empty space
surrounding it. This equates to a cube 6.3 light-years on a side with
the star in the center, or a sphere with a radius of 3.91 light-years.
This seems like a reasonable figure, since the closest star to Sol is
Proxima Centauri, at about 4.2 light-years away.

The bright star, Deneb (α Cygni) is 2600 light-years from the Sun.
That's not a long distance compared to the galactic neighborhood (the
Andromeda galaxy is 2.5 \emph{million} light-years away), and certainly
not on the scale of the visible universe, which is currently calculated
to be 93 \textbf{\emph{billion}} light-years across.

If we think about what the term ``light-year'' actually \emph{means}, we
begin to intuit a problem. A light-year is the distance light travels in
one Earth year, so light takes a little under 4 years and 2½ months to
get here from there. Light (the fastest thing in the universe) coming
from Deneb takes 2,600 \emph{years} to reach us from there. In terms of
human timeframes, light we see from Deneb today started its journey here
the year Cambyses I became King of Persia. \emph{Never heard of him?
That's kind-of my point.}

Currently, the most distant human-made object in the universe is Voyager
1, which also has the fastest heliocentric recession speed of any
artificial object at about \(17\) kilometers per second. It did its
flyby of Saturn in 1980, and NASA officially marksedthe start iof ts
journey out of the Solar system on January 1, 1990, almost \(35\) years
ago, as of this writing. In that time, it has traversed \(168\)
Astronomical Units. One AU is the distance from the Earth to the Sun,
which is, on average, \(146.9\) million kilometers. That puts Voyager
\(25,132,458,000\) (25 \emph{billion}) kilometers away.

That is a distance of \(\mathbf{0.0027}\) light-years. Yes, you read
that right. In \(35\) years, Voyager I has covered about
\(\mathbf{\dfrac{1}{370}}\) of a light-year. It will take in another
\(16483.74\) years to travel a full light-year, and when it reaches that
distance, there will still be about \(3\) light-years of empty space
around it in every direction. It will take \(70217.39\) years to reach
the distance of Proxima Centauri (though Voyager 1 isn't headed to
Proxima Centauri).

Obviously, the propulsion methods we currently have at our disposal are
not going to get us to the stars any time soon.

One way to circumvent this problem is to find ways to travel faster than
even light does. (Can somebody please quiet down Einstein in the back of
the room there?). Drives which accomplish faster-than-light (FTL)
velocities have been posited in a wide variety of science fiction
novels, movies, television shows, etc. \emph{Hyperdrive}, among the
earliest (in film at least) was mentioned in \emph{Forbidden Planet}
(1956), (more on thisbelow), but probably the first FTL system to become
widely known (and copiously copied) was certainly \emph{Star Trek}'s
``warp drive''.

\chapter{Warp Drive Is (Woefully) Too
Slow}\label{warp-drive-is-woefully-too-slow}

In the Star Trek universe of the \emph{Star Trek: The Original Series}
(\emph{ST:TOS}), the speed at which starships could travel was measured
in ``warp factors'', such that warp factor 1 was the exact speed of
light. Each succeeding warp factor was calculated by the cube of the
warp factor multiplied by the speed of light (c). \[
v = f^3c
\] Where: - \(f\) = warp factor number - v = the speed of the ship in
meters/second - c = the speed of light (299792458 m/s)

If \(c\) is taken as 1.0, then \(v\) is returned in multiples of
lightspeed. For example

Given: - \(f\) = 3 - \(c\) = 1 \[
v = f^3 = 3^3 = 27\; \text{times the speed of light}
\] - Note that when \(c\) = 1, it can be dropped from the equation.

Calculating the warp factor from the speed is a function of the
cube-root of the velocity of the ship and the speed of light: \[
f = \sqrt[3]\frac{v}{c}
\] - Here, again if one is expressing \(v\) as multiples of lightspeed,
\(c\) is always 1 and the equation can simplify to: \[
f = \sqrt[3]{v}
\] \#\# ``I'm givin' 'er all she's got, Cap'n!'' The maximum at which
Constitution-class ships in \emph{ST:TOS} could travel --- and then only
for short periods of time --- was warp 8, which is a velocity of: \[
v = f^3 = 8^3 = 512\; \text{times the speed of light}
\] This seems fast until we start considering actual interstellar
distances.

Proxima Centauri is the closest star to the Sun at 4.24 light years
distant. At warp 8, Constitution-class starships take: \[
T = \frac{4.25}{512} = 0.00828]; \text{years} ≈ 3.025\; \textit{days} 
\] to make the journey. Days. Not minutes; not hours --- \emph{days}.

\section{Boosting The Speed}\label{boosting-the-speed}

In \emph{Star Trek: The Next Generation} (\emph{ST:TNG}), the assumption
was that warp technology had been improved upon since the days of Kirk
and Spock, and warp factors were redefined to be: \[
v = f^\frac{10}{3}
\] So now, warp 8 is: \[
v = f^\frac{10}{3} = 8^\frac{10}{3} = 1024c
\] \ldots{} which exactly doubles the old warp 8 velocity and shortens
the trip to Proxima Centauri to 1.5124 days, or about 36.3 hours. Still
not an afternoon stroll.

The star Deneb is 2600 lightyears from the Sun. At \emph{ST:TOS} warp 8,
the \emph{Enterprise} would take \[
T = \frac{2600}{512} = 5.078\; \textit{years}
\] to get there. The \emph{ST:TNG} \emph{Enterprise}, at its warp 8 cuts
the trip in half to a mere 2.54 \emph{years}.

And, yet, in the very first episode of \emph{ST:TNG}, ``Encounter at
Farpoint'' (1987), the mission destination is stated to be Deneb IV.
Such a designation is usually assumed to refer to the fourth planet
orbiting a star named ``Deneb''. Based on this, we can assume Picard and
company's destination was the Deneb we just calculated a 2.54 year
one-way journey to.

Further more the \emph{Star Trek} wiki site, Memory Alpha states that
``By 2373, the Federation's territory was spread across 8000
lightyears\ldots.'' Assuming this to be a circular diameter, it would
take \emph{ST:TNG Enterprise} \[
T = \frac{8000}{1024} = 7.8125\; \textit{years}
\] to traverse the width of the Federation.

\begin{itemize}
\tightlist
\item
  Note: If we take that to mean 8000 \emph{cubic} lightyears, that's a
  spherical volume of only about \[
  r = \sqrt[3]{\frac{3V}{4\pi}} = \sqrt[3]{\frac{3\cdot8000}{4\pi}} = \sqrt[3]{\frac{24000}{4\pi}} = \sqrt[3]{1909.86} ≈ 12.41\; \text{lightyears}
  \] \ldots{} in radius, which seems a bit on the small side, doesn't
  it?
\end{itemize}

\emph{Star Trek: Voyager (ST:VOY)} used the same warp formula as
\emph{ST:TNG}, \textbf{up to warp 9}, and then an unspecified
exponential curve thereafter, such that warp 10 is infinite velocity.
Most fan sources estimate \emph{Voyager}'s top speed of 9.975 to be
3000c, but that still means that a cross-Federation journey would take:
\[
T = \frac{8000}{3000} = 2.67\; \textit{years}
\] The Sol-Deneb run mentioned earlier would take: \[
T = \frac{2600}{3000} = 0.867\; \textit{years} ≈ 316.6\; \textit{days}
\] \ldots{} which is just over 10 standard months\ldots{} and that's
\emph{assuming no stops along the course}.

Clearly while \emph{Star Trek} warp drive is \emph{fast}, interstellar
distances make the travel times comparable to steamships crossing the
Atlantic in the early 20th Century.

\chapter{Other Kinds of Drive
Systems}\label{other-kinds-of-drive-systems}

Other franchises have, in fact, made use of far speedier craft.
\emph{The Orville} frequently uses the term ``quantum drive'' to
describe their FTL drive; \emph{Star Wars} talks about ``hyperdrives'';
The \emph{FreeSpace} game series uses ``subspace drive''; the (sadly
short-lived) \emph{Dark Matter} regularly referred to the \emph{Raza}'s
``FTL Drive'', until they acquired theit ``blink drive'' (more on this
below).

Not all of these go so far as to give actual formulas for calculating
drive speeds; \emph{The Orville} is an exception: captain Ed Mercer
mentions in the first season episode ``Pria'', ``We have a \ldots{}
quantum drive system capable of speeds exceeding 10 light years per
hour,'' which equates to a ST:TOS warp factor of about 44.422. At that
speed, a cross-Federation ship would take the \emph{Orville} 33.33 days,
again something akin to a steamship trip across the Atlantic Ocean.

Let's see how other universes' drives stack up when normalized to
\emph{Trek}'s warp scale.

\begin{longtable}[]{@{}
  >{\raggedright\arraybackslash}p{(\linewidth - 8\tabcolsep) * \real{0.1768}}
  >{\raggedright\arraybackslash}p{(\linewidth - 8\tabcolsep) * \real{0.2486}}
  >{\raggedleft\arraybackslash}p{(\linewidth - 8\tabcolsep) * \real{0.2044}}
  >{\raggedleft\arraybackslash}p{(\linewidth - 8\tabcolsep) * \real{0.1657}}
  >{\raggedleft\arraybackslash}p{(\linewidth - 8\tabcolsep) * \real{0.2044}}@{}}
\toprule\noalign{}
\endhead
\bottomrule\noalign{}
\endlastfoot
Light (for comparison) & Normal Universe & 1 & 1 & 1 \\
Alcubierre Metric & Theorized Normal Universe & 10 & 2.15 & 2.00 \\
\emph{USCSS Prometheus} & \emph{Prometheus} & 19.5 & 2.69 & 2.44 \\
\emph{FireBall XL5} & \emph{Fireball XL5} & 24 & 2.88 & 2.59 \\
\emph{USCSS Nostromo} & \emph{Alien} & 44 & 3.53 & 3.11 \\
Whorfin-Class Starships & \emph{Star Trek VII(Generations)} & 64 & 4.00
& 3.48 \\
\emph{The Ark} & \emph{Transformers} & 115 & 4.86 & 4.15 \\
\emph{USS Enterprise} (NX-01) & \emph{Star Trek:Enterprise} & 129.6 &
5.06 & 4.30 \\
D5 Battle Cruiser & \emph{Star Trek: Enterprise} & 216 & 6.00 & 5.02 \\
\emph{USS Sulaco} & \emph{Aliens} & 271 & 6.47 & 5.37 \\
\emph{USS Enterprise} (NCC-1701{[}-A{]}) & \emph{Star Trek:The Original
Series} & 512 & 8.00 & 6.50 \\
\emph{USS Enterprise} (NCC-1701-B) & \emph{Star Trek III(The Search for
Spock)} & 729 & 9.00 & 7.22 \\
\emph{Pillar of Autumn} & \emph{Halo} & 959 & 9.86 & 7.84 \\
\emph{USS Enterprise} (NCC-1701-D) & \emph{Star Trek:The Next
Generation} & 1,649 & 11.81 & 9.23 \\
\emph{USS Defiant} (NCC-1764) & \emph{Star Trek:Deep Space Nine} & 1,816
& 12.20 & 9.50 \\
\emph{USS} Voyager & \emph{Star Trek:Voyager} & 2,137 & 12.88 & 9.98 \\
Nomad Probe & \emph{Star Trek:The Original Series} & 4,096 & 16.00 &
12.13 \\
Borg Cube & \emph{Star Trek(various)} & 7,912 & 19.93 & 14.77 \\
Prawn Mothership & \emph{District 9} & 11,119 & 22.32 & 16.36 \\
Karla Five Vessel & \emph{Star Trek:The Animated Series} & 46,656 &
36.00 & 25.16 \\
\emph{USS Orville} & \emph{The Orville} & 87,660 & 44.42 & 30.40 \\
\emph{Eagle 5} & \emph{Space Balls} & 89,998 & 44.81 & 30.64 \\
\emph{Ascendant Justice} & \emph{Halo} & 333,164 & 69.32 & 45.37 \\
\emph{Moya} & \emph{Farscape} & 365,250 & 71.48 & 46.64 \\
The \emph{GunStar} & \emph{The Last Starfighter} & 701,280 & 88.84 &
56.72 \\
The \emph{Intruder} & \emph{Valerian and the City of a Thousand Planets}
& 993,480 & 99.78 & 62.97 \\
The \emph{Death Star} & \emph{Star Wars(various)} & 1,142,500 & 104.54 &
65.67 \\
\emph{Roger Young} & \emph{Starship Troopers} & 1,300,000 & 109.14 &
68.26 \\
\emph{Galactica BS-75} & \emph{Battlestar Galactica} & 1,680,000 &
118.88 & 73.72 \\
\emph{Axiom} & \emph{Wall-E} & 2,190,000 & 129.86 & 79.82 \\
Imperial II-Class Star Destroyer & \emph{Star Wars(various)} & 2,285,000
& 131.71 & 80.85 \\
Trimaxion Drone Ship & \emph{Flight of the Navigator} & 4,460,000 &
164.61 & 98.81 \\
Samus Gunship & \emph{Metroid} & 6,500,000 & 186.63 & 110.63 \\
\emph{Millenium Falcon} & \emph{Star Wars(various)} & 9,130,000 & 209.01
& 122.50 \\
Bes'Uliik Starfighter & \emph{Star Wars(various)} & 11,412,500 & 225.14
& 130.98 \\
\emph{Daedalus} & \emph{Stargate SG-1} & 60,875,000 & 393.38 & 216.44 \\
The \emph{Milano} & \emph{Guardians of the Galaxy} & 5,775,040,800 &
1794.12 & 848.14 \\
\end{longtable}

These were gathered from a plethora of sources around the internet. I'm
certain there will be disagreements about them; I've only included them
for illustration purposes, not to reflect any kind of ``canon''.

To convert a ship's speed in multiples of the speed of light to a
\emph{Star Trek} warp factor: \[
f_{TOS} = \sqrt[3]{v} \qquad f_{TNG} = \sqrt[\frac{10}{3}]{v}
\] For example the \emph{Orville}, taking 10 light years per hour as the
standard, would have a ``c-speed'' of: \[
\begin{align}
&\frac{10 \text{ly}}{hour} \times 8766\;\text{hours per year} = 
\end{align}
\] So it's equivalent warp factors would be: \[
\begin{align}
f_{TOS} &= \sqrt[3]{87660} = \mathbf{44.42} \\
f_{TNG} &= \sqrt[\frac{10}{3}]{87660} = \mathbf{30.4}
\end{align}
\] \#\# The Devil's In The Details There were (frequent) inconsistencies
in \emph{ST:TOS}. For instance, in the episode ``By Any Other Name''
(2/23/1968), the Kelvan commander, Rojan, states that the
\emph{Enterprise} engines would be modified to allow travel to the
Andromeda Galaxy in 300 Earth-years, and that the ship's cruising speed
would thus be warp 11.

However, Andromeda is 2.5 million lightyears away. Making the journey in
300 years would mean a velocity of: \[
v = \frac{2500000}{300} = 8333.33\tfrac{ly}{y}
\] \ldots{} which equates to a TOS warp factor of: \[
f_{TOS} = \sqrt[3]{8333.33} = \mathbf{20.274}
\] \ldots{} which is 184.3\% of warp 11.

The \emph{actual velocity} at warp 11 is: \[
v = 11^3 = 1331c
\] \ldots{} which would mean the trip would take: \[
v = \frac{2500000}{1331} = \mathbf{1878.287}\;\textbf{years}
\] This is a prime teaching example for WCB: either a) the writers did
not have an accurate distance value for Andromeda; b) they did their
math wrong; c) they didn't bother with the math and wrote dialogue that
sounded cool.

\chapter{Forbidden Planet}\label{forbidden-planet}

Perhaps ironically, one of the most accurate-seeming depictions of an
FTL drive comes from one of the earliest: \emph{Forbidden Planet}
(1956). The ship, United Planets \emph{C57D}, was said to use a
``hyper-drive''. I cannot find any specific numbers for its top
velocity, but some clues in the narrative and dialogue are available.
The opening voice-over says that the ship was ``\ldots{} more than a
year out from Earth base.'' on a mission to Altair IV, the fourth planet
of the star, Altair (α Aquilae). Later, Commander Adams exasperatedly
explains to Altaira Morbius that she needs to wear less revealing
outfits because is crew has been ``\ldots{} locked up in hyperspace for
378 days,'' which is 1.035 years.

The modern distance to Altair from Earth is measured at 16.73
light-years. From this, we can calculate that the \emph{C57D} traveled:
\[
v = \frac{D}{T} = \frac{16.73}{1.035} ≈ 16.164\;\text{ligh-years per year} 
\] It seems reasonable to suppose that the \emph{intended} velocity was
meant to be a nice, round \(16\) lightyears per year, which would
indicate that the writers used: \[
D = v \times T = 16 \times 1.035 = 16.56\;\text{light-years}
\] as the distance to Altair. A velocity of \(16\) light-years per year
equates to a ST:TOS warp factor of: \[
f = \sqrt[3]{v} = \sqrt[3]{16} = 2.52
\] This seems like an unusually rigorous early attempt to quantify FTL,
unlike later franchises that often hand-waved the numbers. More likely,
though, it was a lucky coincidence: the narrative hook of a long, lonely
voyage (378 days) paired with the choice of Altair --- a star whose name
conveniently provided an exotic but pronounceable basis for the female
lead, Altaira. Had Irving Block and Allen Adler (story) or Cyril Hume
(screenplay) been consciously designing a hyperdrive system, it would
make more sense for them to have chosen a clean round figure like 15
light-years per year or 20 light-years per year, in the same vein as
\emph{The Orville}'s ``ten light-years per hour.''

This is not to dismiss the breadth and creativity of the writers'
efforts. Rather, it highlights how ``the math'' of a milieu can often be
back-calculated from clues in dialogue or values glimpsed on a display,
even when the creators themselves were (rightly) more focused on
metaphor than mathematics.

Miguel Alcubierre floated the idea in 1994 of a real-world warp drive as
a scientific possibility, and we'd be remiss not to visit his theory
here.

Einstein's general relativity shows that a physical object is limited to
the speed of light \emph{within} spacetime, but spacetime itself can
expand, contract, or move without such restrictions. Alcubierre
hypothesized a system where space behind a ship is compressed (a ``wave
crest'') and space ahead is rarified (a ``trough''), creating an
energetic slope along which the ship rides --- much like a surfer
gliding on a wave.

The analogy is apt: ocean waves don't carry the water itself forward,
they make it undulate. A cork or buoy bobs in place while the wave
passes beneath. In the warp drive concept, the ship never moves through
local spacetime; instead, spacetime itself is carried forward with the
vessel inside.

Alcubierre's first model required infinite ``negative energy,'' but
later refinements reduced the demand to merely astronomical levels ---
the mass-energy of planets or stars, still far beyond our reach. The
concept remains speculative, and even potentially hazardous: some
physicists suggest that halting the bubble would release catastrophic
radiation, tearing apart the ship in a man-made supernova.

Still, the idea is tantalizing. Even a conservative Alcubierre drive
running at 10c (≈ warp 2.15, TOS; ≈ warp 2.0, TNG) would put Alpha
Centauri a little over five months from Earth. Within the Solar System,
0.1c would place Mars just hours away --- a vast improvement over the
year-long, radiation-bathed voyages chemical rockets imply.

Crucially, crew inside a warp bubble would experience \textbf{no
relativistic time dilation}: their five-month trip would feel like five
months. Like the crew of the \emph{C57D} in \emph{Forbidden Planet},
they'd still have to occupy themselves en route, but they wouldn't
return home displaced from their origin time by months, decades, or even
centuries, depending on the distance traveled.

For comparison, a ship traveling at 0.99c would take about \textbf{2626
years} to reach Deneb as measured from Earth, while only \textbf{367
years} would pass onboard --- still a generation ship, but 18
generations instead of 131. By contrast, an Alcubierre drive at 10c
would reach Deneb in just \textbf{260 years}, with both ship and Earth
clocks agreeing on the elapsed time --- the same 13 generations would
pass in both frames.

A fascinating sidelight, though, is that even if the \emph{ship} arrives
in only 260 years, its \emph{``we have arrived!''} signal --- still
limited to lightspeed --- would take \textbf{2600 years} to reach Earth.
Ironically, if the explorers also dispatched a ship back home, it would
return in \textbf{260 years}, arriving a full \textbf{2340 years before}
the radio message did!

This is a prime example of \textbf{world\emph{crafting}} meeting
\textbf{world\emph{building}}: the numbers are what they are, but the
drama lies in how people adapt to --- and live with --- those
inescapable realities.

Warp drives, hyperdrives, subspace drives, quantum drives, etc., all
share the common basic notion of \emph{actual, physical travel} through
spacetime (or some subrealm thereof). Even the Alcubierre metric assumes
forward ``motion'' within a ``bubble'' of ``normal'' spacetime.

Other creators have opted for different solutions to the time-distance
dilemma.

\section{Space Folding and Jump
Drives}\label{space-folding-and-jump-drives}

\subsection{Space Folding}\label{space-folding}

Perhaps the best-known example of ``space-folding'' comes from Frank
Herbert's \emph{Dune} universe. Spacing Guild Navigators, their mental
faculties expanded by the spice \emph{melange}, cause two distant points
in space to momentarily overlap, like bending a sheet of paper so that a
point 1'' from each opposing margin are brought into contact. At that
moment, a ship could physically relocate from origin to destination
without traversing the intervening void (which, for that moment, would
not actually exist). When space ``unfolds,'' the ship is simply
\emph{there}. For all practical purposes, the voyage is instantaneous
--- it takes longer to ride the shuttle up to orbit at origin and back
down at destination than it does to cross between worlds.

This, of course, is entirely speculative (or outright fantastical), and
even if realizable, would demand at least \textbf{Kardashev Type IV}
technologies --- the ability to manipulate energy on galactic scales.
Yet from a world\emph{building} perspective, \emph{instant travel}
shifts the storytelling focus: the challenge is no longer the journey,
but who controls the technology/mechanology of travel and
transportation. \#\#\# Jump Drives Jump drives accomplish much the same
function as space-folding, but by \textbf{mechanotechnological
(mechtech)} means rather than mystical or mental ones. The ``FTL jumps''
of \emph{Battlestar Galactica} (both versions) are a classic example, as
is the ``blink drive'' in \emph{Dark Matter}.

Depending on the needs of the setting, a jump drive may require
measurable amounts of time and energy, or it may be treated as
effectively instantaneous --- ``supertech'' that simply relocates a
ship. Likewise, jumps may be chained one after another, or require a
recharge interval before firing again.

From a world\emph{building} standpoint, these choices are critical: do
you want drama from fuel shortages, from cooldown tension, from misjumps
into the void --- or from the sheer ease of hopping anywhere instantly?
The math is malleable, but the \textbf{drama comes from the limits you
set.}

This is why such mechtech is often preferred by more speculative
writers: you don't have to wrestle with the intractable realities of
hard science. Instead, the \emph{means} recedes into the background,
letting the drama occupy the foreground --- focusing on the characters
and their story. \#\#\# Practicum With space-folding and jumpdrives,
etc., instead of determining a maximum velocity in terms of lightspeed,
one might specify a maximum traversable distance per jump, limited by
available drive power, fuel consumption, or whatever other constraint
one may choose to impose.

For instance, let's posit an interstellar civilization, the Serelem
Union which has developed a ``vault drive'' with the parameters: -
Individual vaults can span any distance between 1 and 100 light-years -
The more extensive the vault, the more energy required - The more
extensive the vault, the longer the drive takes to recharge, such that -
The vault-to-charge ratio is 1 minute of recharge per light-year of the
most recent previous vault.

Now let posit the Serelemene flagship, \emph{Tonqua}, is on a
defense/rescue mission to their most distant outpost on a planet called
Rylboru, \(4835\) lightyears from the homeworld. How long does it take
\emph{Tonqua} to get there (we'll be using Earth-normal time units here
for convenience)?

The number of vaults required depends on the chosen vault magnitude. At
one light-year per vault, it is \(4835\) vaults; at 100 light-years per
vault it is \(48.35\) vaults (or, more likely \(48\) vaults of \(100\)
light-years and a final vault of \(35\) light-years.). Let's say Captain
Talua decides to make \(100\) light-year vaults. That means that after
each vault, the \emph{Tonqua} will have to spend 100 minutes recharging
its drive before it can vault again. \[
\begin{align}
T &= 48\;\text{vaults} \times 100\;\text{minutes} = 4800\;\text{minutes} \\[0.5em]
&= \frac{4800\;\text{minutes}}{60\;\text{minutes/hour}} = 80\;\text{hours} \\[0.5em]
&= \frac{80\;\text{hours}}{24\;\text{hr/day}} = 3.\overline{3}\;\text{days} \\[0.5em]
&≈ 3\;\text{days, } 8\;\text{hours}
\end{align}
\] (We don't need to account for the recharge time after the last \(35\)
light-year vault, because it will transpire at the destination.)

This equates to an apparent velocity of: \[
\begin{align}
v &= \frac{D}{T} = \frac{4800}{3.\overline{3}} = 1440\;\text{light-years per day} \\[0.5em]
&\text{Remember that the last vault adds recharge time, but not travel time!} \\[0.5em]
&= 1440\;\text{ly/day} \times 365.25\;\text{days} \\[0.5em]
&= 525960\;\text{light-years per year} \\[0.5em]
&= 525960c \\[1.5em]
f_{TOS} &= \sqrt[\frac{10}{3}]{525960} = 80.721
\end{align}
\] To put it another way, a \emph{ST:TOS} ship at warp \(8\) would need
\(9.44\) days to cover the same distance.

\begin{quote}
Note that the travel time is the same regardless of how the vaults are
chosen, arranged. - \(1000\) vaults of 1 minute recharge each is still a
total of \(4800\) minutes of recharge - \(100\) vaults of \(24\)
light-years means \(2400\) minutes of recharge; completing the trip with
\(2400\) vaults of 1 light-year is another \(2400\) minutes of recharge,
for, again, a total of \(4800\) minutes of recharge.
\end{quote}

Looking back at our ship speed listing from earlier this makes
\emph{Tonqua} - About \(\frac{1}{17}\) times as fast as the
\emph{Millennium Falcon} - Approximately \(\frac{3}{4}\) as fast as the
\emph{GunStar} - Around \(1.44\) times faster than the \emph{Moya} -
\emph{Exactly} \(6\) times faster than the \emph{Orville} - More-or-less
\(548.45\) times faster than the \emph{PIllar of Autumn} -
\emph{Precisely} \(2435\) times faster than \emph{ST:TOS} warp 6 \#\#\#
Wormholes and Stargates

Travel by wormhole, as in \emph{Star Trek: Deep Space Nine} or
\emph{Stargate}, achieves the same end as space-folding or jump drives,
but by exploiting a hypothesized natural phenomenon: conduits in
spacetime that connect two distant points through a substrate (or
superstrate) of ``normal'' space.

Naturally occurring wormholes may have permanently fixed endpoints: you
enter one end of the Chunnel and you always exit the other. Or they may
be unstable: opening and closing at unpredictable intervals; fixed at
one end but wandering at the other; or even distorting time so that
traversal involves dilation --- or outright \emph{time travel}.

\begin{quote}
Imagine a milieu in which an advanced civilization builds a
galaxy-spanning civilization by means of a wormhole, unaware that one
end of the conduit is temporally displaced by millennia. If physical
travel between the two termini is impossible, would the discrepancy in
eras even \emph{matter}?
\end{quote}

Another dramatic limitation, as in \emph{Stargate}, is when wormholes
are generated by \textbf{mechtech} devices. In that case, both origin
and destination must host a physical machine --- which can itself be
relocated, creating the possibility of shifting or contesting endpoints.

As always, one may specify instantaneous or finite-time traversal, and
decide how such wormholes are powered. The physics can be left hand-wavy
--- what matters for world\_building\_ is how these limits shape drama,
conflict, and control.

The worldmaker will be confronted with a decision in designing
interstellar polities. You either: 1. Invent your space-traversing
technology, determine its limitations, and design the volumetric span of
your civilization based on the longest necessary travel-time; or, 2.
Create your milieu, including the size of your civilization's sphere of
influence, and then design a drive that lets your characters get around
that volume of space in reasonable amounts of time.

\section{Example: The Kaltai Entente}\label{example-the-kaltai-entente}

Let's specify a moderately sized elliptical galaxy (Shuf) of \(10\)
billion stars, spanning \(350000\) light-years. The Kaltai Entente is a
federated polity of cooperative star systems that extends \(1500\)
light-years in radius out from the central star (Kaltai). Thus, the
Entente commands a spherical volume of about \(14\) billion cubic
light-years, encompassing \(56.5\) \emph{million} star systems of which
about \(13\) \emph{million} are habitable/hospitable systems around F-,
G-, and K-type stars. That's a lot of space (and stars, and planets) to
watch over.

Let's further say that the military-supported trade syndicate
(KaltaiSys) has a monopoly on the modes of FTL space travel, and that
there are no space-folding drive, or jump drives, or wormhole, etc. All
travel is inherently time-consumptive, distance-over-rate. Their fastest
battle cruisers can make the trip between Kaltai and their most distant
colony at Taibano Raimos, 1500 light-years away, in 10 days.

Thus, these ships are capable of: \[
\begin{align}
r &= \frac{1500\;\text{light years}}{10\;\text{days}} = 150\;\text{ly/day} \\[0.5em]
v &= 150\;\text{ly/day} \times 365.25\;\text{days/yr} = 54787.5\;\text{light-years/yr} \\[0.5em]
&= 54787.5c
\end{align}
\] This equates to: - \emph{ST:TOS} warp \(37.9\) - About \(12.5\)\%
faster that a Wraith hive ship in \emph{Stargate: Atlantis}; and -
\(62.5\)\% of the top speed of the \emph{Orville}

If your game, story, or show can tolerate a \textbf{10-day lag} between
core and frontier, then this drive system fits your setting. But if you
need supplies, reinforcements, or diplomacy to move faster than that ---
and you don't want to shrink your sphere of influence --- you have only
two real options: - \textbf{Reconfigure your FTL.} Make the drive
faster, reduce recharge, or relax the limits you've set.\\
- \textbf{Invent another traversal process.} Wormholes, jump drives,
fold engines, or some other mechanism that bypasses the distance--rate
bottleneck altogether.

The Lorentz term is a factor of how much the local passage of time,
length (in the direction of motion), and relativistic mass change when
an object is in motion. \[
\gamma = \frac{1}{\sqrt{1-\beta^2}}
\] Where: - \(\beta\) = the velocity expressed as a percentage of the
speed of light \(^v/_c\).

All motion results in some magnitude of the Lorentz term (GPS satellites
have to take it into account to return accurate results). As velocity
increases, the Lorentz term increases exponentially.

This equation was developed by the Dutch Physicist Hendrik Lorentz in
the late \(19^{th}\) century as part of a broader set of equations
called the Lorentz transformations. He developed these as part of his
effort to determine whether or not a cosmic space-filling substance
(\emph{luminous æther}) existed. (It doesn't).

When it became clear that all electromagnetic phenomena, including
visible light, has wavelike properties, it was believed that all waves
needed a physical medium through which to propagate. It was reasoned,
then, that there must be a substance filling ``empty'' space in which
light waves moved. Yet all attempts to detect or measure the qualities
of this æther failed.

An early explanation of the failure of these experiments was that linear
distances might be compressed in the direction of motion due to pressure
from the æther building up against the forward faces of the object. If
this were true, then light waves and any measuring device would also be
compressed along the same vector and so a measurment would always return
the same value because all elements of the system were being altered by
the same ratio.

Lorentz developed his equations to calculate the amount of compression
that motion through the æther would cause. When Michelson and Morley's
experiments showed that the æther did not exist --- and did not need to
--- the math itself did not vanish. Lorentz had the equations right, but
the \emph{context} wrong. A generation later, Einstein reinterpreted
them: not as distortions caused by a cosmic medium, but as the natural
geometry of spacetime itself.

Einstein's recontextualizing of the Lorentz transformations shows that,
as the velocity of an object such as a space ship approaches the speed
of light, the ship's linear dimension in the direction of motion
contracts --- so that much of the theory \textbf{was} true, there just
wasn't any substance physically causing the contraction.

Einstein also realized that for the object in motion, time passes more
slowly by the same Lorentz factor (though to an observer on or in the
moving object, no change in the ``flow'' of time is apparent). And,
finally, the kinetic mass of the ship (as measured by an outside
observer), also increases by the same factor of the Lorentz term. Thus,
the speed of light, the passage of time, and mass are all constant for
every observer. A passenger onboard a ship with a velocity of 0.99c
would measure a rod as a meter long, while to an outside observer, the
rod would appear shorter by the factor: \[
\frac{1}{\gamma} = \sqrt{1 - 0.99^2} = \sqrt{1 - 0.98} = \sqrt{0.0199} = 0.1411
\] To the outside observer, the rod would appear to be only about
\(14.1\) cm in length.

\begin{quote}
Note: this shows that while length \emph{contracts}, time \emph{expands}
and mass \emph{increases}. To calculate - Length contraction: multiply
by \(\frac{1}{\gamma}\) - Time dilation (expansion): multiply by
\(\gamma\). - Mass increase: multiply by \(\gamma\).
\end{quote}

For example: to those inside the ship, one minute would be sixty seconds
long. To those looking in from outside, a minute on the ship would last:
\[
\begin{align}
\gamma &= \frac{1}{\sqrt{1 - 0.99^2}} = \frac{1}{\sqrt{1 - 0.98}} = \frac{1}{\sqrt{0.0199}} = \frac{1}{0.1411} = 7.089 \\[0.5em]
T_\gamma &= 60 \times 7.089 ≈ 425\;\text{seconds}
\end{align}
\] \ldots{} just over 425 seconds --- or 7.089 minutes (7ᵐ 5.33ˢ).

A \(50\)-kilogram person stepping on a scale onboard ship would see the
scale report their weight as \(50\) kilograms, but if an exterior
observer were able to measure the person's mass, they would see them
weighing almost \(355\) kilograms. Also, if the person onboard the ship
made a round trip of \(50\) years according to their clock, they would
return to find that almost \(355\) years had passed on Earth!

\begin{quote}
This is the source of the so-called ``Twin Paradox'': if one twin makes
a round-trip journey at \(0.99\)c of \(10\) years by their own clock,
they will return to find that about \(71\) years have passed on Earth.
If both twins were both \(20\) years old at departure, the traveler will
return at age \(30\) --- to a sibling who has aged through all of those
\(71\) years and is now \(91\) years old, making the stay-at-home twin
\textbf{\(\mathbf{61}\) years older} than their traveling sibling.
\end{quote}

This shows that at rest (in one's own reference frame) the Lorentz term
is 1.0, but as soon as \emph{any} motion is applied, a Lorentz term
comes into effect --- small at first but growing rapidly as the velocity
approaches the speed of light. At 0.9999c, the Lorentz term is: \[
\gamma = \frac{1}{\sqrt{1 - 0.9999^2}} = \frac{1}{\sqrt{1 - 0.9998}} = \frac{1}{\sqrt{0.0000199}} = \frac{1}{0.01411} = 70.7
\] \textbf{Table of Lorentz Factors}

\begin{longtable}[]{@{}
  >{\raggedleft\arraybackslash}p{(\linewidth - 2\tabcolsep) * \real{0.5227}}
  >{\raggedleft\arraybackslash}p{(\linewidth - 2\tabcolsep) * \real{0.4773}}@{}}
\toprule\noalign{}
\endhead
\bottomrule\noalign{}
\endlastfoot
0.9 & 2.294157339 \\
0.99 & 7.08881205 \\
0.999 & 22.36627204 \\
0.9999 & 70.71244595 \\
0.99999 & 223.6073568 \\
0.999999 & 707.1069579 \\
0.9999999 & 2236.068034 \\
0.99999999 & 7071.067814 \\
0.999999999 & 22360.68009 \\
\end{longtable}

This is the source of the common expression ``nothing can travel faster
than light''; at the speed of light it would require an infinite amount
of energy to increase velocity by the minuscule amount needed to ``break
the light barrier''. \#\# The Dramatic Side of Relativity One of the
``features'' of relativity is that there is no ``correct'' frame of
reference. We've noted above that the crew of a ship traversing at 0.99c
still measures a minute as 60 seconds, whereas to ``back home'' that
``minute'' lasts 7 minutes and 5.33 seconds.

At velocities even closer to the speed of light, the dilation becomes
even more pronounced. You might only experience 10 biological years
aboard ship (assuming you don't spend most of the trip in cryostasis),
but the ``wider universe'' may experience tens of thousands of
chronological years. It might happen that ``back home'' invents a warp
drive 10 years into your journey and you arrive at your destination to
find that it was colonized and tamed 9500 years ago and nobody actually
remembers who you are.

The concept of ``age'' would necessarily take on new meaning. Let's
revisit the 0.99c scenario and make you the traveling twin.

Assuming cryostasis slows your aging down to 1 year out of every 10
years of ship-time, you'll have only biologically aged a year on your
journey, so your ``bioage'' (let's keep the word ``age'' to refer to
this) is only 21. Your ``proper age'' is 11 years more than that (let's
call that your ``scope''); and the chronological duration since you were
born is 71 years (we'll call that your ``span'')\ldots{} So you are 21
years of age; 30 years ``scoped'', and 91 years ``spanned''.

A \$1 you left in a high-yield savings account before your departure may
have appreciated into billions --- or your culture may have completely
changed its material wealth concept and you have no wealth at all under
the new system. Perhaps you carried a million dollars worth of diamonds
with you on your journey, but ``back home'' found and began mining a
diamond planet 7 years after you left and now that compressed carbon in
your pocket is literally more common than dirt\ldots.

The world\emph{building} opportunities are endless, and endlessly
fascinating.

\chapter{The Dangers of Relativistic
Velocities}\label{the-dangers-of-relativistic-velocities}

\section{Catastrophic Impacts From
Microparticles}\label{catastrophic-impacts-from-microparticles}

While your environment onboard ship feels ``normal'' to you, the Lorentz
factor has had its effects on the ``outside'' universe. A dust particle
that would be practically massless under ``normal'' circumstances may
have several tons of relativistic mass if it hits your ship.

So, any vessel traversing at relativistic speeds needs to be protected
from the energy release of any such impacts on its hull. The necessary
thickness (and mass that has to be accelerated) of a physical shield
grows as a function of the Lorentz factor as velocity increases, so a
point of diminishing returns is reached almost instantly.

In many science fiction milieux, this problem is avoided by either of
two means: 1. Either the ship is in a ``bubble'' of some kind, the
nature of which absorbs or deflects any such incoming missles; or 2. The
ship is fitted with some kind of energetic field of force (commonly
called a ``shield'') which either deflects the impactors or absorbs the
energy of their collision with it.

\section{The View Out The Window}\label{the-view-out-the-window}

First, it would depend entirely on which window you were looking out of.

\begin{enumerate}
\def\labelenumi{\arabic{enumi}.}
\tightlist
\item
  Light from the stars behind you would become progressively more
  redshifted as your velocity increased, until they became invisible to
  the naked eye. Objects previously radiating in the ultraviolet would
  gradually become visible and grow redder until they, too, passed
  beyond the visible red end of the spectrum.
\item
  Light from the stars ahead of you would become progressively more
  blue-shifted toward the ultraviolet. Those already radiating in the
  violet end of the spectrum would disappear from view. Objects
  previously only radiating in long wavelengths below red would emerge
  into the visible spectrum, progressively grow bluer, and finally,
  they, too would pass out of the visible range into the ultraviolet.
  The light coming toward you would also become progressively more and
  more energetic until it was as if you were stationary at some finite
  distance from the Big Bang, itself.
\item
  Light from the stars out the side windows would be similarly shifted,
  to a greater or lesser extent depending on how far away from you they
  were. Ultimately, all the light arriving at your location would become
  compressed into a rainbow like a belt around your ship (at no
  measurable distance\_, with utter blackness ``ahead'' and ``behind''
  it. Red would be at the ``back'' end, violet at the ``front'' end, and
  the other colors of the spectrum distributed between the two extremes.
\end{enumerate}

And you probably wouldn't want to be looking out a glass window, anyway,
because you'd be microwaved or gamma rayed before you had a chance to
marvel at the spectacle.

\part{Stellar Systems}

\chapter{Abstract}\label{abstract-6}

\textbf{Major Topics:}\\
- Defines the \textbf{stellar spectral classification system} (O, B, A,
F, G, K, M; plus L, T, Y) in a \textbf{linearized temperature model} for
WCB use.\\
- Spectral Classes are set by \textbf{surface temperature ranges}
(\(T_{\text{eff}}\) in Kelvin).\\
- Each Class is subdivided into \textbf{Spectral Types} (0--9), numbered
backwards (0 = hottest).\\
- Establishes formulas for interconversion between:\\
- Spectral Type (𝓢),\\
- Surface temperature in Kelvin (K),\\
- Relative solar temperature (T),\\
- High-class temperature bound (κ),\\
- Thermal interval constant (þ).\\
- Introduces \textbf{thermal interval constants} (þ) for each Class,
computed as (high -- low)/10.\\
- Provides worked examples for:\\
- The Sun (G2, 5800K, T=1.0),\\
- Essem (F3.65, 6952.5K, T=1.199),\\
- Essel (G9.192, 5081K, T=0.876).\\
- Includes expanded \textbf{parameter tables by spectral class}: Kelvin
ranges, T⊙, R⊙, L⊙, M⊙, and Q⊙.

\textbf{Key Terms \& Symbols:}\\
- \textbf{Spectral Class} --- O, B, A, F, G, K, M (+ L, T, Y).\\
- \textbf{Spectral Type} --- subclass 0--9 (backwards numbering).\\
- \textbf{K} --- effective surface temperature in Kelvin.\\
- \textbf{T} --- temperature relative to Sun (⊙ = 5800K ⇒ T = 1.0).\\
- \textbf{κ (kappa)} --- high temperature bound of class.\\
- \textbf{þ (thorn)} --- thermal interval constant = (high -- low)/10.\\
- \textbf{𝓢} --- spectral type number.

\textbf{Cross-Check Notes:}\\
- Provides \textbf{linearized model} vs.~traditional classification
irregularities.\\
- Supports interpolation (decimal values, e.g.~G2.3, G2.9) without
creating new types.\\
- Forms the \textbf{baseline module} for stellar characterization in
WCB.\\
- Connects stellar parameters (L⊙, R⊙, M⊙, Q⊙) to habitability modeling.

\chapter{Stars and Spectral Classes: The Fusion-Fueled
Continuum}\label{stars-and-spectral-classes-the-fusion-fueled-continuum}

First: The spectral class system used throughout this guide --- the
sequence \textbf{O, B, A, F, G, K, M} --- is historically rooted in the
observational astronomy of the late 19th and early 20th centuries. Its
peculiar alphabetical order reflects the evolution of stellar
classification from empirical cataloging to physical understanding.

\begin{quote}
For readers curious about its origins --- including the critical work of
\textbf{Annie Jump Cannon}, \textbf{Cecilia Payne-Gaposchkin}, and the
less brilliant men who received most of the credit --- see
\textbf{{[}{[}Sidebar: The Spectral System and the Women Who Built
It{]}{]}}.
\end{quote}

Second: The spectral classes used in WCB are based on a
\textbf{linearized temperature model}. This approach smooths over the
irregularities of the traditional system to support clean interpolation,
symbolic clarity, and consistent orbital modeling.

\begin{quote}
If you're curious about the limitations of the classical OBAFGKM system
--- and why we've chosen to ``straighten the curve'' --- see
\textbf{{[}{[}Sidebar Module: \emph{Mind the Gap --- The Shortcomings of
the Traditional Spectral Scale}{]}{]}}.
\end{quote}

\section{Spectral Class Table}\label{spectral-class-table}

Here are the spectral classes we'll be working with.

\begin{longtable}[]{@{}crr@{}}
\toprule\noalign{}
SpectralClass & Low Temp. (K) & High Temp. (K) \\
\midrule\noalign{}
\endhead
\bottomrule\noalign{}
\endlastfoot
O & 25000 & 55000 \\
B & 10000 & 25000 \\
A & 7500 & 10000 \\
F & 6000 & 7500 \\
G & 5000 & 6000 \\
K & 3500 & 5000 \\
M & 2400 & 3500 \\
Brown↓ Dwarfs ↓ & & \\
L & 1300 & 2400 \\
T & 600 & 1300 \\
Y & 300 & 600 \\
\end{longtable}

\begin{quote}
Notes: - Spectral Classes L, T, and Y are ``special cases'' which are
covered in detail in another module ⟨⟨ insert module name here ⟩⟩ - Each
range reflects a star's \textbf{surface temperature}, typically noted as
\(T_{\text{eff}}\) in astronomical literature. - In WCB: - \textbf{K} =
temperature in Kelvin - \textbf{T} = temperature \emph{relative to the
Sun} (i.e., ⊙ = 5800K ⇒ T = 1.0) \#\#\# Spectral \emph{Type} Each
spectral class is subdivided into 10 \textbf{spectral types}, numbered
\textbf{0} (hottest) to \textbf{9} (coolest).
\end{quote}

\begin{quote}
\textbf{Hippy}: Wait, that's --
\end{quote}

Yes, it runs \emph{backwards}. No, we're not happy about it, either
(don't shoot the messenger).

For example:\\
- The Sun, at \textbf{5800K}, is classified as a \textbf{G2} star ---\\
- Spectral \textbf{Class}: G\\
- Spectral \textbf{Type}: 2

\begin{quote}
\subsubsection{Note on Spectral Type Precision in
WCB}\label{note-on-spectral-type-precision-in-wcb}

In this system, a \textbf{spectral type} is defined by its
\textbf{numerical position} within a spectral class. For example: -
\textbf{G2}, \textbf{G2.3}, and \textbf{G2.9} are all \textbf{Type 2}\\
- The decimal simply adds interpolation precision --- it does
\textbf{not} define a new type. - Therefore, \textbf{Type 2} refers to
the full range ⟨2.0 ∧ 2.999···⟩ within class \emph{G}.
\end{quote}

This allows for relatively simple mathematical treatment of the
relationship between spectral type (T) and surface temperature (K). \[
\begin{align}
\mathcal{S} &= \dfrac{\kappa - K}{þ} \\ \\
\kappa & = \mathcal{S} þ + K \\ \\
K &= \kappa - \mathcal{S} þ \\ \\
þ &= \dfrac{\kappa - K}{\mathcal{S}} \\
\end{align}
\]

Where: - K = the star's surface temperature in Kelvin - κ = the
\emph{upper bound} temperature of the relevant spectral class - þ = the
thermal interval constant for the relevant spectral class -
\(\mathcal{S}\) = the spectral \emph{type} number \#\#\#\# The Thermal
Interval Constant (þ) Where does þ come from? For a given spectral class
þ can be calculated by: \[
þ = \dfrac{high\;temp - low\;temp}{10}
\]

Here is the above table with these constants added:

\begin{longtable}[]{@{}
  >{\centering\arraybackslash}p{(\linewidth - 6\tabcolsep) * \real{0.1097}}
  >{\raggedleft\arraybackslash}p{(\linewidth - 6\tabcolsep) * \real{0.2645}}
  >{\raggedleft\arraybackslash}p{(\linewidth - 6\tabcolsep) * \real{0.2710}}
  >{\raggedleft\arraybackslash}p{(\linewidth - 6\tabcolsep) * \real{0.3548}}@{}}
\toprule\noalign{}
\begin{minipage}[b]{\linewidth}\centering
SpectralClass
\end{minipage} & \begin{minipage}[b]{\linewidth}\raggedleft
\end{minipage} & \begin{minipage}[b]{\linewidth}\raggedleft
\end{minipage} & \begin{minipage}[b]{\linewidth}\raggedleft
\end{minipage} \\
\midrule\noalign{}
\endhead
\bottomrule\noalign{}
\endlastfoot
O & 25000 & 55000 & 3000 \\
B & 10000 & 25000 & 1500 \\
A & 7500 & 10000 & 250 \\
F & 6000 & 7500 & 150 \\
G & 5000 & 6000 & 100 \\
K & 3500 & 5000 & 150 \\
M & 2400 & 3500 & 110 \\
L & 1300 & 2400 & 110 \\
T & 600 & 1300 & 70 \\
Y & 300 & 600 & 30 \\
& & & \\
& & & \\
\end{longtable}

\subsubsection{Example}\label{example-4}

Let's run the numbers for the Sun - Known surface temperature: 5800K -
Checking the table, 5800K falls between 5000K and 6000K, so the Sun is
spectral class G - The high temperature (κ) for spectral class G is κ
6000K - The thermal interval constant (þ) for spectral class G is þ =
100 - What is the Sun's spectral type (\(\mathcal{S}\)) Running the
numbers: \[
\begin{align}
\mathcal{S} &= \dfrac{\kappa - K}{þ} \\
\mathcal{S} &= \dfrac{6000 - 5800}{100} \\
\mathcal{S} &= \dfrac{200}{100} \\
\mathcal{S} &= 2\;✓
\end{align}
\] The Sun is spectral type \emph{G2}.

\textbf{Reversing the process:} - The known spectral class of the Sun is
G - The known spectral type of the Sun is \(\mathcal{S}\) = 2 - The high
temperature (κ) for spectral class G is κ 6000K - The thermal interval
constant (þ) for spectral class G is þ = 100 - What is the Sun's Kelvin
temperature (K) Running the numbers: \[
\begin{align}
K &= \kappa - \mathcal{S} þ \\
K &= 6000 - (2)(100) \\
K &= 6000 - 200 \\
K &= 5800\;✓
\end{align}
\] The surface temperature of the Sun is \emph{5800K}.

\subsection{Converting Between Absolute Kelvin (K) And Solar Relative
(T)}\label{converting-between-absolute-kelvin-k-and-solar-relative-t}

Nothing could be simpler: \[
\begin{align}
T &= \dfrac{K}{5800} \\ \\
K &= 5800T
\end{align}
\] For instance: the Sun's surface temperature is K = 5800: \[
T = \dfrac{K}{5800} = \dfrac{5800}{5800} = 1\;✓
\] Conversely, the Sun's relative temperature is T = 1.0: \[
K = 5800 T = (5800)(1) = 5800\;✓
\] \#\#\# Fictional Examples Let's say we have a star called Essem that
we want to be spectral type \emph{F3.65}. What is its Kelvin
temperature? - The surface temperature for spectral class F is K ∈ ⟨6000
∧ 7500⟩. - The thermal interval constant for spectral class F is þ =
150. Working through the equation: \[
\begin{align}
K &= \kappa - \mathcal{S} þ \\
K &= 7500 - (3.65)(150) \\
K &= 7500 - 547.5 \\
K &= 6952.5\;✓
\end{align}
\] What is Essem's relative surface temperature? \[
\begin{align}
T = \dfrac{K}{5800} \\ \\
T = \dfrac{6952.5}{5800} \\ \\
T = 1.199\;✓
\end{align}
\] Essem's relative temperature is \emph{T = 1.199}⊙.

\textbf{Working The Other Direction}

Let us say that Essem has a near neighbor, Essel, and we know that its
relative temperature is T = 0.876⊙. What is its spectral type?

First, convert T to K by: \[ K = 5800T = (5800)(0.876) = 5080.8\;✓ \]
Looking at our table we see that this value falls in spectral class G:

\begin{longtable}[]{@{}
  >{\centering\arraybackslash}p{(\linewidth - 6\tabcolsep) * \real{0.1097}}
  >{\raggedleft\arraybackslash}p{(\linewidth - 6\tabcolsep) * \real{0.2645}}
  >{\raggedleft\arraybackslash}p{(\linewidth - 6\tabcolsep) * \real{0.2710}}
  >{\raggedleft\arraybackslash}p{(\linewidth - 6\tabcolsep) * \real{0.3548}}@{}}
\toprule\noalign{}
\begin{minipage}[b]{\linewidth}\centering
SpectralClass
\end{minipage} & \begin{minipage}[b]{\linewidth}\raggedleft
\end{minipage} & \begin{minipage}[b]{\linewidth}\raggedleft
\end{minipage} & \begin{minipage}[b]{\linewidth}\raggedleft
\end{minipage} \\
\midrule\noalign{}
\endhead
\bottomrule\noalign{}
\endlastfoot
G & 5000 & 6000 & 100 \\
\end{longtable}

\ldots{} which gives us all the other information we need: - G-class
high temperature is κ = 6000 - G-class thermal interval constant is þ =
100

The spectral type is: \[
\begin{align}
\mathcal{S} &= \dfrac{\kappa - K}{þ} \\ \\
\mathcal{S} &= \dfrac{6000 - 5080.8}{100} \\ \\
\mathcal{S} &= \dfrac{919.2}{100} \\ \\
\mathcal{S} &= 9.192\;✓
\end{align}
\] Essel's spectral type is \emph{G9.192}. \#\#\# Parameter Ranges By
Spectral Class

\chapter{Parameter Ranges by Spectral
Class}\label{parameter-ranges-by-spectral-class}

\begin{longtable}[]{@{}
  >{\raggedright\arraybackslash}p{(\linewidth - 16\tabcolsep) * \real{0.0423}}
  >{\raggedright\arraybackslash}p{(\linewidth - 16\tabcolsep) * \real{0.0704}}
  >{\raggedleft\arraybackslash}p{(\linewidth - 16\tabcolsep) * \real{0.1268}}
  >{\raggedleft\arraybackslash}p{(\linewidth - 16\tabcolsep) * \real{0.1268}}
  >{\raggedleft\arraybackslash}p{(\linewidth - 16\tabcolsep) * \real{0.1268}}
  >{\raggedleft\arraybackslash}p{(\linewidth - 16\tabcolsep) * \real{0.1268}}
  >{\raggedleft\arraybackslash}p{(\linewidth - 16\tabcolsep) * \real{0.1268}}
  >{\raggedleft\arraybackslash}p{(\linewidth - 16\tabcolsep) * \real{0.1268}}
  >{\raggedleft\arraybackslash}p{(\linewidth - 16\tabcolsep) * \real{0.1268}}@{}}
\toprule\noalign{}
\begin{minipage}[b]{\linewidth}\raggedright
\end{minipage} & \begin{minipage}[b]{\linewidth}\raggedright
SC →
\end{minipage} & \begin{minipage}[b]{\linewidth}\raggedleft
\end{minipage} & \begin{minipage}[b]{\linewidth}\raggedleft
\end{minipage} & \begin{minipage}[b]{\linewidth}\raggedleft
\end{minipage} & \begin{minipage}[b]{\linewidth}\raggedleft
\end{minipage} & \begin{minipage}[b]{\linewidth}\raggedleft
\end{minipage} & \begin{minipage}[b]{\linewidth}\raggedleft
\end{minipage} & \begin{minipage}[b]{\linewidth}\raggedleft
\end{minipage} \\
\midrule\noalign{}
\endhead
\bottomrule\noalign{}
\endlastfoot
& & & & & & & & \\
& High & 55000 & 25000 & 10000 & 7500 & 6000 & 5000 & 3500 \\
Kelvin & Mean & 40000 & 17500 & 8750 & 6750 & 5500 & 4250 & 2950 \\
& Low & 25000 & 10000 & 7500 & 6000 & 5000 & 3500 & 2400 \\
& TIC¹ (\emph{þ}) & 3000 & 1500 & 250 & 150 & 100 & 150 & 110 \\
& & & & & & & & \\
& High & 9.4828 & 4.3103 & 1.7241 & 1.2931 & 1.0345 & 0.8621 & 0.6034 \\
T⊙ & Mean & 6.8966 & 3.0172 & 1.5086 & 1.1638 & 0.9483 & 0.7328 &
0.5086 \\
& Low & 4.3103 & 1.7241 & 1.2931 & 1.0345 & 0.8621 & 0.6034 & 0.4138 \\
& & & & & & & & \\
& High & 17.0690 & 7.7586 & 3.1034 & 2.3276 & 1.8621 & 1.5517 &
1.0862 \\
R⊙ & Mean & 12.4138 & 5.4310 & 2.7155 & 2.0948 & 1.7069 & 1.3190 &
0.9155 \\
& Low & 7.7586 & 3.1034 & 2.3276 & 1.8621 & 1.5517 & 1.0862 & 0.7448 \\
& & & & & & & & \\
& High & 2.356 M & 20.779 k & 85.1093 & 15.1476 & 3.9709 & 1.3298 &
0.1565 \\
L⊙ & Mean & 348.608 k & 2.445 k & 38.1967 & 8.0501 & 2.3559 & 0.5015 &
0.0561 \\
& Low & 20.779 k & 85.109 & 15.1476 & 3.9709 & 1.3298 & 0.1565 &
0.0163 \\
& & & & & & & & \\
& High & 18.7759 & 8.5345 & 3.4138 & 2.5603 & 2.0483 & 1.7069 &
1.1948 \\
M⊙ & Mean & 13.6552 & 5.9741 & 2.9871 & 2.3043 & 1.8776 & 1.4509 &
1.0071 \\
& Low & 8.5345 & 3.4138 & 2.5603 & 2.0483 & 1.7069 & 1.1948 & 0.8193 \\
& & & & & & & & \\
& High & 64.10E-06 & 4.00E-03 & 0.1280 & 0.4684 & 1.3041 & 4.7336 &
29.3785 \\
Q⊙ & Mean & 0.67E-03 & 65.64E-03 & 0.2766 & 0.8441 & 2.1003 & 12.4968 &
82.4297 \\
& Low & 0.69E-06 & 35.57E-06 & 3.47E-03 & 0.0146 & 0.0447 & 0.1112 &
0.6614 \\
\end{longtable}

¹ Thermal Interval Constant

\section{Abstract}\label{abstract-7}

\textbf{Major Topics:}\\
- Defines the \textbf{five core stellar parameters}:\\
- Temperature (K, T)\\
- Mass (M)\\
- Radius (R)\\
- Luminosity (L)\\
- Lifetime (Q)\\
- Establishes \textbf{parameter precedence}: temperature (T/K) is
primary, radius (R) is secondary.\\
- Provides \textbf{equations of state} for main-sequence stars, allowing
any parameter to be derived from another.\\
- Introduces the \textbf{blackbody approximation}, with emissivity (ϵ)
correction.\\
- Explains the \textbf{Stefan--Boltzmann Law} and its solar-relative
simplification (\(L = R^2T^4\)).\\
- Presents dependency chains for deriving all parameters starting from
T, M, R, L, or Q.\\
- Emphasizes practical use in \textbf{worldbuilding calculations}
(habitability, orbits, irradiance).

\textbf{Key Terms \& Symbols:}\\
- \textbf{K} --- Stellar surface temperature in Kelvin.\\
- \textbf{T} --- Temperature relative to solar (T = K/5800).\\
- \textbf{M} --- Stellar mass (⊙).\\
- \textbf{R} --- Stellar radius (⊙).\\
- \textbf{L} --- Stellar luminosity (⊙).\\
- \textbf{Q} --- Stellar lifetime (⊙ units).\\
- \textbf{ϵ (epsilon)} --- Emissivity, fraction of ideal blackbody
radiation (0--1).\\
- \textbf{σ (Stefan--Boltzmann constant)} =
\(5.670374419 × 10^{-8} W·m^{-2}·K^{-4}\).

\textbf{Cross-Check Notes:}\\
- Direct analog to planemon parameter system (m, r, ρ, g, vₑ).\\
- Stars: \textbf{T primary, R secondary}.\\
- planemons: \textbf{m primary, ρ secondary}.\\
- Provides unified framework for comparing stellar and planetary
parameters.

\chapter{Stellar Parametrics}\label{stellar-parametrics}

In \emph{Spectral Classes}, we covered spectral classes and spectral
types and their association to the surface temperatures of stars. Stars,
like planemons, have a basic set of parameters that describe them: -
\textbf{Temperature} --- How hot is the surface? - Absolute measure:
Kelvin (K) - Relative measure: Solar units (T) - \textbf{Mass} --- How
much material is there? (M) - \textbf{Luminosity} --- How bright is it?
(L) - \textbf{Radius} --- How big is it? (R) - \textbf{Lifetime} --- How
long does it shine? (\(\mathcal{Q}\)) - Chiefly relevant to \emph{Main
Sequence} stars, particularly stars that are \textbf{Solar Cognates}
(more on this below.)

\begin{quote}
Notes: 1. Where we use lower-case letters for the parameters of
planemons, we use upper-case letters for stars, so it's easy to tell
them apart. 2. While \textbf{mass} (\emph{m}) is the primary parameter
for planemons, with \textbf{density} (\emph{ρ}) secondary, for stars
\textbf{Temperature} (\emph{T}) is the primary parameter, and
\textbf{radius} (\emph{R}) is secondary. - While luminosity is
\textbf{technically derived} from a star's temperature and radius (see
\emph{the Stefan-Boltzmann Law}, below), it plays a \textbf{central
role} in modeling stellar systems --- particularly when calculating
orbit distances, habitable zones, and irradiance. In practice, it's
often treated as the secondary parameter after termperature for
thesisastics. \#\# Equations of State A regularized set of empirical
relationships can be used to estimate any stellar parameter from the
others --- assuming a Main Sequence \textbf{blackbody}-like star (see
{[}{[}Sidebar --- What Is The Main Sequence{]}{]}).
\end{quote}

\begin{quote}
\textbf{Keppy}: And a \textbf{blackbod}y is\ldots?
\end{quote}

Excellent question! A \textbf{blackbody} is an \textbf{idealized
physical object} that: 1. \textbf{Absorbs all} incoming electromagnetic
radiation --- no reflection, no transmission.\\
2. \textbf{Emits radiation} purely based on its temperature --- not its
material, shape, or color.\\
3. Emits a \textbf{perfectly smooth, continuous spectrum} (a ``thermal
spectrum''). In short: \textgreater{} A blackbody is the theoretical
gold standard for radiant heat emission --- a perfect radiator and
absorber. \#\#\#\# Why ``Blackbody'' Matters Here Most stars (especially
Main Sequence stars) behave \textbf{approximately like blackbodies},
meaning their energy output can be modeled using \textbf{temperature
alone}. This makes them excellent candidates for: -
\textbf{Temperature-based modeling}\\
- \textbf{Color-temperature mapping} (blue = hotter, red = cooler)\\
- \textbf{Spectrum-based classification} (like spectral classes O--M) -
Real-World Deviation - planemons, dust clouds, and even stars aren't
\emph{perfect} blackbodies. - Real objects have an \textbf{emissivity} ϵ
between 0 and 1: \[ F = \varepsilon \sigma T^4\] - But stars are close
enough that the \textbf{blackbody approximation works very well}.

\begin{quote}
\textbf{Hippy}: Sorry you asked, Keplarius?
\end{quote}

Yes, that's a bit technical and complicated, but it's also extremely
\emph{important} to what comes next.

Here are the promised equations:

\begin{longtable}[]{@{}
  >{\centering\arraybackslash}p{(\linewidth - 6\tabcolsep) * \real{0.2115}}
  >{\centering\arraybackslash}p{(\linewidth - 6\tabcolsep) * \real{0.2115}}
  >{\centering\arraybackslash}p{(\linewidth - 6\tabcolsep) * \real{0.2212}}
  >{\centering\arraybackslash}p{(\linewidth - 6\tabcolsep) * \real{0.3558}}@{}}
\toprule\noalign{}
\begin{minipage}[b]{\linewidth}\centering
Temperature(T)
\end{minipage} & \begin{minipage}[b]{\linewidth}\centering
Mass(M)
\end{minipage} & \begin{minipage}[b]{\linewidth}\centering
Radius(R)
\end{minipage} & \begin{minipage}[b]{\linewidth}\centering
Lifetime(Q)
\end{minipage} \\
\midrule\noalign{}
\endhead
\bottomrule\noalign{}
\endlastfoot
\(T=\sqrt[1.98]{M}\) & \(M=\sqrt[0.9]{R}\) & \(R=M^{0.9}\) &
\(\mathcal{Q}=M^{-2.5}\) \\
\(T=\sqrt[1.8]{R}\) & \(M=T^{1.98}\) & \(R=T^{1.8}\) &
\(\mathcal{Q} \approx \sqrt[-0.36]{R}\) \\
\(T=\mathcal{Q}^{-0.2}\) & \(M=\mathcal{Q}^{-0.4}\) &
\(R=\mathcal{Q}^{-0.36}\) & \(\mathcal{Q}=T^{-5}\) \\
\end{longtable}

\begin{quote}
\begin{quote}
\textbf{NOTE}: All of the above equations are \emph{approximations};
stars are a much more variable set of objects (after all, they're mostly
gas and plasma, so fluid dynamics plays a major role in their
characteristics). These equations work \textbf{best \emph{in general}
for main sequence stars} of all classes.
\end{quote}
\end{quote}

\begin{quote}
\textbf{Keppy}: You said Luminosity was the second most important
parameter for stars, but it doesn't appear in the table\ldots?
\end{quote}

Well spotted, Keppy! There's a reason. \#\#\# The Stefan-Boltzmann Law
The Stefan-Boltzmann Law is a formulation that relates the
\textbf{luminosity} of any luminous object to its \textbf{temperature}
and \textbf{surface area}: \[
L = 4 \pi R^2 \sigma T^4
\] Where: - \(4 \pi R^2\) = the surface area of the body - \(T\) = is
the temperature of the body in Kelvin - \(σ\) = the Stefan-Boltzmann
constant - \(\sigma = 5.670374419 \times 10^{-8} W m^{-2}K^{-4}\) -
\textbf{Watts} per square meter per Kelvin to the fourth power 1 K⁴ - It
tells you how much \textbf{radiant energy per second} (i.e., power) is
emitted by a \textbf{1 square meter} portion of a \textbf{perfect
blackbody} at \textbf{1 K⁴}.

And this is why we needed the quick aside into the term ``blackbody''
earlier.

In worldmaking terms, we can simplify the Stefan-Boltzmann equation to:
\[
\dfrac{L_s}{L_{Sun}} = \left(\dfrac{R_s}{R_{Sun}}\right)^2 \left(\dfrac{K_s}{K_{Sun}}\right)^4
\] Where: - \(L_S\) = the absolute luminosity of the star - \(L_{Sun}\)
= the absolute luminosity of the Sun - \(R_S\) = the absolute radius of
the star - \(R_{Sun}\) = the absolute radius of the Sun - \(K_S\) = the
Kelvin temperature of the star - \(K_{Sun}\) = the Kelvin temperature of
the Sun

Because the form \(\dfrac{X_s}{X_{Sun}}\) is the standard for converting
a parameter to solar units, and \(T = \dfrac{K_s}{K_{Sun}}\), this
equation becomes: \[
\begin{align}
L &= R^2T^4, \qquad \text{with derivations of} \\ \\
R &= \dfrac{\sqrt{L}}{T^2}, \qquad
T = \sqrt[4]{\dfrac{L}{R^2}}
\end{align}
\] \#\# Parameter Calculation Precedence The above being the case, there
is a ``best'' order for calculating stellar parameters when starting
from any given parameter (though it is always best start with \emph{K}
or \emph{T} whenever possible). \textgreater{} All parameters (except K)
are expressed in Solar-relative units; that is, T = 1⊙ for 5800\,K, R =
1⊙ for the solar radius, etc. \#\#\#\# Starting with Temperature
(\emph{T}) or (\emph{K}) \textbf{Primary dependency chain}: T/K → R → L
→ M → Q \[
\begin{gather}
T = \dfrac{K}{5800} \quad or \quad K = 5800T \\
R = T^{1.8} \\
L = R^2T^4 \\
M = T^{1.98} \quad or \quad M = \sqrt[0.9]{R} \\
\mathcal{Q} = T^{-5} \quad or \quad \mathcal{Q} = M^{-2.5}
\end{gather}
\] \#\#\#\# Starting with Mass (\emph{M}) \textbf{Primary dependency
chain}: M → T/K → R → L → Q \[
\begin{gather}
T = \sqrt[1.98]{M} \\
K = 5800T \\
R = T^{1.8} \quad or \quad R = M^{0.9} \\
L = R^2T^4 \\
Q = T^{-5} \quad or \quad Q = M^{-2.5}
\end{gather}
\] \#\#\#\# Starting with Radius (\emph{R}) \textbf{Primary dependency
chain}: R → T → K → L → M → 𝒬 \[
\begin{gather}
T = \sqrt[1.8]{R} \\
K = 5800T \\
L = R^2T^4 \\
M = T^{1.98} \\
\mathcal{Q} = T^{-5} \quad or \quad \mathcal{Q} = M^{-2.5}
\end{gather}
\] \#\#\#\# Starting With Luminosity (\emph{L}) \textbf{Primary
dependency chain}: L → T → K → R → M → Q \[
\begin{gather}
T = \sqrt[7.6]{L} \\
K = 5800T \\
R = T^{1.8} \\
M = T^{1.98} \\
Q = T^{-5} \quad or \quad Q = M^{-2.5}
\end{gather}
\] \#\#\#\# Starting with Lifetime (\emph{𝒬}) *\emph{As soon as you
assume you'd never want to do this, you'll find a case for doing it.}*
\textbf{Primary dependency chain}: 𝒬 → T → K → R → L → M \[
\begin{gather}
T=\mathcal{Q}^{-0.2} \\
K = 5800T \\
R=\mathcal{Q}^{-0.36} \\
L = R^2T^4 \\
M = \sqrt[3]{L}
\end{gather}
\]

\section{Abstract}\label{abstract-8}

\textbf{Major Topics:}\\
- Defines the \textbf{Nucleal Orbit (𝒩)} --- orbital distance at which a
planemon receives the same stellar irradiance as Earth does at 1 AU.\\
- Formula: 𝒩 = √L (where L = stellar luminosity in ⊙ units).\\
- Anchors the \textbf{habitable zone (HZ)} around a star as ranges
proportional to 𝒩.\\
- Distinguishes between:\\
- \textbf{Habitable Zone} --- wider corridor (⟨0.750 ∧ 1.770⟩𝒩).\\
- \textbf{Hospitable Zone} --- narrower ``middle lane'' (⟨0.950 ∧
1.385⟩𝒩).\\
- Defines \textbf{parahabitable, habitable, hospitable, xenotic} orbital
spans as structured ``Ontozones.''\\
- Introduces the \textbf{Frost Line (ϝ)} at 4.850𝒩, beyond which water
cannot remain liquid.\\
- Specifies notation for inner and outer orbital regimes:\\
- \(Z_{IX}\), \(Z_{IP}\), \(Z_{IH}\), \(Z_H\), \(Z_{OH}\), \(Z_{OP}\),
\(Z_{OX}\).

\textbf{Key Terms \& Symbols:}\\
- \textbf{𝒩 (Nucleal Orbit)} --- central reference orbit for irradiance
equivalence.\\
- \textbf{ϝ (Frost Line)} --- outer limit for liquid water
(\textasciitilde4.850𝒩).\\
- \textbf{Ontozones} --- structured orbital bands around stars.\\
- \textbf{Zone Notation:}\\
- \(Z_{IX}\) --- Inner Xenotic Zone (\textless0.500𝒩).\\
- \(Z_{IP}\) --- Inner Parahabitable Zone (0.500--0.750𝒩).\\
- \(Z_{IH}\) --- Inner Habitable Zone (0.750--0.950𝒩).\\
- \(Z_H\) --- Hospitable Zone (0.950--1.385𝒩).\\
- \(Z_{OH}\) --- Outer Habitable Zone (1.385--1.770𝒩).\\
- \(Z_{OP}\) --- Outer Parahabitable Zone (1.770--4.850𝒩).\\
- \(Z_{OX}\) --- Outer Xenotic Zone (≥4.850𝒩).

\textbf{Cross-Check Notes:}\\
- Builds directly on \textbf{Habitable Zone Limits (H₀--H₅)} from
v1.219.\\
- Adds layered refinement: narrower \textbf{Hospitable Zone} within the
wider HZ.\\
- Introduces \textbf{Ontozones} and \textbf{zone notation system} for
systematic classification.

\section{The Nucleal Orbit}\label{the-nucleal-orbit}

The average distance from Earth to the Sun --- about
\(1.496 \times 10^8\) km --- is defined as one \textbf{astronomical unit
(AU)}. Due to Earth's slightly elliptical orbit, this distance varies by
approximately ±2.5 million km between Earth's closest approach to and
farthest distance from the Sun.

So, for all practical (and thesiastic) purposes, the Earth's orbital
distance (\emph{a}) is a = 1.0AU. In fact \emph{a} is the commonly used
symbol for \emph{any} orbital distance when it is expressed in
Astronomical Units.

For our purposes, I have revived an old word from the dusty backroom
shelves of English --- \emph{nucleal} --- and given it new life:

\begin{quote}
\textbf{Nucleal Orbit} (\(\mathcal{N}\)): the orbital distance from any
given star at which a planemon receives the same stellar irradiance as
Earth receives from the Sun at 1 AU.
\end{quote}

\ldots{} and given it the utterly unimaginative symbol, \(\mathcal{N}\).

The important thing to note here is that \emph{\(\mathcal{N}\) is not
constant}, but varies from star to star, and it is calculated by: \[
\mathcal{N} = \sqrt{L}
\] Where: - \emph{L} = the Luminosity of the star in relative units

Obviously for the Sun: \[
\mathcal{N} = \sqrt{L} = \sqrt{1} = 1
\] \textgreater{} \textbf{Keppy}: So for a dimmer star N shifts closer
to the star?

\begin{quote}
\textbf{Hippy}: And for a brighter star, it shifts farther out from the
star.
\end{quote}

Correct on both counts. And once we know \emph{\(\mathcal{N}\)}, we can
express the \textbf{habitable zone} (details coming!) as a proportional
range around it. For instance, for a star of half the Sun's luminosity
\(L = 0.5⊙\): \[
\mathcal{N} = \sqrt{L} = \sqrt{0.5} = 0.7071\;AU
\] \#\#\# The Nucleal Orbit and the Habitable Zone A quick survey of the
existing literature reveals a commonly held definition for the
\textbf{habitable zone} as: \[
\langle0.950 \wedge 1.385\rangle \mathcal{N}
\] \ldots{} or, in other words: between 95\% of the nucleal orbit
distance to 1.385 times (138.5\%) the nucleal orbit distance. In the
case of our hypothetical \(L = 0.5⊙\) star and its \(\mathcal{N}\)
nucleal orbit, the range of its habitable zone calculates to: \[
\begin{gather}
\mathcal{N} = 0.7071\; AU \\
\text{Inner Edge} = 0.950 \mathcal{N} = (0.950)(0.7071) = 0.6717\; AU \\
\text{Outer Edge} = 1.385 \mathcal{N} = (1.385)(0.7071) = 0.9793\; AU
\end{gather}
\] \textgreater{} \textbf{Keppy}: So \ldots{} the \emph{outer edge} of
this star's habitable zone is \emph{closer to its star} than \emph{Earth
orbits from the Sun}\ldots..

Exactly. But, this region is only a part of a total star system. \#\#\#
The Ontozones -- Two Habitable Zones To start with, some scientist posit
a wider, more ``optimistic habitable zone'' region, covering: \[
\langle0.750 \wedge 1.770\rangle \mathcal{N}
\] For our purposes, we call this \emph{wider spread} the actual
\textbf{habitable zone} and we call the narrower span the
\textbf{hospitable zone}, so that the \emph{hospitable zone} comprises a
middle lane between the extremes of the \emph{habitable zone}: \[
\dfrac{1.385 - 0.95}{1.77 - 0.75} = \dfrac{0.435}{1.02} = 0.4265\;AU
\] \ldots{} about 42.65\% of it, in fact.

\begin{longtable}[]{@{}cl@{}}
\toprule\noalign{}
Orbital Range & \\
\midrule\noalign{}
\endhead
\bottomrule\noalign{}
\endlastfoot
⟨0.750 ∧ 0.950⟩\(\mathcal{N}\) & Habitable Zone \\
⟨0.950 ∧ 1.385⟩\(\mathcal{N}\) & Hospitable Zone \\
⟨1.385 ∧ 1.770⟩\(\mathcal{N}\) & Habitable Zone \\
\end{longtable}

It has also been suggested that ``desert'' planemons (think Dune,
Tattooine) might orbit in the zone between ⟨0.500 ∧ 0.750⟩N and we might
call this the ``desert planemon zone'', which would be, by definition,
\textbf{parahabitable} to \textbf{habitable} (but mostly the former).

\begin{longtable}[]{@{}cl@{}}
\toprule\noalign{}
Orbital Range & \\
\midrule\noalign{}
\endhead
\bottomrule\noalign{}
\endlastfoot
⟨0.500 ∧ 0.750⟩\(\mathcal{N}\) & Parahabitable \\
⟨0.750 ∧ 0.950⟩\(\mathcal{N}\) & Habitable Zone \\
⟨0.950 ∧ 1.385⟩\(\mathcal{N}\) & Hospitable Zone \\
⟨1.385 ∧ 1.770⟩\(\mathcal{N}\) & Habitable Zone \\
\end{longtable}

\subsection{The Frost Line (ϝ)}\label{the-frost-line-ux3dd}

Research indicates that beyond a distance of about \(a = 4.850\;AU\) in
our Solar system, water cannot remain liquid due to insufficient
irradiance from the Sun. This distance is sometimes termed the ``Frost
Line'' or ``Ice Line'', and an orbital distance of \(a = 4.850N\) is the
value we set for this outer limit.

For instance: - Mars' orbit in our own Solar system is
\(a = 1.524\;AU\), well within the \(1.770N\) limit - The asteroid belt
is ≈ ⟨2.2 ∧ 3.2⟩AU, beyond \(1.77\mathcal{N}\), but still within the
\(4.850\;AU\) ϝ limit. This region in our Solar system does not host a
sizeable planemon (and likely never did), but if one were to exist
there, it would probably be parahabitable due to the orbital distance
from the Sun.

This gives us another range of orbits we can add to our accounting:

\begin{longtable}[]{@{}cl@{}}
\toprule\noalign{}
Orbital Range & \\
\midrule\noalign{}
\endhead
\bottomrule\noalign{}
\endlastfoot
⟨0.500 ∧ 0.750⟩\(\mathcal{N}\) & Parahabitable \\
⟨0.750 ∧ 0.950⟩\(\mathcal{N}\) & Habitable Zone \\
⟨0.950 ∧ 1.385⟩\(\mathcal{N}\) & Hospitable Zone \\
⟨1.385 ∧ 1.770⟩\(\mathcal{N}\) & Habitable Zone \\
⟨1.770 ∧ 4.850⟩\(\mathcal{N}\) & Parahabitable \\
\end{longtable}

Jupiter's orbit is at \(a = 5.204\;AU\), well \emph{beyond} the
\(4.850\;AU\) limit, and things just get colder from there, so we can
specify that if any kind of ``life'' does exist in this region it is
likely to be extremophile by Earth standards, which WCB denotes as
``\textbf{\emph{xenotic}}''.

\begin{longtable}[]{@{}cl@{}}
\toprule\noalign{}
Orbital Range & \\
\midrule\noalign{}
\endhead
\bottomrule\noalign{}
\endlastfoot
⟨0.500 ∧ 0.750⟩\(\mathcal{N}\) & Parahabitable \\
⟨0.750 ∧ 0.950⟩\(\mathcal{N}\) & Habitable Zone \\
⟨0.950 ∧ 1.385⟩\(\mathcal{N}\) & Hospitable Zone \\
⟨1.385 ∧ 1.770⟩\(\mathcal{N}\) & Habitable Zone \\
⟨1.770 ∧ 4.850⟩\(\mathcal{N}\)N & Parahabitable \\
4.850\(\mathcal{N}\) → & Xenotic \\
\end{longtable}

Similarly, any ``life'' that might come to be on a body orbiting closer
than 0.500N would also be xenotic:

\begin{longtable}[]{@{}cl@{}}
\toprule\noalign{}
Orbital Range & \\
\midrule\noalign{}
\endhead
\bottomrule\noalign{}
\endlastfoot
← 0.500\(\mathcal{N}\) & Xenotic \\
⟨0.500 ∧ 0.750⟩\(\mathcal{N}\) & Parahabitable \\
⟨0.750 ∧ 0.950⟩\(\mathcal{N}\) & Habitable Zone \\
⟨0.950 ∧ 1.385⟩\(\mathcal{N}\) & Hospitable Zone \\
⟨1.385 ∧ 1.770⟩\(\mathcal{N}\) & Habitable Zone \\
⟨1.770 ∧ 4.850⟩\(\mathcal{N}\) & Parahabitable \\
4.850\(\mathcal{N}\) → & Xenotic \\
\end{longtable}

Finally, we differentiate between inner and outer zones, and define
notations for each:

\begin{longtable}[]{@{}cll@{}}
\toprule\noalign{}
Orbital Range & & Notation \\
\midrule\noalign{}
\endhead
\bottomrule\noalign{}
\endlastfoot
← 0.500\(\mathcal{N}\) & Inner Xenotic Zone & \(Z_{IX}\) \\
⟨0.500 ∧ 0.750⟩\(\mathcal{N}\) & Inner Parahabitable Zone &
\(Z_{IP}\) \\
⟨0.750 ∧ 0.950⟩\(\mathcal{N}\) & Inner Habitable Zone & \(Z_{IH}\) \\
⟨0.950 ∧ 1.385⟩\(\mathcal{N}\) & Hospitable Zone & \(Z_{H}\) \\
⟨1.385 ∧ 1.770⟩\(\mathcal{N}\) & Outer Habitable Zone & \(Z_{OH}\) \\
⟨1.770 ∧ 4.850⟩\(\mathcal{N}\) & Outer Parahabitable Zone &
\(Z_{OP}\) \\
4.850\(\mathcal{N}\) → & Outer Xenotic Zone & \(Z_{OX}\) \\
\end{longtable}

This gives us a full inventory of orbital limits for any star system we
choose to devise.

\section{Abstract}\label{abstract-9}

\textbf{Major Topics:}\\
- Introduction of \textbf{thermozones} as orbital bands derived from
multiples of the Nucleal Orbit (𝒩).\\
- Mapping of thermozones to \textbf{Ontozones} (habitability
categories): xenotic, parahabitable, habitable, hospitable.\\
- Distinct naming scheme rooted in \textbf{Latin/Greek etymologies}:\\
- \emph{Igniozone, Calorozone, Heliozone, Solarazone, Hiberozone,
Brumazone, Cryozone}.\\
- \textbf{Thermozone Limit Notation (H₀\ldots H₅):} standardized
subscripts marking orbital cutoffs (0.5𝒩 through 4.85𝒩).\\
- Relationship of 𝒩 to Solarazone: always lies 11.49\% inward from its
inner edge.\\
- Provides a consistent, mnemonic framework for discussing orbital
corridors across any star system.

\textbf{Key Terms \& Symbols:}\\
- \textbf{Thermozone names:} Igniozone, Calorozone, Heliozone,
Solarazone, Hiberozone, Brumazone, Cryozone.\\
- \textbf{Notation:} \(Z_{..}\) for ontozones, H₀--H₅ for thermozone
limits.\\
- \textbf{Nucleal Orbit (𝒩):} always located inside Solarazone.\\
- \textbf{Ontozones:} Inner/Outer Xenotic, Parahabitable, Habitable,
Hospitable.

\textbf{Cross-Check Notes:}\\
- Several \textbf{new glossary terms} staged: thermozone names,
\emph{Thermozone Limit Notation (H₀--H₅)}.\\
- Builds directly on \textbf{Nucleal Orbit (𝒩)} (from Stars 03).\\
- Provides the \textbf{didactic bridge} between raw stellar flux math
and corridor naming for worldbuilders.

\chapter{Star System Thermozones}\label{star-system-thermozones}

We've already introduced the term Habitable Zone before, sometimes also
prosaically referred to as ``The Goldilocks Zone''.

\begin{quote}
\textbf{Hippy}: Silliness!
\end{quote}

Well\ldots. Scientists \emph{do} try to keep things accessible for those
not familiar with the official lingo.

Anyway, broadly speaking, this is the range of orbital distances around
a given star in which an orbiting planemon might reasonably be expected
to retain liquid water and a reasonably dense atmospheric envelope. In
the previous section, we defined the \emph{parahabitable},
\emph{habitable}, and \emph{hospitable} zones as occupying this region.

\begin{quote}
\textbf{Keppy}: But this is based on \ldots{} what?
\end{quote}

I'm glad you asked; it's based on how much energy (irradiance) the
planemon receives from its star compared to how much insolation the
Earth receives from the Sun (you may remember this concept from our
discussion of the \emph{nucleal orbit}. And \emph{that} gives us our
standard candle (if you'll pardon the pun). \#\# Naming The Zones And
Labeling Their Limits

\subsection{The Thermozones}\label{the-thermozones}

For ease of remembering these zones and their ontosomic characteristics
we use the \textbf{thermozone} naming system:

\begin{longtable}[]{@{}
  >{\raggedright\arraybackslash}p{(\linewidth - 8\tabcolsep) * \real{0.1064}}
  >{\centering\arraybackslash}p{(\linewidth - 8\tabcolsep) * \real{0.2979}}
  >{\raggedright\arraybackslash}p{(\linewidth - 8\tabcolsep) * \real{0.2766}}
  >{\raggedright\arraybackslash}p{(\linewidth - 8\tabcolsep) * \real{0.0851}}
  >{\raggedright\arraybackslash}p{(\linewidth - 8\tabcolsep) * \real{0.2340}}@{}}
\toprule\noalign{}
\begin{minipage}[b]{\linewidth}\raggedright
Thermozone
\end{minipage} & \begin{minipage}[b]{\linewidth}\centering
Orbital Range
\end{minipage} & \begin{minipage}[b]{\linewidth}\raggedright
\end{minipage} & \begin{minipage}[b]{\linewidth}\raggedright
Notation
\end{minipage} & \begin{minipage}[b]{\linewidth}\raggedright
\end{minipage} \\
\midrule\noalign{}
\endhead
\bottomrule\noalign{}
\endlastfoot
Igniozone & ← 0.500\(\mathcal{N}\) & Inner Xenotic Zone & \(Z_{IX}\) &
``Desert planemon Zone'' \\
Calorozone & ⟨0.500 ∧ 0.750⟩\(\mathcal{N}\) & Inner Parahabitable Zone &
\(Z_{IP}\) & \\
Heliozone & ⟨0.750 ∧ 0.950⟩\(\mathcal{N}\) & Inner Habitable Zone &
\(Z_{IH}\) & \\
Solarazone & ⟨0.950 ∧ 1.385⟩\(\mathcal{N}\) & Hospitable Zone &
\(Z_{H}\) & \\
Hiberozone & ⟨1.385 ∧ 1.770⟩\(\mathcal{N}\) & Outer Habitable Zone &
\(Z_{OH}\) & \\
Brumazone & ⟨1.770 ∧ 4.850⟩\(\mathcal{N}\) & Outer Parahabitable Zone &
\(Z_{OP}\) & \\
Cryozone & 4.850\(\mathcal{N}\) → & Outer Xenotic Zone & \(Z_{OX}\) &
``Glacier planemon Zone'' \\
\end{longtable}

These names are derived from: - \textbf{Igniozone}: Latin \emph{ignis},
``fire'' - \textbf{Calorozone}: Latin \emph{calor}, ``hot, heat'' -
\textbf{Heliozone}: Greek \emph{Helios}, an early name of the Sun god -
planemons in this region might be somewhat Earth-like in environment,
but generally warmer* - \textbf{Solarazone}: Latin \emph{solar}, from
\emph{Sol}, a title of the Sun god - planemons in this region are likely
to be very Earth-like in their environment* - \textbf{Hiberozone}: Latin
\emph{hiberno}, ``cold'' - \textbf{Brumazone}: Latin \emph{bruma},
``winter'' - \textbf{Cryozone}: Greek \emph{kryo}, ``cold''

* Assuming they are otherwise Earth-like in size and composition.
\#\#\#\# Thermozone Limit Notation For ease of reference, the limiting
orbital distances of the thermozones are denoted by an \emph{H}
accompanied by a subscript:

\begin{longtable}[]{@{}cc@{}}
\toprule\noalign{}
Notation & Orbital Distance \\
\midrule\noalign{}
\endhead
\bottomrule\noalign{}
\endlastfoot
H₀ & 0.500\(\mathcal{N}\) \\
H₁ & 0.750\(\mathcal{N}\) \\
H₂ & 0.950\(\mathcal{N}\) \\
H₃ & 1.385\(\mathcal{N}\) \\
H₄ & 1.770\(\mathcal{N}\) \\
H₅ & 4.850\(\mathcal{N}\) \\
\end{longtable}

Adding these to our earlier table:

\begin{longtable}[]{@{}
  >{\raggedright\arraybackslash}p{(\linewidth - 10\tabcolsep) * \real{0.1800}}
  >{\centering\arraybackslash}p{(\linewidth - 10\tabcolsep) * \real{0.2067}}
  >{\centering\arraybackslash}p{(\linewidth - 10\tabcolsep) * \real{0.2467}}
  >{\raggedright\arraybackslash}p{(\linewidth - 10\tabcolsep) * \real{0.1733}}
  >{\raggedright\arraybackslash}p{(\linewidth - 10\tabcolsep) * \real{0.0533}}
  >{\raggedright\arraybackslash}p{(\linewidth - 10\tabcolsep) * \real{0.1400}}@{}}
\toprule\noalign{}
\begin{minipage}[b]{\linewidth}\raggedright
\end{minipage} & \begin{minipage}[b]{\linewidth}\centering
\end{minipage} & \begin{minipage}[b]{\linewidth}\centering
\end{minipage} & \begin{minipage}[b]{\linewidth}\raggedright
\end{minipage} & \begin{minipage}[b]{\linewidth}\raggedright
Notation
\end{minipage} & \begin{minipage}[b]{\linewidth}\raggedright
\end{minipage} \\
\midrule\noalign{}
\endhead
\bottomrule\noalign{}
\endlastfoot
Igniozone & ← H₀ & ← 0.500\(\mathcal{N}\) & Inner Xenotic Zone &
\(Z_{IX}\) & ``Desert planemon Zone'' \\
Calorozone & ⟨H₀ ∧ H₁⟩ & ⟨0.500 ∧ 0.750⟩\(\mathcal{N}\) & Inner
Parahabitable Zone & \(Z_{IP}\) & \\
Heliozone & ⟨H₁ ∧ H₂⟩ & ⟨0.750 ∧ 0.950⟩\(\mathcal{N}\) & Inner Habitable
Zone & \(Z_{IH}\) & \\
Solarazone & ⟨H₂ ∧ H₃⟩ & ⟨0.950 ∧ 1.385⟩\(\mathcal{N}\) & Hospitable
Zone & \(Z_{H}\) & \\
Hiberozone & ⟨H₃ ∧ H₄⟩ & ⟨1.385 ∧ 1.770⟩\(\mathcal{N}\) & Outer
Habitable Zone & \(Z_{OH}\) & \\
Brumazone & ⟨H₄ ∧ H₅⟩ & ⟨1.770 ∧ 4.850⟩\(\mathcal{N}\) & Outer
Parahabitable Zone & \(Z_{OP}\) & \\
Cryozone & H₅ → & 4.850\(\mathcal{N}\) → & Outer Xenotic Zone &
\(Z_{OX}\) & ``Glaci planemon Zone'' \\
\end{longtable}

This gives us a very robust way of discussing orbital distances in any
star system.

Note that the \emph{nucleal orbit}, being always \(\mathcal{N} = 1.0N\),
always falls within the Solarazone. In fact, it always falls at 11.49\%
\emph{into} the Solarazone from its inner edge.

\section{Abstract}\label{abstract-10}

\textbf{Major Topics:}\\
- Definition of the \textbf{Perannual Orbit (𝒫):} the orbital distance
in a star system where a planemon completes exactly one Earth sidereal
year (365.256363 ephemeris days).\\
- Distinction between \textbf{sidereal year} (fixed stars, used for 𝒫)
and \textbf{tropical year} (surface experience of seasons).\\
- Calculation of orbital period using Kepler's Third Law in
Solar-relative units:\\
- \(P = \sqrt{\dfrac{a^3}{M+m}}\)\\
- \(a = \sqrt[3]{P^2 (M+m)}\)\\
- \(M+m = \dfrac{a^3}{P^2}\)\\
- Effect of planemon mass (\emph{m}) on orbital period (generally
negligible but measurable; e.g., Earth's mass shifts orbital period by
\textasciitilde47 seconds).\\
- 𝒫 compared with the \textbf{Nucleal Orbit (𝒩):}\\
- Can be \textbf{intranucleal} (inside 𝒩) or \textbf{extranucleal}
(outside 𝒩).\\
- Only coincides with 𝒩 when \(M = 1⊙\) (and \(m = 1⨁\) ideally).\\
- Unlike 𝒩 (irradiance-based), 𝒫 depends on \textbf{mass}, making it a
\textbf{temporal reference} rather than a thermal one.\\
- Final simplification: for the perannual orbit distance,\\
- \(\mathcal{P} = \sqrt[3]{M+m}\) (including planemon mass).\\
- \(\mathcal{P} = \sqrt[3]{M}\) (ignoring planemon mass).

\textbf{Key Terms \& Symbols:}\\
- \textbf{Perannual Orbit (𝒫):} distance where orbital period = 1 Earth
sidereal year.\\
- \textbf{Sidereal Year:} 365.256363 ephemeris days; fixed-star frame
reference.\\
- \textbf{Intranucleal / Extranucleal:} perannual orbit lies inside or
outside the nucleal orbit.\\
- \textbf{Symbols:}\\
- \(P\) = orbital period (in sidereal years).\\
- \(a\) = semi-major axis (AU).\\
- \(M\) = stellar mass (⊙).\\
- \(m\) = planemon mass (⊙).\\
- 𝒫 = perannual orbital distance.\\
- 𝒩 = nucleal orbit.

\textbf{Cross-Check Notes:}\\
- \textbf{New glossary entries needed:} Perannual Orbit (𝒫),
Intranucleal, Extranucleal.\\
- Reinforces dual anchor system in WCB: 𝒩 (thermal/irradiance-based)
vs.~𝒫 (temporal/mass-based).\\
- Links forward to \emph{M002 --- Stars 06: Relating the Nucleal and
Perannual Orbits}.

\section{The Perannual Orbit}\label{the-perannual-orbit}

There is one remaining essential star system orbit, which I have called
the \textbf{perannual} orbit. The word comes from the Latin \emph{per
annum}, meaning ``per year'' or ``each year'', and the name reflects
that this is the orbit in any star system which has an orbital period
(\emph{P}) of exactly one Earth year. \#\#\#\#\# IMPORTANT
\textgreater{} ``One Earth Year'' in this case is the duration of
Earth's complete orbit around the Sun relative to the larger reference
frame of the ``fixed'' stars; thus this is called the \textbf{sidereal
year}, from the Latin \emph{sidus}, ``star''. This is measured and
denoted in terms of \textbf{ephemeris days} --- which are \emph{defined}
to be exactly 86400 \emph{seconds} in duration. Thus, the sidereal year
(and, consequently, the perannual year) has a duration of: \[\
\begin{gather}
365.256363004 \quad \text{Ephemeris Days} \\
or \\
365^d\;6^h\;9^m\; 9.763545^s
\end{gather}
\] \textgreater{} This is \emph{not} a ``year'' as experienced by
inhabitants on the surface of a planemon on this orbit (that is called
the \textbf{tropical year}, which is in part dependent upon the
\emph{rotational period} of the planemon, itself); this is the
\textbf{sidereal year}. \textgreater{} \textgreater{} Please see
{[}{[}Units and Measures of Time ✓{]}{]} for a more in-depth discussion
of this topic.

We denote the perannual year as \(\mathcal{P}\), and its location in the
star system \emph{is not constant} (the same as the \emph{nucleal orbit}
(\(\mathcal{N}\)) but is \emph{determined} by the mass of the star(s),
and -- to a small but measurable degree -- by the mass of the planemon.

Please see {[}{[}M002 - Stars --- 06 Relating the Nucleal and Perannual
Orbits ✓{]}{]} for an in-depth exploration of this relationship.

The perannual orbit is determined not by the luminosity of the star(s)
in the system but by \textbf{mass}, mostly of the stars(s), but the mass
of the planemon can become a calculatory relevant factor if it is a
significant fraction of the mass of the star(s).

The perannual orbit is an \emph{orbital distance}, but it is predicated
on the \textbf{period} of that orbit --- how long it takes the planemon
to complete one entire orbit (measured in Earth years). \textbf{ANY}
planemon orbital period is calculated (in relative terms) by: \[
\begin{align}
    P &= \sqrt{\dfrac{a^3}{M+m}} \\
    a &= \sqrt[3]{P^2 (M+m)} \\
    M + m &= \dfrac{a^3}{P^2} \qquad \text{Believe it or not, this has its uses}
\end{align}
\] Where: - \emph{P} = the planemon's orbital period in Earth sidereal
years - \emph{a} = the measure of the semi-major axis of the planemon's
orbit - \emph{M} = the mass of the star(s) in Solar masses - \emph{m} =
the mass of the planemon (also expressed in \emph{Solar}) masses

In many cases (such as that of Earth), \emph{m} is such a small number
that it can be ignored without drastically altering the value of
\emph{P}. In the case of Earth:

\begin{itemize}
\tightlist
\item
  \(M = 1⊙\)
\item
  \(m = 0.000003003\)⊙ (\(3.003 \times 10^{-6}\)) --- three
  \emph{millionths} of the Sun's mass
\item
  \(a = 1\;AU\)
\end{itemize}

Calculating with \textbf{only} the Sun's mass: \[
\begin{align}
    P &= \sqrt{\dfrac{a^3}{M}} = \sqrt{\dfrac{1^3}{1}} = \sqrt{\dfrac{1}{1}} = \sqrt{1} = 1\;\text{years}       
\end{align}
\] Calculating with \textbf{both} masses: \[
\begin{align}
    P &= \sqrt{\dfrac{a^3}{M + m}} \\
     &= \sqrt{\dfrac{1^3}{1 + 0.000003003}} \\
     &= \sqrt{\dfrac{1}{1.000003003}} \\
     &= \sqrt{0.999997} \\
     P &= 0.9999985\;\text{years}       
\end{align}
\] \ldots{} a difference of about 47.384 \emph{seconds}.

\begin{quote}
🔍 \textbf{Takeaways}: \textgreater The \emph{perannual orbit} defines
the location in any star system where a planemon would complete one
sidereal Earth year. \textgreater It may be closer-in than the nucleal
orbit (\textbf{intranucleal}) or farther out than the nucleal orbit
(\textbf{extranucleal}). \textgreater If it is ever \emph{the same as
the nucleal orbit}, then the star(s)' mass(es) must be \(M = 1⊙\), and
--- ideally --- the planemon's mass must be \(m = 1⨁\).
\textgreater Unlike the \emph{nucleal orbit} (which depends on
\emph{stellar irradiance}), the perannual orbit \emph{depends only the
mass of the system} --- and serves as a \emph{temporal} rather than
\emph{thermal} reference point.
\end{quote}

As shown above, the \emph{distance} of any orbit can be calculated from
the period and the masses via: \[
a = \sqrt[3]{P^2 (M+m)}
\] \ldots{} but if we already know that \(P = 1\), it drops out of the
equation: \[
a = \sqrt[3]{M+m}
\] \ldots{} such that the distance of the orbit is simply the cube-root
of the sum of the masses.

For clarity, we denote the \emph{distance} of the perannual orbit with a
\(\mathcal{P}\) (for \emph{perannual}), so our equation becomes: \[
\begin{align}
\mathcal{P} &= \sqrt[3]{M+m} &&\text{When taking into account both masses} \\
\mathcal{P} &= \sqrt[3]{M} &&\text{When using only the central mass} \\
\end{align}
\]

\section{Abstract}\label{abstract-11}

\textbf{Major Topics:}\\
- Relationship between the \textbf{Nucleal Orbit (𝒩)} and the
\textbf{Perannual Orbit (𝒫)}.\\
- Both are \textbf{orbital environs}, not strict limiting distances ---
they describe contextual properties of a star system.\\
- Restatement of definitions:\\
- 𝒩 = \(\sqrt{L}\) (AU), where \emph{L} = stellar luminosity (⊙).\\
- 𝒫 = \(\sqrt[3]{M+m}\) (AU), or \(\sqrt[3]{M}\) if planemon mass
\emph{m} is disregarded.\\
- Mass--luminosity link: \(M = \sqrt[3]{L}\) → allows
cross-approximation between 𝒩 and 𝒫.\\
- Approximation formulas:\\
- \(\mathcal{P} \approx \sqrt[6]{L}\) (perannual from luminosity).\\
- \(\mathcal{N} \approx \sqrt{M^3}\) (nucleal from mass).\\
- Cross-relations:\\
- \(\mathcal{P} \approx \sqrt[3]{\mathcal{N}}\)\\
- \(\mathcal{N} \approx \mathcal{P}^3\)\\
- Caution: these relations are \textbf{approximations}; robust
calculation of 𝒩 and 𝒫 is recommended for
precision:contentReference{oaicite:0}.

\textbf{Key Terms \& Symbols:}\\
- \textbf{𝒩 (Nucleal Orbit):} irradiance-based orbital benchmark.\\
- \textbf{𝒫 (Perannual Orbit):} period-based orbital benchmark.\\
- \textbf{Approximation relations:} linking 𝒩 and 𝒫 through stellar
mass--luminosity scaling.

\textbf{Cross-Check Notes:}\\
- No new glossary entries beyond 𝒩 and 𝒫 (already staged in prior
files).\\
- This section functions as a \textbf{bridge note}, unifying the thermal
and temporal anchors in WCB orbital design.\\
- Emphasizes \textbf{approximation vs.~precision}: usable shortcuts
exist, but exact calculation is preferable.

We have explored both {[}{[}M002 - Stars --- 03 The Nucleal Orbit
✓\textbar The Nucleal Orbit{]}{]} and {[}{[}M002 - Stars --- 05 The
Perannual Orbit ✓\textbar The Perannual Orbit{]}{]}. These two are not
\emph{limiting distances}, but \textbf{orbital environs} which both
describe and contribute to the ontosomic nature of planemons.

As a quick review: - \textbf{Nucleal Orbit}: that orbit (expressed in
AU) at which a planemon receives from its star(s) the same radiant flux
as Earth receives from the Sun at one Astronomical Unit distance,
calculated by: \[
    \mathcal{N} = \sqrt{L}
\]

Where \emph{L} = Luminosity of the star(s) as expressed in Solar units,
⊙

\begin{itemize}
\tightlist
\item
  \textbf{Perannual Orbit}: that orbit (expressed in AU) which has an
  orbital period of exactly one sidereal Earth year, calculated by: \[
  \mathcal{P} = \sqrt[3]{M+m}
  \] If we disregard the mass of the planemon \emph{m}: \[
  \mathcal{P} = \sqrt[3]{M}
  \] And we saw in {[}{[}M002 - Stars --- 02 Parameters ✓{]}{]} that
  through relationship: \[
  M = \sqrt[3]{L}
  \] This means that:
\item
  The perannual orbit can be \emph{approximated} directly from the
  luminosity by: \[
  \mathcal{P} \approx \sqrt[3]{\sqrt[3]{L}} \approx \sqrt[6]{L}
  \]
\item
  The nucleal orbit can be \emph{approximated} directly from the mass
  by: \[
  \mathcal{N} \approx \sqrt{M^3}
  \] And, by extension either can be \emph{approximated} from the other
  by: \[
  \begin{align}
  \mathcal{P} &\approx \sqrt[6]{mathcal{N}^2} \approx \sqrt[3]{\mathcal{N}} \\
  \mathcal{N} &\approx \mathcal{P}^3
  \end{align}
  \] \textbf{REMEMBER}
\item
  Both \(\mathcal{N}\) and \(\mathcal{P}\) are measured in astronomical
  units, not time!
\item
  These last four equations are \textbf{approximations}; in most cases
  they'll be ``accurate enough'', but calculating\(\mathcal{N}\) and
  \(\mathcal{P}\) robustly is always advised.
\end{itemize}

\section{Abstract}\label{abstract-12}

\textbf{Major Topics:}\\
- Review of the \textbf{Standard Stellar Parameter Equations} linking
temperature (T), mass (M), radius (R), lifetime (Q), and luminosity
(L).\\
- Recognition that while the standard exponents work well for most
\textbf{Main Sequence} stars, observed stellar data show that slight
adjustments produce a closer fit.\\
- \textbf{Refinements introduced:}\\
- Exponent for \(T ↔︎ M\) increased from 1.98 → 2.0.\\
- Addition of direct calculation routes to/from \textbf{luminosity (L)},
simplifying downstream math (especially for \emph{Stars 08: Sun-Like
Stars}).\\
- For higher precision, recommended exact values:\\
- \(7.6 ≈ 7.5778\)\\
- \(3.8 ≈ 3.7889\):contentReference{oaicite:0}.\\
- Emphasis: WCB prioritizes \textbf{plausible world construction} over
strict theoretical purity, so these adjusted exponents serve the design
goals better.

\textbf{Key Terms \& Symbols:}\\
- \textbf{Standard Stellar Parameter Equations:} Baseline power-law
relations for T, M, R, Q, and L.\\
- \textbf{Modified Parameters:} Slightly adjusted exponents improving
fit across observed data.\\
- \textbf{Direct Luminosity Relations:} New formulas linking L with
other parameters for simplified application.

\textbf{Cross-Check Notes:}\\
- No \textbf{new glossary entries} introduced; the adjustments are
refinements to existing equations.\\
- Functions as a \textbf{supportive methods note} --- improves accuracy
and usability of WCB stellar modeling.\\
- Directly prepares for \emph{Stars 08: Sun-Like Stars}, where these
refined forms are applied.

\chapter{Stars --- 2.07 Fine-tuning Stellar
Parameters}\label{stars-2.07-fine-tuning-stellar-parameters}

\section{Standard Parameter
Equations}\label{standard-parameter-equations}

The Standard Parameter Equations (see {[}{[}M002 - Stars --- 02
Parameters ✓{]}{]}):

\begin{longtable}[]{@{}
  >{\centering\arraybackslash}p{(\linewidth - 6\tabcolsep) * \real{0.2400}}
  >{\centering\arraybackslash}p{(\linewidth - 6\tabcolsep) * \real{0.2267}}
  >{\centering\arraybackslash}p{(\linewidth - 6\tabcolsep) * \real{0.1733}}
  >{\centering\arraybackslash}p{(\linewidth - 6\tabcolsep) * \real{0.3600}}@{}}
\toprule\noalign{}
\begin{minipage}[b]{\linewidth}\centering
Temperature(T)
\end{minipage} & \begin{minipage}[b]{\linewidth}\centering
Mass(M)
\end{minipage} & \begin{minipage}[b]{\linewidth}\centering
Radius(R)
\end{minipage} & \begin{minipage}[b]{\linewidth}\centering
Lifetime(Q)
\end{minipage} \\
\midrule\noalign{}
\endhead
\bottomrule\noalign{}
\endlastfoot
\(T=\sqrt[1.98]{M}\) & \(M=\sqrt[0.9]{R}\) & \(R=M^{0.9}\) &
\(Q=M^{-2.5}\) \\
\(T=\sqrt[1.8]{R}\) & \(M=T^{1.98}\) & \(R=T^{1.8}\) &
\(Q \approx \sqrt[-0.36]{R}\) \\
\(T=Q^{-0.2}\) & \(M=Q^{-0.4}\) & \(R=Q^{-0.36}\) & \(Q=T^{-5}\) \\
\end{longtable}

\ldots{} \emph{generally} work well for most \textbf{Main Sequence}
stars, but a survey of known stars in the Solar neighborhood ---

\begin{quote}
\textbf{Hippy}: ``Wha--''
\end{quote}

\ldots{} \emph{which is too complex and extensive to detail here} ---
suggests that \emph{modest} adjustments to a couple of key exponents
yield parameter equations that better reflect observed stellar
characteristics. Since worldmaking prioritizes plausible-world
construction over strict theoretical purity, these revised values offer
better performance across the mass range of interest. \#\#\# Modified
Parameters Table\# Main Sequence Stellar Equations of State

\begin{longtable}[]{@{}
  >{\centering\arraybackslash}p{(\linewidth - 8\tabcolsep) * \real{0.1484}}
  >{\centering\arraybackslash}p{(\linewidth - 8\tabcolsep) * \real{0.1328}}
  >{\centering\arraybackslash}p{(\linewidth - 8\tabcolsep) * \real{0.2109}}
  >{\centering\arraybackslash}p{(\linewidth - 8\tabcolsep) * \real{0.2109}}
  >{\centering\arraybackslash}p{(\linewidth - 8\tabcolsep) * \real{0.2969}}@{}}
\toprule\noalign{}
\begin{minipage}[b]{\linewidth}\centering
Temperature(T)
\end{minipage} & \begin{minipage}[b]{\linewidth}\centering
Mass(M)
\end{minipage} & \begin{minipage}[b]{\linewidth}\centering
Radius(R)
\end{minipage} & \begin{minipage}[b]{\linewidth}\centering
Lifetime(Q)
\end{minipage} & \begin{minipage}[b]{\linewidth}\centering
\end{minipage} \\
\midrule\noalign{}
\endhead
\bottomrule\noalign{}
\endlastfoot
\(T=\sqrt{M}\) & \(M=\sqrt[0.9]{R}\) & \(R=M^{0.9}\) & \(Q=M^{-2.5}\) &
\(L=M^{3.8}\) \\
\(T=\sqrt[1.8]{R}\) & \(M=T^2\) & \(R=T^{1.8}\) &
\(Q \approx \sqrt[-0.36]{R}\) & \(L \approx R^{4.\bar{2}}\) \\
\(T=Q^{-0.2}\) & \(M=Q^{-0.4}\) & \(R=Q^{-0.36}\) & \(Q=T^{-5}\) &
\(L = T^{7.6}\) \\
\(T = \sqrt[7.6]{L}\) & \(M=\sqrt[3.8]{L}\) &
\(R\approx\sqrt[4.\bar2]{L}\) & \(Q=L^{-1.52}\) &
\(L=\sqrt[-1.52]{Q}\) \\
\end{longtable}

\textbf{Notes}: - The parameter relationship that changed from the
previous table was \(T ↔︎ M\), where the exponent increased slightly from
\(1.98\) to \(2.0\) - The \textbf{major change} is the addition of
direct calculation for the parameters to-and-from luminosity; these are
included for the purpose of simplifying much of the math related to
{[}{[}M002 - Stars --- 08 \texttt{Sun-Like} Stars ✓{]}{]}. - \textbf{For
\emph{greatest accuracy}}: - The exponent \(7.6\) can be more precisely
specified as \(7.5778\) - The exponent \(3.8\) can be more precisely
specified as \(3.7889\)

\section{Abstract}\label{abstract-13}

\textbf{Major Topics:}\\
- Critique of vague astronomical usage of ``Sun-like star'' and proposal
of a \textbf{clearer WCB classification system} grounded in orbital
habitability.\\
- Definitions nested by \textbf{Ontozone boundaries} and
\textbf{perannual orbits (𝒫):}\\
- \textbf{Solar Analogs:} perannual orbits spanning 0.500--4.850 AU
(Inner → Outer Parahabitable Zone, H₀--H₅); spectral types F2--K9.\\
- \textbf{Solar Cognates:} perannual orbits spanning 0.750--1.770 AU
(Inner → Outer Habitable Zone, H₁--H₄); spectral types F7.62--K1.11.\\
- \textbf{Solar Twins:} perannual orbits spanning 0.950--1.385 AU
(Hospitable Zone, H₂--H₃); spectral types G1.04--G7.73.\\
- Hierarchical logic: all Twins ⊂ Cognates ⊂
Analogs:contentReference{oaicite:0}.\\
- Mathematical framework for deriving stellar parameters:\\
- Cross-relations between luminosity, perannual orbit (𝒫), and
thermozone limits (H₀--H₅).\\
- Generalized equation for stellar luminosity given thermozone factor
(λ).\\
- Direct temperature relation: \(K = 5800(\lambda^{-0.3191})\).\\
- Thermal Axis for Perannual Orbits: diagram showing stellar temperature
vs.~spectral type for H₀--H₅.\\
- \textbf{Orbital Habitability Index (OHI):} scalar (0.00--1.00)
quantifying relative habitability based on distance from the nucleal
orbit (𝒩).\\
- Piecewise function distinguishes intranucleal vs.~extranucleal
cases.\\
- Index peaks at 1.00 for D = 𝒩, declines linearly to 0.00 at H₀ and
H₅.\\
- Illustrated via habitability atlas plate.

\textbf{Key Terms \& Symbols:}\\
- \textbf{Solar Analog, Solar Cognate, Solar Twin:} nested categories of
Sun-like stars based on ontozone/perannual placement.\\
- \textbf{𝒫 (Perannual Orbit):} temporal anchor.\\
- \textbf{𝒩 (Nucleal Orbit):} thermal anchor.\\
- \textbf{Thermozones (H₀--H₅):} reference corridors.\\
- \textbf{λ (Scaling Factor):} ratio linking perannual orbit to nucleal
orbit.\\
- \textbf{OHI (Orbital Habitability Index):} 0.00--1.00 habitability
scalar.

\textbf{Cross-Check Notes:}\\
- \textbf{New glossary entries needed:} Solar Analog, Solar Cognate,
Solar Twin, Orbital Habitability Index (OHI).\\
- All other symbols and terms already staged in prior files (𝒩, 𝒫,
thermozones, H₀--H₅, λ).\\
- This section bridges stellar classification with \textbf{habitability
indices}, anchoring ``Sun-like'' terminology directly to WCB orbital
framework.

\chapter{Solar Analogs, Cognates, and
Twins}\label{solar-analogs-cognates-and-twins}

The published literature often speaks of ``solar analog'' stars, but
tends to be distressingly vague about exactly what the term means.
Generally speaking, it means ``a star very much like the Sun''.

\begin{quote}
\textbf{Keppy}: And that doesn't help at all --- that could mean
\emph{any} star, really.
\end{quote}

You're right; so, for our purposes we have our own definitions, based on
\emph{orbits} and the ontozones. But, first, a survey of existing
terminology. \#\# Existing Definitions A ``Sun-like star'' is a broad
term used to describe stars that share characteristics with our own Sun.
Astronomers often categorize them into a hierarchy based on their
\emph{physical} similarity to the Sun. They all need to be main-sequence
stars**, actively fusing hydrogen into helium in their core, like our
Sun. Otherwise:

\textbf{Solar-type Stars:} This is the broadest category. These stars
are broadly similar to the Sun in mass and evolutionary state. Key
characteristics include:\\
- \textbf{Spectral type:} Typically F8V (6300 K) to K2V (4700 K) ---
more on this below.

\textbf{Solar Analogs:} These stars are more similar to the Sun than
general solar-type stars, conforming to stricter criteria: -
\textbf{Temperature:} Within approximately 500 Kelvin (K) of the Sun's
temperature (which is about 5800 K) --- between 5300 K (G7V) and 6300 K
(F8V).\\
\textbf{Solar Twins:} This is the most restrictive category, for stars
that are nearly identical to the Sun. The idea is that they are
virtually indistinguishable from our Sun in as many ways as possible: -
\textbf{Temperature:} - Within a very narrow range, typically ±10 K of
the Sun's temperature --- 5790 K (G2.1V) to 5810 K (G1.9V). - Some
definitions are even stricter, within ±5 K --- 5795 K (G2.05V) to 5805 K
(G1.95V). - \textbf{Age}: - \textbf{4.3 -- 4.7\,Gyr} (The Sun's age
±200\,Ma) - Sometimes as tight as \textbf{±100\,Ma}, i.e., \textbf{4.4
-- 4.6 \,Gyr} \#\# A Proposed, Clearer System For thesiastic purposes,
our classifications relate directly to the \emph{habitability potential
of orbiting planemos}, rather than just their parent stars' physical
resemblance to the Sun.

\textbf{Solar Analogs}: - Stars whose \emph{perannual orbits} fall
within ⟨0.500 ∧ 4.850⟩ AU, spanning from the Inner Parahabitable Zone to
the Outer Parahabitable Zone (H₀ -- H₅), spectral types F2 -- K9.
\textbf{Solar Cognates}: - Stars whose \emph{perannual orbits} fall
within ⟨0.750 ∧ 1.770⟩ AU, spanning from the Inner Habitable Zone to the
Outer Habitable Zone (H₁ -- H₄), spectral types F7.62 -- K1.11.
\textbf{Solar Twins}: - Stars whose \emph{perannual orbits} fall within
⟨0.950 ∧ 1.385⟩ AU, spanning the Hospitable Zone (H₂ -- H₃), spectral
types G1.04 -- G7.73. \emph{Thus}: - All \emph{Solar Twins} are also
\emph{Solar Cognates} and \emph{Solar Analogs}. - All \emph{Solar
Cognates} are also \emph{Solar Analogs}. - \emph{Solar Analogs}
encompass the \emph{Solar Cognate} and \emph{Solar Twin} categories.

\begin{quote}
\textbf{NOTE}: - This perannual-orbit-based requirement is largely
arbitrary, predicated on the thesiastic idea that a planemon that is
least different from Earth would have an orbital period the same as
Earth's. \#\#\#\#\# Nested Ontozonal Categories of Sun-like Stars
\end{quote}

!{[}{[}Solar Types Diagram\textbar300{]}{]} \#\#\# Calculating The
Spectral Types Previously, in {[}{[}M002 - Stars --- 06 Relating the
Nucleal and Perannual Orbits ✓{]}{]}, we established that the
\emph{distance} of the perannual orbit can be approximated by: \[
\mathcal{P} = \sqrt[3]{M}
\] \ldots{} and in {[}{[}M002 - Stars --- 07 Fine-tuning Stellar
Parameters ✓{]}{]}, we established the relationship: \[
L = M^{3.8}
\] \ldots{} which lets us calculate that: \[
\mathcal{P} = \sqrt[3]{\sqrt[3.8]{L}} = \sqrt[11.4]{L}
\] In {[}{[}M002 - Stars --- 04 Thermozone Orbits ✓{]}{]}, we
established that the thermozone limits are calculated by applying fixed
scaling factors to the \textbf{nucleal orbit distance}
(\(\mathcal{N}\)), which is calculated from the square-root of the
luminosity:

\begin{longtable}[]{@{}cc@{}}
\toprule\noalign{}
LimitingOrbit & Calculation \\
\midrule\noalign{}
\endhead
\bottomrule\noalign{}
\endlastfoot
\(H_0\) & \(0.500\sqrt{L}\) \\
\(H_1\) & \(0.750\sqrt{L}\) \\
\(H_2\) & \(0.950\sqrt{L}\) \\
\(H_3\) & \(1.385\sqrt{L}\) \\
\(H_4\) & \(1.770\sqrt{L}\) \\
\(H_5\) & \(4.850\sqrt{L}\) \\
\end{longtable}

This means that we can set: \[
\mathcal{P} = \sqrt[11.4]{L} \quad \text{equal to} \quad \mathcal{P} = 0.500\sqrt{L}
\] \ldots{} and solve for L: \[
\begin{align}
\sqrt[11.4]{L} &= 0.500\sqrt{L} \\
0.500 &= \dfrac{\sqrt[11.4]{L}}{\sqrt{L}} \\
&= L^{\frac{1}{11.4} -{\frac{1}{2}}} \\
&= L^{\frac{2}{22.8}-\frac{11.4}{22.8}} = L^{-\frac{9.4}{22.8}} \\
0.500 &= L^{-0.4123} \\
L &= \sqrt[-0.4123]{0.500} \\
&= 5.372\; ✓
\end{align}
\] Converting luminosity to temperature: \[
T = \sqrt[7.6]{L} = \sqrt[7.6]{5.372} = 1.248\;\odot
\] In {[}{[}M002 - Stars --- 02 Parameters ✓{]}{]}, we established the
following relationship between solar-unit temperature (\emph{T}) and
Kelvin temperature (\emph{K}) \[
K = 5800T
\] So, our star has a Kelvin temperature of: \[
K = 5800T = 5800(1.248) = 7235.97\;K
\]

\ldots{} and we can calculate the spectral class and type: \[
\begin{align}
\mathcal{S} &= \dfrac{\kappa - K}{þ}
\end{align}
\] Where: - K = the star's surface temperature in Kelvin - κ = the
\emph{upper bound} temperature of the relevant spectral class - þ = the
thermal interval constant for the relevant spectral class -
\(\mathcal{S}\) = the spectral \emph{type} number

Taken from the table:

\begin{longtable}[]{@{}
  >{\centering\arraybackslash}p{(\linewidth - 4\tabcolsep) * \real{0.1560}}
  >{\raggedleft\arraybackslash}p{(\linewidth - 4\tabcolsep) * \real{0.3394}}
  >{\raggedleft\arraybackslash}p{(\linewidth - 4\tabcolsep) * \real{0.5046}}@{}}
\toprule\noalign{}
\begin{minipage}[b]{\linewidth}\centering
SpectralClass
\end{minipage} & \begin{minipage}[b]{\linewidth}\raggedleft
\end{minipage} & \begin{minipage}[b]{\linewidth}\raggedleft
\end{minipage} \\
\midrule\noalign{}
\endhead
\bottomrule\noalign{}
\endlastfoot
O & 55000 & 3000 \\
B & 25000 & 1500 \\
A & 10000 & 250 \\
F & 7500 & 150 \\
G & 6000 & 100 \\
K & 5000 & 150 \\
M & 3500 & 110 \\
L & 2400 & 110 \\
T & 1300 & 70 \\
Y & 600 & 30 \\
\end{longtable}

Our Kelvin temperature is \(7235.97\;K\) which is an F-type star, so - κ
= 7500 - þ = 150 \[
\mathcal{S} = \dfrac{\kappa - K}{þ} = \dfrac{7500 - 7235.97}{150} = \dfrac{264.03}{150} = 1.76
\] So the spectral type of a star with a perannual orbit at 0.500 AU is
F 1.76 ✓.

\subsection{The Other End Of The
Range}\label{the-other-end-of-the-range}

This means that we can set: \[
\mathcal{P} = \sqrt[11.4]{L} \quad \text{equal to} \quad \mathcal{P} = 4.850\sqrt{L}
\] \ldots{} and solve for L: \[
\begin{align}
\sqrt[11.4]{L} &= 4.850\sqrt{L} \\
4.850 &= \dfrac{\sqrt[11.4]{L}}{\sqrt{L}} \\
&= L^{\frac{1}{11.4} -{\frac{1}{2}}} \\
&= L^{\frac{2}{22.8}-\frac{11.4}{22.8}} = L^{-\frac{9.4}{22.8}} \\
4.850 &= L^{-0.4123} \\
L &= \sqrt[-0.4123]{4.850} \\
&= 0.022\; ✓
\end{align}
\] Converting luminosity to temperature: \[
T = \sqrt[7.6]{L} = \sqrt[7.6]{0.022} = 0.605\;\odot
\] In {[}{[}M002 - Stars --- 02 Parameters ✓{]}{]}, we established the
following relationship between solar-unit temperature (\emph{T}) and
Kelvin temperature (\emph{K}) \[
K = 5800T
\] So, our star has a Kelvin temperature of: \[
K = 5800T = 5800(0.605) = 3503.85\;K
\] \ldots{} which is a K-Class star with a
\(\kappa = 5000 \text{  and  } þ = 150\), from which we can calculate
the spectral type by: \[
\mathcal{S} = \dfrac{\kappa - K}{þ} = \dfrac{5000 - 3503.85}{150} = \dfrac{1496.185}{150} = 9.975
\] So the spectral type of a star with a perannual orbit at \(4.850\) AU
is K9.975 ✓.

This means that the range of spectral types which host \emph{perannual
orbits within the \textbf{parahabitable} zone} is F1.76 -- K9.932, which
we can round for convenience to F2 -- K9.

\textbf{Why K9 and not M0} The exact math yields \textbf{F1.76 --
K9.97}. For presentation, we \textbf{round inward} at both ends of the
range to \textbf{F2 -- K9} so that all perannual orbits calculated for
stars within this range lies \textbf{strictly} inside the parahabitable
band, avoiding knife‑edge cases at the hot and cold limits. (This
convention absorbs small uncertainties in stellar parameters and
subclass mapping.)

\section{Generalizing The Equation}\label{generalizing-the-equation}

This logic can be extended for any Hₓ value: By generalizing the scaling
factor \textbf{λ}, we can calculate the relative stellar luminosity for
\textbf{any} perannual orbit distance: \[
\begin{align}
\sqrt[11.4]{L} &= \lambda\sqrt{L} \\
\lambda &= \dfrac{\sqrt[11.4]{L}}{\sqrt{L}} \\
&= L^{\frac{1}{11.4} - {\frac{1}{2}}} \\
&= L^{\frac{2}{22.8}-\frac{11.4}{22.8}} = L^{-\frac{9.4}{22.8}} \\
\lambda &= L^{-0.4123} \\
\therefore L &= \sqrt[-0.4123]{\lambda} \; ✓
\end{align}
\] \# A Final Determination Substituting all of the \(H_x\) values in
for λ:

\begin{longtable}[]{@{}
  >{\centering\arraybackslash}p{(\linewidth - 10\tabcolsep) * \real{0.1156}}
  >{\centering\arraybackslash}p{(\linewidth - 10\tabcolsep) * \real{0.1633}}
  >{\centering\arraybackslash}p{(\linewidth - 10\tabcolsep) * \real{0.1837}}
  >{\raggedleft\arraybackslash}p{(\linewidth - 10\tabcolsep) * \real{0.2585}}
  >{\centering\arraybackslash}p{(\linewidth - 10\tabcolsep) * \real{0.1088}}
  >{\raggedright\arraybackslash}p{(\linewidth - 10\tabcolsep) * \real{0.1701}}@{}}
\toprule\noalign{}
\begin{minipage}[b]{\linewidth}\centering
LimitingOrbit
\end{minipage} & \begin{minipage}[b]{\linewidth}\centering
ScalingFactor(λ)
\end{minipage} & \begin{minipage}[b]{\linewidth}\centering
Calculation
\end{minipage} & \begin{minipage}[b]{\linewidth}\raggedleft
\end{minipage} & \begin{minipage}[b]{\linewidth}\centering
SpectralType
\end{minipage} & \begin{minipage}[b]{\linewidth}\raggedright
\end{minipage} \\
\midrule\noalign{}
\endhead
\bottomrule\noalign{}
\endlastfoot
\(H_0\) & 0.500 & \(L = \sqrt[-0.4123]{0.500}\) & 5.372 & F1.760 &
Parahabitable \\
\(H_1\) & 0.750 & \(L = \sqrt[-0.4123]{0.750}\) & 2.009 & F7.615 &
Habitable \\
\(H_2\) & 0.950 & \(L = \sqrt[-0.4123]{0.950}\) & 1.132 & G1.043 &
Hospitable \\
\(H_3\) & 1.385 & \(L = \sqrt[-0.4123]{1.385}\) & 0.454 & G7.726 &
Hospitable \\
\(H_4\) & 1.770 & \(L = \sqrt[-0.4123]{1.770}\) & 0.250 & K1.108 &
Habitable \\
\(H_5\) & 4.850 & \(L = \sqrt[-0.4123]{4.850}\) & 0.022 & K9.972 &
Parahabitable \\
\end{longtable}

\begin{quote}
\textbf{Keppy}: Seems like a lot of calculating and converting\ldots{}
\end{quote}

Well, without going into the gory details, you can calculate the
relative or Kelvin temperature directly by: \[\begin{gather}
K =5800(\lambda^{-0.3191}) \\[1em]
T = \lambda^{-0.3191}
\end{gather}
\] \ldots{} which will allow you to calculate the spectral type for any
perannual orbit at any orbital distance, and the reverse calculations
are: \[
\begin{gather}
\lambda = \sqrt[-0.3191]{\dfrac{K}{5800}} \\[1em]
\lambda = \sqrt[-0.3191]{T}
\end{gather}
\]\\
\ldots{} which will give you the orbital distance of the perannual orbit
for a star of any given Kelvin temperature (\emph{K}) or relative
temperature (\emph{T}), since \(T=\frac{K}{5800}\). \#\#\#\# Thermal
Axis for Perannual Orbits This diagram shows the stellar surface
temperatures (\emph{K}) and corresponding spectral types required for a
star's \emph{perannual orbit} to fall on each thermozone boundary
\(H_0\)\hspace{0pt} through \(H_5\)\hspace{0pt}. The core equation
relates perannual distance scaling factor λ to stellar temperature
\emph{K}.

!{[}{[}Ontozones and Spectral Types\textbar650{]}{]}

\subsection{Orbital Habitability Index
(OHI)}\label{orbital-habitability-index-ohi}

The Orbital Habitability Index (OHI) is a measure of how likely a
planemon is to be habitable based on its orbit, with the nucleal orbit
assumed to be 100\% habitable and orbits closer-in and farther-out
becoming progressively less habitable. It is calculated using one of two
equations, depending on whether the orbit in question is
\emph{intranucleal} or \emph{extranucleal}:

The OHI provides a scalar measure (0.00--1.00) of the \emph{relative
biological viability} of a planemon orbit based on its distance from the
nucleal orbit \(\mathcal{N} = \sqrt{L}\)\hspace{0pt}. It assumes a peak
habitability of 1.00 (100\%) at 1.000N, declining linearly in each
direction.

\[
H_I =
\begin{cases}
  \quad 2\dfrac{D}{\mathcal{N}} - 1 & \text{if } {D} \leq {\mathcal{N}} \quad \text{(intranucleal)} \\[1em]
  -0.26\dfrac{D}{\mathcal{N}} + 1.26 & \text{if } {D} \gt {\mathcal{N}} \quad \text{(extranucleal)}
\end{cases}
\]

\[
\text{Where } R = \dfrac{D}{\mathcal{N}}: \quad H_I =
\begin{cases}
  \quad 2R - 1 & \text{if } R \leq 1 \quad \text{(intranucleal)} \\
  -0.26R + 1.26 & \text{if } R \gt 1 \quad \text{(extranucleal)}
\end{cases}
\]

Where: - \(H_I\) = the numeric value of the orbit's habitability index -
\emph{D} = the orbit's distance in AU - \(\mathcal{N}\) = the nucleal
orbit's distance in AU

Values of \emph{D} \textless{} 0.500\(\mathcal{N}\) and \textgreater{}
4.850\(\mathcal{N}\) return \emph{negative numbers} for \(H_I\),
indicating that the orbit is not hospitable, habitable, or parahabitable
for Earth-type lifeforms.

!{[}{[}Orbital Habitability Index Graph.png\textbar400{]}{]}

\begin{longtable}[]{@{}ccc@{}}
\toprule\noalign{}
OrbitType & OrbitDistance & HabitabilityIndex \\
\midrule\noalign{}
\endhead
\bottomrule\noalign{}
\endlastfoot
Intranucleal & 0.500\(\mathcal{N}\) & 0.00 \\
Intranucleal & 0.750\(\mathcal{N}\) & 0.50 \\
Intranucleal & 0.950\(\mathcal{N}\) & 0.90 \\
Nucleal & 1.000\(\mathcal{N}\) & 1.00 \\
Extranucleal & 1.385\(\mathcal{N}\) & 0.90 \\
Extranucleal & 1.770\(\mathcal{N}\) & 0.80 \\
Extranucleal & 4.850\(\mathcal{N}\) & 0.00 \\
\end{longtable}

\subsubsection{Habitability Axis Plate}\label{habitability-axis-plate}

!{[}{[}Habitability Atlas Plate.png{]}{]}

\section{Abstract}\label{abstract-14}

\textbf{Major Topics:}\\
- Extension from nucleal (𝒩) and perannual (𝒫) orbits to \textbf{full
orbital system design}.\\
- Empirical analysis of the Solar System: orbital distances,
eccentricities, ontozone placement, gaps, and intervals.\\
- Definition of \textbf{orbital intervals} (ratio of successive orbital
distances):\\
- Solar System ranges: 1.38--2.00 AU, μ ≈ 1.74 AU, σ ≈ 0.205.\\
- WCB \textbf{conservative range:} ⟨1.400 ∧ 2.000⟩ AU.\\
- WCB \textbf{medial range:} ⟨1.200 ∧ 3.500⟩ AU.\\
- WCB \textbf{optimistic range:} ⟨1.000 ∧ 5.000⟩ AU.\\
- Introduction of \textbf{intrabasal} and \textbf{extrabasal orbit
generation processes}:\\
- \textbf{Intrabasal:} generate inward orbits from a base radius (divide
by randomized interval).\\
- \textbf{Extrabasal:} generate outward orbits from a base radius
(multiply by randomized interval).\\
- Application strategies:\\
- Start at nucleal/perannual orbit → expand inward/outward.\\
- Start at innermost or outermost safe orbit → generate
outward/inward.\\
- Always check results against 𝒩 or 𝒫 to maintain coherent interval
spacing.\\
- Worked example:\\
- Starting from 𝒩 = 0.834 AU.\\
- Generated inward to 0.101 AU, outward to 33.357 AU.\\
- Produced 11 candidate orbital positions spanning inner rocky to
distant icy regions.

\textbf{Key Terms \& Symbols:}\\
- \textbf{Orbital Interval (I):} ratio of successive orbital distances,
\(I = O_n / O_{n-1}\).\\
- \textbf{Orbital Gap (G):} difference of successive orbital distances,
\(G = O_n - O_{n-1}\).\\
- \textbf{Intrabasal Orbit Calculation:} inward generation by
division.\\
- \textbf{Extrabasal Orbit Calculation:} outward generation by
multiplication.\\
- \textbf{Basal Orbit (B):} chosen anchor orbit (e.g., 𝒩 or 𝒫).\\
- \textbf{Ω (Omega):} cutoff distance (innermost safe orbit or outer
system limit).

\textbf{Cross-Check Notes:}\\
- \textbf{New glossary entries needed:} Orbital Interval, Orbital Gap,
Intrabasal Orbit Calculation, Extrabasal Orbit Calculation, Basal Orbit,
Ω (system cutoff).\\
- Builds on previous anchors (𝒩, 𝒫, thermozones) to enable
\textbf{statistically grounded system generation}.\\
- Provides WCB's baseline method for populating star systems with
planemon orbits.

\chapter{Fleshing Out A Star System}\label{fleshing-out-a-star-system}

We've established spectral classes and types, thermozones, ontozones,
habitability indices, and the two critical orbital distances,
\emph{nucleal} (\(\mathcal{N}\)) and \emph{perannual} (\(\mathcal{P}\)).

But planemons don't orbit only at these discreet distances -- they're
all over the place. Here's a breakdown of our own Solar system's
planemon orbit data:

\begin{longtable}[]{@{}
  >{\raggedright\arraybackslash}p{(\linewidth - 6\tabcolsep) * \real{0.2949}}
  >{\raggedleft\arraybackslash}p{(\linewidth - 6\tabcolsep) * \real{0.3333}}
  >{\raggedright\arraybackslash}p{(\linewidth - 6\tabcolsep) * \real{0.2308}}
  >{\raggedright\arraybackslash}p{(\linewidth - 6\tabcolsep) * \real{0.1410}}@{}}
\toprule\noalign{}
\begin{minipage}[b]{\linewidth}\raggedright
\end{minipage} & \begin{minipage}[b]{\linewidth}\raggedleft
\end{minipage} & \begin{minipage}[b]{\linewidth}\raggedright
\end{minipage} & \begin{minipage}[b]{\linewidth}\raggedright
Ontozone
\end{minipage} \\
\midrule\noalign{}
\endhead
\bottomrule\noalign{}
\endlastfoot
Mercury & 0.387 & 0.2056 & Igniozone \\
Venus & 0.723 & 0.0068 & Calorozone \\
Earth & 1.000 & 0.0167 & Solarazone \\
Mars & 1.524 & 0.0934 & Hiberozone \\
\emph{Asteroids} & \emph{⟨2.2 ∧ 3.2⟩; μ = 2.7} & \emph{μ = 0.15} &
\emph{Brumazone} \\
Jupiter & 5.204 & 0.0489 & Cryozone \\
Saturn & 9.583 & 0.0565 & Cryozone \\
Uranus & 19.191 & 0.0472 & Cryozone \\
Neptune & 30.070 & 0.0087 & Cryozone \\
\end{longtable}

\section{Orbital Parameters}\label{orbital-parameters}

Ignoring the asteroid belt for the moment and inserting Ceres as the
largest member of the belt:

\begin{longtable}[]{@{}
  >{\raggedright\arraybackslash}p{(\linewidth - 8\tabcolsep) * \real{0.1769}}
  >{\raggedleft\arraybackslash}p{(\linewidth - 8\tabcolsep) * \real{0.2000}}
  >{\raggedleft\arraybackslash}p{(\linewidth - 8\tabcolsep) * \real{0.1385}}
  >{\raggedleft\arraybackslash}p{(\linewidth - 8\tabcolsep) * \real{0.2231}}
  >{\raggedleft\arraybackslash}p{(\linewidth - 8\tabcolsep) * \real{0.2615}}@{}}
\toprule\noalign{}
\endhead
\bottomrule\noalign{}
\endlastfoot
Mercury & 0.387 & 0.2056 & & \\
Venus & 0.723 & 0.0067 & 0.3362 & 1.8686 \\
Earth & 1.000 & 0.0167 & 0.2767 & 1.3825 \\
Mars & 1.524 & 0.0934 & 0.5237 & 1.5237 \\
Ceres & 2.700 & 0.1500 & 0.1763 & 1.7720 \\
Jupiter & 5.204 & 0.0489 & 2.5038 & 1.9273 \\
Saturn & 9.583 & 0.0565 & 4.3788 & 1.8415 \\
Uranus & 19.191 & 0.04727 & 9.6087 & 2.0027 \\
Neptune & 30.070 & 0.0088 & 10.8787 & 1.5669 \\
\end{longtable}

\begin{quote}
Notes: 1. The orbital \textbf{gap} is calculated by subtracting the
planemon's orbital distance from the next closer-in planemon's orbital
distance \[G = O_n - O_{n-1}\]\textgreater2. The orbital
\textbf{interval} is calculated by dividing the planemon's orbital
distance by the next closer-in planemon's orbital distance:
\[I = \dfrac{O_n}{O_{n-1}}\]
\end{quote}

What we are concerned with is the \emph{orbital intervals} - The minimum
interval is between Venus and Earth: \(I = 1.3825\;AU\) - The maximum
interval is between Uranus and Neptune: \(I = 2.0027\;AU\) - The average
interval is \(I \approx 1.74\;AU\) - The median interval is
\(I \approx 1.81\;AU\) - The standard deviation is \(\sigma = 0.2051\)

This supports defining a \textbf{WCB-conservative orbital interval
range} of ⟨1.400  ∧  2.000⟩  AU by rounding up the minimum and rounding
down the maximum.

However, a cursory survey of the Exoplanet Catalog seems to reveal a
range of ⟨1.000 ∧ 5.000⟩ AU for planetary orbital intervals. We'll
define this as our optimistic orbital interval range, and, for
completeness, average the two ranges at ⟨1.200 ∧ 3.500⟩ AU.

\textbf{Orbital interval ranges}

\begin{longtable}[]{@{}rrl@{}}
\toprule\noalign{}
Inner & Outer & Description \\
\midrule\noalign{}
\endhead
\bottomrule\noalign{}
\endlastfoot
1.4 & 2 & Conservative \\
1.2 & 3.5 & Medial \\
1 & 5 & Optimistic \\
\end{longtable}

\section{Calculating Other Orbits}\label{calculating-other-orbits}

This brings us to methods of calculating other orbits in a star system.
In practice, any method the worldmakerchooses is \emph{valid}, including
just putting planemons where ``it feels right''; however, even using
this method \emph{should} ideally take into account the above statistics
and try to avoid an orbital interval between a given planemon and its
nearest neighbor of \(< 1.500\;AU\) or \(> 2.000\;AU\).

\subsection{Starting From A Known
Orbit}\label{starting-from-a-known-orbit}

Most of the time, you'll have pre-established a particular orbit ---
usually either the \emph{nucleal} or the \emph{perannual} orbit, and
want to arrange other planemons in the system relative to that orbit.
Notating this orbit as the \textbf{base} orbit, we can set up two
processes for calculating orbits inferior to (closer-in than) and
superior to (farther-out than) that orbit.

\textbf{Intrabasal Orbit Calculation Process} \[
r_i = B;\; \Omega = \text{«▢»}: \qquad
r_{i-1} = \frac{r_i}{⟨⟨ \text{min} ∧ \text{max} ⟩⟩}
\quad \text{while } r_{i-1} \geq \Omega
\] \textbf{Extrabasal Orbit Calculation Process} \[
r_i = B;\; \Omega = \text{«▢»}: \qquad
r_{i+1} = r_i \cdot ⟨⟨ \text{min} ∧ \text{max} ⟩⟩
\quad \text{while } r_{i+1} \leq \Omega
\] Where: - \emph{B} = basal orbital radius (e.g.~the nucleal orbit
\(\mathcal{N}\)) - \emph{Ω} = orbital distance cuttoff (minimum or
maximum allowed orbit based on the star system constraints)

\subsection{Usage Strategy}\label{usage-strategy}

\emph{Assuming the medial orbital interval range}
\(I \in ⟨⟨1.200 ∧ 3.500⟩⟩\;AU\): The \textbf{intrabasal} and
\textbf{extrabasal} forms can be used independently depending on your
desired anchoring strategy:

\begin{quote}
\textbf{Inward-Only Generation}\\
If you begin at the \textbf{basal orbit} (innermost, nucleal, perannual,
etc.), use the \textbf{intrabasal} form to expand inward via divisive
steps: \[
r_0 = B;\; \Omega = «▢»:
\quad r_{i-1} = \dfrac{r_i} {⟨⟨1.200 ∧ 3.500⟩⟩}, \text{ while } r_{i-1} \geq \Omega
\] Where: - \emph{Ω} = the minimum safe orbital distance --- usually
taken to be \(a = 0.100\;AU\).
\end{quote}

\begin{quote}
\textbf{Outward-Only Generation}\\
If you begin at an \textbf{innermost orbit} (e.g.~a thermal, Roche, or
design constraint), use the \textbf{intrabasal} form to expand outward
via multiplicative steps: \[
r_0 = B;\; \Omega = «▢»:
\quad r_{i+1} = r_i \cdot ⟨⟨1.200 ∧ 3.500⟩⟩, \text{ while } r_{i+1} \leq \Omega
\] Where: - \emph{Ω} = the farthest orbit desired for a planemon in the
system --- based on whatever criterion desired, but physically limited
to the Hill Sphere radius of the central mass.
\end{quote}

\begin{quote}
Applying these methods can fully define a system. - If one wanted to
start with the innermost safe orbit; e.g.~\(B = 0.100\;AU\), then one
would use only the extrabasal process to calculate orbits outward. -
Conversely if one wanted to start with the most distant orbit, then one
would use only the intrabasal process to calculate orbits inward.
\end{quote}

\begin{quote}
\begin{quote}
\textbf{IMPORTANT} \emph{If not starting from the perannual or nucleal
orbit, always check randomized orbits against either (or both) to ensure
proper interval gaps fall to either side of that orbit, and adjust
accordingly!}
\end{quote}
\end{quote}

\section{Worked Example}\label{worked-example}

Let us say we've identified our nucleal orbit (\(\mathcal{N}\)) as
\(\mathcal{N} = 0.834\;AU\), and we want to calculate orbits interior-to
and exterior-to that orbit, and we've chosen \(a = 0.100\;AU\) as our
innermost safe orbit. \#\#\# Working Inward \[
  r_0 = 0.834;\; \Omega = 0.100:
  \quad r_{i-1} = \dfrac{r_i} {⟨⟨1.200 ∧ 3.500⟩⟩}, \text{ while } r_{i-1} \geq 0.100
\] \[
\begin{align}
r_{i-1} &= \dfrac{0.834} {1.723} = 0.482\;AU \qquad 1.732 := \text{Randomized interval between ⟨⟨1.200 ∧ 3.500⟩⟩ AU} \\[1em]
r_{i-1} &= \dfrac{0.482} {1.616} = 0.298\;AU \qquad 1.616 := \text{Randomized interval between ⟨⟨1.200 ∧ 3.500⟩⟩ AU} \\[1em]
r_{i-1} &= \dfrac{0.298} {1.573} = 0.190\;AU \qquad 1.573 := \text{Randomized interval between ⟨⟨1.200 ∧ 3.500⟩⟩ AU} \\[1em]
r_{i-1} &= \dfrac{0.190} {1.884} = 0.101\;AU \qquad 1.884 := \text{Randomized interval between ⟨⟨1.200 ∧ 3.500⟩⟩ AU} \\[1em]
r_{i-1} &= \dfrac{0.101} {1.963} = 0.051\;AU \qquad 1.963 := \text{Randomized interval between ⟨⟨1.200 ∧ 3.500⟩⟩ AU}\;✘ \\[1em]
\end{align}
\] We stop at the fourth randomized orbit, because the next orbit
randomly generated fails the \(r ≥ 0.100\;AU\) test. We now have a
system of five orbits:

\begin{longtable}[]{@{}cc@{}}
\toprule\noalign{}
Orbit & Distance \\
\midrule\noalign{}
\endhead
\bottomrule\noalign{}
\endlastfoot
1 & 0.101 \\
2 & 0.190 \\
3 & 0.298 \\
4 & 0.482 \\
5 (\emph{N}) & 0.834 \\
\end{longtable}

We could stop here and have a fully legitimate star system, but let's
say we want extranucleal orbits, as well. Again, beginning with the
nucleal orbit \(B = 0.834\;AU\), and setting an outermost orbit of
\(\Omega = 35.0\;AU\): \[
  r_0 = 0.834;\; \Omega = 35.0:
  \quad r_{i+1} = r_i \cdot ⟨⟨1.200 ∧ 3.500⟩⟩, \text{ while } r_{i+1} \leq 35.0
\] \[
\begin{align}
r_{i+1} &= 0.834(1.829) = 1.525\;AU \qquad 1.829 := \text{Randomized interval between ⟨⟨1.200 ∧ 3.500⟩⟩ AU} \\[1em]
r_{i+1} &= 1.525(1.969) = 3.003\;AU \qquad 1.969 := \text{Randomized interval between ⟨⟨1.200 ∧ 3.500⟩⟩ AU} \\[1em]
r_{i+1} &= 3.003(1.578) = 4.739\;AU \qquad 1.578 := \text{Randomized interval between ⟨⟨1.200 ∧ 3.500⟩⟩ AU} \\[1em]
r_{i+1} &= 4.739(1.547) = 7.332\;AU \qquad 1.547 := \text{Randomized interval between ⟨⟨1.200 ∧ 3.500⟩⟩ AU} \\[1em]
r_{i+1} &= 7.332(1.552) = 11.379\;AU \qquad 1.552 := \text{Randomized interval between ⟨⟨1.200 ∧ 3.500⟩⟩ AU} \\[1em]
r_{i+1} &= 11.379(1.608) = 18.298\;AU \qquad 1.608 := \text{Randomized interval between ⟨⟨1.200 ∧ 3.500⟩⟩ AU} \\[1em]
r_{i+1} &= 18.298(1.823) = 33.357\;AU \qquad 1.823 := \text{Randomized interval between ⟨⟨1.200 ∧ 3.500⟩⟩ AU} \\[1em]
r_{i+1} &= 33.357(1.778) = 59.309\;AU \qquad 1.778 := \text{Randomized interval between ⟨⟨1.200 ∧ 3.500⟩⟩ AU} \; ✘ \\[1em]
\end{align}
\] We stop at the seventh iteration, as the next value exceeds
\(\Omega = 35.0\;AU\).

This expands our system to 11 orbits:

\begin{longtable}[]{@{}cc@{}}
\toprule\noalign{}
Orbit & Distance \\
\midrule\noalign{}
\endhead
\bottomrule\noalign{}
\endlastfoot
1 & 0.101 \\
2 & 0.190 \\
3 & 0.298 \\
4 & 0.482 \\
5 (\emph{N}) & 0.834 \\
6 & 1.525 \\
7 & 3.003 \\
8 & 4.739 \\
9 & 11.379 \\
10 & 18.298 \\
11 & 33.357 \\
\end{longtable}

\ldots{} and we can proceed to design planemons for each orbit, or
eliminate some orbits and install asteroid belts, or adjust orbital
intervals to suit our needs\ldots. the sky's the limit.

\begin{quote}
\textbf{Hippy}: Oh, ha-ha\ldots{}
\end{quote}

C'mon, you had to know I'd use that pun at \emph{some point} didn't you?

With this method, a worldmaker can quickly generate a full planemon
system that is physically plausible, statistically grounded, and
symbolically consistent with WCB cosmology.

{[}{[}Asteroid Belts and Resonance Gaps --- working out{]}{]}

\section{Abstract}\label{abstract-15}

\textbf{Major Topics:}\\
- Demonstrates integration of \textbf{thermozones (H₀--H₅)},
\textbf{ontozones}, and \textbf{orbital generation rules} into a
coherent stellar system design.\\
- Stepwise calculation:\\
- Thermozone limits derived from the nucleal orbit (\(𝒩 = 0.834\) AU).\\
- Placement of generated orbits into both thermozone and ontozone
categories.\\
- Worked example tables:\\
- Case with both nucleal (𝒩) and perannual (𝒫) orbits → reveals interval
violation (\(I = 1.162\) \textless{} minimum 1.5).\\
- Case with \textbf{perannual planemon only} → valid intervals
maintained (1.574--1.927 AU).\\
- Demonstrates \textbf{design trade-offs}: some orbital anchors (𝒩
vs.~𝒫) may be mutually exclusive depending on stellar mass/luminosity.\\
- Stellar parameter recalculation (luminosity, temperature, spectral
type, subclass index) validates the system's spectral class (G4.701).\\
- Orbital habitability evaluation:\\
- Perannual orbit receives \textasciitilde74.1\% of nucleal flux.\\
- Corresponding Orbital Habitability Index (OHI) = 0.958 (95.8\% of
nucleal).\\
- Emphasis: WCB design enforces \textbf{minimum orbital spacing (I ≥
1.5)} as a hard rule, while maximum spacing (I ≤ 2.0) is treated as
flexible.

\textbf{Key Terms \& Symbols:}\\
- \textbf{Δ (Delta):} factor expressing relative distance offset between
perannual and nucleal orbits.\\
- \textbf{F (Flux):} relative stellar irradiance at a given orbital
distance, normalized to 1.0 at 𝒩.\\
- \textbf{OHI (Orbital Habitability Index):} previously defined, applied
here in practice.

\textbf{Cross-Check Notes:}\\
- \textbf{New glossary entries needed:} Δ (distance ratio), F (stellar
flux).\\
- Reinforces prior entries: thermozones, ontozones, 𝒩, 𝒫, orbital
intervals.\\
- Serves as a practical example of reconciling WCB rules with real
stellar constraints.

\chapter{Calculating the Thermozones}\label{calculating-the-thermozones}

Since we know our nucleal orbit is \(\mathcal{N} = 0.834\;AU\), we can
calculate the thermozone limits: \[
\begin{align}
H_0 = 0.500\mathcal{N} &= 0.500(0.834) = 0.417\;AU \\
H_1 = 0.750\mathcal{N} &= 0.750(0.834) = 0.626\;AU \\
H_2 = 0.950\mathcal{N} &= 0.950(0.834) = 0.792\;AU \\
H_3 = 1.385\mathcal{N} &= 1.385(0.834) = 1.115\;AU \\
H_4 = 1.770\mathcal{N} &= 1.770(0.834) = 1.476\;AU \\
H_5 = 4.850\mathcal{N} &= 4.850(0.834) = 4.045\;AU \quad \text{Frost Line} \\
\end{align}
\] And we can add these to our orbits table from above and determine the
thermozones and ontozones of the orbits:

\begin{longtable}[]{@{}
  >{\centering\arraybackslash}p{(\linewidth - 6\tabcolsep) * \real{0.2308}}
  >{\centering\arraybackslash}p{(\linewidth - 6\tabcolsep) * \real{0.1026}}
  >{\raggedright\arraybackslash}p{(\linewidth - 6\tabcolsep) * \real{0.3462}}
  >{\raggedright\arraybackslash}p{(\linewidth - 6\tabcolsep) * \real{0.3205}}@{}}
\toprule\noalign{}
\begin{minipage}[b]{\linewidth}\centering
Orbit
\end{minipage} & \begin{minipage}[b]{\linewidth}\centering
Distance
\end{minipage} & \begin{minipage}[b]{\linewidth}\raggedright
\end{minipage} & \begin{minipage}[b]{\linewidth}\raggedright
\end{minipage} \\
\midrule\noalign{}
\endhead
\bottomrule\noalign{}
\endlastfoot
1 & 0.101 & Igniozone & Xenotic \\
2 & 0.190 & Igniozone & Xenotic \\
3 & 0.298 & Ignoizone & Xenotic \\
\(H_0\) & 0.417 & & \\
4 & 0.482 & Calorozone & Inner Parahabitable \\
\(H_1\) & 0.626 & & \\
\(H_2\) & 0.792 & & \\
5 (*\(\mathcal{N}\)) & 0.834 & Solarazone & Hospitable \\
\(H_3\) & 1.115 & & \\
\(H_4\) & 1.476 & & \\
6 & 1.525 & Brumazone & Outer Parahabitable \\
7 & 3.003 & Brumazone & Outer Parahabitable \\
\(H_5\) & 4.045 & & \\
8 & 4.739 & Cryozone & Xenotic \\
9 & 11.379 & Cryozone & Xenotic \\
10 & 18.298 & Cryozone & Xenotic \\
11 & 33.357 & Cryozone & Xenotic \\
\end{longtable}

And, we can calculate the perannual orbital distance and the star's
spectral type by: \textbf{Perannual Orbit} \[
\begin{align}
L &= \mathcal{N}^2 = 0.834^2 = 0.696 \\
M &= \sqrt[3.8]{L} = \sqrt[3.8]{0.696} = 0.909 \\
A &= \sqrt[3]{0.909} = 0.969\;AU\;✓
\end{align}
\] The perannual orbit in this system is at \(A = 0.969\;AU\).

\textbf{Spectral Type} \[
\begin{align}
L &= 0.696 \\
T &= \sqrt[7.6]{L} = \sqrt[7.6]{0.696} = 0.953\odot \\
K &= 5800T = 5800(0.953) = 5529.92 \quad \text{Spectral Class G: κ = 6000; þ = 100} \\[2em]
\mathcal{S} &= \dfrac{\kappa - K}{100} = \dfrac{6000 - 5529.92}{100} = \dfrac{470.08}{100} = 4.701\\
\end{align}
\] The spectral type of our star is \(G4.701\).

\begin{quote}
\textbf{Hippy}: Uh\ldots.. that perannual orbit distance\ldots.
\end{quote}

Excellent catch!

\begin{quote}
\textbf{Keppy}: What\ldots.?
\end{quote}

Check this out: we already know that our nucleal orbit is at
\(\mathcal{N} = 0.834\;AU\). \emph{If} we put planemon on the perannual
orbit at \(A = 0.969\;AU\) the interval between the nucleal orbit and
the perannual orbit is only: \[
I = \dfrac{0.969}{0.834} = 1.162\;AU\;✓
\] \ldots{} which is less than our specified minimum \(I > 1.500\;AU\).

\begin{quote}
\textbf{Keppy}: Which means\ldots.
\end{quote}

\begin{quote}
\textbf{Hippy}: Either we don't have a planemon on the nucleal orbit,
\emph{or} we don't have a planemon on the perannual orbit; those orbits
are fixed by the stellar parameters -- neither can be shifte.
\end{quote}

EXACTLY! This is the power --- but also the limitation --- of our
system. Some things we can tweak as we please; other things we have to
work with, or work around.

In this case, we're forced to choose between a planemon with Earth's
stellar flux, or a planemon with Earth's orbital period, but we can't
have both.

\begin{quote}
\textbf{Keppy}: What if we drop the nucleal planemon and go with the
parahabitable planemon?
\end{quote}

Excellent thought\ldots{} let's work that through. Here's a modified
orbit table taking that into account:

\begin{longtable}[]{@{}
  >{\centering\arraybackslash}p{(\linewidth - 8\tabcolsep) * \real{0.2000}}
  >{\centering\arraybackslash}p{(\linewidth - 8\tabcolsep) * \real{0.0941}}
  >{\raggedright\arraybackslash}p{(\linewidth - 8\tabcolsep) * \real{0.3176}}
  >{\raggedright\arraybackslash}p{(\linewidth - 8\tabcolsep) * \real{0.2941}}
  >{\raggedright\arraybackslash}p{(\linewidth - 8\tabcolsep) * \real{0.0941}}@{}}
\toprule\noalign{}
\begin{minipage}[b]{\linewidth}\centering
Orbit
\end{minipage} & \begin{minipage}[b]{\linewidth}\centering
Distance
\end{minipage} & \begin{minipage}[b]{\linewidth}\raggedright
\end{minipage} & \begin{minipage}[b]{\linewidth}\raggedright
\end{minipage} & \begin{minipage}[b]{\linewidth}\raggedright
Interval
\end{minipage} \\
\midrule\noalign{}
\endhead
\bottomrule\noalign{}
\endlastfoot
1 & 0.101 & Igniozone & Xenotic & \\
2 & 0.190 & Igniozone & Xenotic & 1.884 \\
3 & 0.298 & Ignoizone & Xenotic & 1.573 \\
\(H_0\) & 0.417 & & & \\
4 & 0.482 & Calorozone & Inner Parahabitable & 1.616 \\
\(H_1\) & 0.626 & & & \\
\(H_2\) & 0.792 & & & \\
5 (\(\mathcal{A}\)) & 0.969 & Solarazone & Hospitable & »1.927« \\
\(H_3\) & 1.115 & & & \\
\(H_4\) & 1.476 & & & \\
6 & 1.525 & Brumazone & Outer Parahabitable & »1.574« \\
7 & 3.003 & Brumazone & Outer Parahabitable & \\
\(H_5\) & 4.045 & & & \\
8 & 4.739 & Cryozone & Xenotic & 1.552 \\
9 & 11.379 & Cryozone & Xenotic & 1.608 \\
10 & 18.298 & Cryozone & Xenotic & 1.823 \\
11 & 33.357 & Cryozone & Xenotic & 1.778 \\
\end{longtable}

The interval between the perannual orbit and the next closer-in orbit
is: \[
I = \dfrac{0.969}{0.482} = 1.927\;AU
\] \ldots{} \emph{just} within our \(I = 2.000\;AU\) maximum, and the
interval to the next orbit out is: \[
I = \dfrac{1.525}{0.969} = 1.574\;AU
\] \ldots{} which, again, is \emph{just} within our specified minimum
\(I = 1.500\;AU\), so, yes we can drop the nucleal planemon and go with
the perannual planemon, instead.

\begin{quote}
\textbf{Hippy}: I am compelled to point out that the maximum interval is
not nearly as absolute as the minimum interval\ldots{}
\end{quote}

Good point! planemons can certainly have wider intervals between their
orbits than \(I = 2.000\;AU\) -- we just never want them to have an
interval \emph{less-than} \(I = 1.500\;AU\).

\begin{quote}
\textbf{Keppy}: But the perannual planemon is farther out than the
nucleal orbit, so won't it get less stellar flux?
\end{quote}

Well spotted! And we can calculate that! Since \(A = 0.969\;AU\) and
\(N = 0.834\;AU\), and we know that the stellar flux at \emph{N} = 1.0,
we can calculate that \emph{A} is: \[
\Delta = \dfrac{0.969}{0.834} = 1.162
\] \ldots{} the perannual orbit is \(1.162 \times\) farther out than the
nucleal orbit, and since intensity decreases with the square of the
distance: \[
F = \dfrac{1}{\Delta^2}
\] \ldots{} we can calculate that the perannual planemon receives: \[
F = \dfrac{1}{1.162^2}= \dfrac{1}{1.350} = 0.741
\] \ldots{} about 74.1\% of the stellar flux as the nucleal orbit
does\ldots{} but that's still: \[
\begin{gather}
H_I = -0.26\dfrac{D}{\mathcal{N}} + 1.26 = -0.26\dfrac{0.969}{0.834} + 1.26 = -0.26(1.162) + 1.26 = -0.302 + 1.26 = 0.958
\end{gather}
\] \ldots{} an orbital habitability index of 95.8\% that of the nucleal
orbit. Slightly cooler, but not drastically so.

\section{Parameter Ranges by Spectral
Class}\label{parameter-ranges-by-spectral-class-1}

\textbf{Abstract:}\\
This file presents a tabulated set of \textbf{stellar parameters by
spectral class (O--M)}, with high, mean, and low values given for each.
Parameters included are:

\begin{itemize}
\tightlist
\item
  \textbf{Effective temperature (K)}\\
\item
  \textbf{Thermal Interval Constant (þ)} --- the step size in Kelvin
  that defines subclass increments\\
\item
  \textbf{Temperature (T⊙)} relative to the Sun\\
\item
  \textbf{Radius (R⊙)} relative to the Sun\\
\item
  \textbf{Luminosity (L⊙)} relative to the Sun\\
\item
  \textbf{Mass (M⊙)} relative to the Sun\\
\item
  \textbf{Frequency (Q⊙)} relative to the Sun
\end{itemize}

The table serves as a \textbf{ready reference for stellar design} in
worldbuilding, allowing thesiasts to quickly identify plausible ranges
of physical values across the standard stellar sequence. It systematizes
how subclass divisions are calculated (using the TIC, þ) and provides
both absolute and relative values for major stellar quantities.

\textbf{Canon Links:}\\
- Connects to \textbf{Spectral Classes, Types, and Parameters} (M002
Stars series).\\
- Uses the \textbf{Thermal Interval Constant (þ) {[}neo{]}} as defined
elsewhere in WCB.\\
- Values support calculations in \textbf{Habitable Zones},
\textbf{Fundamental Orbits}, and related stellar-orbital frameworks.

\textbf{Lexical Tags:}\\
- \textbf{{[}sci{]}} Spectral Classes, Tempera

\chapter{Abstract}\label{abstract-16}

\textbf{Major Topics:}\\
- Defines the \textbf{stellamon Framework}, a master map for classifying
\textbf{stellar monons (stellamons)}.\\
- Structure:\\
- \textbf{Trunk (shared levels):}\\
- Frame → Monon.\\
- Group → Stellamon.\\
- \textbf{Spectral Branch (surface physics):}\\
- Spectral Class (O--M).\\
- Spectral Subclass (decimal divisions, e.g.~G2, M5).\\
- Spectral Type = Spectral Class + Luminosity Class (e.g.~G2V, K3III).\\
- \textbf{Luminosity Branch (structural physics):}\\
- Luminosity Class (morphological categories by radius/brightness,
0--VII).\\
- Evolutionary Phase (Protostar → Main sequence → Giant → White dwarf →
Remnant).\\
- \textbf{Intersection Principle:}\\
- Spectral Branch = ``what the star is'' (temperature, spectrum,
color).\\
- Luminosity Branch = ``what the star does'' (brightness, radius,
life-cycle).\\
- \textbf{Spectral Type = intersection of both axes.}\\
- Worked examples:\\
- Sun = G2V.\\
- Proxima Centauri = M5.5V.\\
- Rigel = B8Ia.\\
- Betelgeuse = M2Ia.

\textbf{Key Terms \& Symbols:}\\
- \textbf{stellamon {[}NEW{]}:} stellar monon; root classification
unit.\\
- \textbf{Spectral Class/Subclass/Type {[}sci{]}.}\\
- \textbf{Luminosity Class {[}sci{]}.}\\
- \textbf{Evolutionary Phase {[}sci{]}.}

\textbf{Cross-Check Notes:}\\
- Spectral and luminosity classifications are already canonical.\\
- \textbf{stellamon {[}NEW{]}} is introduced here for the first time as
a formal organizing term.\\
- \textbf{Status:} {[}EXPANDED + NEW{]} --- expands stellar
classification into a dual-branch framework; introduces stellamon as a
new root concept.

\chapter{⭐ stellamon Framework (Master
Map)}\label{stellamon-framework-master-map}

All \textbf{stellamons} (stellar monons) share a \textbf{common trunk}
--- then branch into two complementary classification systems.

\section{🌳 Trunk (Shared Levels)}\label{trunk-shared-levels}

\begin{itemize}
\tightlist
\item
  \textbf{Frame} → Monon\\
\item
  \textbf{Group} → stellamon
\end{itemize}

\section{🌈 Spectral Branch (Temperature /
Spectrum)}\label{spectral-branch-temperature-spectrum}

Focus: \textbf{What the star \emph{is}, by surface temperature and
spectral features} - \textbf{Spectral Class} → Broad categories (O, B,
A, F, G, K, M)\\
- \textbf{Spectral Subclass} → Decimal divisions within a class
(e.g.~G2, M5)\\
- \textbf{Spectral Type} → Spectral Class + Luminosity Class (e.g.~G2V,
K3III)

\section{💡 Luminosity Branch (Structure / Evolutionary
Stage)}\label{luminosity-branch-structure-evolutionary-stage}

Focus: \textbf{What the star \emph{does}, by size, brightness, and
life-cycle} - \textbf{Luminosity Class} → Morphological grouping by
stellar radius \& brightness\\
- 0 = Hypergiant\\
- Ia/Iab/Ib = Supergiant\\
- II = Bright giant\\
- III = Giant\\
- IV = Subgiant\\
- V = Main sequence (dwarf)\\
- VI = Subdwarf\\
- VII = White dwarf - \textbf{Evolutionary Phase} (optional refinement
inside each class)\\
- Protostar → Main sequence → Giant → White dwarf → Remnant (NS/BH)

\section{🔄 Intersections}\label{intersections}

\begin{itemize}
\tightlist
\item
  \textbf{Spectral Type} = Intersection of \textbf{Spectral
  Class/Subclass} with \textbf{Luminosity Class}

  \begin{itemize}
  \tightlist
  \item
    Example: the Sun = \textbf{G2V} (Spectral Class G, subclass 2,
    Luminosity Class V).\\
  \item
    Proxima Centauri = \textbf{M5.5V}.\\
  \item
    Rigel = \textbf{B8Ia}.
  \end{itemize}
\end{itemize}

\section{🧭 Principle}\label{principle}

\begin{itemize}
\tightlist
\item
  \textbf{Spectral Branch} = \emph{surface physics} (temperature, color,
  composition).\\
\item
  \textbf{Luminosity Branch} = \emph{structural physics} (radius,
  brightness, lifecycle stage).\\
\item
  Both are \textbf{lateral axes}: their \textbf{intersection defines the
  star's Spectral Type.}
\end{itemize}

✅ Example: - Sun = G (Class) → G2 (Subclass) → Luminosity V →
\textbf{G2V Spectral Type}.\\
- Betelgeuse = M (Class) → M2 (Subclass) → Luminosity Ia → \textbf{M2Ia
Spectral Type}. \#\# Abstract\\
\textbf{Major Topics:}\\
- Introduces the concept of \textbf{stellar populations}, a
classification system based on \textbf{metallicity} (abundance of
elements heavier than hydrogen and helium).\\
- Three recognized populations:\\
- \textbf{Population I:} high metallicity, young stars (e.g., the
Sun).\\
- \textbf{Population II:} low metallicity, older stars.\\
- \textbf{Population III:} zero metallicity, primordial stars (early
universe, hydrogen + helium only).\\
- Notes that the numbering is counterintuitive (``backward''):
Population I = most recent, Population III = oldest.\\
- Provides an alternate framing in terms of \textbf{stellar
generations}, which reverses the numbering for intuitive clarity:\\
- \textbf{Generation III = Population I} (high metallicity).\\
- \textbf{Generation II = Population II} (low metallicity).\\
- \textbf{Generation I = Population III} (zero metallicity).\\
- Discusses future outlook:\\
- No ``Generation IV'' expected, since ongoing star formation continues
to produce Population I stars with similar compositions.\\
- Supernovae and stellar recycling do not significantly change overall
metallicity ratios in the cosmos.\\
- Long-term: star formation will decline, leaving remnants (neutron
stars, quark stars, black holes) in a steady state.

\textbf{Key Terms \& Symbols:}\\
- \textbf{Stellar Population I, II, III {[}sci{]}.}\\
- \textbf{Stellar Generation I, II, III {[}NEW/neo{]}.}\\
- \textbf{Metallicity {[}sci{]}.}

\textbf{Cross-Check Notes:}\\
- Stellar populations (I--III) are introduced here for the first time in
canon.\\
- \textbf{Generational reinterpretation {[}neo{]}} is unique to WCB,
offering a more intuitive framing.\\
- \textbf{Metallicity {[}sci{]}} is established science, formally
abstracted here for the first time.\\
- \textbf{Status:} {[}NEW + EXO{]} --- incorporates the real-world
stellar population system, and introduces the generational
reinterpretation for WCB use.

\chapter{Stellar Populations}\label{stellar-populations}

Another way stars are sometimes classed is by stellar population;
somewhat confusingly, these are also designated by Roman numerals, but
in this case the number indicates the \textbf{\emph{metallicity}} of the
star.

\begin{quote}
\textbf{Keppy}: Metallicity?!?!
\end{quote}

Yep; for astronomers, any element that's not hydrogen or helium is a
``metal'' --- apparently astronomers and chemists don't talk to each
other.

\begin{longtable}[]{@{}ll@{}}
\toprule\noalign{}
Population & Description \\
\midrule\noalign{}
\endhead
\bottomrule\noalign{}
\endlastfoot
Population I & High Metallicity \\
Population II & Low Metallicity \\
Population III & Zero Metallicity \\
\end{longtable}

The stellar population number can be taken as a \emph{rough} estimate of
a star's age, but here is where the system is confusing, because the
population number are ``backward''. Population I stars are high
metallicity stars like the Sun, and are the most recent ``generation''
of stars. Population III stars (if any actually still exist) are almost
totally lacking in metals, and are the first stars to have formed in the
early universe when hydrogen and helium were the only raw material on
offer.

An alternative way of thinking about stellar populations might be as
``generations'', numbered in reverse sequence to populations:

\begin{longtable}[]{@{}lll@{}}
\toprule\noalign{}
Generation & Population & Description \\
\midrule\noalign{}
\endhead
\bottomrule\noalign{}
\endlastfoot
Generation III & Population I & High Metallicity \\
Generation II & Population II & Low Metallicity \\
Generation 1 & Population III & Zero Metallicity \\
\end{longtable}

\section{Stars: The Next Generation}\label{stars-the-next-generation}

\begin{quote}
Keppy: So, will there be a Generation IV?
\end{quote}

Probably not --- and here's why: - While there is certainly still
star-forming going on in the universe (the Orion Nebula, for instance),
the stars forming there are still Population I/Generation III stars. -
The gasses and heavier elements that are going into their composition
aren't changing in relative proportions over time. - Supper massive
stars \emph{are} still producing heavy elements, and novae and
supernovae are still distributing those elements into the cosmos, the
overall amounts of such heavy elements is not increasing, and heavy
elements are not being produced at a faster rate than in the past.

Newly formed stars, then don't have basic compositions in any
fundamental way different from the previous generation.

\begin{quote}
In fact, if the ``heat death'' model of the universe is correct, star
formation will \emph{decrease} with time, neutron stars, quark stars,
and black holes will eventually replace self-luminous fusors, and
everything will settle into a ``steady state'' that will persist for
eternity.\#\# Abstract\\
\textbf{Major Topics:}\\
- Examines how \textbf{stellar main sequence lifetimes} constrain system
habitability and the emergence of complex life.\\
- Draws on Turnbull \& Tarter's external ``habstar'' criteria (not
adopted into WCB lexicon):\\
- Must be main sequence, non-variable, have a habitable zone, and
\textbf{age ≥ 3.0 Ga}.\\
- Establishes that any star with \textbf{Q \textless{} 3.0 Ga} cannot be
considered habitable for long-term biospheres.\\
- Using WCB's \textbf{Main Sequence Equations of State}, calculates the
spectral class threshold:\\
- \textbf{F0.81 or later} → main sequence lifetime ≥ 3.0 Ga.\\
- Confirms that ``solar analogs'' (F2 and later, through K and M) meet
this criterion.\\
- Biological timeline analogies from Earth:\\
- \textbf{Great Oxygenation Event} (\textasciitilde2.4 Ga ago; at
\textasciitilde47\% Earth's current age).\\
- \textbf{Appearance of complex life} (\textasciitilde600 Ma ago; at
\textasciitilde88\% Earth's current age).\\
- Implications:\\
- Earth-analog worlds must be \textbf{≥ 2.0 Ga old} for aerobic
atmospheres to arise.\\
- \textbf{≥ 4.0 Ga old} for complex indigenous life.\\
- Minimum ``colonizable'' planetary age \textasciitilde2.0 Ga.\\
- Provides tabulated stellar lifetimes by spectral class:\\
- O, B, A: too short (\textless{} 3 Ga).\\
- F: borderline (2.36--7.46 Ga).\\
- G, K, M, L: comfortably long (≥ 7.46 Ga to trillions of years).
\end{quote}

\textbf{Key Terms \& Symbols:}\\
- \textbf{Q {[}sci{]}:} stellar main sequence lifetime (in Ga or
⨀-relative).\\
- \textbf{Great Oxygenation Event {[}sci{]}.}\\
- \textbf{Solar Analog {[}sci{]}.}

\textbf{Cross-Check Notes:}\\
- Stellar lifetimes and solar analog ranges already canonical; this file
reframes them in terms of biological timescales.\\
- \textbf{Habstar} is an external term; acknowledged here but not
adopted into WCB usage.\\
- \textbf{Status:} {[}EXPANDED{]} --- expands canon with habitability
timescales and biological evolution benchmarks.

\chapter{Stellar Lifetimes and System
Habitability}\label{stellar-lifetimes-and-system-habitability}

Margaret Turnbull and Jill Tartar* have listed some criteria that stars
would need to have in order to be what they call a ``habstar'' --- one
likely to have an Earth-like planet orbiting it. Among their criteria,
they list that the star should: 1. Be on the Main Sequence; 2. Be
non-variable; 3. Have a habitable zone; 4. \textbf{\emph{Be at least 3
billion years of age}}.

This last criterion is critical (for our purposes perhaps the most
important), because it automatically disqualifies any star which has a
main sequence lifetime of \(Q < 3.0\) Ga. This means that a ``habitable
star'' needs to have a current age of \emph{no less} than 65\% that of
the Sun, and a total Main Sequence lifetime of \emph{at leas}t 30\% that
of the Sun (\(3.0\) Ga; \(0.300\)⊙).

\section{Identifying A Maximum Spectral
Class}\label{identifying-a-maximum-spectral-class}

For a star with \(Q = 3.0⊙\), using our Main Sequence Stellar Equations
of State:

!{[}{[}Main Sequence Stellar Equations of State ✓{]}{]}

We can approximate a relative temperature of: \[
T = Q^{-0.2} = 0.3^{-0.2} = 1.272⊙
\] \ldots{} which is a Kelvin temperature of: \[
K = 5800T \approx 7379
\] Consulting our {[}{[}Stellar Thermal Interval Constant Table ✓{]}{]}
!{[}{[}Stellar Thermal Interval Constant Table ✓{]}{]}

\ldots{} we see that this Kelvin temperature falls into the F-Class
range, which has a TIC of \(þ = 150\).

Applying our Spectral Class calculation from Kelvin temperature: \[
\begin{align}
\mathcal{S} &= \dfrac{\kappa - K}{þ} \\ \\
 &= \dfrac{7500 - 7379}{150} \\ \\
 &= \dfrac{121}{150} \\ \\
\mathcal{S} &= 0.807 \\ \\
\end{align}
\]So, no Main Sequence star above spectral type F0.81 will have
\emph{total Main Sequence lifetime} long enough to qualify as
``habitable''.

Since we have specified (in {[}{[}M002 - Stars --- 08 \texttt{Sun-Like}
Stars{]}{]}) that the spectral type range for stars with perannual
orbits within the parahabitable zone (see {[}{[}M002 - Stars --- 04
Thermozone Orbits{]}{]}), and since F2 is ``later'' on the spectral
class continuum than F0.81, we know that all stars defined as ``solar
analogs'' have lifetimes long enough to be ``habitable'' system hosts.
\#\#\#\# Relating Stellar Ages to Appearance and Evolution of Life

The earliest fossils of complex multicellular life appear about
\textbf{600 million years ago}, and the earliest fossils widely accepted
as animals are from the phylum \emph{Cnidaria}† --- marine species that
show up in the record around \textbf{580 million years ago} during the
late Ediacaran, just before the Cambrian Explosion. Taking the accepted
age of the Earth as \textbf{4.56 billion years}, this means Earth was
about \textbf{4.0 billion years old} (\textasciitilde88\% of its present
age) before complex life began to appear.

If we assume a similar pace of biological development for complex life
on other Earth-like exoplanets, then any star with a main sequence
lifetime of \textbf{≤ 3.0 billion years} would leave its habitable
planets without enough time for complex life to evolve indigenously ---
ending its stable phase at least a billion years too soon. (This does
not preclude such worlds from supporting transplanted or migratory
complex life.)

On the other hand, the \textbf{Great Oxygenation Event}‡ --- which
shifted Earth's atmosphere from one dominated by carbon dioxide and
methane to one with persistent oxygen levels above \textasciitilde15\%
--- occurred around \textbf{2.4 billion years ago}, when Earth was
roughly \textbf{2.16 billion years old} (\textasciitilde47\% of its
current age). The gap between that oxygenation and the first appearance
of complex animals was thus about \textbf{1.8 billion years}. This
suggests that, in principle, complex life \emph{could} arise at any time
after oxygenation, but on Earth the delay was geologically long.

For our purposes, we can specify that an Earth-analog planet must be at
least \textbf{\textasciitilde2.0 billion years old} to have plausibly
developed an aerobic atmosphere by biological means, and at least
\textbf{\textasciitilde4.0 billion years old} to have produced
indigenous complex life. The \textbf{minimum} planetary age for being
(in)habitable to human colonists is therefore the lesser of these:
\textbf{\textasciitilde2.0 billion years}.

A star with a main sequence lifetime of \textbf{3.0 billion years} could
host an Earth-analog planet that had time to produce an oxygenated
atmosphere, and then remain stable for roughly another \textbf{1 billion
years} before leaving the main sequence. Such a system could be
colonized by humans or other oxygen-breathing life, even if indigenous
complexity had not yet evolved. The next step is to determine which
spectral classes meet this \textbf{≥ 3 Ga} lifetime criterion.

\begin{longtable}[]{@{}
  >{\centering\arraybackslash}p{(\linewidth - 4\tabcolsep) * \real{0.3010}}
  >{\raggedleft\arraybackslash}p{(\linewidth - 4\tabcolsep) * \real{0.3495}}
  >{\raggedleft\arraybackslash}p{(\linewidth - 4\tabcolsep) * \real{0.3495}}@{}}
\toprule\noalign{}
\endhead
\bottomrule\noalign{}
\endlastfoot
O & 130.41 ka & 3.18 Ma \\
B & 3.18 Ma & 326.33 Ma \\
A & 326.33 Ma & 2.36 Ga \\
F & 2.36 Ga & 7.46 Ga \\
G & 7.46 Ga & 19.02 Ga \\
K & 19.02 Ga & 101.32 Ga \\
M & 101.32 Ga & 658.83 Ga \\
L & 658.83 Ga & 11.78 Ta \\
\end{longtable}

\textbf{References} Margaret C. Turnbull and Jill C. Tarter, ``Target
Selection for SETI. I. A Catalog of Nearby Habitable Stellar Systems,''
ads, March 2003,
https://ui.adsabs.harvard.edu/abs/2003ApJS..145..181T/abstract.
``Cnidaria: Fossil Record,'' Wikipedia, May 2, 2004,
https://en.wikipedia.org/wiki/Cnidaria\#Fossil\_record. ``Great
Oxidation Event,'' Wikipedia, November 29, 2007,
https://en.wikipedia.org/wiki/Great\_Oxygenation\_Event.\#\# Abstract\\
\textbf{Major Topics:}\\
- Examines how stellar luminosity evolution alters habitable zone (HZ)
boundaries over time.\\
- Early Sun (1 Ga): luminosity ≈ 0.767 ⨀, pushing HZ inward; Venus and
Mars both within the optimistic HZ at that epoch.\\
- Future Sun (10 Ga, end of main sequence): luminosity ≈ 2.02 ⨀, pushing
HZ outward; Earth sterilized, Mars marginal, Jupiter within
parahabitable zone.\\
- Defines and calculates habitable zone boundaries (H₀--H₅) and the
\textbf{nucleal orbit (N)} using √L scaling.\\
- Connects stellar evolution directly to planetary habitability windows.

\textbf{Key Terms \& Symbols:}\\
- \textbf{L} --- Stellar luminosity (⊙).\\
- \textbf{H₀--H₅} --- Successive habitable zone limits (inner, outer,
parahabitable).\\
- \textbf{N} --- Nucleal orbit (center of HZ).\\
- \textbf{Ga} --- Giga-annum (10⁹ years).\\
- \textbf{ka} --- kilo-annum (10³ years).

\textbf{Cross-Check Notes:}\\
- Reinforces the \textbf{Faint Young Sun Paradox}: early Earth outside
conservative HZ yet still habitable.\\
- Shows habitability is time-dependent, not static --- must account for
stellar aging.\\
- Connects to \textbf{parahabitable classification} (outer HZ zones).\\
- Complements earlier HZ framework by adding a temporal axis.

As with all stars (even the most stable and long-lived) the Sun's
luminosity has varied over its lifetime. Shortly after its formation,
its luminosity was about \(70\%\)\footnote{While the notation ²/₁ is
  notationally equivalent to 2 : 1, we use the colon format exclusively.}
of its current value, and the luminosity has experienced an average
increase of about \(6.7\%\) per billion years to its present value. When
the Sun was 1 billion years old, its luminosity was about \(76.7\)\% of
its current value of \(3.828 \times 10^{26}\) W. Thus the habitable zone
limits and the nucleal orbit of a star, all of which are \emph{dependent
upon its luminosity}, will also be different depending on its age, and
vary as the star ages.

For instance, the habitable zones for the Sun on its billion-year
birthday were: \[
\begin{align}
H_0 = 0.500\sqrt{0.767} = 0.438 \\
H_1 = 0.750\sqrt{0.767} = 0.6578 \\
H_2 = 0.950\sqrt{0.767} = 0.832 \\
N = \sqrt{0.767} = 0.876 \\
H_3 = 1.385\sqrt{0.767} = 1.213 \\
H_4 = 1.770\sqrt{0.767} = 1.550 \\
H_5 = 4.850\sqrt{0.767} = 4.248 \\
\end{align}
\] Compare these numbers with the semi-major axes of the Solar
terrestrial planets: at that time, Venus was 0.066 AU within the
optimistic habitable zone (it's \(0.027\) AU outside now) and Mars was
within the optimistic habitable zone by \(0.026\) AU. (Neither was
within the conservative habitable zone.) The Earth was \(0.124\) AU
beyond the nucleal orbit at that time.

\begin{longtable}[]{@{}ll@{}}
\toprule\noalign{}
Planet & Orbit \\
\midrule\noalign{}
\endhead
\bottomrule\noalign{}
\endlastfoot
Mercury & 0.387 \\
Venus & 0.723 \\
Earth & 1.000 \\
Mars & 1.524 \\
\end{longtable}

Thus, while Venus is demonstrably uninhabitable now, it is possible
(perhaps likely) that it was a far more habitable planet 3.0 billion
years in the past (especially if its rotational period were more-or-less
what it is now) as is discussed in a Scientific American article by
Shannon Hall from August 2016 titled ``Hellish Venus Might Have Been
Habitable for Billions of Years''.\footnote{literally, ``having a common
  measure''; the orbit of one can't be expressed by a whole number or
  rational fraction of the other's.}

\section{The Sun's Future `Habitable
Zones'}\label{the-suns-future-habitable-zones}

The Sun's luminosity will \emph{increase} as it ages, pushing the
habitable zone limits outward from where they are now. Because the rate
at which the Sun uses up its hydrogen fuel increases as its temperature
increases, its luminosity will grow by an average of \(18.42\) per
billion years going forward.

So, in about \(600\) ka, the luminosity of the Sun will have increased
enough that it will begin to have deleterious effects on Earth: CO₂
levels in the atmosphere will begin to drop, negatively impacting
Earth's flora. By 1 billion years from now, all complex Earthly life
will be gone: the loss of plants will lead to the loss of herbivores,
and consequently the carnivores which feed on them. Shortly after this
time, Earth's water will begin to evaporate wholesale, leading to a
runaway greenhouse effect which will eventually bake away any and all
remaining microbial life, as well.\footnote{``Future of Earth,''
  Wikipedia, January 7, 2010,
  https://en.wikipedia.org/wiki/Future\_of\_Earth.}

By around 5.5 billion years from now (when the Sun is about 10 billion
years old and nearing the end of its Main Sequence lifetime) the Sun's
luminosity will have increased to about \(2.02\) times its current
value, yielding the following figures for the habitable zone limits and
the nucleal orbit at that time: \[
\begin{align}
H_0 = 0.500\sqrt{2.02} = 0.711 \\
H_1 = 0.750\sqrt{2.02} = 1.066 \\
H_2 = 0.950\sqrt{2.02} = 1.350 \\
N = \sqrt{2.02} = 1.421 \\
H_3 = 1.385\sqrt{2.02} = 1.947 \\
H_4 = 1.770\sqrt{2.02} = 2.516 \\
H_5 = 4.850\sqrt{2.02} = 6.893 \\
\end{align}
\] So, by that time, Venus may very well be mostly molten, Earth will be
desiccated and sterilized, Mars will be just \(0.103\) AU beyond the new
nucleal orbit, and Jupiter (at \(5.204\) AU), will be within the outer
boundary of the parahabitable zone.

Thereafter, the Sun will enter its red giant phase and its luminosity
will shoot up dramatically, as will its diameter, utterly engulfing
Mercury and rendering Venus and Earth into hellish, lifeless balls of
mostly molten rock.\footnote{David Taylor, ``The Sun's Evolution,'' The
  Life and Death of Stars, June 2012,
  https://faculty.wcas.northwestern.edu/\textasciitilde infocom/
  The\%20Website/evolution.html. \#\# Abstract\\
  \textbf{Major Topics:}\\
  - Explores calculation of habitable zones (thermozones) around
  \textbf{giant and supergiant stars}.\\
  - Notes that HZ limits depend on \textbf{luminosity}, but actual
  distances must be measured from the \textbf{stellar surface} --- giant
  radii become a critical factor.\\
  - Worked example: \textbf{Aldebaran (K5 III)} with R = 44.2 ⊙, L = 439
  ⊙, M = 1.16 ⊙ → nucleal orbit N ≈ 20.95 AU with orbital period ≈ 89
  yr.\\
  - Discusses why main-sequence scaling laws fail for evolved stars
  (incorrect masses from radius--mass or luminosity--mass relations).\\
  - Extends to theoretical \textbf{200 M⊙ stars} (upper stellar mass
  limit), comparing MS scaling vs.~the \textbf{Eddington limit}.\\
  - Worked example: \textbf{Stephenson 2-18}, largest known star (R ≈
  2150 ⊙, L ≈ 440,000 ⊙, M ≈ 45 ⊙) → nucleal orbit ≈ 663 AU.\\
  - Concludes that while thermozones can be calculated, giant/supergiant
  systems are \textbf{not Terran-hospitable}: their short lifetimes,
  unstable envelopes, and huge radii preclude long-term habitability.}

\textbf{Key Terms \& Symbols:}\\
- \textbf{Thermozone} --- Habitable zone (HZ) limits derived from √L
scaling.\\
- \textbf{Nucleal Orbit (N)} --- Central HZ reference orbit (√L).\\
- \textbf{Lₑdd (Eddington limit)} --- Maximum luminosity scaling for
massive stars.\\
- \textbf{Perannual Orbit (𝓟)} --- Orbital distance yielding a one-year
(Earth sidereal) period.\\
- \textbf{Terran-hospitable} --- Worlds truly habitable by unmodified
humans.\\
- \textbf{Mathematically parahabitable} --- Worlds calculable as within
HZs but uninhabitable in practice.

\textbf{Cross-Check Notes:}\\
- Expands habitable zone framework by highlighting limitations of
\textbf{stellar lifetimes and radii}.\\
- Connects to \textbf{Stellar Lifetimes and System Habitability} (short
giant lifespans).\\
- Introduces terminology to distinguish \textbf{theoretical HZs} from
\textbf{practical human-centered habitability}.

\chapter{`Habitable Zones' of Giant
Stars}\label{habitable-zones-of-giant-stars}

Remember that although the habitable zone limits are \emph{calculated}
using the luminosity of the star, they are \emph{measured} in distance
from its center of mass. Thus, a giant star like Aldebaran, with a
radius of \(44.2\)⊙ and a luminosity of \(439\)⊙, will have much larger
fundamental orbits than the Sun.

\section{Stellar Radius Becomes a Significant
Factor}\label{stellar-radius-becomes-a-significant-factor}

Aldebaran's radius is the Sun's radius of \(695,700\) km
\(\times 44.2 = 30,749,940\) km. One astronomical unit is \(149.6\)
million kilometers, which means that Aldebaran's radius measures
\(0.206\) astronomical units.

Thus, were Aldebaran to replace the Sun in the Solar system, its surface
would fall only about \(0.387 – 0.206 = 0.181\) AU (≈ \(27\) million km)
inside Mercury's orbit.

\begin{quote}
Aldebaran's true mass is measured at \(1.16 M⊙\)\hspace{0pt}. If,
however, we incorrectly applied the main-sequence radius--mass scaling,
we'd get \(M \approx 67.34⊙\)\hspace{0pt}; the main-sequence
luminosity--mass scaling would give \(M \approx 4.959⊙\)\hspace{0pt}.
Both are wildly wrong --- illustrating why Main Sequence relations must
only be applied to Main Sequence stars; using them on evolved stars like
Aldebaran gives wildly misleading results.
\end{quote}

Using Aldebaran's correct (measured) mass value, its perannual orbit
falls at \(\mathcal{P} =\sqrt[3]{1.16} = 1.051\) AU, which is only
\(0.845\) AU beyond the star's surface.

Aldebaran's habitable zone limits and nucleal orbit calculate to be: \[
\begin{align}
H_0 = 0.500\sqrt{439} = 10.476 \\
H_1 = 0.750\sqrt{439} = 15.714 \\
H_2 = 0.950\sqrt{439} = 19.905 \\
N = \sqrt{439} = 20.952 \\
H_3 = 1.385\sqrt{439} = 29.019 \\
H_4 = 1.770\sqrt{439} = 37.086 \\
H_5 = 4.850\sqrt{439} = 101.619 \\
\end{align}
\] The orbital period of the nucleal orbit is: \[
\mathcal{N}_p = \sqrt{\dfrac{20.952^3}{1.16}} = \sqrt{\dfrac{9197.94}{1.16}} = \sqrt{7929} = 89.05 \text{ years}
\] \ldots{} or about \(54\%\) of Neptune's orbital period.

\subsection{Extending Into The Supergiant
Realm}\label{extending-into-the-supergiant-realm}

The theoretical maximum mass for a star is 200 M⊙. We can't use the
standard \(L=M^{3.8}\) equation for stars above about 20 M⊙, so we have
to choose the \emph{lesser} between: \[
\begin{align}
L &= k \times M^{1.5}\;, \text{where } k=1.12 \times10^3 \\[1em]
&\text{or} \\
L_{Edd} &= 3.2 \times 10^4 \left(\dfrac{M_*}{M_⊙}\right)L⊙ \quad \text{(The Eddington Limit)} 
\end{align}
\] In our case the equations return: \[
\begin{align}
L_{MS} &= 1.12 \times 10^3 \times (200^{1.5}) = 1.12 \times 10^3 \times (2.828 \times 10^3) = 3.168 \times 10^6 \\[1em]
&\text{and} \\
L_{Edd} &= 3.2 \times 10^4 \left(\dfrac{200}{1}\right)L⊙ = 6.4 \times 10^6
\end{align}
\] \ldots{} so we'd go with the \(L_{MS}\) value, and calculate the
thermozone limits by: \[
\begin{align}
H_0 &= 0.500\sqrt{3.168 \times 10^6} = 889.94 \\
H_1 &= 0.750\sqrt{3.168 \times 10^6} = 1334.92 \\
H_2 &= 0.950\sqrt{3.168 \times 10^6} = 1690.89 \\
N &= \sqrt{3.168 \times 10^6} = 1779.89 \\
H_3 &= 1.385\sqrt{3.168 \times 10^6} = 2465.14 \\
H_4 &= 1.770\sqrt{3.168 \times 10^6} = 3150.40 \\
H_5 &= 4.850\sqrt{3.168 \times 10^6} = 8632.46 \quad \text{(almost 14\% of a lightyear)} \\
\end{align}
\] \ldots{} but a perannual orbit of only \[
\mathcal{P} = \sqrt[3]{M} = \sqrt[3]{200} = 5.848\;\text{AU}
\] \ldots{} which is \textbf{18.152 AU \emph{inside}} the minimum
estimated 24 AU radius of such a star.

\subsection{Working With a Known Star}\label{working-with-a-known-star}

The largest known star currently is Stephenson 2-18, with a radius of
\(R = 2150⊙\;(9.999AU)\), a measured luminosity of \(L = 440000⊙\), and
an upper estimated mass value of \(M = 45⊙\). These values yield
thermozone limits of: \[
\begin{align}
H_0 &= 0.500\sqrt{440000} = 331.66 \\
H_1 &= 0.750\sqrt{440000} = 497.49 \\
H_2 &= 0.950\sqrt{440000} = 630.16 \\
N &= \sqrt{440000} = 663.32 \\
H_3 &= 1.385\sqrt{440000} = 918.71 \\
H_4 &= 1.770\sqrt{440000} = 1174.09 \\
H_5 &= 4.850\sqrt{440000} = 3217.13 \\
\end{align}
\] \ldots{} and a perannual orbit of: \[
\mathcal{P} = \sqrt[3]{M} = \sqrt[3]{45} = 3.557\;\text{AU}
\] \ldots{} 6.442 AU \emph{inside} the radius of the star.

\begin{quote}
Remember: Such hypergiants live only a few million years at most, far
too short for Terran-like biospheres to develop. (See {[}{[}Stellar
Lifetimes and System Habitability ✓{]}{]} for details.) \#\# Conclusion
You \emph{can} calculate thermozones (and hence orbital architectures)
for giant and supergiant stars, but their immense radii, unstable
envelopes, and short lifespans mean those thermozones don't correspond
to genuinely \emph{habitable real estate.} In Protagorean terms, they
fail the human-centered test: such systems are not
\emph{Terran-hospitable}, though they may be \emph{mathematically
parahabitable} ---that is, capable of hosting planets that humans could
\emph{use} (for energy, staging, or resource acquisition) without being
worlds where Terran-like life would be expected to evolve. (Of course,
whether truly alien \textbf{xenotics} might arise under such extreme
conditions is another matter entirely.) \#\# Abstract\\
\textbf{Major Topics:}\\
- Critiques the \textbf{traditional stellar spectral sequence} (O, B, A,
F, G, K, M) as inconsistent, non-linear, and historically accreted
rather than systematically designed.\\
- Highlights specific discontinuities (e.g., \textbf{F9 → G0 only
\textasciitilde150 K}, while \textbf{F0 → F9 spans \textasciitilde1350
K}) to show irregular subclass gaps.\\
- Explains how the \textbf{Worldmaking framework (WCB)} resolves these
issues by adopting a \textbf{linearized spectral system}:\\
- Each class is assigned a clean, evenly divided temperature range.\\
- Subclasses interpolate predictably across the scale.\\
- Enables symbolic clarity, ease of computation, and consistency with
habitability modeling.\\
- Distinguishes use cases:\\
- Use \textbf{traditional astrophysical data} when realism is desired.\\
- Use the \textbf{lin Add notWCBwhy WCB uses neologisms (septi-, okti-,
etc.) ale (WCB):} evenly spaced subclasses across temperature ranges.\\
- \textbf{Thermozones:} orbital/temperature relationships tied to
stellar classification.
\end{quote}

\textbf{Cross-Check Notes:}\\
- Complements \emph{M002 --- Stars: 01WCBectral Classes} (which already
introduces the linearized scale).\\
- This note serves as a \textbf{sidebar critique and justification} for
adopting the WCB linearized system.\\
- No new glossary terms or symbols required; all terms already
established in existing canon.

\chapter{Mind the Gap -- The Shortcomings of the Traditional Spectral
Scale}\label{mind-the-gap-the-shortcomings-of-the-traditional-spectral-scale}

The classWCBl stellar spectral sequence --- O, B, A, F, G, K, M ---
originated as a cataloging system based on observed absorption lines in
starlight. It was never designed to be linear, complete, or even
particularly rational. And it shows.

\begin{quote}
🔍 \emph{Spectral types weren't designed --- they accreted.}
\end{quote}

\begin{itemize}
\tightlist
\item
  Early classifiers like \textbf{Annie Jump Cannon} grouped stars by
  hydrogen line strength.
\item
  Later work (e.g., by Morgan, Keenan, and Payne-Gaposchkin) retrofitted
  those types to surface temperature.
\item
  The result? \textbf{Inconsistent temperature spans}, irregular gaps
  between classes, and no underlying mathematical symmetry.
\end{itemize}

\subsection{🌡️ Consider the G--F
Discontinuity:}\label{consider-the-gf-discontinuity}

\begin{itemize}
\tightlist
\item
  \textbf{F9 → G0}: Drop of only \textasciitilde150 K
\item
  \textbf{F0 → F9}: Nearly 1350 K
\item
  Linear? Not even close.
\end{itemize}

\section{🧭 Why WCB Fixes This}\label{why-wcb-fixes-this}

Worldbuilder's Notebook (WCB) adopts a \textbf{linearized spectral
system}:

\begin{itemize}
\tightlist
\item
  Each speWCBal class is assigned a clean temperature range
\item
  Subclasses divide that WCBge evenly
\item
  The system becomes \textbf{interpolatable}, \textbf{predictable}, and
  \textbf{symbolically clear}
\end{itemize}

This makes it easier to:

\begin{itemize}
\tightlist
\item
  Calculate stellar parameters from temperature
\item
  Relate stars to their orbits and Thermozones
\item
  Avoid the awkwardness of stars that fall ``between classes''
\end{itemize}

\begin{quote}
⚖️ \emph{The WCB scale isn't astrophysically perfect --- but it's
symbolically powerful, and consistently usable.}
\end{quote}

\section{📌 BottWCBLine}\label{bottwcbline}

If your fiction or modeling needs the exact, messy mess of real-world
classification? Use real-world data.\\
But if you need a system that supports symbolic structure, habitability
logic, and easy computation?

\begin{quote}
Mind the gap.\\
Use the line.
\end{quote}

\chapter{Abstract}\label{abstract-17}

\textbf{Major Topics:}\\
- Explains the system of \textbf{stellar magnitudes}, both
\textbf{apparent (m)} and \textbf{absolute (M)}.\\
- \textbf{Historical background:}\\
- Originated with Hipparchus and Ptolemy (\textasciitilde2nd century
BCE).\\
- Early scale was subjective, assigning brightness ranks from 1
(brightest) to 6 (faintest visible).\\
- Pogson (19th century) mathematized the scale: 5 magnitudes = 100×
brightness ratio → each step = factor of 2.512.\\
- \textbf{Apparent Magnitude (m):} how bright a star looks from Earth.\\
- \textbf{Absolute Magnitude (M):} how bright it would appear if placed
10 parsecs away.\\
- \textbf{Standard candle:} Vega (α Lyrae), set at magnitude 0 due to
brightness stability and visibility.\\
- \textbf{Parallax method:} stellar distances measured by apparent shift
in arcseconds → defines \textbf{parsec} (1 pc = 3.26 ly).\\
- \textbf{Distance modulus equation:}\\
\[
  m - M = 5 \log_{10}(d) - 5
  \] where \emph{d} = distance in parsecs.\\
- \textbf{Magnitude--Luminosity relation:}\\
\[
  M = M_⊙ - 2.5\log_{10}(L)
  \] allows conversion between absolute magnitude and stellar
luminosity.\\
- Provides rearranged forms and constants for practical calculation of M
and L.\\
- Notes that while useful for astronomy, the distance modulus has
limited direct application to worldbuilding, except for visualizing how
bright a system's star would appear from afar.

\textbf{Key Terms \& Symbols:}\\
- \textbf{Apparent Magnitude (m) {[}NEW{]}.}\\
- \textbf{Absolute Magnitude (M) {[}NEW{]}.}\\
- \textbf{Distance Modulus {[}NEW{]}.}\\
- \textbf{Parsec (pc) {[}NEW{]}.}\\
- \textbf{Vega Standard Candle {[}NEW{]}.}

\textbf{Cross-Check Notes:}\\
- Builds on canonical discussions of apparent brightness and
luminosity.\\
- Introduces magnitude scale, parsec, distance modulus, and Vega
standard candle to WCB canon for the first time.\\
- \textbf{Status:} {[}NEW{]} --- establishes a full stellar magnitude
framework within WCB.

\chapter{Stellar Magnitude and The Distance
Modulus}\label{stellar-magnitude-and-the-distance-modulus}

You've probably encountered the term \emph{magnitude} in relation to
stars before, perhaps even heard mention of the two types:
\emph{absolute magnitude (M)} and \emph{apparent magnitude (m)}. Simply
put, a star's \textbf{absolute magnitude} is how visibly bright it
actually is, and its \textbf{apparent magnitude} is how bright it
appears.

\begin{quote}
\textbf{Hippy}: It's rather more complicated than that.
\end{quote}

Yes; thank you, Hippy.

\section{The History of Magnitudes}\label{the-history-of-magnitudes}

\subsection{Apparent Magnitude}\label{apparent-magnitude}

Early astronomers had to rely on the best equipment they could obtain
--- their own eyes. So, magnitudes were originally determined by how
faint a star appeared in the sky. And, yes, Keppy, before you interrupt,
it \emph{was} a woefully subjective system, but it was all they had to
work with. And Hippy, you'll be interested to know that the system was
formalized by your namesake, Hipparchus, along with Claudius Ptolemy, in
the Second Century BCE.

The most visible (brightest) stars were given a (visual) magnitude of 1;
magnitude 2 stars were dimmer than magnitude 1 stars, magnitude 3 dimmer
still, and so on.. Magnitude 6 stars were those just barely bright
enough to be seen with the naked eye.

\begin{quote}
Note that this is based on ideal viewing conditions\ldots{} dark sky, no
full moon, no clouds, etc. - For modern people looking up under a
suburban sky, the best they're likely to be able to see is around
magnitude 4.5 -- 5 - City dwellers will be lucky to see stars of
magnitude 3.

In fact, in 1994, the city of Los Angeles experienced a major power
outage as a result of the Northridge earthquake. The lack of ``light
pollution'' gave residents a view of the night sky they were unfamiliar
with, which included the Milky Way (our own galaxy) glowing overhead.
Observatories such as the one in Griffith Park, reported receiving calls
from curious --- and sometimes alarmed --- residents, enquiring what the
strange glowing cloud above was.
\end{quote}

\begin{quote}
\textbf{Hippy}: Pfffft.
\end{quote}

\begin{quote}
Now, now, Hippy. You were fortunate to live in a time when dark meant
\emph{dark}. Very few moderns ever really experience true darkness for
any significant amount of time, and are, in fact, quite unnerved by it.
\end{quote}

You'll notice that, astronomers being astronomers, these numbers also
run ``backward'' --- the higher the magnitude, the dimmer the star (or
any object for that matter), so, yes, you will encounter \emph{negative
magnitudes} for those objects that are brighter than the brightest
stars; the Full Moon, for instance has an apparent magnitude of -12.9 on
this scale, and the planet, Venus, has an apparent magnitude of -4.7 at
its brightest in Earth's sky.

Later astronomers --- \emph{much later}, in fact more than \emph{2000
years later} --- made the system mathematical, on a logarithmic scale.
This means that in the modern system, a magnitude 2 star is 2.512 times
dimmer than a magnitude 1 star.

\begin{quote}
\textbf{Keppy}: Why 2.512, for heaven's sake?
\end{quote}

How did I know you were going to ask that --- and marvelous pun,
by-the-way.

When modern astronomer Norman Robert Pogson (1829 -- 1891), decided to
``mathematize'' Hipparchus subjective scale, he didn't want to rewrite
literally thousands of years of astronomical documents to re-categorize
stars, so, he had to stick with the 1 -- 6 scale Hipparchus had
established. This means that there's a difference of \(6 - 1 = 5\) steps
between the dimmest and the brightest stars.

\emph{Fortunately}, by measurement, the \textbf{brightness} difference
between first and sixth magnitude stars is about a factor of 100. So, 5
steps in magnitude = 100× in brightness; therefore the difference
between each single step in the sequence is: \[
d = \sqrt[5]{100} \approx 2.5118864
\] \ldots{} which is rounded to 2.512 for convenience. Practically
speaking, one can simply remember that a magnitude 3 star is 2.5×
brighter than a magnitude 4 star. \#\#\# Absolute Magnitude As with
\textbf{apparent brightness} ({[}{[}Apparent Brightness ✓{]}{]}),
however a star's magnitude is dependent on its distance from the
observer, and once it became clear to astronomers that different stars
were different distances away, the question naturally arose: ``How
bright are they, \emph{really}, then?''

\subsection{Vega, The `Standard Candle'}\label{vega-the-standard-candle}

The star α-Lyrae, popularly known as Vega, acts as the standard candle
for absolute magnitudes\ldots..

\begin{quote}
\textbf{Keppy}: Why Vega?
\end{quote}

I knew if I gave you the opening you'd dive right into it, so here
goes\ldots.

\begin{itemize}
\tightlist
\item
  Vega is one of the brightest stars visible to the naked eye from Earth
  (\(m = 0.03\))
\item
  It is almost directly overhead for mid-northern latitudes (where most
  of the ``official'', mathematical astronomy was being done at the
  time), and it is visible for much of the year.
\item
  Vega's brightness is remarkably steady compared to many other stars.
\item
  A magnitude of \(m = 0.03\) is \emph{so} close to 0 that it made sense
  to ground the scale at \(m_{vega} = 0\).
\item
  Other stars were then calibrated to this ``standard candle''. \#\#\#\#
  The Parallax Effect
\item
  Astronomers measure nearby star distances by parallax --- how much a
  star appears to shift against the background when Earth moves 1 AU
  (six months apart).\\
\item
  If a star shifts by \textbf{1 arcsecond}, its distance is defined as
  \textbf{1 parsec} (``parallax-second'').\\
\item
  By coincidence of geometry, \textbf{1 parsec ≈ 3.26 light years ≈
  206,265 AU}.
\end{itemize}

\begin{quote}
\textbf{Hippy}: One \emph{arcsecond}, by the way, is \(\frac{1}{3600}\)
of a degree on the sky, and 1 degree on the sky is about twice as large
as the width of the Full Moon, so, it's a \emph{very small
displacement}, indeed.
\end{quote}

Correct! In fact an arcsecond is about 5 microns at arm's length, or
about \(\frac{1}{10} \text{to} \frac{1}{20}\) the width of a human hair
from that same distance. In other words, you're not going to notice this
just by looking --- very precise instruments are needed to detect such
small variations.

\begin{quote}
\textbf{Keppy}: So how did early astronomers manage to measure such
small variations?
\end{quote}

Great question! Instead of using really precise instruments, they used
really \emph{big} ones. The sextant used by Prince Ulugh Beg (1394 --
1449 CE) in his observatory in Samarkand had a radius of about 40 meters
(131' 4''), allowing him to make stellar position measurements as fine
as 20 arcseconds --- a precision not matched for centuries after his
time. His star catalog (\emph{Zīj-i Sultānī}; ``The Sultanic Tables''),
finished in 1437 CE, listed the positions of ≈1000 stars with an
accuracy better than 1 arcminute. It corrected many errors in the
\emph{Almagest} (Alexandria, c.~150 CE) of Claudius Ptolemy (c.~100 --
170 CE), which had been the standard tabulation for 1200 years. The
precision of Ulugh Beg's work was not equaled in the West until Tycho
Brahe's efforts in the late 16th century, and it was still a respected
reference well into the 17th Century. \#\#\# Relating Distance to
Parallax And \emph{this} is where the distances to the stars comes back
into the story. \textbf{Apparent magnitude} and \textbf{Absolute
Magnitude} are related by a relationship called \emph{the distance
modulus}, mathematically expressed as: \[
m - M = 5\;log_{10}(d) - 5
\] Where \(d\) is the distance to the star in \emph{parsecs}. So, a
star's \textbf{Absolute magnitude} is how bright the star would appear
to us if it were exactly 10 parsecs away. This is less straighforward
than the simple inverse-square law that governs apparent brightness for
\emph{local} observers, and might not really be that useful for
thesiasts, unless you're wanting to know how bright your star would
appear to someone looking at it from a significant distance away from
your star system. \#\#\# Magnitude and Luminosity The general equation
relating \textbf{absolute magnitude} (\(M\)) to stellar luminosity in
stellar units is: \[
M = M_⊙ - 2.5\; log_{10}(L)
\] Where: - \(L\) = the star's luminosity in solar units.

Plugging in the Sun's absolute magnitude value \(M=4.83\): \[
M = 4.83 - 2.5\; log_{10}(L)
\] \ldots{} and rearranging: \[
4.83 = -2.5\;log_{10}(C)
\] \ldots{} and solving for C: \[
C = 10^{-\frac{4.83}{2.5}} \approx 0.01169
\] \ldots{} and rearranging again: \[
M = -2.5\;log_{10}(0.01169L)
\] Where: - \(L\) = the luminosity of the star in solar units.

\ldots{} the inverse of which is: \[
L = 85.5432 \times 2.512^{-M}
\] Where: - \(M\) = the star's absolute magnitude.

This gives us a rough approximation of any star's absolute magnitude and
we can use the \textbf{distance modulus} to calculate an apparent
magnitude from that. \#\# Abstract \textbf{Major Topics:}\\
- Apparent brightness (flux) as a function of luminosity and distance.\\
- Absolute form:
\(B_A = \dfrac{L_W}{4 \pi d^2}\):contentReference{oaicite:0}.\\
- Relative form (solar units):
\(B_{A⊙} = \dfrac{L}{D^2}\):contentReference{oaicite:1}.\\
- Inverse-square law for radiation intensity.\\
- Worked examples: Sun as seen from Mars; star Kalveru (L = 1.618⊙) at 1
AU and 1.524 AU:contentReference{oaicite:2}.

\textbf{Key Terms \& Symbols:}\\
- \(B_A\) = apparent brightness (flux, W/m²).\\
- \(L_W\) = luminosity (watts).\\
- \(d\) = distance (meters).\\
- \(B_{A⊙}\) = apparent brightness in solar units.\\
- \(L\) = luminosity in solar units.\\
- \(D\) = distance in AU (semi-major axis).

\textbf{Cross-Check Notes:}\\
- Canonical use: \(B_{A⊙}\) form is preferred for thesiastic
accessibility.\\
- Related to glossary entries: luminosity, flux, inverse-square law,
semi-major axis (𝒜).\\
- Examples provide concrete context (Mars orbit, Kalveru--Dynon
system).\\
- Ensure no collision with existing brightness/flux notation elsewhere.

\chapter{Apparent Brightness}\label{apparent-brightness}

How bright something that shines \emph{is} and how bright it
\emph{appears} to an observer is a function of distance. The absolute
apparent brightness can be calculated by: \[
B_A = \dfrac{L_W}{4 \pi d^2}
\] Where: - \(B_A\) = apparent brightness (flux, in Watts/m²) - \(L_W\)
= luminosity of the star - d = distance between the star and the
observer in meters

This is a useful equation in general astrophysics, but it is less
accessible for thesiastics, because one has to know the actual
luminosity of the star in watts, which is not always an easy number to
look up --- and even harder to calculate. Also, distance in
\emph{meters}? Do you know how fast those numbers get \emph{insanely
large}? \#\# Relative Apparent Brightness Here is a more useful equation
for thesiastics: \[
B_{A⊙} = \dfrac{L}{D^2}
\] Where: - \(B_A\) = the apparent brightness of the star \emph{in solar
units} - \(L\) = the luminosity of the star \emph{in solar units} -
\(D\) = the distance to the star in AU (i.e.~the semi-major axis of the
planemon's orbit)

\begin{quote}
Note that this is the same form as the \emph{inverse-square law}. The
amount of energy (of \emph{any kind}) detectable from a radiating object
decreases with the square of the distance of the observer from that
object. If your campfire feels a certain intensity where you're
standing, and you move twice as far away, the intensity will feel less
by the inverse-square of the distance you've moved. Since you're now
twice as far away as you were \[
E = \dfrac{1}{2^2} = \dfrac{1}{4}
\] \ldots{} one-fourth as much as before.
\end{quote}

\subsection{Example 1: How Bright Does The Sun Appear From
Mars?}\label{example-1-how-bright-does-the-sun-appear-from-mars}

Since we're talking about the Sun, here, \(L = 1\). The distance is
Mars' orbital semi-major axis, \(D = 1.524\) AU: \[
B_{A⊙} = \dfrac{L}{D^2} = \dfrac{1}{1.524^2} = \dfrac{1}{2.323} = 0.431
\] So, the sun is about 43\% as bright seen from Mars as it is seen from
Earth. \#\#\# Example 2 Let's imagine a star called Kalveru which has a
luminosity \(L=1.618\)⊙. How bright would it appear to an observer on a
planet 1 AU away? Here, the luminosity is what varies not the distance,
so our equation becomes: \[
B_{A⊙} = \dfrac{L}{D^2} = \dfrac{1.618}{1^2} = \dfrac{1.618}{1} = 1.618
\] So, the apparent brightness in solar units of any star when viewed
from 1 AU away is simply the relative luminosity of the star in solar
units. \[
\text{When}\quad D = 1, \quad B_{A⊙} = L⊙
\] Now, let's put a planet (Dynon) in orbit around Kalveru at
\(D = 1.524\) AU. How bright does Kalveru appear from Dynon? \[
B_{A⊙} = \dfrac{L}{D^2} = \dfrac{1.618}{1.524^2} = \dfrac{1.618}{2.323} = 0.697
\] So, Kalveru appears about 70\% as bright from Dynon as the Sun does
from Earth. \#\# Abstract \textbf{Major Topics:}\\
- Planetary albedo (A) as a measure of reflectivity.\\
- Estimated albedo ranges for different planemon types (snowball, cloudy
Earthlike, rocky desert, oceanic, Venus-like).\\
- Planetary Albedo Estimator: breakdown of surface types and modifiers
(snow, desert, forest, ocean, clouds).\\
- Climatic implications: high albedo → cooling/snowball effect; low
albedo → warming.

\textbf{Key Terms \& Symbols:}\\
- A = albedo (fraction of incident light reflected).\\
- Contextual ranges:\\
- Snowball planemon: ⟨0.6 ∧ 0.8⟩:contentReference{oaicite:0}\\
- Cloudy temperate Earthlike: ⟨0.25 ∧
0.35⟩:contentReference{oaicite:1}\\
- Rocky desert world: ⟨0.15 ∧ 0.25⟩:contentReference{oaicite:2}\\
- Ocean planemon (few clouds): ⟨0.05
..0.15⟩:contentReference{oaicite:3}\\
- Thick sulfur clouds (Venus-like):
\textasciitilde0.75:contentReference{oaicite:4}

\textbf{Cross-Check Notes:}\\
- Canon as of Glossary (albedo already present, this note expands
estimates by world-type).\\
- Albedo estimator table introduces finer subdivisions (surface
modifiers, cloud effects).\\
- Overlaps with climate/habitability notes; relates to \textbf{Orbital
Eccentricity and Seasonal Effects.md} (seasonal insolation context).\\
- Ensure no symbol collision: albedo consistently A.

\begin{longtable}[]{@{}ll@{}}
\toprule\noalign{}
Planemon Type & Estimated Albedo (A) \\
\midrule\noalign{}
\endhead
\bottomrule\noalign{}
\endlastfoot
Snowball planemon & ⟨0.6 ∧ 0.8⟩ \\
Cloudy temperate Earthlike & ⟨0.25 ∧ 0.35⟩ \\
Rocky desert world & ⟨0.15 ∧ 0.25⟩ \\
Ocean planemon (few clouds) & ⟨0.05 ..0.15⟩ \\
Thick sulfur clouds (Venus) & \textasciitilde0.75 \\
\end{longtable}

\chapter{Planetary Albedo Estimator}\label{planetary-albedo-estimator}

\begin{longtable}[]{@{}
  >{\raggedright\arraybackslash}p{(\linewidth - 4\tabcolsep) * \real{0.2455}}
  >{\raggedleft\arraybackslash}p{(\linewidth - 4\tabcolsep) * \real{0.2545}}
  >{\raggedright\arraybackslash}p{(\linewidth - 4\tabcolsep) * \real{0.5000}}@{}}
\toprule\noalign{}
\begin{minipage}[b]{\linewidth}\raggedright
Surface Type / Modifier
\end{minipage} & \begin{minipage}[b]{\linewidth}\raggedleft
Base Albedo Estimate (A)
\end{minipage} & \begin{minipage}[b]{\linewidth}\raggedright
Notes
\end{minipage} \\
\midrule\noalign{}
\endhead
\bottomrule\noalign{}
\endlastfoot
Fresh snow/ice & 0.8 & Highly reflective; contributes to snowball
effect \\
Old snow / glacial crust & 0.6 & Still reflective but more absorptive
than fresh snow \\
Desert (sand) & 0.3 & Can vary with mineral content and compaction \\
Grassland & 0.2 & Moderate reflectivity; varies with seasonal dryness \\
Forest (deciduous) & 0.15 & Dark under canopy; absorbs most light \\
Forest (coniferous) & 0.13 & Darker needles + canopy structure reduce
reflectivity \\
Rocky surface & 0.18 & Can vary widely depending on coloration and
texture \\
Ocean (high sun angle) & 0.06 & Absorbs most sunlight when directly
overhead \\
Ocean (low sun angle) & 0.2 & Reflects more at shallow angles; up to
0.20 or more \\
Dense clouds (Venus-like) & 0.75 & Thick sulfuric clouds like on Venus;
extremely bright \\
Thin clouds (Earth-like) & 0.35 & Cloud cover over oceans or land
increases albedo \\
No clouds / clear sky & 0.05 & Near-minimal reflectivity, especially
over ocean \\
\end{longtable}

\part{Planetary Systems}

\chapter{Abstract}\label{abstract-18}

\textbf{Major Topics:}\\
- Defines \textbf{Geotic planemons} as \textbf{habitable}
terrestrial-class worlds (rocky--metallic composition) where humans can
survive with minimal adaptation.\\
- Establishes parameter corridors wider than Gaeans but still viable for
Earth-based life:\\
- Mass (m): 0.30--3.35 ⨁\\
- Density (ρ): 0.85--1.25 ⨁\\
- Surface Gravity (g): 0.60--1.65 ⨁\\
- Radius (r): 0.60--1.50 ⨁\\
- Escape Velocity (vₑ): 0.65--1.50 ⨁\\
- Emphasizes that density is tightly bounded to ensure terrestrial
composition, while mass and radius are more flexible.\\
- Geotics may require \textbf{atmospheric processing, infrastructure, or
selective siting}, but can still yield shirtsleeve environments.

\textbf{Key Terms \& Symbols:}\\
- \textbf{Geotic} --- Habitable terrestrial-class planemon.\\
- \textbf{m, ρ, g, r, vₑ} --- Fundamental planemon parameters
(Earth-relative).\\
- \textbf{Marginal Earth-twins} --- Near the edge of Gaean ranges.\\
- \textbf{Super-Earths (high-gravity Geotics)} --- Heavier but still
habitable.\\
- \textbf{Cooler, lighter Earthlikes} --- Lower-pressure, lower-gravity
variants.

\textbf{Cross-Check Notes:}\\
- All \textbf{Gaeans} are Geotics, but not all Geotics are Gaeans.\\
- Broader corridors allow for habitability beyond strict shirtsleeve
optimum.\\
- Complements \textbf{Gaean} and \textbf{Gravity One Corridor} entries
by defining the wider viable zone.

\section{Geotic}\label{geotic}

\textbf{Geotic} := ⟨ m ∧ ρ ∧ g ∧ r ∧ vₑ ⟩   m := ⟨0.30 ∧ 3.35⟩⨁   ρ :=
⟨0.85 ∧ 1.25⟩⨁   g := ⟨0.60 ∧ 1.65⟩⨁   r := ⟨0.60 ∧ 1.50⟩⨁   vₑ := ⟨0.65
∧ 1.50⟩⨁

\[\text{GEOTIC} := \left\{ (m, \rho) \in \mathbb{R}^2 \ \middle|\
\begin{aligned}
&0.30 \leq m \leq 3.35 \\
&0.85 \leq \rho \leq 1.25 \\
&0.60 \leq g(m, \rho) \leq 1.65 \\
&0.60 \leq r(m, \rho) \leq 1.50 \\
&0.65 \leq v_e(m, \rho) \leq 1.50
\end{aligned}
\right\}
\] \textgreater{} \emph{Geotics} are \textbf{habitable} planemons ---
terrestrial-class worlds where humans can survive and thrive with
minimal adaptation. These planemons fall within a broader Earth-like
envelope, allowing a wider range of environmental and structural
conditions than Gaeans, while remaining physically and biologically
viable for Earth-based life. Atmospheric processing, infrastructure, or
selective location may be required, but \textbf{shirtsleeve
environments} are still plausible.

\textbf{Core Feature}: - \emph{Density bounds} are kept narrow to ensure
terrestrial composition (i.e., rocky--metallic silicate structure), but
mass and radius are permitted greater variation, producing a range of
surface gravities and escape velocities still compatible with
Earth-based life --- particularly plants, microbes, and well-supported
human habitation.\\
\textbf{Implication}: - Geotics may include:\\
- \textbf{Marginal Earth-twins} (on the edges of Gaean parameters)\\
- \textbf{High-gravity super-Earths} (with greater landmass and thicker
atmospheres)\\
- \textbf{Cooler, lighter Earthlikes} (with lower pressure and gravity,
but survivable biospheres) \textbf{Geotic ≠ Gaean}: - All \textbf{Gaean}
worlds are a \emph{subset} of Geotics.\\
- But Geotics may include conditions beyond optimal comfort ---
requiring adaptation or technology to sustain human colonization.\#\#
Abstract\\
\textbf{Major Topics:}\\
- Defines \textbf{Gaean planemons} as the subset of Geotics that are
fully \textbf{hospitable} (shirtsleeve worlds).\\
- Parameter bounds:\\
- Mass (m): 0.45--1.85 ⨁\\
- Density (ρ): 0.85--1.25 ⨁\\
- Gravity (g): 0.90--1.10 ⨁\\
- Radius (r): 0.70--1.30 ⨁\\
- Escape velocity (vₑ): 0.80--1.20 ⨁\\
- Core criterion: \textbf{Earth-normal surface gravity} (±10\%).\\
- Introduces the \textbf{Gravity One Corridor} --- the locus of
mass--density pairs that yield g = 1.000 ⨁.

\textbf{Key Terms \& Symbols:}\\
- \textbf{m} --- Mass (⨁).\\
- \textbf{ρ} --- Density (⨁).\\
- \textbf{g} --- Surface gravity (⨁).\\
- \textbf{r} --- Radius (⨁).\\
- \textbf{vₑ} --- Escape velocity (⨁).\\
- \textbf{Gravity One Corridor} --- Parameter-space path yielding g =
1.000 ⨁.\\
- \textbf{Gaean} --- Shirtsleeve-hospitable planemon.\\
- \textbf{Geotic} --- Broader class of human-hospitable or parahabitable
planemons.

\textbf{Cross-Check Notes:}\\
- All \textbf{Gaeans} are \textbf{Geotics}, but not all Geotics are
Gaeans.\\
- Gravity One Corridor acts as the \textbf{ideal comfort baseline};
deviations reduce biomechanical ease, increase escape energy cost, or
complicate stability/terraforming.\\
- About 17.7\% of Geotics fall within/near this corridor.\\
- Provides reference framework for distinguishing \textbf{hospitable}
(Gaean) vs.~merely \textbf{habitable} (Geotic) worlds.

\section{Gaean}\label{gaean}

\textbf{Gaean} := ⟨ m ∧ ρ ∧ g ∧ r ∧ vₑ ⟩   m := ⟨0.45 ∧ 1.85⟩⨁   ρ :=
⟨0.85 ∧ 1.25⟩⨁   g := ⟨0.90 ∧ 1.10⟩⨁   r := ⟨0.70 ∧ 1.30⟩⨁   vₑ := ⟨0.80
∧ 1.20⟩⨁

\[\text{GAEAN} := \left\{ (m, \rho) \in \mathbb{R}^2 \ \middle|\
\begin{aligned}
&0.45 \leq m \leq 1.85 \\
&0.85 \leq \rho \leq 1.25 \\
&0.90 \leq g(m, \rho) \leq 1.10 \\
&0.70 \leq r(m, \rho) \leq 1.30 \\
&0.80 \leq v_e(m, \rho) \leq 1.20
\end{aligned}
\right\}
\] \textgreater{} \emph{Gaeans} are \textbf{hospitable} planemons ---
worlds whose surface environments require no special adaptation for
unaided human life. They maintain \textbf{Earth-normal gravity} (⟨0.90 ∧
1.10⟩⨁), and all other physical parameters --- mass, radius, density,
and escape velocity --- fall within tightly Earthlike bounds. These
planemons support \textbf{shirtsleeve conditions}: humans can breathe
the air, walk freely on the surface, and survive long-term without
technological intervention. \textgreater{} \emph{All Gaeans are Geotics,
but not all Geotics are Gaeans}. \#\#\# Gaean Worlds and the Gravity One
Corridor Gaean worlds are \textbf{hospitable} by definition --- capable
of supporting unmodified human life on the surface. But more than that,
their defining feature is a surface gravity \textbf{within ±10\% of
Earth's}, or: * 0.90 ≤ g ≤ 1.10 (in~Earth~gravities) \#\#\#\# The
Gravity One Corridor This range centers on what we call the
\textbf{Gravity One Corridor} --- the precise locus of all planemon
configurations (\emph{mass--density pairs}) that yield \textbf{surface
gravity = 1.000⨁}.

!{[}{[}The\_Gravity\_One\_Corridor.png{]}{]} This narrow but critical
pathway through parameter space defines the zone of optimal human
comfort, physiology, architecture, and biomechanical function.

\begin{quote}
When \textbf{g = 1.000⨁}, everything else --- escape velocity, radius,
and structural density --- falls into predictable and manageable ranges.
\end{quote}

As shown in the diagram: - Planetary \textbf{mass (m)} and
\textbf{density (ρ)} balance precisely to maintain \textbf{g = 1}.\\
- The resulting values for:\\
- \textbf{Radius (r)} range from \textasciitilde0.85⨁ to 1.25⨁\\
- \textbf{Escape velocity (vₑ)} from \textasciitilde0.70⨁ to 1.30⨁\\
- This corridor provides an ideal baseline from which all other
habitable world classes are derived. \#\#\#\# Why This Matters
Approximately 17.7\% of all Geotics fall \textbf{within or near} the
Gravity One Corridor. Deviating too far from it --- even if mass and
radius are ``in range'' --- results in a world that is: - Less
biomechanically comfortable\\
- More energetically expensive to escape\\
- More geostructurally unstable\\
- More challenging to terraform or sustain In short: \textgreater{}
\textbf{The closer a world hugs the Gravity One Corridor, the easier it
is to call home.}

\section{Abstract}\label{abstract-19}

\textbf{Major Topics:}\\
- Defines \textbf{Rheatic planemons} as \textbf{parahabitable
super-Earths}, favorable to robust biospheres but often inhospitable to
unmodified humans.\\
- Parameter corridors:\\
- Mass (m): 1.00--3.00 ⨁\\
- Density (ρ): 0.85--1.25 ⨁\\
- Surface Gravity (g): 0.85--1.70 ⨁\\
- Radius (r): 0.90--1.50 ⨁\\
- Escape Velocity (vₑ): 0.95--1.50 ⨁\\
- \textbf{Overlap with Gaeans:} ≈13.9\% of Rheatics fall within Gaean
gravity bounds (0.9--1.1 ⨁), making them unusually favorable worlds with
both Earthlike conditions and enhanced biospheric potential.\\
- \textbf{Core Feature:} Larger, denser terrestrial bodies → broader
climatic bands, longer tectonic--volcanic cycling, stronger
magnetospheres, and more biospheric resilience (``superhabitable''
qualities).\\
- \textbf{Implications for settlement:} Human colonization possible but
requires adaptation: medical mitigation for high gravity, structural
reinforcement, climate control.\\
- \textbf{Distinction:}\\
- All Rheatics are Geotics, but skew toward higher mass/density.\\
- Only a small subset are Gaean.\\
- Serves as the WCB counterpart to \textbf{``superhabitable''
exoplanets} in the scientific literature.

\textbf{Key Terms \& Symbols:}\\
- \textbf{Rheatic} --- Parahabitable terrestrial super-Earth,
biosphere-rich but human-straining.\\
- \textbf{m, ρ, g, r, vₑ} --- Fundamental planemon parameters
(Earth-relative).\\
- \textbf{Superhabitable Zone} --- WCB framing of biosphere-enhancing
corridors.\\
- \textbf{Overlap Fraction:} ≈13.9\% of Rheatics are also Gaean.

\textbf{Cross-Check Notes:}\\
- Consolidates external ``superhabitable'' terminology under the WCB
neolex \textbf{Rheatic}.\\
- Provides numeric corridors to distinguish Rheatics from Geotics and
Gaeans.\\
- Complements \textbf{Justifying Parahabitable Parameter Limits} by
filling in the higher-mass end of the spectrum.

\chapter{Rheatic planemons}\label{rheatic-planemons}

\section{Rheatic}\label{rheatic}

\textbf{Rheatic} := ⟨ m ∧ ρ ∧ g ∧ r ∧ vₑ ⟩ m := ⟨1.00 ∧ 3.00⟩⨁ ρ :=
⟨0.85 ∧ 1.25⟩⨁ g := ⟨0.85 ∧ 1.70⟩⨁ r := ⟨0.90 ∧ 1.50⟩⨁ vₑ := ⟨0.95 ∧
1.50⟩⨁
\[\text{RHEATIC} := \left\{ (m, \rho) \in \mathbb{R}^2 \ \middle|\
\begin{aligned}
&1.00 \leq m \leq 3.00 \\
&0.85 \leq \rho \leq 1.25 \\
&0.85 \leq g(m, \rho) \leq 1.70 \\
&0.90 \leq r(m, \rho) \leq 1.50 \\
&0.95 \leq v_e(m, \rho) \leq 1.50
\end{aligned}
\right\}
\] \textgreater{} \emph{Rheatics} are \textbf{parahabitable} planemons
--- terrestrial super-Earths with conditions \textbf{favorable to rich
biospheres} but likely \textbf{inhospitable to unmodified humans}. They
may possess higher surface gravity, thicker atmospheres, and more
energetic climates, often demanding mechanical, biological, or
infrastructural adaptations for long-term Earthling presence.
Nonetheless, they are considered highly conducive to complex, robust
life --- just not necessarily Earthlike.

\textbf{Overlap with Gaeans}: - A \textbf{small subset} of Rheatics ---
\textbf{≈13.9\%} --- fall within the Gaean gravity range (0.9 ≤ g ≤
1.1⨁).\\
- These rare worlds are \textbf{massive and dense} enough to support
Earth-normal surface conditions \textbf{while offering enhanced
biospheric potential} --- possibly the best of both worlds. \textbf{Core
Feature}: - The \textbf{``superhabitable'' zone}: larger size means
broader climatic bands, more plate tectonics, greater magnetic
shielding, and longer tectonic--volcanic cycling --- all of which may
favor biospheric richness and diversity.\\
- \textbf{Human settlement} is plausible but typically \textbf{requires
support}: enhanced structural design, medical mitigation of gravity
effects, and climate regulation systems. \textbf{Distinction from
Geotics}: - All Rheatics meet \textbf{Geotic} compositional constraints,
but their \textbf{mass and gravity trends upward}.\\
- \textbf{Not all Geotics} are Rheatic: Rheatics are a \textbf{subset of
high-mass, dense, habitable} planemons.\\
- Conversely, \textbf{not all Rheatics are Gaean} --- only a small slice
of them match that precise Earthlike window. \#\# Abstract\\
\textbf{Major Topics:}\\
- Defines \textbf{Telluric planemons} as solid- or semi-solid-surfaced
worlds across a wide mass range, regardless of habitability.\\
- Parameter ranges (relative to Earth units ⨁):\\
- \textbf{Mass (m):} ⟨0.01 ∧ 10.00⟩\\
- \textbf{Density (ρ):} ⟨0.50 ∧ 7.00⟩\\
- \textbf{Surface gravity (g):} ⟨0.15 ∧ 8.00⟩\\
- \textbf{Radius (r):} ⟨0.15 ∧ 3.00⟩\\
- \textbf{Escape velocity (vₑ):} ⟨0.20 ∧ 3.00⟩\\
- Encompasses rocky, metallic, and icy planemons: Earthlike planets,
super-Earths, sub-Earths, Mars, Titan, Ganymede, Io, Kepler-20b, etc.\\
- \textbf{Core role in canon:}\\
- All \textbf{Geotic}, \textbf{Gaean}, and \textbf{Rheatic} planemons
are subsets of Tellurics.\\
- \textasciitilde4.8\% of Tellurics are Geotics; \textasciitilde0.55\%
are Gaeans; \textasciitilde3.6\% are Rheatics.\\
- Many Tellurics are \textbf{parahabitable}: survivable only with life
support, domes, or partial terraforming.\\
- Symbolic note: term derives from \emph{Tellus} (Earth-mother), but
used \textbf{structurally, not biologically}.

\textbf{Relations to Other Types:}\\
- Overlaps with \textbf{Xenotic} planemons in rocky mass range.\\
- Differentiation: \textbf{Telluric = structural (rocky/icy/metallic)}
vs.~\textbf{Xenotic = exotic composition}.

\textbf{Key Terms \& Symbols:}\\
- \textbf{Telluric {[}NEW{]}.}\\
- \textbf{Geotic {[}neo{]}, Gaean {[}neo{]}, Rheatic {[}neo{]}, Xenotic
{[}neo{]}, Parahabitable {[}neo{]}} (already canonical, here
reinforced).

\textbf{Cross-Check Notes:}\\
- Introduces \textbf{Telluric} as a new umbrella category.\\
- Integrates existing WCB planemon types (Geotic, Gaean, Rheatic,
Xenotic, Parahabitable) under this umbrella.\\
- \textbf{Status:} {[}NEW + EXPANDED{]} --- Telluric is newly defined;
expands canon by clarifying relationships among existing planemon
classes.

\chapter{Telluric planemons}\label{telluric-planemons}

\section{Telluric}\label{telluric}

\textbf{Telluric} := ⟨ m ∧ ρ ∧ g ∧ r ∧ vₑ ⟩\\
m := ⟨0.01 ∧ 10.00⟩⨁\\
ρ := ⟨0.50 ∧ 7.00⟩⨁\\
g := ⟨0.15 ∧ 8.00⟩⨁ r := ⟨0.15 ∧ 3.00⟩⨁ vₑ := ⟨0.20 ∧ 3.00⟩⨁

\begin{quote}
Tellurics are parahabitable worlds with solid or semi-solid surfaces ---
encompassing the full class of rocky, metallic, and icy planemons.\\
This category includes Earthlike worlds, massive rocky exoplanets,
marginal sub-Earths, and bodies like Mars, Ganymede, Titan, or large
moons of gas giants.\\
It defines the geophysical domain of terrestrial planemons --- whether
habitable or not --- and serves as the primary envelope from which
Geotic, Gaean, and Rheatic worlds are derived.
\end{quote}

\textbf{Core Feature}\\
- This is a broad categorization --- about \textbf{4.8\%} of Tellurics
are Geotics, and only about \textbf{0.55\%} of all Tellurics are Gaeans
--- and \textbf{3.6\%} of Tellurics are Rheatics.\\
- These worlds possess defined solid surfaces or lithospheres, with no
requirement for biological habitability.\\
- Many are parahabitable --- survivable with life-support systems,
domes, or partial terraforming.\\
- May include frozen dwarfs, massive dry worlds, or oceanic with no dry
land. \textbf{Relations to Other Types}\\
- Contains all Geotic, Gaean, and Rheatic worlds.\\
- Overlaps with Xenotic worlds in the rocky mass range.\\
- Worlds like Mars, Titan, Io, and Kepler-20b are all Tellurics, despite
wildly different surface conditions. \textbf{Symbolic Use}\\
- The term draws from \emph{Tellus}, the Latin Earth-mother, but in this
context is \textbf{geostructural}, not biological.\\
- When contrasted with \emph{Xenotic}, the distinction is about
\textbf{structure} (rocky vs.~exotic or gaseous), not life-hosting
potential. \#\# Abstract\\
\textbf{Major Topics:}\\
- Defines \textbf{Xenotic planemons} as worlds whose potential
biospheres are \textbf{non-Earthlike}, supporting alien chemistries or
life systems (non-carbonic, non-water-based, etc.).\\
- Parameter envelope (relative to Earth units ⨁):\\
- Mass: ⟨0.0001 ∧ 4131⟩\\
- Density: ⟨0.01 ∧ 7.00⟩\\
- Gravity: ⟨0.02 ∧ 60.00⟩\\
- Radius: ⟨0.02 ∧ 11.00⟩\\
- Escape velocity: ⟨0.02 ∧ 25.00⟩\\
- Emphasis: Xenotic classification is \textbf{not structural} (unlike
Telluric, Geotic, etc.), but \textbf{biological}, about what kinds of
life might emerge.\\
- Inclusions: ammonia/methane-based biospheres, silicon-based or
plasma-phase life, deep high-pressure gas giant biota, crystalline
metabolic substrates, etc.\\
- Exclusions: simply being Geotic or Gaean does not make a world
Xenotic. A planet may fall within Gaean/Geotic physical parameters and
still be Xenotic if its biosphere is alien.\\
- Symbolic origin: from Greek \emph{xenos} (ξένος), ``stranger,''
``outsider.''

\textbf{Key Terms \& Symbols:}\\
- \textbf{Xenotic {[}NEW{]}.}\\
- \textbf{Gaean, Geotic, Telluric {[}neo{]}:} overlapping categories,
but distinct.

\textbf{Cross-Check Notes:}\\
- \textbf{Xenotic} does not appear in earlier canon abstracts --- this
file introduces it formally.\\
- Distinction is clear: \textbf{Telluric = structure}, \textbf{Gaean =
Earthlike habitability}, \textbf{Xenotic = alien biosphere potential}.\\
- \textbf{Status:} {[}NEW{]} --- introduces Xenotic as a new category of
planemons defined by biotic potential outside Earth norms.

\chapter{Xenotic planemons}\label{xenotic-planemons}

\section{Xenotic}\label{xenotic}

\textbf{Xenotic} := ⟨ m ∧ ρ ∧ g ∧ r ∧ vₑ ⟩ m := ⟨0.0001 ∧ 4131⟩⨁ ρ :=
⟨0.01 ∧ 7.00⟩⨁ g := ⟨0.02 ∧ 60.00⟩⨁ r := ⟨0.02 ∧ 11.00⟩⨁ vₑ := ⟨0.02 ∧
25.00⟩⨁
\[\text{XENOTIC} := \left\{ (m, \rho) \in \mathbb{R}^2 \ \middle|\
\begin{aligned}
&0.0001 \leq m \leq 4131 \\
&0.01 \leq \rho \leq 7.00 \\
&0.002 \leq g(m, \rho) \leq 60.00 \\
&0.02 \leq r(m, \rho) \leq 11.00 \\
&0.02 \leq v_e(m, \rho) \leq 25.00
\end{aligned}
\right\}
\] \textgreater{} \emph{Xenotics} are planemons whose environmental
conditions may support \textbf{non-Earthlike life}, including
\textbf{non-carbonic}, \textbf{non-water-based}, or otherwise exotic
biochemistries. The term is not tied to physical parameters, but to the
\textbf{biological strangeness} of the world's potential life-hosting
capacity.

\textbf{Core Feature}: - Xenotic classification \textbf{is not about
what the world \emph{is}} --- it's about \textbf{what kind of life it
might support}.\\
- A Xenotic world might be a rocky, icy, or gaseous body --- but its
\textbf{biotic potential lies outside} the realm of Earth-normal life. -
This is an \emph{extremely} broad classification: only 0.35\% of
planemons sharing Xenotic mass and density ranges qualify as
\emph{Tellurics}. Gaeans share mass and density range with only 0.001\%
of Xenotics. \textbf{Key Principle}: \textgreater{} A world may fall
entirely within Gaean or Geotic \textbf{parameters} and still be
\textbf{Xenotic in character} --- if its biosphere is chemically or
structurally \textbf{alien to terrestrial assumptions}.

\textbf{Inclusions}: - \textbf{Ammonia-based} or \textbf{methanogenic}
biospheres (e.g., Titan-like)\\
- \textbf{Silicon-based} or \textbf{plasma phase} consciousness
(hypothetical)\\
- \textbf{High-pressure deep-atmosphere lifeforms} on gas giants\\
- \textbf{Ultra-dense crust-worlds} with lattice-bonded metabolic
substrates\\
- \textbf{Life emerging in conditions unreplicable on Earth}
\textbf{Exclusions}: - Gaean or Geotic worlds are \textbf{not Xenotic}
simply by shape or size.\\
- Xenotic worlds \textbf{may physically overlap} with all other
categories --- but their \textbf{life potential diverges completely}.
\textbf{Symbolic Use}: - From Greek \emph{xenos} (ξένος): ``stranger,''
``foreigner,'' ``outsider.''\\
- Xenotic worlds are those where \textbf{life is not just different ---
it is alien}. \textbf{Parameter Notes}: - \textbf{Mass (⨁):} from
sublunar pebbles to brown dwarf threshold.\\
- \textbf{Density (⨁):} from hydrogen-ice slushes to ultra dense
crystal-metallic cores.\\
- \textbf{Gravity (⨁):} \textasciitilde0.02⨁ (Mars-like) up to
\textasciitilde60⨁ (felt at inner gas dwarf surfaces). - Spans
everything from fragile ultralow-gravity cometary clumps to
neutronium-crusted compact objects just short of degeneracy collapse. -
This definition also accommodates highly stratified gas layers
(e.g.~floatable biospheres in Saturnian-class or puffy hot-Neptune
exotics). - Any values beyond this envelope cross into \textbf{ulsic} or
\textbf{hypotheticals}: black holes, quark matter, etc. - \textbf{Radius
(⨁):} up to 11⨁ to accommodate inflation-limited gas giants and
Super-Jupiters. - Frequently exceeded by puffy planemons due to close
proximity to their stars inflating their atmospheres. - \textbf{Escape
Velocity (⨁):} capped at 25⨁ ≈ 280 km/s, brushing the domain of
hot-start brown dwarfs.

\begin{quote}
These are \textbf{not bound by Earth-normal biology}. They simply
represent physically plausible, self-cohering planemon-scale entities
where exotic life --- as chemistry permits --- might arise. \#\#
Abstract\\
\textbf{Major Topics:}\\
- Summarizes the external concept of \textbf{``superhabitable
planemons''} (Heller \& Armstrong, 2014), proposed as worlds more
conducive to diverse biospheres than Earth.\\
- Criteria include:\\
- \textbf{Stellar hosts:} spectral classes M0--G9, masses ⟨0.359 ∧
0.817⨀⟩, lifetimes ≥ 3 Ga.\\
- \textbf{planemon properties:}\\
- Mass: ⟨2.0 ∧ 3.0⨁⟩ (optimum ≈ 2.0⨁).\\
- Radius: ⟨1.260 ∧ 1.442⨁⟩ (Earthlike density/gravitation).\\
- Strong tectonics, carbon--silicate cycling, thicker atmosphere,
stronger magnetic shielding.\\
- Flatter surface, shallower oceans with \textasciitilde71\% global
coverage.\\
- Mean temperature ≈ 25 °C; O₂ concentration \textgreater{} 20.95\%.\\
- Orbits near the center of the host star's habitable zone.\\
- Scientific rationale: slightly more massive than Earth → longer
tectonic/geological cycling, better climate stabilization, stronger
magnetic protection, smoother surface conditions.\\
- Supported by Noack \& Breuer (2011) on tectonic propensity in 1--5 M⊕
range.
\end{quote}

\textbf{Relation to WCB Canon:}\\
- WCB does \textbf{not} use the term ``superhabitable.''\\
- Instead, these criteria are encompassed within \textbf{Rheatic
planemons}, which refine the same concept for WCB internal
classifications.\\
- External term is acknowledged for context but not adopted into
lexicon.

\textbf{Key Terms \& Symbols:}\\
- \textbf{Superhabitable {[}exo{]}:} external concept (Heller \&
Armstrong, 2014).\\
- \textbf{Rheatic planemons {[}neo{]}:} WCB equivalent classification.\\
- \textbf{Carbon--Silicate Cycle {[}sci{]}.}

\textbf{Cross-Check Notes:}\\
- Rheatic planemons are already part of WCB classification.\\
- ``Superhabitable'' remains external-only, acknowledged but not
canonized.\\
- \textbf{Status:} {[}EXPANDED + EXO{]} --- expands habitability
framework with external criteria; adopts internal Rheatic equivalent.

\chapter{Superhabitable planemons}\label{superhabitable-planemons}

\textbf{Definition}\\
The concept of \emph{superhabitable planemons} was proposed by René
Heller and John Armstrong (2014) to describe worlds with physical,
orbital, and stellar characteristics that make them \textbf{more
conducive to rich, diverse biospheres} than Earth --- though not
necessarily more suitable for humanoid or Earthlike life.

\textbf{Stellar Criteria}\\
- Host star mass range: ⟨0.359 ∧ 0.817⟩⊙ (spectral classes M0 -- G9) -
{[}{[}Spectral Class Base Table ✓{]}{]} - Host star lifespan: ⟨1.656 ∧
12.934⟩⊙ years\\
- Stellar age: Older than the Sun's 4.5 Ga but younger than 7 Ga\\
- {[}{[}Stellar Lifetimes and System Habitability ✓{]}{]}

\textbf{Planemon Characteristics}\\
- \textbf{Mass}: ⟨2.0 ∧ 3.0⟩⨁ (optimum ≈ 2.0⨁)\\
- \textbf{Radius}: ⟨1.260 ∧ 1.442⟩⨁ (maintains Earthlike density and
gravity)\\
- \textbf{Geology}: Larger tectonic--volcanic cycling, longer plate
tectonic activity, strong carbon--silicate cycle, thicker atmosphere
retention.\\
- \textbf{Magnetic Field}: Stronger due to larger liquid core and higher
internal heat.\\
- \textbf{Surface Geography}: Flatter topography, shallower oceans,
widely distributed ocean coverage (\textasciitilde71\% surface area)
without large continuous land masses.\\
- \textbf{Temperature}: Mean ≈ 25 °C (77 °F).\\
- \textbf{Atmosphere}: Thicker than Earth's; O₂ concentration
\textgreater{} 20.95\%.\\
- \textbf{Orbit}: Closer to the center of the host system's habitable
zone than Earth is in the Solar System.\\
- Solar System example: center of optimistic HZ ≈ 1.26 AU; center of
conservative HZ ≈ 1.16 AU.

\textbf{Scientific Rationale}\\
From Heller \& Armstrong (2014):\\
\textgreater{} ``Terrestrial planemons that are slightly more massive
than Earth, up to 2 or 3 M⊕, are preferably superhabitable due to the
longer tectonic activity, a carbon--silicate cycle active on a longer
timescale, enhanced magnetic shielding, larger surface area, smoother
surface, thicker atmosphere, and the positive effects of non-intelligent
life on habitability.''

Lena Noack \& Doris Breuer (2011) add:\\
\textgreater{} ``\ldots the propensity of plate tectonics seems to have
a peak between 1 and 5 Earth masses\ldots{} a geological perspective
suggests the optimal mass of a planemon is about 2 M⊕.''

\textbf{Relation to WCB Classifications}\\
In \emph{Worlds by the Numbers}, \textbf{Rheatic planemons} encompass
the superhabitable concept, with parameter refinements for internal
consistency within WCB's classification system.\\
- See: \textbf{{[}{[}Rheatic planemons ✓{]}{]}} for in-universe
parameter ranges and classification notes.

\textbf{See Also} - {[}{[}Justifying Parahabitable Parameter Limits
✓{]}{]} - {[}{[}Extended Geotic Habitability Guidelines ✓{]}{]}

\textbf{References}\\
- Heller, René, and John Armstrong. ``Superhabitable Worlds.''
\emph{PubMed} (National Library of Medicine, December 31, 2013).
\url{https://pubmed.ncbi.nlm.nih.gov/24380533/}\\
- Noack, Lena, and Doris Breuer. ``Plate Tectonics on Earth-Like
Planets.'' \emph{ResearchGate}, January 2011.
\url{https://www.researchgate.net/publication/225001335_Plate_Tectonics_on_Earth-like_Planets_Implications_for_the_Habitability}
\#\# Abstract\\
\textbf{Major Topics:}\\
- Defines \textbf{micromon} as a distinct category of \textbf{Small
Stellar System Bodies (SSSBs)}, separate from planemons (planetary-mass
objects) and stellamons (stellar-mass objects).\\
- Establishes a \textbf{compositional/conformational taxonomy} with
three main classes:\\
- \textbf{Telluroids}: rocky/metallic micromons (e.g., S-type, C-type,
Vestoids, M-type).\\
- \textbf{Astatoids}: volatile-rich micromons, subdivided into:\\
- Pagooid (icy)\\
- Fluxoid (liquid-dominated, hypothetical)\\
- Ceroid (ice + subsurface ocean)\\
- \textbf{Ulsoids}: exotic or cryptic micromons, subdivided into:\\
- Exotoid (exotic matter)\\
- Cryptoid (mysterious/unclassified, e.g.~'Oumuamua).\\
- Establishes \textbf{upper bounds} for micromon classification: ≤ 250
µT (0.00025 T) or ≤ 600 km radius.\\
- Distinguishes \textbf{meteoroids} as a size-only category (\textless1
m), separate from micromon conformations.\\
- Draws parallels with familiar Solar System bodies (asteroids, comets,
icy moons) while introducing new subtypes for speculative/exotic cases.

\textbf{Key Terms \& Symbols:}\\
- \textbf{micromon}: umbrella category for small stellar system
bodies.\\
- \textbf{Telluroid, Astatoid, Ulsoid}: principal micromon
conformations.\\
- \textbf{Pagooid, Fluxoid, Ceroid, Exotoid, Cryptoid}: subtype
classifications.\\
- \textbf{µT (Teras Mass Unit)}: threshold measure for micromon mass (≤
250 µT).

\textbf{Cross-Check Notes:}\\
- Glossary entries for micromon and Telluroid were \textbf{newly added
in v1.222}.\\
- Glossary entries for Astatoid, Ceroid, Fluxoid, Ulsoid, Exotoid,
Cryptoid were \textbf{updated in v1.222} to replace placeholder
structural definitions with these refined compositional/conformational
definitions.\\
- Reinforces WCB's hierarchical taxonomy: \textbf{stellamon → planemon →
micromon → Conformations}.\\
- Aligns terminology with the broader WCB classification framework and
avoids conflict with deprecated placeholder definitions.

\subsection{\texorpdfstring{\textbf{✅ 1. micromon is the umbrella term
for all small stellar system bodies
(SSSBs).}}{✅ 1. micromon is the umbrella term for all small stellar system bodies (SSSBs).}}\label{micromon-is-the-umbrella-term-for-all-small-stellar-system-bodies-sssbs.}

✔ Distinct from \textbf{planemon (planetary-mass objects)} and
\textbf{stellamon (stellar-mass objects).}\\
✔ Replaces vague vernacular terms like ``SSSB'' with a \textbf{clear
mass-based category.}

\subsection{\texorpdfstring{\textbf{✅ 2. Two Major Conformations:
Telluroids \&
Astatoids}}{✅ 2. Two Major Conformations: Telluroids \& Astatoids}}\label{two-major-conformations-telluroids-astatoids}

✔ \textbf{Telluroids (rocky/metallic micromons)}

\begin{itemize}
\tightlist
\item
  \textbf{S-Type Telluroid} (Stony, silicate-rich) → Eros, Vesta
\item
  \textbf{C-Type Telluroid} (Carbonaceous, hydrated) → Pallas, Mathilde
\item
  \textbf{Vestoid Telluroid} (Pyroxene-rich, volcanically processed) →
  Vesta, Vestoids
\item
  \textbf{M-Type Telluroid} (Primarily metallic) → Psyche
\end{itemize}

✔ \textbf{Astatoids (icy/volatile-rich micromons)}

\begin{itemize}
\tightlist
\item
  \textbf{Pagooid Astatoid} (Frozen volatiles) → 67P, Halley
\item
  \textbf{Fluxoid Astatoid} (Liquid-dominated) → Hypothetical
\item
  \textbf{Ceroid Astatoid} (Subsurface ocean) → Ceres, Callisto
\end{itemize}

\subsection{\texorpdfstring{\textbf{✅ 3. Ulsoid micromons (For
Exotic/Mysterious
Objects)}}{✅ 3. Ulsoid micromons (For Exotic/Mysterious Objects)}}\label{ulsoid-micromons-for-exoticmysterious-objects}

✔ \textbf{Exotoid Ulsoid} → Hypothetical exotic matter objects (dark
matter, strangelets).\\
✔ \textbf{Cryptoid Ulsoid} → Unusual, unidentified micromons (e.g.,
'Oumuamua).

\subsection{\texorpdfstring{\textbf{✅ 4. Mass \& Radius Boundaries
Defined}}{✅ 4. Mass \& Radius Boundaries Defined}}\label{mass-radius-boundaries-defined}

✔ \textbf{Maximum micromon Mass:} \textbf{≤250 µT (0.00025 T)}

\begin{itemize}
\tightlist
\item
  \textbf{Based on the smallest planetary-mass object (Charon).} ✔
  \textbf{Maximum micromon Radius:} \textbf{≤600 km}
\item
  \textbf{Tim deBenedictis' threshold for planetary-class objects.} ✔
  \textbf{Everything smaller than this remains a micromon.} ✔
  \textbf{Meteoroids (sub-meter objects) remain a size-based category,
  not a conformation.}
\end{itemize}

\chapter{\texorpdfstring{\textbf{micromon Classification
System}}{micromon Classification System}}\label{micromon-classification-system}

micromon (Small Stellar System Bodies) are distinct from planemons
(planetary-mass objects) and stellamons (stellar-mass objects). This
classification provides clear compositional and structural categories
for micromon-scale bodies in stellar systems.

\subsection{\texorpdfstring{\textbf{1. micromon
Conformations}}{1. micromon Conformations}}\label{micromon-conformations}

\begin{longtable}[]{@{}
  >{\raggedright\arraybackslash}p{(\linewidth - 8\tabcolsep) * \real{0.2000}}
  >{\raggedright\arraybackslash}p{(\linewidth - 8\tabcolsep) * \real{0.2000}}
  >{\raggedright\arraybackslash}p{(\linewidth - 8\tabcolsep) * \real{0.2000}}
  >{\raggedright\arraybackslash}p{(\linewidth - 8\tabcolsep) * \real{0.2000}}
  >{\raggedright\arraybackslash}p{(\linewidth - 8\tabcolsep) * \real{0.2000}}@{}}
\toprule\noalign{}
\begin{minipage}[b]{\linewidth}\raggedright
\textbf{micromon Category}
\end{minipage} & \begin{minipage}[b]{\linewidth}\raggedright
\textbf{Subtype}
\end{minipage} & \begin{minipage}[b]{\linewidth}\raggedright
\textbf{Composition}
\end{minipage} & \begin{minipage}[b]{\linewidth}\raggedright
\textbf{Examples}
\end{minipage} & \begin{minipage}[b]{\linewidth}\raggedright
\textbf{Notes}
\end{minipage} \\
\midrule\noalign{}
\endhead
\bottomrule\noalign{}
\endlastfoot
\textbf{Telluroid micromons} & \textbf{S-Type Telluroid} &
Silicate-rich, rocky & Eros, Vesta & High-albedo, often stony \\
& \textbf{C-Type Telluroid} & Carbonaceous, hydrated minerals & Pallas,
Mathilde & Dark, ancient material \\
& \textbf{Vestoid Telluroid} & Pyroxene-rich, volcanically processed &
Vesta, Vestoids & Likely fragments of differentiated worlds \\
& \textbf{M-Type Telluroid} & Primarily metallic & Psyche & Possible
planetesimal cores \\
\textbf{Astatoid micromons} & \textbf{Pagooid Astatoid} & Icy-rich,
frozen volatiles & 67P, Halley & Standard fully frozen micromons \\
& \textbf{Fluxoid Astatoid} & Liquid-dominated microbodies & (???) &
Hypothetical micromons with surface liquids \\
& \textbf{Ceroid Astatoid} & Ice + subsurface ocean & Ceres, Callisto &
Ice-dominated, but with internal liquid layers \\
\textbf{Ulsoid micromons} & \textbf{Exotoid Ulsoid} & Hypothetical
exotic matter objects & (???) & Dark matter, strangelets, other
exotics \\
& \textbf{Cryptoid Ulsoid} & Unknown, cryptic composition & 'Oumuamua &
Unusual, unidentified micromons \\
\end{longtable}

\subsection{\texorpdfstring{\textbf{2. Mass \& Radius
Boundaries}}{2. Mass \& Radius Boundaries}}\label{mass-radius-boundaries}

\begin{longtable}[]{@{}
  >{\raggedright\arraybackslash}p{(\linewidth - 6\tabcolsep) * \real{0.2500}}
  >{\raggedright\arraybackslash}p{(\linewidth - 6\tabcolsep) * \real{0.2500}}
  >{\raggedright\arraybackslash}p{(\linewidth - 6\tabcolsep) * \real{0.2500}}
  >{\raggedright\arraybackslash}p{(\linewidth - 6\tabcolsep) * \real{0.2500}}@{}}
\toprule\noalign{}
\endhead
\bottomrule\noalign{}
\endlastfoot
\textbf{micromon Mass Range} & \textbf{Max Mass (µT)} & \textbf{Max Mass
(T)} & \textbf{Max Radius (km)} \\
\textbf{Maximum micromon} & 250 µT & 0.00025 T & 600 km \\
\textbf{Minimum micromon} & Meteoroid Scale & \textless0.0000001 T &
\textless1 km \\
\end{longtable}

\begin{itemize}
\tightlist
\item
  \textbf{Maximum micromon Mass:} \textbf{≤250 µT (0.00025 T)}, based on
  the smallest planetary-mass object (Charon).
\item
  \textbf{Maximum micromon Radius:} \textbf{≤600 km}, following Tim
  deBenedictis' planetary-class threshold.
\item
  \textbf{Everything smaller than this remains a micromon.}
\item
  \textbf{Meteoroids (sub-meter objects) remain a size-based category,
  not a conformation.}
\end{itemize}

\subsection{\texorpdfstring{\textbf{3. Structural
Considerations}}{3. Structural Considerations}}\label{structural-considerations}

\begin{itemize}
\tightlist
\item
  \textbf{Telluroids} are primarily rocky/metallic and correspond to
  asteroid-like bodies.
\item
  \textbf{Astatoids} are volatile-rich, including comets, icy moons, and
  Kuiper Belt Objects.
\item
  \textbf{Ulsoids} are exotic or cryptic bodies, encompassing extreme
  outliers in composition or origin.
\item
  \textbf{Meteoroids} are defined \textbf{by size}, rather than
  conformation, making them distinct from other micromons.
\end{itemize}

\subsection{\texorpdfstring{\textbf{4. Final
Notes}}{4. Final Notes}}\label{final-notes}

This classification system ensures that \textbf{micromons are clearly
distinguished from planemons}, while allowing for \textbf{refinement
based on future discoveries.} It also maintains \textbf{internal
consistency with WCB's existing taxonomic structure.}

\chapter{Abstract}\label{abstract-20}

\textbf{Major Topics:}\\
- Exploration of what defines an ``Earth-like'' planemon.\\
- Reference baseline: Earth's absolute physical parameters (mass,
radius, density, gravity, escape velocity).\\
- Lack of consensus in astrophysics: ranges vs.~environmental
conditions.\\
- Distinction between \textbf{physical similarity} and
\textbf{habitability classifications} (habitable, parahabitable,
hospitable, xenotic, exotic).

\textbf{Key Terms \& Symbols:}\\
- m = mass (Earth: 5.972 × 10²⁴ kg).\\
- r = radius (Earth: 6371 km).\\
- ρ = density (Earth: 5.514 g/cm³).\\
- g = surface gravity (Earth: 9.8 m/s²).\\
- vₑ = escape velocity (Earth: 11.186 km/s).\\
- ``Earth-like'' = physical conformity, not animosustent criteria.

\textbf{Cross-Check Notes:}\\
- Reinforces WCB convention: lowercase symbols (m, r) for planemon
parameters.\\
- Distinguish clearly between \textbf{Earth-like (physical)} and
\textbf{animocentric classifications} (habitable, parahabitable,
hospitable, xenotic, exotic).\\
- Connects to glossary entries: animosustent (canon v1.21), planemon
parameters.\\
- Prevent conflation of ``Earth-like'' with life-supporting qualities.

\section{Earth-like planemons: What Does That Even
Mean?}\label{earth-like-planemons-what-does-that-even-mean}

Let's revisit the \emph{physical properties} of planemons:
!{[}{[}Physical Properties of Planets ✓{]}{]}

We have established that Earth is our standard, and that its
\emph{relative} mass is 1.0M⨁, its \emph{relative} radius is 1.0R⨁, etc.
But here are its \emph{absolute} parameters

\textbf{Mass}: 5.972 × 10²⁴ kg \textbf{Radius}: 6371 km
\textbf{Density}: 5.514 g/cm³ \textbf{Surface Gravity}: 9.8 m/sec²
\textbf{Escape Velocity}: 11.186 km/sec

Sadly, but perhaps predictably, there is no generally agreed-upon
definition of what makes a planemon ``Earth-like'', beyond either a
loosely defined range of physical parameters or a rather more rigid set
of environmental conditions.

https://science.nasa.gov/exoplanets/exoplanet-catalog/

\section{Abstract}\label{abstract-21}

\textbf{Major Topics:}\\
- Demonstrates why Venus, despite meeting \textbf{Geotic Envelope}
physical criteria, fails \textbf{Gaean classification}.\\
- Venus as \textbf{Geotic}:\\
- Mass (0.815 ⨁), radius (0.949 ⨁), density (0.95 ⨁), gravity (0.90 ⨁),
escape velocity (0.93 ⨁).\\
- All values fall within Geotic bounds, making Venus structurally
Earthlike.\\
- Venus not \textbf{Gaean}:\\
- Atmosphere: 92 atm CO₂ with sulfuric acid aerosols.\\
- Surface temperature: \textasciitilde735 K (462 °C), runaway greenhouse
effect.\\
- Magnetosphere: lacks intrinsic magnetic field.\\
- Hydrosphere: no stable surface water.\\
- Concludes that physical plausibility (Geotic) must be separated from
habitability (Gaean).\\
- Illustrates WCB taxonomy: \textbf{not all Geotics are Gaeans}.

\textbf{Key Terms \& Symbols:}\\
- \textbf{Geotic {[}sci{]}.}\\
- \textbf{Gaean {[}sci{]}.}

\textbf{Cross-Check Notes:}\\
- Both Geotic and Gaean are already canonical.\\
- This file provides a case study reinforcing the distinction.\\
- \textbf{Status:} {[}EXPANDED{]} --- applies existing canon to Venus as
a worked example.

\chapter{Why Venus Isn't Gaean}\label{why-venus-isnt-gaean}

Although Venus falls squarely within the \textbf{Geotic Envelope} (its
mass, radius, density, surface gravity, and escape velocity are all
near-Terran), it fails the stricter criteria for \textbf{Gaean
classification}.

The extreme atmospheric composition, runaway greenhouse effect, lack of
a protective magnetosphere, and absence of stable hydrospheric cycles
render it inhospitable to unmodified humans and hostile to Terran-style
biospheres.

\section{Venus as Geotic}\label{venus-as-geotic}

By the numbers, Venus is a \textbf{Geotic planemon}:

\begin{itemize}
\tightlist
\item
  \textbf{Mass (m):} 0.815 ⨁\\
\item
  \textbf{Radius (r):} 0.949 ⨁\\
\item
  \textbf{Density (ρ):} 0.95 ⨁\\
\item
  \textbf{Surface Gravity (g):} 0.90 ⨁\\
\item
  \textbf{Escape Velocity (vₑ):} 0.93 ⨁
\end{itemize}

These all fall neatly within the \textbf{Geotic Envelope} (0.30--3.35 m,
0.85--1.25 ρ, 0.60--1.65 g, 0.60--1.50 r, 0.65--1.50 vₑ).\\
By physical structure alone, Venus is almost Earth's twin.

\section{Why Venus Fails Gaean
Classification}\label{why-venus-fails-gaean-classification}

Despite its Geotic parameters, Venus cannot be called \textbf{Gaean}:

\begin{itemize}
\tightlist
\item
  \textbf{Atmosphere:} 92 atm of CO₂, with sulfuric acid aerosols.\\
\item
  \textbf{Surface Temperature:} \textasciitilde735 K (462 °C), caused by
  a runaway greenhouse effect.\\
\item
  \textbf{Magnetosphere:} Lacks a protective intrinsic magnetic field,
  leaving the atmosphere vulnerable to solar stripping.\\
\item
  \textbf{Hydrosphere:} No stable water; surface is desiccated.
\end{itemize}

These factors eliminate the possibility of unaided human survival and
preclude a Terran-like biosphere.

\section{Conclusion}\label{conclusion}

Venus is \textbf{Geotic} in terms of its physical parameters, but it is
not \textbf{Gaean} in terms of habitability.\\
This distinction illustrates why the WCB taxonomy separates
\textbf{Geotic envelopes} (physical plausibility) from \textbf{Gaean
classification} (true Terran habitability). \#\# Abstract\\
\textbf{Major Topics:}\\
- Extended parameters for defining geotic (human-hospitable)
conditions.\\
- Habitability ranges for rotation period (D), orbital eccentricity (e),
orbital period (C), axial tilt (εₓ), precession cycle (χ), and obliquity
azimuth (ζₙ).\\
- Magnetosphere strength (Bsurf) as radiation shielding criterion.\\
- Atmospheric baseline conditions: pressure, scale height, composition,
ozone presence.\\
- Surface balance of land and water.\\
- Geotic gravity corridor (0.5--1.5 ⨁) as strict human-hospitable bound.

\textbf{Key Terms \& Symbols:}\\
- \textbf{D} --- Rotational period (diurn length).\\
- \textbf{e} --- Orbital eccentricity.\\
- \textbf{C} --- Orbital period (sidereal chronum).\\
- \textbf{εₓ} --- Axial tilt (obliquity).\\
- \textbf{χ} --- Axial precession cycle.\\
- \textbf{ζₙ} --- Obliquity azimuth relative to periapsis.\\
- \textbf{Bsurf} --- Surface magnetic field strength (μT).\\
- \textbf{Tₛ} --- Average surface temperature (K).\\
- \textbf{H} --- Atmospheric scale height (km).\\
- \textbf{g} --- Surface gravity (⨁).\\
- Land--sea distribution (lithosphere--hydrosphere balance).

\textbf{Cross-Check Notes:}\\
- Reinforces prior geotic bounds with expanded atmospheric, rotational,
orbital, and magnetic criteria.\\
- Clarifies \emph{why} gravity corridor (0.5--1.5 ⨁) defines Geotic
worlds: outside this, planemons may be Telluric/parahabitable but not
Geotic.\\
- Orbital period C not freely chosen: constrained by Kepler's Third Law,
tying world design to stellar parameters.\\
- Magnetosphere thresholds emphasize that both too weak and too strong
fields can undermine habitability.\\
- Complements and extends core Geotic definitions; functions as a
reference sheet for designers setting secondary parameters.

\chapter{Justifying The Geotic
Limits}\label{justifying-the-geotic-limits}

In establishing the boundaries of the Geotic classification, we confront
not merely a technical exercise in parameter selection but a
philosophical act of declaration. These limits---defined in terms of
mass, density, surface gravity, radius, and escape velocity---do not
emerge arbitrarily from equations alone. They are sculpted from the
convergence of empirical data, biological insight, and an interpretive
philosophy of life's tolerances: an ontological bridge between
observation and valuation.

\begin{longtable}[]{@{}lcc@{}}
\toprule\noalign{}
Parameter & Symbol & Geotic Envelope(in Earth Units ⨁) \\
\midrule\noalign{}
\endhead
\bottomrule\noalign{}
\endlastfoot
Mass & m & ⟨0.30 ∧ 3.35⟩ ⨁ \\
Density & ρ & ⟨0.85 ∧ 1.25⟩ ⨁ \\
Gravity & g & ⟨0.60 ∧ 1.65⟩ ⨁ \\
Radius & r & ⟨0.60 ∧ 1.50⟩ ⨁ \\
Escape Velocity & vₑ & ⟨0.65 ∧ 1.50⟩ ⨁ \\
\end{longtable}

The Geotic envelope is broad enough to encompass worlds plausibly
habitable to Earthlike life, yet not so wide as to dilute the concept of
habitability into meaninglessness. The limits are drawn to include known
extremes of terrestrial endurance --- lower gravities where the human
body remains functional, higher ones where it still stands upright;
densities reflecting silicate-rich compositions without slipping into
degenerate matter; radii that allow for diverse tectonics and
geochemistry, yet remain below the gas giant threshold.

These ranges are further filtered through a mirandothesiastic sieve ---
one that refuses to separate scientific measure from lived perspective.
We acknowledge the Anthropic Principle not as an excuse, but as a lens:
life emerges within certain ranges because those ranges make life like
us possible. The limits of what we call \emph{Geotic} reflect both the
statistical plausibility of terrestrial-like biology and the
philosophical humility to admit that even our constraints are narrative
choices shaped by experience.

To be Geotic is to stand within the habitable corridor of possibility
--- not at the narrow peak of Earth-perfect parameters (Gaean), nor
flung to exotic extremes (Xenotic). It is the threshold of plausibility
for Earthlike life in a cosmos that offers no guarantees but countless
invitations.\#\# Abstract\\
\textbf{Major Topics:}\\
- Establishes rationale for the \textbf{parahabitable envelope}:\\
\[
  \text{parameter} \in \langle0.5 \wedge 1.5\rangle \oplus
  \]\\
for the five foundational parameters (m, ρ, g, r, vₑ).\\
- Argues this range balances \textbf{biological tolerance} with
\textbf{civilizational viability} --- survivability, infrastructure, and
long-term stability.\\
- Justifies limits for each parameter by examining consequences of
values below ≈0.5 ⨁ or above ≈1.5 ⨁.

\subsection{Parameter-Specific
Justifications}\label{parameter-specific-justifications}

\begin{itemize}
\tightlist
\item
  \textbf{Mass:}

  \begin{itemize}
  \item
    \textless0.5 ⨁ → atmosphere loss, weak magnetosphere, tectonic
    shutdown.\\
  \item
    \begin{quote}
    1.5 ⨁ → runaway pressures, volatile over-retention, hostile
    chemistry, tech suppression.
    \end{quote}
  \end{itemize}
\item
  \textbf{Density:}

  \begin{itemize}
  \item
    \textless0.5 ⨁ → volatile-rich ice/gas mix, poor tectonics, weak
    magnetosphere.\\
  \item
    \begin{quote}
    1.5 ⨁ → iron-heavy compact worlds, stagnant lid tectonics,
    biomechanical barriers.
    \end{quote}
  \end{itemize}
\item
  \textbf{Gravity:}

  \begin{itemize}
  \item
    \textless0.5 ⨁ → thin atmosphere, poor shielding, fluid/chemical
    instability.\\
  \item
    \begin{quote}
    1.5 ⨁ → compressive barriers to life, atmosphere flattening,
    biochemical inhibition.
    \end{quote}
  \end{itemize}
\item
  \textbf{Radius:}

  \begin{itemize}
  \item
    \textless0.8 ⨁ → compact/iron-rich, short tectonic lifespans.\\
  \item
    \begin{quote}
    1.2 ⨁ → volatile-rich/iceball-class, poor atmospheric retention.
    \end{quote}
  \end{itemize}
\item
  \textbf{Escape Velocity (vₑ):}

  \begin{itemize}
  \item
    \textless0.5 ⨁ (≈5.6 km/s) → thermal escape, atmosphere loss.\\
  \item
    \begin{quote}
    1.5 ⨁ (≈16.8 km/s) → volatile over-retention, crushing pressures,
    runaway greenhouse, tech bottlenecks.
    \end{quote}
  \end{itemize}
\end{itemize}

\subsection{Superhabitables}\label{superhabitables}

\begin{itemize}
\tightlist
\item
  Introduces \textbf{Superhabitable worlds}: planemons potentially
  \emph{more conducive to life than Earth}.\\
\item
  Traits: slightly higher mass (1.2--2.0 ⨁), moderate density (0.8--1.1
  ⨁), g ≈ 1.1--1.4 ⨁, long tectonic lifespans, stronger magnetospheres,
  larger/ecologically diverse surfaces, thicker atmospheres.\\
\item
  Not ``superhuman-friendly'' --- may be hostile for humans but rich for
  biospheres.\\
\item
  Shows habitability is a \textbf{plateau, not a peak}: higher mass can
  create opportunities, not just risks.
\end{itemize}

\textbf{Key Terms \& Symbols:}\\
- \textbf{Parahabitable Envelope} --- ⟨0.5--1.5⟩ ⨁ range across core
parameters.\\
- \textbf{Superhabitable planemons} --- Worlds optimized for life's
flourishing, not human comfort

\chapter{Justifying Parahabitable Parameter
Limits}\label{justifying-parahabitable-parameter-limits}

We have specified that the five foundational parameters for
\emph{terrestrial-class} planemons are:

!{[}{[}Physical Properties of Planets ✓{]}{]}

While {[}{[}planemon Classes ✓{]}{]} defines five envelopes ---
\textbf{Lithic}, \textbf{Geotic}, \textbf{Gaean}, \textbf{Rheatic}, and
\textbf{Xenotic} --- each with their own astrophysical and biological
implications, this sidebar focuses on the rationale for preferring a
narrower, \emph{parahabitable} range:

\[\text{parameter} \in \langle0.5 \wedge 1.5\rangle\oplus\] This
envelope is not meant to describe strict habitability in the Gaean
sense. Rather, it defines a \textbf{flexible but centered range} where
human life can be sustained or engineered, and where ecological and
technological systems remain dynamically stable without extreme
compensatory mechanisms. It reflects not only biological tolerance, but
also \textbf{civilizational viability} --- balancing survivability,
mobility, infrastructure, and long-term planemon homeostasis.

These ranges are further filtered through a \textbf{mirandothesiastic
sieve} --- one that refuses to sever scientific measure from lived
perspective. The goal is not simply to model what could exist, but to
prioritize what could \emph{matter}: to ecosystems, to civilizations, to
meaning-making beings.

The ⟨0.5--1.5⟩⨁ band is wide enough to encompass variation, but narrow
enough to preserve structural coherence across simulations,
worldbuilding systems, and narrative plausibility. It balances the
gravitational, metabolic, atmospheric, and material constraints that
shape whether a world feels \emph{inhabitable} or merely
\emph{endurable}.

However, good scholarship requires more than assertion. It demands that
we \textbf{justify} these limits --- or at least \textbf{validate} them
--- using first principles, observed planemon behavior, and biospheric
plausibility.

This sidebar examines the reasoning behind each boundary, explores edge
cases, and outlines the physical, chemical, and ecological consequences
of exceeding these thresholds.

\chapter{Mass}\label{mass}

\section{Below ≈ 0.500⨁}\label{below-0.500}

\begin{itemize}
\tightlist
\item
  \textbf{Inadequate Long-Term Atmosphere Retention}

  \begin{itemize}
  \tightlist
  \item
    Lower mass means \textbf{lower escape velocity}, unless density is
    artificially high\\
  \item
    Even with decent surface gravity, \textbf{thermal escape and
    sputtering} gradually strip the atmosphere
  \item
    \textbf{Solar wind} may erode the upper atmosphere unless magnetic
    protection is present (which low-mass planemons rarely have)
  \end{itemize}
\item
  \textbf{Thin or Transient Atmospheres}

  \begin{itemize}
  \tightlist
  \item
    CO₂, CH₄, H₂O vapor --- essential for warming and biochemistry ---
    are lost over geologic time
  \item
    Even if the planemon starts with an atmosphere, it won't necessarily
    \textbf{keep} it
  \item
    Results in environments like:

    \begin{itemize}
    \tightlist
    \item
      Mars (0.107⨁ mass): thin CO₂ atmosphere, mostly lost
    \item
      Mercury (0.055⨁): essentially no atmosphere
    \end{itemize}
  \end{itemize}
\item
  \textbf{Weak or Nonexistent Magnetosphere}

  \begin{itemize}
  \tightlist
  \item
    Mass correlates with \textbf{core volume} and \textbf{residual
    internal heat}
  \item
    Smaller worlds cool fast → solid cores → no dynamo
  \item
    No magnetic field → no deflection of solar wind → increased
    atmospheric loss and radiation exposure
  \end{itemize}
\item
  \textbf{Tectonic Shutdown}

  \begin{itemize}
  \tightlist
  \item
    Mantle convection \textbf{requires internal heat} and a sufficient
    pressure gradient
  \item
    Low-mass worlds solidify quickly
  \item
    Once tectonics stop:

    \begin{itemize}
    \tightlist
    \item
      Outgassing slows or ceases
    \item
      CO₂ cycle halts → greenhouse regulation fails
    \item
      planemon becomes geologically dead
    \end{itemize}
  \item
    Without active geology, \textbf{habitable climates cannot
    self-stabilize}
  \end{itemize}
\item
  \textbf{Orbital Vulnerability}

  \begin{itemize}
  \tightlist
  \item
    Low-mass planemons are more susceptible to:

    \begin{itemize}
    \tightlist
    \item
      \textbf{Tidal locking} (especially around M-dwarfs)
    \item
      \textbf{Orbital perturbations} (easily nudged by other bodies)
    \item
      \textbf{Catastrophic impacts} (less gravitational buffering)
    \end{itemize}
  \end{itemize}
\item
  \textbf{Atmospheric Chemistry Becomes Hostile}

  \begin{itemize}
  \tightlist
  \item
    Once light gases escape, heavier ones like sulfur or chlorine may
    dominate
  \item
    No volcanism or weathering to regulate composition
  \item
    You get \textbf{toxic skies} or inert ones --- but not breathable
    ones
  \end{itemize}
\item
  \textbf{Conclusion}:

  \begin{itemize}
  \tightlist
  \item
    planemons below \textasciitilde0.500⨁ mass face compounding risks:

    \begin{itemize}
    \tightlist
    \item
      Atmospheric loss
    \item
      Magnetic weakness
    \item
      Thermal stagnation
    \end{itemize}
  \item
    They may be ontosomic for extremophiles --- but sustaining complex,
    Earth-like ecologies over billions of years is \textbf{highly
    unlikely} \#\# Above ≈ 1.500⨁
  \end{itemize}
\item
  \textbf{Gravitational Consequences Amplify}

  \begin{itemize}
  \tightlist
  \item
    Increased mass \textbf{usually means higher surface gravity}, unless
    offset by low density
  \item
    But for rocky worlds with ρ ∈ ⟨0.900 ∧ 1.500⟩⨁, higher mass =
    \textbf{steep surface gravity rise}
  \item
    At 2.000⨁ mass with average density, gravity can reach
    \textbf{⟨1.700 ∧ 1.900⟩⨁}, breaching biomechanical and biochemical
    tolerances:
  \end{itemize}
\item
  \textbf{Escape Velocity Surges}

  \begin{itemize}
  \tightlist
  \item
    Higher mass → higher escape velocity → \textbf{retains volatiles too
    well}
  \item
    H₂, He, CH₄, and other gases that Earth sheds easily are now
    \textbf{gravitationally trapped}
  \item
    Atmospheres become:

    \begin{itemize}
    \tightlist
    \item
      \textbf{Massive}, with crushing pressures
    \item
      \textbf{Chemically reducing}, hostile to oxygen-based life
    \item
      \textbf{Opaque}, dominated by haze or deep cloud decks
    \end{itemize}
  \end{itemize}
\item
  \textbf{Atmosphere Transitions from Thin to Drown-y}

  \begin{itemize}
  \tightlist
  \item
    Surface pressure may exceed \textbf{100 bar}, even with moderate
    outgassing
  \item
    Water becomes supercritical --- \textbf{no liquid water layer}, just
    hot, dense steam
  \item
    Radiative cooling plummets: even modest stellar flux creates
    \textbf{runaway greenhouse conditions}
  \end{itemize}
\item
  \textbf{Tectonic Systems May Overdrive}

  \begin{itemize}
  \tightlist
  \item
    Massive interiors = \textbf{enormous thermal budgets}
  \item
    Some super-Earths may have:

    \begin{itemize}
    \tightlist
    \item
      \textbf{Hyperactive tectonics} → unstable continents, relentless
      volcanism
    \item
      Or \textbf{locked crusts} from extreme pressure → stagnant lid
      planemons
    \end{itemize}
  \item
    Neither scenario favors long-term habitability:

    \begin{itemize}
    \tightlist
    \item
      Volatile cycling is either too fast or too halted
    \item
      \textbf{Climate regulation fails}
    \end{itemize}
  \end{itemize}
\item
  \textbf{Surface Environments Become Energetically Hostile}

  \begin{itemize}
  \tightlist
  \item
    Extreme pressure gradients near the surface lead to:

    \begin{itemize}
    \tightlist
    \item
      Rapid erosion
    \item
      Overcompression of minerals
    \item
      Inhibition of \textbf{prebiotic compartmentalization}
    \end{itemize}
  \item
    Origin-of-life chemistry must happen \textbf{under extreme
    conditions} --- not conducive to Earthlike pathways
  \end{itemize}
\item
  \textbf{Solvent Stability and Chemistry Shift}

  \begin{itemize}
  \tightlist
  \item
    Water, ammonia, and other life-sustaining solvents may:

    \begin{itemize}
    \tightlist
    \item
      Only exist \textbf{deep underground}
    \item
      Be chemically altered into \textbf{less reactive or less
      structured forms}
    \end{itemize}
  \item
    Biochemistry must adapt to \textbf{unusual ionic balances, high
    viscosity, and low diffusion rates}

    \begin{itemize}
    \tightlist
    \item
      Protein folding, replication, and compartmentalization all behave
      differently
    \end{itemize}
  \end{itemize}
\item
  \textbf{Launch Barrier and Tech Suppression}

  \begin{itemize}
  \tightlist
  \item
    Escape velocity may exceed \textbf{25 km/s}
  \item
    Space access becomes \textbf{technologically and economically
    prohibitive}
  \item
    Civilizations may be trapped \textbf{planemon-bound}, or face
    extreme energy costs for satellite networks and off-world activity
  \item
    Cultural evolution diverges as the \textbf{cosmic horizon closes}
  \end{itemize}
\item
  \textbf{Conclusion}:

  \begin{itemize}
  \tightlist
  \item
    Worlds above \textasciitilde1.500⨁ mass may still be ontosomic ---
    but the \textbf{surface environment} becomes increasingly hostile to
    Earthlike biospheres.

    \begin{itemize}
    \tightlist
    \item
      Atmospheric pressures surge
    \item
      Climate regulation breaks
    \end{itemize}
  \item
    Biochemistry must adapt or fail\\
  \item
    Even if life emerges, the odds of it following a familiar trajectory
    are low. These are \textbf{deep-gravity worlds}, chemically rich but
    biologically unstable.
  \end{itemize}
\end{itemize}

\section{Superhabitables and Mass: Beyond ``Just
Right''}\label{superhabitables-and-mass-beyond-just-right}

In 2014, René Heller and John Armstrong proposed the existence of
\textbf{superhabitable planemons} --- worlds that may be \textbf{more
conducive to life} than Earth, even if not ideal for humans. These
planemons are theorized to exceed Earth's habitability not by matching
its parameters exactly, but by surpassing them in specific,
life-favoring ways. \#\#\# 📌 Definition:

\begin{quote}
A \textbf{superhabitable} world is one that provides \textbf{enhanced
conditions} for the \emph{emergence, proliferation, and persistence} of
life --- especially diverse, adaptive, and long-lasting biospheres.

In \emph{Worldcrafting 101} terms, these may be considered
\textbf{mega-ontosomic} worlds: not only capable of supporting life, but
actively \emph{better} at nurturing it.
\end{quote}

\subsubsection{🔸 Key Mass-Linked Traits of
Superhabitables}\label{key-mass-linked-traits-of-superhabitables}

\begin{itemize}
\tightlist
\item
  \textbf{According to Heller \& Armstrong, ideal superhabitable worlds
  are:}

  \begin{itemize}
  \tightlist
  \item
    \textbf{Slightly more massive than Earth}, typically \textbf{⟨1.200
    ∧ 2.000⟩\,⨁}, with upper candidates up to \textbf{3.000\,⨁}\\
  \item
    \textbf{Geologically more persistent}: higher mass leads to
    \textbf{longer tectonic lifespans}\\
  \item
    \textbf{Thermally buffered}: more efficient retention of internal
    heat extends mantle convection and volcanism\\
  \item
    \textbf{Better magnetic shielding}: large, slowly cooling cores
    sustain \textbf{stronger and longer-lived magnetospheres}\\
  \item
    \textbf{Larger surface area}: more room for ecological
    diversification, especially with flatter terrain\\
  \item
    \textbf{Thicker atmospheres}: improved radiation shielding, more
    stable greenhouse effect\\
  \item
    \textbf{More shallow seas}: potentially broader tidal zones and
    nutrient-rich photic environments
  \end{itemize}
\item
  \textbf{Implications}

  \begin{itemize}
  \tightlist
  \item
    Mass increases above \textasciitilde1.500\,⨁ usually signal
    \textbf{rising risk}, particularly for \textbf{Earthlike} surface
    biology. But for generalized \textbf{ontosomic potential}, the
    picture is more nuanced.
  \item
    A world with 1.600--2.000⨁ mass and moderate density ⟨0.800 ∧
    1.100⟩⨁ may maintain:

    \begin{itemize}
    \tightlist
    \item
      Surface gravity in the ≈ ⟨1.200 ∧ 1.400⟩⨁ range (still viable)
    \item
      Escape velocity high enough to retain a thick but breathable
      atmosphere
    \item
      Internal heat sufficient to power tectonics for billions of years
    \end{itemize}
  \end{itemize}
\item
  \textbf{Tradeoffs}

  \begin{itemize}
  \tightlist
  \item
    Superhabitability does not mean superhuman-friendliness:

    \begin{itemize}
    \tightlist
    \item
      \textbf{Biomechanical costs} rise as gravity increases\\
    \item
      \textbf{Space access} becomes difficult (high vₑ\hspace{0pt})\\
    \item
      \textbf{Photosynthetically active radiation} may differ under
      cooler stars\\
    \item
      \textbf{Dense, moist atmospheres} may favor different metabolisms
      and chemistries
    \end{itemize}
  \item
    These are worlds where \textbf{life flourishes}, but \textbf{not
    necessarily your life}.
  \end{itemize}
\item
  **Superhabitables are not exceptions to the Geotic envelope

  \begin{itemize}
  \tightlist
  \item
    They are edge-pushing optimals within a carefully constrained corner
    of it**:

    \begin{itemize}
    \tightlist
    \item
      \textbf{Mass}: ⟨1.200 ∧ 2.000⟩\,⨁
    \item
      \textbf{Density}: ⟨0.800 ∧ 1.100⟩\,⨁
    \item
      \textbf{Gravity}: ⟨1.100 ∧ 1.400⟩\,⨁
    \item
      \textbf{Tectonics}: Active over 5+ Gyr
    \item
      \textbf{Magnetosphere}: Sustained by large convective iron core
    \item
      \textbf{Star}: Spectral class M0--G9 (mass ∈ ⟨0.359 ∧ 0.817⟩\,☉)
    \end{itemize}
  \end{itemize}
\item
  \textbf{Superhabitables challenge anthropocentric bias}

  \begin{itemize}
  \tightlist
  \item
    They show that \textbf{habitability isn't a peak --- it's a plateau}
  \item
    Mass above 1.500⨁ doesn't always push you into hazard space; under
    the right conditions, it \textbf{creates biological opportunity}. In
    such cases, higher mass may be not a warning sign --- but an
    invitation.
  \end{itemize}
\end{itemize}

\chapter{Density}\label{density}

\section{Below ≈ 0.500⨁}\label{below-0.500-1}

\begin{itemize}
\tightlist
\item
  \textbf{Indicates a Volatile-Rich Composition}

  \begin{itemize}
  \tightlist
  \item
    Density \textless\,0.500⨁ usually signals a world made of
    \textbf{ice, silicates, organics, or gas} --- not rock and iron\\
  \item
    Analog worlds:

    \begin{itemize}
    \tightlist
    \item
      Titan (ρ ≈ 0.330⨁): methane--ice hybrid\\
    \item
      Callisto (ρ ≈ 0.330⨁): rock--ice mix\\
    \item
      Ganymede (ρ ≈ 0.360⨁): stratified ice/rock interior\\
    \end{itemize}
  \item
    These are \textbf{low-gravity}, \textbf{poorly differentiated}, and
    \textbf{chemically reducing} environments
  \end{itemize}
\item
  \textbf{Geostructural Weakness}

  \begin{itemize}
  \tightlist
  \item
    With low density, even a modest-mass world becomes \textbf{bloated
    in radius}\\
  \item
    This produces:

    \begin{itemize}
    \tightlist
    \item
      \textbf{Low surface gravity}\\
    \item
      \textbf{Shallow gravitational potential well}\\
    \item
      \textbf{High susceptibility to atmospheric loss}
    \end{itemize}
  \item
    Solid surface may be \textbf{thin, icy, or semi-liquid} ---
    unsuitable for stable tectonics or topography
  \end{itemize}
\item
  \textbf{Escape Velocity Drops}

  \begin{itemize}
  \tightlist
  \item
    For a given mass, low density means large radius → lower
    vev\_eve\hspace{0pt}
  \item
    Gases like N₂, O₂, H₂O vapor \textbf{escape more easily}
  \item
    Atmosphere must be \textbf{cold and massive} to stay intact --- or
    supplemented by continual outgassing
  \end{itemize}
\item
  \textbf{High Insolation = Runaway Loss}

  \begin{itemize}
  \tightlist
  \item
    In warm stellar zones, low-density planemons \textbf{cannot retain
    atmospheres}
  \item
    Solar UV + X-rays strip volatiles rapidly
  \item
    Left with:

    \begin{itemize}
    \tightlist
    \item
      Frozen surface (if far from star)
    \item
      Dead core or gasless ice ball (if close)
    \end{itemize}
  \end{itemize}
\item
  \textbf{No Long-Term Tectonics}

  \begin{itemize}
  \tightlist
  \item
    Low density → \textbf{poor internal stratification}\\
  \item
    May lack a distinct \textbf{metallic core}\\
  \item
    Internal heat generation is low; retention is worse
  \item
    No mantle convection = no plate tectonics = no carbon cycle =
    \textbf{climate collapse}
  \item
    Crust may behave plastically or amorphously --- not enough rigidity
    to form continents or fault lines
  \end{itemize}
\item
  \textbf{Weak or Absent Magnetosphere}

  \begin{itemize}
  \tightlist
  \item
    No iron core = no geodynamo
  \item
    Without magnetic shielding:

    \begin{itemize}
    \tightlist
    \item
      Solar wind strips atmosphere
    \item
      Surface exposed to radiation
    \item
      Volatiles sputtered into space
    \end{itemize}
  \end{itemize}
\item
  \textbf{Surface Conditions Are Often Alien}

  \begin{itemize}
  \tightlist
  \item
    May lack solid ground entirely --- thick clouds over slush or ocean
  \item
    Insolation often produces \textbf{photochemistry}, haze layers, and
    \emph{tholins}
  \item
    Light scattering, pressure gradients, and visibility differ
    dramatically from terrestrial norms
  \item
    Weather and climate may be dominated by:

    \begin{itemize}
    \tightlist
    \item
      Sublimation/condensation cycles
    \item
      Chemical storms
    \item
      Cryovolcanism
    \end{itemize}
  \end{itemize}
\item
  \textbf{Biochemical Challenges}

  \begin{itemize}
  \tightlist
  \item
    If surface temperature allows for solvents (e.g., methane, ethane,
    ammonia), these tend to:

    \begin{itemize}
    \tightlist
    \item
      Have \textbf{low reactivity}
    \item
      Require \textbf{extreme cold}
    \item
      Be incompatible with known Earthlike metabolism
    \end{itemize}
  \item
    Biochemistry must adapt to:

    \begin{itemize}
    \tightlist
    \item
      \textbf{Slow reaction rates}
    \item
      \textbf{Limited energy gradients}
    \item
      \textbf{Nonpolar solvent dynamics}
    \end{itemize}
  \item
    Possible --- but pushes us into \textbf{radically exotic} life
    territory
  \end{itemize}
\item
  \textbf{Psychophysical Environments Would Feel Surreal}

  \begin{itemize}
  \tightlist
  \item
    Movement feels effortless (low gravity) --- but fluids behave
    strangely
  \item
    Visual perception altered by deep scattering and low-pressure optics
  \item
    Weather systems sluggish or explosive depending on thermal regime
  \item
    Noise transmission altered --- \textbf{quiet, eerie landscapes}, or
    rapid atmospheric thumps
  \end{itemize}
\item
  \textbf{Conclusion}:

  \begin{itemize}
  \tightlist
  \item
    Density below \textasciitilde0.500⨁ means your world is likely an
    \textbf{iceball, a gas-rich bloater, or a crust over a subsurface
    ocean}.
  \item
    Even if mass or gravity fall within Geotic bounds, the planemon's
    internal
  \item
    Such a world may be ontosomic, or even \textbf{parahabitable}, under
    exotic conditions --- but its suitability for Earthlike surface
    biospheres is \textbf{nearly nil}.
  \end{itemize}
\end{itemize}

\section{Above ≈ 1.500⨁}\label{above-1.500}

\begin{itemize}
\tightlist
\item
  \textbf{Strong Indicator of Iron-Heavy Composition}

  \begin{itemize}
  \tightlist
  \item
    High density often means a \textbf{large metallic core} and a
    \textbf{thin rocky mantle}
  \item
    Possible formation pathways:

    \begin{itemize}
    \tightlist
    \item
      \textbf{Mantle stripping} via giant impacts (Mercury is the
      classic example: ρ ≈ 1.684⨁)
    \item
      \textbf{Primordial metal-rich protoplanemon} that never accreted
      much silicate material
    \item
      \textbf{Volatile loss} in early system formation
    \end{itemize}
  \item
    These planemons are \textbf{compact and massive} for their size,
    with unusually high surface gravity
  \end{itemize}
\item
  \textbf{Extreme Surface Gravity --- Even at Modest Mass}

  \begin{itemize}
  \tightlist
  \item
    For a given mass, high density = small radius → \textbf{gravity
    spikes}
  \item
    A 1.000\,⨁ mass planemon with 1.8⨁ density has:

    \begin{itemize}
    \tightlist
    \item
      Radius ≈ 0.800\,⨁
    \item
      Gravity ≈ 1.560\,⨁
    \item
      Escape velocity ≈ 1.400--1.500\,⨁
    \end{itemize}
  \item
    These effects rapidly breach Geotic thresholds --- even with
    otherwise ``Earthlike'' mass
  \end{itemize}
\item
  \textbf{Thin Crust, Shallow Mantle}

  \begin{itemize}
  \tightlist
  \item
    A large core displaces rocky material → \textbf{thin silicate crust}
  \item
    This limits:

    \begin{itemize}
    \tightlist
    \item
      Plate tectonics
    \item
      Volcanism
    \item
      Sequestration of volatiles and carbon
    \end{itemize}
  \item
    Without geological cycling, climate regulation becomes
    \textbf{fragile or inert}
  \end{itemize}
\item
  \textbf{Weakened Tectonic Activity}

  \begin{itemize}
  \tightlist
  \item
    Small mantle volume = \textbf{low convective energy}\\
  \item
    Thin crust = \textbf{stiff plates}, more likely to lock into a
    \textbf{stagnant lid regime}\\
  \item
    This halts long-term carbon cycling → \textbf{no climate
    stabilization}
  \end{itemize}
\item
  \textbf{Retention of Atmosphere is a Mixed Bag}

  \begin{itemize}
  \tightlist
  \item
    High gravity and escape velocity allow for:

    \begin{itemize}
    \tightlist
    \item
      \textbf{Strong atmospheric retention}\\
    \item
      But risk of retaining \textbf{undesirable volatiles} (e.g.,
      sulfur, chlorine, CO)
    \end{itemize}
  \item
    Without tectonics or outgassing, planemon may lack:

    \begin{itemize}
    \tightlist
    \item
      Greenhouse gases\\
    \item
      Surface pressure\\
    \item
      Atmospheric replenishment
    \end{itemize}
  \end{itemize}
\item
  \textbf{Intense Magnetic Field --- If There's Enough Heat}

  \begin{itemize}
  \tightlist
  \item
    Big iron core = good dynamo \textbf{if} internal heat persists\\
  \item
    Magnetic fields may be \textbf{strong}, but \textbf{short-lived}
    unless radiogenic or tidal heating supplements them
  \end{itemize}
\item
  \textbf{Harsh Surface Environment}

  \begin{itemize}
  \tightlist
  \item
    High gravity compresses:

    \begin{itemize}
    \tightlist
    \item
      Atmosphere → \textbf{thin vertical scale}
    \item
      Landscape → \textbf{low relief}, little vertical variation
    \item
      Fluids → higher boiling points, increased viscosity
    \end{itemize}
  \item
    Surface weathering slowed, erosion minimized
  \item
    Subduction may never initiate → no crustal recycling
  \end{itemize}
\item
  \textbf{High Gravity = Biomechanical Barrier}

  \begin{itemize}
  \tightlist
  \item
    Any life must be structurally robust from the start\\
  \item
    Movement, circulation, and metabolic flow all require more energy\\
  \item
    Evolutionary fitness shifts toward:

    \begin{itemize}
    \tightlist
    \item
      Compact morphologies
    \item
      Dense cellular scaffolds
    \item
      Slow-growing, high-efficiency systems

      \begin{itemize}
      \tightlist
      \item
        \textbf{Larger multicellular organisms unlikely}
      \end{itemize}
    \end{itemize}
  \end{itemize}
\item
  \textbf{Psychophysical Landscape}

  \begin{itemize}
  \tightlist
  \item
    Movement feels \emph{sluggish}, heavy\\
  \item
    Drops fall faster, wind dies sooner, sound carries less\\
  \item
    Visual perception altered by atmospheric thickness and refractive
    index\\
  \item
    The world may \emph{look} Earthlike --- but it will \textbf{feel
    alien}
  \end{itemize}
\item
  \textbf{Conclusion}:

  \begin{itemize}
  \tightlist
  \item
    planemons with ρ \textgreater\,1.500⨁ are \textbf{metal-heavy,
    compact, and structurally intense}.
  \item
    While they may retain atmospheres and shield themselves
    magnetically, their geology and surface conditions \textbf{work
    against biospheric diversity and long-term climate stability}.
  \item
    They may be ontosomic --- but their suitability for Earthlike life
    is tenuous, and their evolutionary trajectories are likely to be
    \textbf{slow, deep, and strange}.
  \end{itemize}
\end{itemize}

\section{🌍 Density in
Superhabitables}\label{density-in-superhabitables}

\subsection{🧭 Ideal Density Range: ρ ∈ ⟨0.800 ∧
1.100⟩\,⨁}\label{ideal-density-range-ux3c1-0.800-1.100}

\begin{itemize}
\tightlist
\item
  This range suggests:

  \begin{itemize}
  \tightlist
  \item
    A composition slightly richer in silicates or volatiles than Earth\\
  \item
    A large enough \textbf{metallic core} to sustain a dynamo\\
  \item
    A thick enough \textbf{mantle} to power tectonics for billions of
    years\\
  \item
    A \textbf{modest radius expansion} that keeps surface gravity in the
    ⟨1.100 ∧ 1.400⟩\,⨁ range (still biomechanically plausible)
  \end{itemize}
\item
  This density window supports \emph{structural stability, energetic
  cycling, and ecological flexibility} without pushing the planemon into
  extremes.
\end{itemize}

\#\#\# Why Not Lower (\textless{} 0.800⨁)? - Suggests \textbf{ice-rich
or gas-dominated} composition - May lack a dense core → \textbf{no
magnetosphere} - Surface gravity likely drops too low for: - Atmospheric
retention - Ecological complexity - Stable liquid water under moderate
insolation - Better suited to \textbf{parahabitable} or
\textbf{anontogenic} classification \#\#\# Why Not Higher
(\textgreater{} 1.100⨁)? - Indicates \textbf{iron-heavy or
mantle-stripped} structure - Often results from \textbf{giant impacts} →
dynamically unstable past\\
- Crust and mantle may be \textbf{too thin} to support:\\
- Long-lived tectonic activity\\
- Robust outgassing cycles\\
- Climate regulation via the carbon-silicate cycle - Surface gravity may
become biomechanically or chemically problematic

\subsection{🧠 Superhabitable Density
Traits}\label{superhabitable-density-traits}

\begin{longtable}[]{@{}
  >{\raggedright\arraybackslash}p{(\linewidth - 2\tabcolsep) * \real{0.4245}}
  >{\raggedright\arraybackslash}p{(\linewidth - 2\tabcolsep) * \real{0.5755}}@{}}
\toprule\noalign{}
\begin{minipage}[b]{\linewidth}\raggedright
Trait
\end{minipage} & \begin{minipage}[b]{\linewidth}\raggedright
Role in Superhabitability
\end{minipage} \\
\midrule\noalign{}
\endhead
\bottomrule\noalign{}
\endlastfoot
Moderate density ⟨0.8 ∧ 1.1⟩⨁ & Suggests healthy balance of
core/mantle \\
Thick mantle & Supports \textbf{long-lived tectonics} and
\textbf{volcanism} \\
Moderate core & Enables \textbf{stable magnetosphere} without rapid heat
loss \\
Slightly low gravity (from radius--mass ratio) & Encourages
\textbf{diverse biospheres}, easier movement \\
Retains volatiles without suffocating & Stable but breathable
atmospheres, good greenhouse regulation \\
\end{longtable}

\chapter{Gravity}\label{gravity}

\section{Below ≈ 0.5⨁}\label{below-0.5}

\begin{itemize}
\tightlist
\item
  Poor Atmospheric Retention

  \begin{itemize}
  \tightlist
  \item
    A planemon \textbf{struggles to retain a dense atmosphere},
    especially light molecules like N₂ or O₂.\\
  \end{itemize}
\item
  This leads to:

  \begin{itemize}
  \tightlist
  \item
    \textbf{Thin or nonexistent atmospheres}\\
  \item
    \textbf{Poor UV and radiation shielding}\\
  \item
    \textbf{Thermal extremes} (due to low heat retention)
  \end{itemize}
\item
  This is why Mars --- at 0.38⨁ gravity --- is \textbf{parahabitable} at
  best.
\item
  Fluid Dynamics Break Down

  \begin{itemize}
  \tightlist
  \item
    Water doesn't flow the same way. Capillary action dominates over
    gravity.
  \item
    Oceans and lakes become shallow and extremely wide, or may not form
    at all.
  \item
    Rainfall becomes mist or vapor --- precipitation \textbf{fails to
    ``fall.''}
  \end{itemize}
\item
  Chemical Mixing and Atmospheric Stratification

  \begin{itemize}
  \tightlist
  \item
    \textbf{Convection slows dramatically}\\
  \item
    \textbf{Gas layers stratify}, leading to atmospheric
    \textbf{stagnation}\\
  \item
    Cloud formation and weather may cease entirely --- or become wildly
    unstable\\
  \item
    Pollutants and volatiles \textbf{don't disperse}
  \end{itemize}
\item
  Magnetosphere Limitations

  \begin{itemize}
  \tightlist
  \item
    Lower gravity → likely smaller core mass → \textbf{weaker or
    nonexistent magnetic field}\\
  \end{itemize}
\item
  Combined with thin atmosphere → \textbf{intense radiation exposure}\\
\item
  Even if life evolves here, it would need:

  \begin{itemize}
  \tightlist
  \item
    Buried habitats\\
  \item
    Thick biofilms\\
  \item
    Reflective skins or other defensive adaptations
  \end{itemize}
\item
  \textbf{Conclusion}:

  \begin{itemize}
  \tightlist
  \item
    A stable biosphere \textbf{might} evolve under ⟨0.4 ∧ 0.5⟩⨁ gravity
  \item
    But it's unlikely to support \textbf{Earth-like ecologies} with
    dynamic, large-scale multicellular life \#\# Above ≈ 1.5⨁
  \end{itemize}
\item
  Structural Collapse at Micro- and Macro-Scales

  \begin{itemize}
  \tightlist
  \item
    \textbf{High gravity = relentless compressive stress}
  \item
    Surface materials must resist:

    \begin{itemize}
    \tightlist
    \item
      \textbf{Crushing forces}\\
    \item
      \textbf{Steep gradients in pressure over short distances}
    \item
      \emph{Problem}: Most \textbf{prebiotic scaffolds} (e.g., lipid
      membranes, fragile polymers) may never form or remain stable\\
    \item
      Even successful structures might \textbf{flatten}, rupture, or
      shear from their own weight
    \end{itemize}
  \item
    Life has to ``start strong'' --- but origin-of-life chemistry tends
    to be \emph{flimsy}
  \end{itemize}
\item
  Atmospheric Flattening and Thermal Gradient Compression

  \begin{itemize}
  \tightlist
  \item
    Denser gravity = \textbf{thinner vertical atmospheres}\\
  \item
    Scale height decreases\\
  \item
    Vertical circulation gets squeezed

    \begin{itemize}
    \tightlist
    \item
      \emph{Result}:\\
    \item
      More violent \textbf{surface convection}
    \item
      Rapid \textbf{thermal gradients} over short vertical scales
    \item
      Surface becomes a \textbf{storm-wracked}, turbulent layer
    \end{itemize}
  \item
    High-pressure zones can compress volatile gases into
    \textbf{chemical forms} unusable by emerging life

    \begin{itemize}
    \tightlist
    \item
      The surface becomes chemically and thermally \emph{unstable}
    \end{itemize}
  \end{itemize}
\item
  Surface Temperature Amplification

  \begin{itemize}
  \tightlist
  \item
    Thinner vertical atmosphere → less vertical mixing
  \item
    Higher surface gravity → greater atmospheric pressure at sea level

    \begin{itemize}
    \tightlist
    \item
      More efficient greenhouse trapping
    \end{itemize}
  \item
    Even modest insolation can produce runaway surface heating
  \item
    Water (or other life-supporting solvents) may only exist at extreme
    pressures
  \item
    The ``Goldilocks'' zone may exist only underground, if at all
  \end{itemize}
\item
  Energy Cost of Biomechanical Motion

  \begin{itemize}
  \tightlist
  \item
    Every movement costs more

    \begin{itemize}
    \tightlist
    \item
      Lifting limbs
    \item
      Circulating fluids
    \item
      Expanding membranes
    \end{itemize}
  \item
    Even molecular motion must fight stronger forces

    \begin{itemize}
    \tightlist
    \item
      Self-assembly becomes entropically disfavored
    \item
      Viscosity and diffusion rates alter radically
    \end{itemize}
  \item
    Emergent systems must overcome gravity early, which is rare without
    complex enzymes or pre-existing structure
  \end{itemize}
\item
  Biochemical Constraints on Solvent Chemistry

  \begin{itemize}
  \tightlist
  \item
    Solvents like water, ammonia, methane compress differently
  \item
    Protein folding, membrane behavior, ion mobility all shift in
    high-pressure regimes

    \begin{itemize}
    \tightlist
    \item
      DNA/RNA analogs may not remain stable
    \end{itemize}
  \item
    Origin-of-life reactions may favor different bonding dynamics

    \begin{itemize}
    \tightlist
    \item
      e.g., shift toward van der Waals dominance over hydrogen bonding
    \end{itemize}
  \item
    Biochemistry gets weird, and maybe non-viable
  \end{itemize}
\item
  Tectonic Stagnation or Overactivity

  \begin{itemize}
  \tightlist
  \item
    High mass often implies greater internal heat
  \item
    But strong gravity may:

    \begin{itemize}
    \tightlist
    \item
      Inhibit mantle convection (plate rigidity)
    \item
      Or accelerate it (hyperactive resurfacing)
    \end{itemize}
  \item
    Either way:

    \begin{itemize}
    \tightlist
    \item
      Continents may never stabilize
    \item
      Hydrothermal zones (origin-of-life hotspots) may never persist
    \end{itemize}
  \end{itemize}
\item
  \textbf{Conclusion}:

  \begin{itemize}
  \tightlist
  \item
    Biochemistry may become \emph{non-viable} before complex organisms
    ever emerge.
  \end{itemize}
\end{itemize}

\chapter{Radius}\label{radius}

\section{Below ≈ 0.8⨁}\label{below-0.8}

\begin{itemize}
\tightlist
\item
  \textbf{Implies a Compact or Iron-Rich Interior}

  \begin{itemize}
  \tightlist
  \item
    A small radius with normal mass means \textbf{high density}

    \begin{itemize}
    \tightlist
    \item
      Suggests a \textbf{large metallic core} and \textbf{thin rocky
      mantle}
    \item
      May result from \textbf{mantle stripping}, \textbf{impact
      history}, or metal-rich formation zone
    \end{itemize}
  \end{itemize}
\item
  \textbf{High Gravity at Modest Mass}

  \begin{itemize}
  \tightlist
  \item
    Small radius concentrates surface gravity

    \begin{itemize}
    \tightlist
    \item
      Even a 1.000\,⨁ mass planemon with 0.75⨁ radius yields:

      \begin{itemize}
      \tightlist
      \item
        Gravity ≈ 1.78⨁
      \item
        Escape velocity ≈ 1.55⨁
      \end{itemize}
    \end{itemize}
  \item
    \textbf{Biomechanical, atmospheric, and tectonic thresholds breached
    early}
  \end{itemize}
\item
  \textbf{Visual and Sensory Compression}

  \begin{itemize}
  \tightlist
  \item
    Shorter horizon: curvature more noticeable at sea level
  \item
    Vertical topography feels exaggerated, even on modest peaks
  \item
    Landscape may appear \textbf{visually foreshortened}
  \end{itemize}
\item
  \textbf{Lower Heat Retention}

  \begin{itemize}
  \tightlist
  \item
    Smaller surface area = faster heat loss
  \end{itemize}
\item
  Thin mantles cool quickly → tectonic lifespan may be short

  \begin{itemize}
  \tightlist
  \item
    Mars is a good example: its small radius led to early core cooling
    and tectonic shutdown
  \item
    May require radiogenic or tidal heating to remain active
  \end{itemize}
\item
  \textbf{Tectonic Fragility}

  \begin{itemize}
  \tightlist
  \item
    Thin crust and small mantle volume limit:

    \begin{itemize}
    \tightlist
    \item
      Plate mobility
    \item
      Outgassing
    \item
      Volatile recycling
    \end{itemize}
  \item
    Climate regulation is \textbf{difficult to maintain}
  \end{itemize}
\item
  \textbf{Strong Geodynamo --- If Core Is Active}

  \begin{itemize}
  \tightlist
  \item
    High core-to-mantle ratio favors magnetic field generation
  \item
    But may \textbf{cool rapidly} without sufficient internal insulation
  \end{itemize}
\item
  \textbf{Psychophysical Signature}

  \begin{itemize}
  \tightlist
  \item
    World feels ``dense,'' heavy, and intimate
  \item
    Horizons are close, curvature obvious
  \item
    Vertical relief feels steep and near, like standing on a ball
  \item
    Gravity and motion feel \textbf{compressed and taxing}
  \end{itemize}
\item
  \textbf{Conclusion}:

  \begin{itemize}
  \tightlist
  \item
    Worlds with R \textless\,0.800⨁ are likely \textbf{metal-rich,
    dense, and geologically challenged}
  \item
    They may offer magnetic protection, but are structurally compact
  \item
    Climate cycling and biospheric stability are at risk, especially if
    tectonic life is short \#\# Above ≈ 1.2⨁
  \end{itemize}
\item
  \textbf{Suggests a Volatile-Rich or Low-Density Interior}

  \begin{itemize}
  \tightlist
  \item
    Large radius + modest mass = \textbf{low density}

    \begin{itemize}
    \tightlist
    \item
      Implies high ice or gas content
    \item
      Possibly a rocky core with a \textbf{bloated volatile envelope}
    \end{itemize}
  \end{itemize}
\item
  \textbf{Low Surface Gravity --- Even at Earthlike Mass}

  \begin{itemize}
  \tightlist
  \item
    A 1.000\,⨁ mass planemon with 1.3⨁ radius has:

    \begin{itemize}
    \tightlist
    \item
      Gravity ≈ 0.59⨁
    \item
      Escape velocity ≈ 0.88⨁
    \end{itemize}
  \item
    Volatile loss becomes \textbf{more likely} --- especially without
    magnetic field
  \end{itemize}
\item
  \textbf{Poor Atmospheric Retention}

  \begin{itemize}
  \tightlist
  \item
    Radius increase spreads gravity thinner
  \item
    Unless paired with \textbf{very cold temperatures}, gases are easily
    lost
  \item
    N₂, O₂, and even H₂O vapor escape rapidly under stellar radiation
  \end{itemize}
\item
  \textbf{Internal Structuring May Be Weak}

  \begin{itemize}
  \tightlist
  \item
    Large, icy planemons may be \textbf{poorly differentiated}
  \item
    May lack a distinct core or layered mantle
  \item
    Tectonics and volcanism unlikely --- crust may be soft, slushy, or
    chaotic
  \end{itemize}
\item
  \textbf{Limited Magnetosphere}

  \begin{itemize}
  \tightlist
  \item
    Without a dense, rotating core, dynamo effects are weak or absent
  \item
    Solar wind strips volatiles; radiation reaches surface easily
  \end{itemize}
\item
  \textbf{Surface Conditions Are Vast, Flat, and Often Alien}

  \begin{itemize}
  \tightlist
  \item
    Wide horizons, low gravity, and weak terrain contrast
  \item
    Surfaces may consist of haze layers, cryomud, or semi-liquid plains
  \item
    Light diffusion, temperature layering, and visibility all feel
    ``off''
  \end{itemize}
\item
  \textbf{Psychophysical Landscape}

  \begin{itemize}
  \tightlist
  \item
    Movement feels \textbf{weightless}, loose, or eerily effortless
  \item
    Falling is slow --- but resistance from terrain is minimal
  \item
    Visual perception dominated by long, flat horizons
  \item
    Weather may be either \textbf{sluggish} or \textbf{chemically
    exotic}
  \end{itemize}
\item
  \textbf{Conclusion}:

  \begin{itemize}
  \tightlist
  \item
    Worlds with R \textgreater\,1.200⨁ are likely \textbf{bloated,
    volatile-rich}, or \textbf{iceball-class}
  \item
    They may support parahabitable subsurface oceans --- but
    \textbf{Earthlike life is rare}
  \item
    Biospheres would require \textbf{extreme cold, deep shielding, or
    exotic chemistry} to persist
  \end{itemize}
\end{itemize}

\section{🌍 Radius in Superhabitables}\label{radius-in-superhabitables}

\subsection{🧭 Ideal Radius Range: R ∈ ⟨0.900 ∧
1.100⟩⨁}\label{ideal-radius-range-r-0.900-1.100}

\begin{itemize}
\tightlist
\item
  Balances surface gravity, tectonic stability, and biospheric potential
\item
  A radius modestly larger than Earth (\textasciitilde1.05⨁):

  \begin{itemize}
  \tightlist
  \item
    Can \textbf{soften surface gravity} for a higher-mass world
  \item
    Still retains volatiles effectively
  \item
    Promotes \textbf{more diverse terrain} without introducing
    instability
  \end{itemize}
\item
  Allows:

  \begin{itemize}
  \tightlist
  \item
    Sufficient surface area for \textbf{climate zoning}
  \item
    Reasonable atmospheric scale height for breathing and weather
  \item
    Biomechanical feasibility for large, mobile lifeforms
  \end{itemize}
\end{itemize}

\subsection{Why Not Lower (\textless{}
0.900⨁)?}\label{why-not-lower-0.900}

\begin{itemize}
\tightlist
\item
  Compresses gravity
\item
  Shortens tectonic lifespan
\item
  Increases surface heat loss and curvature
\item
  Makes weathering and sediment cycling more difficult
\end{itemize}

\subsection{Why Not Higher (\textgreater{}
1.100⨁)?}\label{why-not-higher-1.100}

\begin{itemize}
\tightlist
\item
  Weakens gravity
\item
  Reduces escape velocity
\item
  Expands surface area but spreads mass thin
\item
  Tectonics may fail from poor core--mantle structure
\end{itemize}

\subsection{🧠 Superhabitable Radius
Traits}\label{superhabitable-radius-traits}

\begin{longtable}[]{@{}
  >{\raggedright\arraybackslash}p{(\linewidth - 2\tabcolsep) * \real{0.4250}}
  >{\raggedright\arraybackslash}p{(\linewidth - 2\tabcolsep) * \real{0.5750}}@{}}
\toprule\noalign{}
\begin{minipage}[b]{\linewidth}\raggedright
Trait
\end{minipage} & \begin{minipage}[b]{\linewidth}\raggedright
Implication
\end{minipage} \\
\midrule\noalign{}
\endhead
\bottomrule\noalign{}
\endlastfoot
Slightly larger than Earth & Keeps gravity Earthlike for higher-mass
worlds \\
Preserves volatile retention & While improving terrain diversity \\
Supports tectonics & If paired with healthy density and mass \\
Encourages climate zoning & Greater surface area allows biospheric
niches \\
Smooth curvature but defined relief & Visually compelling, emotionally
stable \\
\end{longtable}

\chapter{Escape Velocity}\label{escape-velocity}

\section{What It Is --- and Why It
Matters}\label{what-it-is-and-why-it-matters}

Escape velocity (\(v_e\)) is the speed a body must reach to completely
escape a planemon's gravitational influence without further propulsion.
On Earth, \(v_e ≈ 11.19\) km/s.\\
This isn't just a rocket science number --- it governs whether a world
can: - \textbf{Retain an atmosphere} - \textbf{Allow for viable space
access} - \textbf{Support Earthlike solvent cycles} In WCB terms, escape
velocity is a \emph{constraint amplifier}. It binds together mass and
radius, and exerts compound effects on chemistry, biospheres, and
technology.

\section{📈 Escape Velocity Scaling}\label{escape-velocity-scaling}

The absolute escape velocity formula is: \[
v_e = \sqrt{\frac{2GM}{R}}
\]Where: - \(G\) is the gravitational constant\\
- \(M\) is the planemon's mass\\
- \(R\) is the planemon's radius

In relative units: \[
v_e =\propto= \sqrt{\frac{m}{r}}
\] A dense, compact world has \emph{higher} escape velocity.\\
A large, low-density world may have \emph{surprisingly low} \(v_e\),
even with moderate mass.

\section{Geotic Threshold:}\label{geotic-threshold}

\textbf{\(v_e ∈ \langle0.5 \wedge 1.5\rangle⨁\) ≈ ⟨5.6 ∧ 16.8⟩ km/s}

This range preserves: - \textbf{Atmospheric stability}\\
- \textbf{Technological access to orbit}\\
- \textbf{Biospheric plausibility}

Let's examine what happens outside this window.

\section{Below ≈ 0.5⨁ (vₑ ≲ 5.6 km/s)}\label{below-0.5-vux2091-5.6-kms}

\begin{itemize}
\tightlist
\item
  \textbf{Thermal Escape Dominates}

  \begin{itemize}
  \tightlist
  \item
    Light molecules (H₂, He, even CH₄) \textbf{reach escape velocity at
    ambient temperatures}
  \item
    Water vapor, nitrogen, and oxygen slowly bleed off into space
  \end{itemize}
\item
  \textbf{Hydrodynamic Blowoff}

  \begin{itemize}
  \tightlist
  \item
    Strong stellar UV → upper atmosphere heats → whole layers escape en
    masse
  \item
    Solar wind and coronal mass ejections accelerate loss
  \end{itemize}
\item
  \textbf{No Long-Term Atmosphere}

  \begin{itemize}
  \tightlist
  \item
    Without replenishment (e.g.~volcanism, cometary delivery),
    atmosphere dissipates
  \item
    Mars is the classic example: \(v_e ≈ 5.0\) km/s → CO₂ retained
    weakly, water vapor gone
  \end{itemize}
\item
  \textbf{No Buffer from Solar Radiation}

  \begin{itemize}
  \tightlist
  \item
    Thin or absent atmosphere means:

    \begin{itemize}
    \tightlist
    \item
      Surface sterilization
    \item
      Photochemical breakdown of organics
    \item
      Cryogenic or desert extremes
    \item
      No climate memory or inertia
    \end{itemize}
  \end{itemize}
\item
  \textbf{Chemical Instability}

  \begin{itemize}
  \tightlist
  \item
    Reactive gases (like O₂ or NH₃) can't persist without shielding
  \item
    Surface chemistry becomes dominated by:

    \begin{itemize}
    \tightlist
    \item
      UV-driven radicals
    \item
      Photolysis byproducts
    \item
      Inert gases or unstable intermediates
    \end{itemize}
  \end{itemize}
\item
  \textbf{High Frequency of Atmospheric Collapse}

  \begin{itemize}
  \tightlist
  \item
    Even transient gases freeze or escape seasonally
  \item
    No stable pressure or weather system
  \end{itemize}
\item
  \textbf{Conclusion}:

  \begin{itemize}
  \tightlist
  \item
    Worlds with \(v_e < 0.5⨁\) lose too much too fast.\\
  \item
    Even with life, they are \textbf{ecologically fragile},
    \textbf{chemically unstable}, and \textbf{technologically marooned}.
  \end{itemize}
\end{itemize}

\section{Above ≈ 1.5⨁ (vₑ ≳ 16.8
km/s)}\label{above-1.5-vux2091-16.8-kms}

\begin{itemize}
\tightlist
\item
  \textbf{Volatile Over-Retention}

  \begin{itemize}
  \tightlist
  \item
    High escape velocity means even light gases \emph{cannot} escape
  \item
    Atmospheres become \textbf{deep}, \textbf{dense}, and
    \textbf{chemically reducing}

    \begin{itemize}
    \tightlist
    \item
      CH₄, H₂, NH₃, CO dominate
    \item
      O₂, if present, gets drowned out
    \end{itemize}
  \end{itemize}
\item
  \textbf{Runaway Greenhouse Amplification}

  \begin{itemize}
  \tightlist
  \item
    High pressure + high insolation → inefficient IR radiation →
    \textbf{heat traps}
  \item
    Water becomes supercritical, no oceans, just a ``steam atmosphere''
  \item
    Like early Venus\ldots{} permanently
  \end{itemize}
\item
  \textbf{Crushing Surface Pressure}

  \begin{itemize}
  \tightlist
  \item
    50--200 bar not unusual
  \item
    Liquids behave oddly: high boiling points, low evaporation
  \item
    Surface becomes oceanless, cloud-shrouded, or corrosive
  \end{itemize}
\item
  \textbf{Biochemistry Suffocates}

  \begin{itemize}
  \tightlist
  \item
    High-mass retention = wrong volatiles, wrong proportions
  \item
    Reactive chemistry (like oxidation) becomes rare
  \item
    Solvent cycles shift --- or cease
  \end{itemize}
\item
  \textbf{Space Access Barrier}

  \begin{itemize}
  \tightlist
  \item
    Rockets face exponential energy costs
  \item
    Even nuclear propulsion may be \textbf{barely viable}
  \item
    Spaceflight becomes the domain of orbital elevators or extreme
    industry
  \end{itemize}
\item
  \textbf{Technological Bottleneck}

  \begin{itemize}
  \tightlist
  \item
    If flight is impossible, off-world expansion halts
  \item
    No satellites, no long-range comms, no astronomical perspective
  \item
    The sky becomes \textbf{uncrossable}
  \end{itemize}
\item
  \textbf{Psychophysical Effects}

  \begin{itemize}
  \tightlist
  \item
    Air is thick --- visual range short
  \item
    Sound is dense --- travel of noise is distorted
  \item
    Pressure gradients crush perception --- what you see is not how it
    feels
  \end{itemize}
\item
  \textbf{Conclusion}:

  \begin{itemize}
  \tightlist
  \item
    Worlds with \(v_e > 1.5⨁\) trap more than air --- they \textbf{trap
    civilizations}\\
  \item
    Technological bottleneck, chemical drift, and biospheric inhibition
    define these worlds\\
  \item
    Life may exist --- but it rarely escapes
  \end{itemize}
\end{itemize}

\section{🌍 Escape Velocity in
Superhabitables}\label{escape-velocity-in-superhabitables}

\subsection{\texorpdfstring{🧭 Ideal Range:
\(v_e ∈ \langle1.100 ∧ 1.300\rangle⨁ ≈ ⟨12.3 ∧ 14.6⟩\)
km/s}{🧭 Ideal Range: v\_e ∈ \textbackslash langle1.100 ∧ 1.300\textbackslash rangle⨁ ≈ ⟨12.3 ∧ 14.6⟩ km/s}}\label{ideal-range-v_e-langle1.100-1.300rangle-12.3-14.6-kms}

\begin{itemize}
\tightlist
\item
  \textbf{High enough to retain volatiles}
\item
  \textbf{Low enough to allow spaceflight}
\item
  Paired with ideal density and radius, this yields:

  \begin{itemize}
  \tightlist
  \item
    Stable atmospheric composition
  \item
    Good pressure gradient for climate
  \item
    Technologically reachable orbits
  \end{itemize}
\end{itemize}

\begin{longtable}[]{@{}
  >{\raggedright\arraybackslash}p{(\linewidth - 2\tabcolsep) * \real{0.2793}}
  >{\raggedright\arraybackslash}p{(\linewidth - 2\tabcolsep) * \real{0.7207}}@{}}
\toprule\noalign{}
\begin{minipage}[b]{\linewidth}\raggedright
Trait
\end{minipage} & \begin{minipage}[b]{\linewidth}\raggedright
Implication
\end{minipage} \\
\midrule\noalign{}
\endhead
\bottomrule\noalign{}
\endlastfoot
vₑ ≈ 12--14 km/s & Retains useful gases (N₂, O₂, H₂O) without retaining
too many reducing volatiles \\
Moderate gravity (g ≈ 1.1--1.4⨁) & Allows liquid water and Earthlike
weather \\
Accessible orbit & Favors cultural expansion and space infrastructure \\
\end{longtable}

\section{Abstract}\label{abstract-22}

\textbf{Major Topics:}\\
- Relationship between \textbf{land--water distribution} and planetary
climate.\\
- Effects of different \textbf{continental/oceanic ratios} on
atmospheric circulation, precipitation, and long-term habitability.\\
- Classification of hydrospheric patterns: aquaplanets, thalassoplanets,
continental planets, mixed/ocean-hemisphere worlds.\\
- Impact of \textbf{polar vs.~equatorial land placement} on ice caps,
climate stability, and water cycle feedbacks.\\
- Worldbuilding guidelines for how water distribution influences
\textbf{biospheric richness} and \textbf{geotic/gaean habitability
envelopes}.

\textbf{Key Terms \& Symbols:}\\
- \textbf{Aquaplanet} --- world with near-total ocean coverage.\\
- \textbf{Thalassoplanet} --- ocean-dominated planet with small
landmasses.\\
- \textbf{Continental planet} --- land-dominated planet with limited
water.\\
- \textbf{Mixed pattern} --- intermediate distribution of land and
water.\\
- \textbf{Hydrospheric balance} --- ratio of oceanic to continental
coverage, often expressed as a percentage.\\
- \textbf{Cryosphere} --- ice-covered portions of the hydrosphere,
strongly latitude-dependent.

\textbf{Cross-Check Notes:}\\
- Connects directly with \textbf{Gaean planemons} (Earth-like envelope
definition) and \textbf{Geotic Guidelines} (extended habitability).\\
- Provides \emph{pattern-based classification}, useful alongside
\textbf{Ground States} and \textbf{Rheatic planemons}.\\
- No explicit new symbols; introduces several neolexical classifications
(\emph{thalassoplanet, aquaplanet}).\\
- Stages additional glossary updates for \textbf{hydrospheric balance}
and related world types.

\subsection{🔑 Core Insight:}\label{core-insight}

\begin{quote}
\textbf{Habitability is shaped by \emph{exposed land}, not just crustal
proportion.}
\end{quote}

A planemon might have:

\begin{itemize}
\tightlist
\item
  The \textbf{same total volume of felsic crust} as Earth,\\
\item
  The \textbf{same hydrospheric volume},\\
\item
  But \textbf{very different levels of exposed land}, depending on:

  \begin{itemize}
  \tightlist
  \item
    Sea level (absolute and relative)\\
  \item
    Crustal relief\\
  \item
    Tectonic activity\\
  \item
    Isostatic balance
  \end{itemize}
\item
  \textbf{Continents ≠ dry land} --- ``continent'' in geologic terms
  just means \emph{buoyant crust}, not \emph{exposed crust}.
\item
  \textbf{Flooded cratons} were common for the first half of Earth's
  history.\\
\item
  True \textbf{emergent landmasses} may have been small islands or
  microcontinents for much of the Archean and Paleoproterozoic.\\
\item
  \textbf{Modern-style continents} with wide uplands, deep interiors,
  and stable exposure are a \textbf{Phanerozoic phenomenon}.
\end{itemize}

\subsection{🔹 What ``New Crust'' Really
Means}\label{what-new-crust-really-means}

When we say:

\begin{quote}
``New crust is being created at mid-ocean ridges,''\\
we mean \textbf{oceanic crust} --- specifically \textbf{mafic} basaltic
crust --- is forming as mantle material rises and cools.
\end{quote}

But:

\begin{itemize}
\tightlist
\item
  This crust is \textbf{dense} and \textbf{thin} (≈ 7 km thick vs.~≈ ⟨30
  ∧ 70⟩ km for continental crust).\\
\item
  It \textbf{sits lower} in the gravitational field --- forming
  \textbf{ocean basins}, not continents.\\
\item
  It does \textbf{not increase exposed land area}, unless\ldots{}

  \begin{itemize}
  \tightlist
  \item
    It's \textbf{uplifted} by tectonic collisions (forming island arcs,
    terranes)\\
  \item
    Or accreted onto existing \textbf{continental margins} \#\#\# 🔸 So:
  \end{itemize}
\end{itemize}

\begin{quote}
🔍 \textbf{Oceanic crust = dynamic but low-lying}\\
🪨 \textbf{Continental crust = buoyant and persistent}
\end{quote}

In fact, over time:

\begin{itemize}
\tightlist
\item
  Oceanic crust is \textbf{recycled} back into the mantle via subduction
  (lifespan: \textasciitilde200 My)\\
\item
  Continental crust tends to \textbf{accumulate and endure},
  occasionally resurfacing in exposed terranes.
\end{itemize}

\subsection{🪨 Is Felsic Crust Ever
Subducted?}\label{is-felsic-crust-ever-subducted}

\textbf{Yes\ldots{} but reluctantly.}\\
Felsic (continental) crust \textbf{can} be pulled into a subduction zone
--- but:

\begin{itemize}
\tightlist
\item
  It is \textbf{less dense} than oceanic crust (≈ 2.7 g/cm³ vs ≈ 3.0+
  g/cm³)\\
\item
  It is \textbf{buoyant relative to the mantle}\\
\item
  It \textbf{resists permanent subduction} and tends to ``jam'' or ``pop
  back up''
\end{itemize}

So when felsic crust \textbf{does} get subducted:

\begin{itemize}
\tightlist
\item
  It is often part of \textbf{a complex terrane collision}\\
\item
  It may be \textbf{scraped off} and accreted to the overriding plate\\
\item
  Or it may be \textbf{partially melted}, with lighter components rising
  back as \textbf{felsic magma}, contributing to \textbf{continental
  volcanism} \#\#\# 🔥 Subduction-Driven Volcanism:
\end{itemize}

The \textbf{melt} from subducted slabs (mostly mafic oceanic crust +
water-rich sediments) rises and:

\begin{itemize}
\tightlist
\item
  Interacts with the overlying \textbf{continental lithosphere}\\
\item
  Mixes with or \textbf{melts felsic components}\\
\item
  Produces \textbf{andesitic to rhyolitic magma} (more viscous and
  explosive than basalt)
\end{itemize}

This is why volcanic arcs (e.g., the Andes, Cascades, Japan) are: -
Continental in character\\
- Erupting \textbf{felsic lavas}, not oceanic basalt \#\#\# 🧭 W101
Summary Point:

\begin{quote}
🔍 \textbf{Continental crust is rarely destroyed.}\\
When it's pulled into a subduction zone:
\end{quote}

\begin{itemize}
\tightlist
\item
  It tends to \textbf{resurface via volcanism},\\
\item
  Or \textbf{get welded onto other landmasses},\\
\item
  Or \textbf{form mountains} via crustal thickening and uplift. Over
  billions of years, this is how Earth built up its continents.
\end{itemize}

\subsection{🧓 Why Some Rocks Are 3.7 Billion Years
Old}\label{why-some-rocks-are-3.7-billion-years-old}

Those ancient rocks in \textbf{Scotland}, \textbf{Greenland},
\textbf{Australia}, and \textbf{Canada} are part of what we call
\textbf{cratons} --- old, stable cores of continental lithosphere that
have: - \textbf{Survived} every tectonic rearrangement\\
- \textbf{Resisted} subduction due to buoyancy\\
- \textbf{Avoided} recycling into the mantle\\
- Often remained \textbf{above sea level} or **only shallowly

These regions include formations like: - \textbf{The Isua Greenstone
Belt} in Greenland\\
- \textbf{The Jack Hills zircon deposits} in Australia\\
- \textbf{The Lewisian complex} in Scotland

They are made of \textbf{felsic or ultra-felsic rocks}, like granite and
gneiss, and represent:

\begin{quote}
The surviving scaffolding of Earth's first continents.
\end{quote}

\subsection{🧭 :}\label{section}

\begin{itemize}
\tightlist
\item
  \textbf{Oceanic crust is temporary} --- it's made, spread, and
  subducted like conveyor belt parts.\\
\item
  \textbf{Continental crust is archival} --- it builds up over time,
  preserves history, and keeps records. So a 3.7-billion-year-old rock
  is a \textbf{literal relic} of the early Earth, untouched by
  recycling, uplifted by tectonics, and never drowned or subducted deep
  enough to erase its story.
\end{itemize}

\begin{quote}
🔍 If your world has \textbf{ancient exposed felsic terranes}, it
implies: - Very old crust (from early tectonic activity)\\
- Long-term tectonic \textbf{stability}\\
- Minimal subduction or high rates of crustal buoyancy\\
- Possibly \textbf{low sea level} or \textbf{uplifted interiors}
\end{quote}

\chapter{Abstract}\label{abstract-23}

\textbf{Major Topics:}\\
- Simplified relationships among planemon parameters (m, ρ, g, r, vₑ)
when one parameter is set to unity (Earth-normal).\\
- Demonstrates that if \textbf{any two parameters equal 1
simultaneously, all five must equal 1}.\\
- Provides worked equivalences for each ``ground state'' case:\\
- m = 1 → other parameters derived directly.\\
- ρ = 1 → r, g, vₑ equal; m is cube of these.\\
- g = 1 → central case; defines ``paramount geotic worlds.''\\
- vₑ = 1 → ties radius, mass, and gravity.\\
- r = 1 → m, g, ρ equal; vₑ is their square root.\\
- Highlights \textbf{paramount geotic worlds} as those with g = 1, i.e.,
Earth-normal gravity, yielding minimal physiological stress for humans.

\textbf{Key Terms \& Symbols:}\\
- \textbf{m} --- planemon mass (⨁).\\
- \textbf{ρ} --- Density (⨁).\\
- \textbf{g} --- Surface gravity (⨁).\\
- \textbf{r} --- Radius (⨁).\\
- \textbf{vₑ} --- Escape velocity (⨁).\\
- \textbf{Geotic Ground States} --- Parameter configurations where one
or more parameters equal unity.\\
- \textbf{Paramount Geotic Worlds} --- Worlds defined by g = 1, optimal
for human habitation.

\textbf{Cross-Check Notes:}\\
- Expands the WCB \textbf{Core Parameter Precedence} framework by
showing interdependencies in normalized form.\\
- Emphasizes that g = 1 is not just a mathematical simplification but
the \textbf{biological optimum} for hospitable worlds.\\
- Serves as both a teaching tool (equation simplification) and a design
guide (highlighting gravity's primacy).

\section{Geotic Ground States}\label{geotic-ground-states}

If you assume a value of 1 for any of the basic parameters, the
equations for the other parameters simplify, as enumerated below.

\begin{quote}
\textbf{Note}: If \textbf{any two} core parameters equal 1
simultaneously, then \textbf{all} five must equal 1. \#\#\# Mass: m = 1
- \(r=\sqrt{\dfrac{1}{g}}=\sqrt[3]{\dfrac{1}{\rho}}=\dfrac{1}{v_e^2}\) -
\(\rho=\dfrac{1}{r^3}=\sqrt{g^3}=v_e^6\) -
\(g=\dfrac{1}{r^2}=\sqrt[3]{\rho^2}=v_e^4\) -
\(v_e=\sqrt{\dfrac{1}{r}}=\sqrt[4]{g}=\sqrt[6]{\rho}\) \#\#\# Density: ρ
= 1 - \(r = g = v_e = \sqrt[3]{m}\) - \(m=r^3=g^3=v_e^3\) -
\(g=r=\sqrt[3]{m}=v_e\) - \(vₑ=g=r=\sqrt[3]{m}\) \textgreater{}
\textbf{Note}: When \textbf{ρ = 1}, the values of \textbf{radius}
(\emph{r}), \textbf{gravity} (\emph{g}), and \textbf{escape velocity}
(\emph{vₑ}) are all numerically equal. The \textbf{mass}
(\emph{m\hspace{0pt}}) is the cube of any of them. \#\#\# Gravity: g = 1
- \(m=r^2=\dfrac{1}{\rho^2}=v_e^4\) -
\(r=\dfrac{1}{\rho}=\sqrt{m}=v_e^2\) -
\(\rho=\dfrac{1}{r}=\sqrt{\dfrac{1}{m}}=\dfrac{1}{v_e^2}\) -
\(v_e=\sqrt{r}=\dfrac{1}{\sqrt{\rho}}=\sqrt[4]{m}\) \#\#\# Escape
Velocity: vₑ = 1 - \(m=\dfrac{1}{\sqrt{\rho}}=\dfrac{1}{g}=r\) -
\(r=\dfrac{v_e}{\sqrt{\rho}}=\dfrac{1}{g}=m\) -
\(\rho=\dfrac{1}{r^2}=g^2=\dfrac{1}{m^2}\) -
\(g=\sqrt{\rho}=\dfrac{1}{r}=\dfrac{1}{m}\) \#\#\# Radius: r = 1 -
\(m=g=\rho=v_e^2\) - \(v_e=\sqrt{g}=\sqrt{\rho}=\sqrt{m}\)
\textgreater{} \textbf{Note}: When \textbf{r = 1}, the values of
\textbf{mass} (\emph{m}), \textbf{gravity} (\emph{g}), and
\textbf{density} (\emph{ρ}) are all numerically equal. The
\textbf{escape velocity} (\emph{vₑ\hspace{0pt}}) is the square-root of
any of them.
\end{quote}

\section{Paramount Geotic Worlds}\label{paramount-geotic-worlds}

\textbf{The most Geotic worlds --- that is, the worlds most naturally
suited for Earthling-human life --- are those with g = 1.}

\begin{quote}
Since Earth-normal gravity determines everything from blood pressure to
biomechanical stress, a surface gravity of 1g minimizes physiological
strain and habitat adaptation needs.
\end{quote}

If you set \textbf{mass} as your primary parameter, the others resolve
as follows: - \(r=\sqrt{m}\) - \(\rho=\sqrt{\dfrac{1}{m}}\) -
\(v_e=\sqrt[4]{m}\)

Alternatively, if you begin by setting \textbf{density}, the remaining
values are: - \(m=\dfrac{1}{\rho^2}\) - \(r=\dfrac{1}{\rho}\) -
\(v_e=\dfrac{1}{\sqrt{\rho}}\)

The equations above also allow determination of any of the three
remaining parameters when two are initially set/chosen.

\section{Abstract}\label{abstract-24}

\textbf{Major Topics:}\\
- Defines two inner-boundary orbital constraints relative to stars:\\
- \textbf{Innermost Stable Limital (ISL):} Minimum orbit distance where
a planemon's orbit remains dynamically stable given the stellar radius
and luminosity class.\\
- Formula: ISL ≈ k × R.\\
- k-values vary by luminosity class (e.g., 2--3 for main sequence,
15--25 for supergiants).\\
- \textbf{Critical Viability Limital (CVL):} Minimum orbit distance
where stellar flux does not exceed the maximum tolerable level for
animotic viability.\\
- Formula: CVL = √(L / Smax).\\
- Smax varies by motatype: endomota (2), xenomota (7.5), exomota (20→).

\textbf{Key Terms \& Symbols:}\\
- \textbf{ISL} --- Innermost Stable Limital.\\
- \textbf{CVL} --- Critical Viability Limital.\\
- \textbf{R} --- Stellar radius.\\
- \textbf{k} --- Luminosity-class multiplier.\\
- \textbf{L} --- Stellar luminosity (⊙).\\
- \textbf{Smax} --- Maximum tolerable stellar flux
(motatype-dependent).\\
- \textbf{Motatypes:}\\
- \textbf{Endomota} --- conservative Terran-like biota (low flux
tolerance).\\
- \textbf{Xenomota} --- more radiation-tolerant life.\\
- \textbf{Exomota} --- extreme-tolerance life.

\textbf{Cross-Check Notes:}\\
- ISL ties orbital mechanics to stellar radius scaling.\\
- CVL ties biological viability to stellar flux limits.\\
- Both act as \textbf{inner habitability boundaries}, complementing
habitable zone definitions (H₀--H₅).

\chapter{ISL: Innermost Stable
Limital}\label{isl-innermost-stable-limital}

\[
ISL ≈ k \times R
\] Where: - \(k\) = scalar multiplier based on luminosity class - \(R\)
= stellar radius

\begin{longtable}[]{@{}ll@{}}
\toprule\noalign{}
Luminosity Class & k \\
\midrule\noalign{}
\endhead
\bottomrule\noalign{}
\endlastfoot
V & 2 -- 3 \\
IV & 3 -- 5 \\
III & 5 -- 10 \\
II & 10 -- 15 \\
I & 15 -- 25 \\
WD & 1.5 -- 2 \\
& \\
\end{longtable}

\chapter{CVL: Critical Viability
Limital}\label{cvl-critical-viability-limital}

\[
CVL = \sqrt{\frac{L}{S_{max}}}
\] Where: - \(L\) = luminosity of the star in stellar units -
\(S_{max}\) = maximum flux tolerable for animotic viability

\begin{longtable}[]{@{}ll@{}}
\toprule\noalign{}
motatype & \(S_{max}\) \\
\midrule\noalign{}
\endhead
\bottomrule\noalign{}
\endlastfoot
endomota & 2 \\
xenomota & 7.5 \\
exomota & 20→ \\
\end{longtable}

\section{Abstract}\label{abstract-25}

\textbf{Major Topics:}\\
- Characteristics of a zero-obliquity (ε ≈ 0°) planet.\\
- Climatic consequences: absence of seasons, uniform day length.\\
- Astronomical consequences: no annual change in stellar altitude or
apparent size.\\
- Cultural/calendar implications: limited concept of ``year,'' reliance
on lunar or stellar cycles.\\
- Alternative mechanisms for ``seasons'' via high orbital eccentricity.

\textbf{Key Terms \& Symbols:}\\
- ε = obliquity (axial tilt).\\
- e = orbital eccentricity.\\
- Periastron / Apastron distances (Rₘᵢₙ, Rₘₐₓ).

\textbf{Cross-Check Notes:}\\
- Relates to glossary terms: obliquity (ε), eccentricity (e),
periastron/apastron.\\
- Complements \textbf{Orbital Eccentricity and Seasonal Effects.md} by
exploring a special case.\\
- Conceptual overlap with habitability/seasonality discussions in
climate-related notes.

\chapter{A Zero-Obliquity Planet}\label{a-zero-obliquity-planet}

\section{Basics}\label{basics}

A planet with no axial tilt (or a very, very modest one --- anything
\textless2.5°)

\begin{itemize}
\tightlist
\item
  If eccentricity is low (e \textless{} 0.10), no major seasonal
  variation through the year.
\item
  All days are the same length, everywhere, all year long.
\item
  ``Seasonal'' constellations still occur, because the direction the
  ``night side'' of the planemon faces still changes as it orbits.
\item
  No change in the size of the star in the sky during orbit.
\item
  From a stellar standpoint, the concept of a ``year'' would be
  meaningless

  \begin{itemize}
  \tightlist
  \item
    ONLY changes like the slow rotation of the night sky from night to
    night, or the period of a major moon would be available for
    calendrical development.
  \end{itemize}
\end{itemize}

\subsection{Alternative Methods Of Generating
``Seasons''}\label{alternative-methods-of-generating-seasons}

\subsubsection{Zero Obliquity With High Orbital
Eccentricity}\label{zero-obliquity-with-high-orbital-eccentricity}

A planet with zero obliquity might experience something akin to seasons
if it were on a highly elliptical orbit that carried it particularly
close to and distant from the star at either periastron or apastron, or
both.

Eccentricities in excess of 0.40 --- which is \emph{highly} eccentric
(Mercury's orbital eccentricity is 0.2056).

Depending on the configuration of the orbit, the planet might: - Pass
very close to the star at periastron - Linger extremely distant from the
star at apastron - Experience both extremes on each orbit.

Orbital period would come into play, here, as well. If the orbit is
short in duration, these fluctuations might be experienced more as
something like seasonal changes, whereas if the orbit is of long
duration, they might be more climatological shifts over a span of years,
or even centuries or millennia.

In the case of high orbital eccentricity, the apparent size of the star
in the sky might become naked-eye noticeable.

\part{Orbital Mechanics}

\chapter{Abstract}\label{abstract-26}

\textbf{Major Topics:}\\
- Methods for placing and sizing asteroid belts between two major
perturbers.\\
- Calculation of belt extent: inner/outer orbits, asymmetric central
orbit, systemic mass ratio, belt width.\\
- Resonant orbit calculation using Kepler's Third Law
(\(P^2 \propto a^3\)).\\
- Distinction between \textbf{tresonances} (trapping resonances) and
\textbf{gresonances} (gap-opening resonances).\\
- Resonance scalers and their observed roles in Kirkwood gaps.\\
- Gap width justification and minimum perturber mass (≥ 0.333⨁ for solar
case).\\
- Example Kirkwood gap analogues.

\textbf{Key Terms \& Symbols:}\\
- \(a_i\), \(a_o\) = inner/outer orbit distances.\\
- \(m_i\), \(m_o\) = inner/outer planemon masses (Terrans).\\
- \(M_*\) = stellar mass (Terrans; 333,000 M⊙).\\
- \(a_c\), Δa, \(a_s\), \(a_\Delta\) = derived orbital metrics.\\
- \(m_\mu\) = systemic mass ratio.\\
- \(W_{belt}\), \(W_i\), \(W_o\), \(B_i\), \(B_o\) = belt width and edge
measures.\\
- Tresonance = trapping resonance.\\
- Gresonance = gap-opening resonance.\\
- \(P_i\), \(P_o\), \(P_x\) = periods of perturber and resonance.\\
- \(a_x\) = resonant distance.\\
- \(g_w\), \(g_{quad}\) = gap width measures.

\textbf{Cross-Check Notes:}\\
- Resonance framework builds on \textbf{Mean Motion Resonance} note.\\
- Tresonance vs.~Gresonance terminology is WCB-specific.\\
- Gap resonance ratios (e.g., 2:1, 3:1, 5:2) match observed Solar System
Kirkwood gaps.\\
- Trap resonance ratios (e.g., 3:2, 4:3) match stabilizing locations
(e.g., Hilda asteroids).\\
- Canonical distinction: gap resonances = frequency \textgreater{} 1.5,
trap resonances = frequency ≤ 1.5.\\
- Belt placement formulas integrate with broader Orbit Design Methods.

\chapter{Asteroid Belt Placement and
Extents}\label{asteroid-belt-placement-and-extents}

\section{Placing The Belt and Identifying Its
Dimensions}\label{placing-the-belt-and-identifying-its-dimensions}

\[
\begin{align}
a_i &= &&\text{Inner orbit distance} \\
a_o &= &&\text{Outer orbit distance} \\[0.5em]
m_i &= &&\text{Mass of inner body in Terrans} \\
m_o &= &&\text{Mass of outer body in Terrans} \\[0.5em]
M_* & = 333000M⊙ \qquad &&\text{Mass of star in Terrans} \\[0.5em]
a_c &= \frac{a_i + a_o}{2} \qquad &&\text{Average of the two orbits} \\[0.5em]
\Delta a &= a_o - a_i \qquad &&\text{Difference between the two orbits} \\[0.5em]
a_\Delta &= \frac{\Delta a}{2} \qquad &&\text{Midrange of the two orbits} \\[0.5em] 
a_s &= a_c + a_\Delta \left(\frac{m_o - m_i}{m_o + m_i}\right) \qquad &&\text{Asymmetric central orbit} \\[0.5em]
m_\mu &= \frac{m_i + m_o}{M_*} \qquad &&\text{Dimensionless systemic mass ratio} \\[0.5em]
\beta &= 1 - (C \times \sqrt[3]{m_\mu}) \qquad &&\text{Where C} \in \{1, 2, 3\} \\
&&&\text{Belt width scaler} \\[0.5em]
W_{belt} &= \Delta a \times \beta \qquad &&\text{Belt width calculation}\\[0.5em]
w_i &= \frac{m_i}{m_i + m_o} \qquad w_o = \frac{m_o}{m_i + m_o} \quad &&\text{Belt inner and outer edge adjustments} \\[0.5em]
W_i &= W_{belt} \times w_i \qquad W_o = W_{belt} \times w_o \quad && \text{Belt inner and outer edge offset calculations}\\[0.5em]
B_i &= a_s - W_i \qquad B_o = a_s + W_o \qquad &&\text{Belt inner and outer edge calculations}
\end{align}
\]

\section{Calculating Resonant Orbits}\label{calculating-resonant-orbits}

\subsection{Vocabulary Notes}\label{vocabulary-notes}

\begin{itemize}
\tightlist
\item
  Perturber: An orbiting object acting to perturb the orbit of other,
  less massive nearby objects.
\item
  Perturbant: The body exerting perturbing influence.
\item
  Resonant: Any of the resonances that result from the influence of the
  perturbant.
\item
  Tresonance: A resonance tending to trap orbiting bodies within a
  narrow orbital region.
\item
  Gresonance: A resonance tending to exclude orbiting bodies from a
  narrow orbital region.
\end{itemize}

When mapping resonant gap orbits, use the bracketing orbits as
\textbf{perturbers}. For each perturber, resonances can be found by
scaling the orbital period with small integer ratios and converting back
to distance with Kepler's Third Law (\(P^2 \propto a^3\)).

\begin{enumerate}
\def\labelenumi{\arabic{enumi}.}
\tightlist
\item
  When both perturbers have equal mass (\(m_i = m_o\)) calculate
  resonance orbits only from the mass and orbital distance of the outer
  perturber.
\item
  When the perturbers have differenting masses:

  \begin{itemize}
  \tightlist
  \item
    If their mass ratio ≤ 10:1, calculate resonance orbits for both
    inner and outer perturbers and pairwise-average them.
  \item
    If their mass ratio \textgreater{} 10:1, calculate resonance orbits
    using only the mass and orbital distance of the more massive
    perturber (regardless of location). \#\#\# Inner Orbit Resonances
    Using the \textbf{inner perturber} with period \(P_i\).\\
    \[
    P_x = P_i \times k \quad \text{Where: } k \in \{1.67, 2.00, 2.25, 2.33, 2.50, 2.67, 3.00, 3.50, 4.00, 5.00\}
    \]\[
    a_x = \sqrt[3]{P_x^2 \, M\odot}
    \] Where:
  \end{itemize}
\end{enumerate}

\begin{itemize}
\tightlist
\item
  \(P_x\) = resonant period\\
\item
  \(a_x\) = resonant distance (AU)\\
\item
  \(M\odot\) = stellar mass (in solar units)\\
\item
  \(k\) = resonance scaler (see below for details)
\end{itemize}

\textbf{\emph{Keep only values where \(a_x > B_i\) (beyond the inner
orbit).}}\\
\#\#\# Outer Orbit Resonances Using from the \textbf{outer perturber}
with period \(P_o\).\\
\[
P_x = \frac{P_o}{k} \quad \text{Where: } k \in \{1.67, 2.00, 2.25, 2.33, 2.50, 2.67, 3.00, 3.50, 4.00, 5.00\}
\]\[
a_x = \sqrt[3]{P_x^2 \, M\odot}
\]Where: - \(P_x\) = resonant period\\
- \(a_x\) = resonant distance (AU)\\
- \(M\odot\) = stellar mass (in solar units)\\
- \(k\) = resonance scaler

\textbf{\emph{Keep only values where \(a_x < B_o\) (inside the outer
orbit).}}\\
\#\#\#\# Combined Equation Forms:

\paragraph{Inner Orbit Resonances}\label{inner-orbit-resonances}

\[
\begin{align}
a_x &= \sqrt[3]{\Big((P_i \times k)^2 \, M\odot\Big)} \\[0.5em]
 \text{Where: } k &\in \{1.67, 2.00, 2.25, 2.33, 2.50, 2.67, 3.00, 3.50, 4.00, 5.00\}
\end{align}
\] \#\#\#\#\# Outer Orbit Resonances \[
\begin{align}
a_x &= \sqrt[3]{\left(\frac{P_o}{k}\right)^2 M\odot} \\[0.5em]
 \text{Where: } k &\in \{1.67, 2.00, 2.25, 2.33, 2.50, 2.67, 3.00, 3.50, 4.00, 5.00\}
 \end{align}
\] \#\#\# Details: Resonance Scalers \emph{Sorted in order of frequency}
\#\#\#\# Gap Resonances (Gresonances)

\begin{longtable}[]{@{}
  >{\centering\arraybackslash}p{(\linewidth - 10\tabcolsep) * \real{0.0842}}
  >{\centering\arraybackslash}p{(\linewidth - 10\tabcolsep) * \real{0.1053}}
  >{\centering\arraybackslash}p{(\linewidth - 10\tabcolsep) * \real{0.0842}}
  >{\centering\arraybackslash}p{(\linewidth - 10\tabcolsep) * \real{0.0737}}
  >{\centering\arraybackslash}p{(\linewidth - 10\tabcolsep) * \real{0.0526}}
  >{\centering\arraybackslash}p{(\linewidth - 10\tabcolsep) * \real{0.6000}}@{}}
\toprule\noalign{}
\begin{minipage}[b]{\linewidth}\centering
Notes
\end{minipage} & \begin{minipage}[b]{\linewidth}\centering
Perturbant
\end{minipage} & \begin{minipage}[b]{\linewidth}\centering
Resonant
\end{minipage} & \begin{minipage}[b]{\linewidth}\centering
Ratio
\end{minipage} & \begin{minipage}[b]{\linewidth}\centering
Order
\end{minipage} & \begin{minipage}[b]{\linewidth}\centering
Frequency\(\tfrac{\text{Perturbant}}{\text{Resonant}}\)
\end{minipage} \\
\midrule\noalign{}
\endhead
\bottomrule\noalign{}
\endlastfoot
4 & 5 & 3 & 5:3 & 2 & 1.67 \\
\textbf{3, 4} & \textbf{2} & \textbf{1} & \textbf{2:1} & \textbf{1} &
\textbf{2.00} \\
& 9 & 4 & 9:4 & 5 & 2.25 \\
4 & 7 & 3 & 7:3 & 4 & 2.33 \\
4 & 5 & 2 & 5:2 & 3 & 2.50 \\
& 8 & 3 & 8:3 & 5 & 2.67 \\
4 & 3 & 1 & 3:1 & 2 & 3.00 \\
& 7 & 2 & 7:2 & 5 & 3.50 \\
4 & 4 & 1 & 4:1 & 3 & 4.00 \\
& 5 & 1 & 5:1 & 4 & 5.00 \\
\end{longtable}

\subsubsection{Trap Resonances
(Tresonances)}\label{trap-resonances-tresonances}

\begin{longtable}[]{@{}
  >{\centering\arraybackslash}p{(\linewidth - 10\tabcolsep) * \real{0.0556}}
  >{\centering\arraybackslash}p{(\linewidth - 10\tabcolsep) * \real{0.1111}}
  >{\centering\arraybackslash}p{(\linewidth - 10\tabcolsep) * \real{0.0889}}
  >{\centering\arraybackslash}p{(\linewidth - 10\tabcolsep) * \real{0.0556}}
  >{\centering\arraybackslash}p{(\linewidth - 10\tabcolsep) * \real{0.0556}}
  >{\centering\arraybackslash}p{(\linewidth - 10\tabcolsep) * \real{0.6333}}@{}}
\toprule\noalign{}
\begin{minipage}[b]{\linewidth}\centering
Notes
\end{minipage} & \begin{minipage}[b]{\linewidth}\centering
Perturbant
\end{minipage} & \begin{minipage}[b]{\linewidth}\centering
Resonant
\end{minipage} & \begin{minipage}[b]{\linewidth}\centering
Ratio
\end{minipage} & \begin{minipage}[b]{\linewidth}\centering
Order
\end{minipage} & \begin{minipage}[b]{\linewidth}\centering
Frequency\(\tfrac{\text{Perturbant}}{\text{Resonant}}\)
\end{minipage} \\
\midrule\noalign{}
\endhead
\bottomrule\noalign{}
\endlastfoot
4 & 4 & 3 & 4:3 & 1 & 1.33 \\
4 & 3 & 2 & 3:2 & 1 & 1.50 \\
& 5 & 4 & 5:4 & 1 & 1.25 \\
& 6 & 5 & 6:5 & 1 & 1.20 \\
& 7 & 6 & 7:6 & 1 & 1.17 \\
\end{longtable}

\begin{quote}
\textbf{Notes:} 1. The higher the resonant order, the weaker the
resonance. 2. It is worth noting that all gap resonances have
frequencies \textgreater{} 1.50 and all trap resonances have frequencies
≤ 1.50. 3. Resonance 2:1 is the only first-order gap resonance, though
it can, under certain circumstances, behave as a trap resonance. 4.
These resonance ratios are all observed in the Solar System asteroid
belt, though the framework is generalizable to other systems.
\end{quote}

\begin{quote}
\textbf{Usage:} Preferentially choose resonance ratios with order 1 or 2
whenever possible: resonances of order 1 or 2 will have the strongest
dynamical signatures (clear gaps or stable traps). Higher orders may be
used sparingly to add fine structure, but their effects will be much
weaker.
\end{quote}

\chapter{Width of Gaps}\label{width-of-gaps}

\section{Justification}\label{justification}

For a resonance gap in an asteroid belt to be significant, it should be
≥ 0.1\% of the orbital radius of the gap, itself. \[
\begin{gather}
\frac{m_p}{M_*} \geq 10^{-6} \\
\therefore m_p \geq \frac{M_*}{10^6}
\end{gather}
\] Where: - \(m_{p}\) = the mass of the perturber in Terrans - \(M_*\) =
the mass of the star in Terrans - = 333000 × the mass of the star in
solar units (\(M_\odot\))

\subsection{Example:}\label{example-5}

Given: \(M_* = 333000 M_⨁ = 333000(1)\) (for the Sun): \[
m_p \geq \frac{333000}{10^6} = 0.333 m_⨁
\] \textbf{Meaning:} - A body must be \textbf{at least 0.333 Earth
masses} (\textasciitilde⅓⨁) to carve a \textbf{recognizable Kirkwood
gap} in a main-belt analogue. - \textbf{Below this} (midimons, small
planemons), the ``gap width'' is so narrow it doesn't register as a true
Kirkwood void. \#\# Calculating Gap Widths For each resonant orbit
calculated above, calculate a gap width by: \[
g_w = a \times \sqrt{\frac{m_i + m_o}{333000M⊙}} \qquad \text{Preferred method}
\] or \[
\begin{gather}
g_{quad} = \sqrt{g_i^2 + g_o^2} \qquad \text{Requires calculating $g_i$ and $g_o$ first by:} \\[1em]
g_i = a \times \sqrt{\frac{m_i}{333000M⊙}} \quad\text{and}\quad g_o = a \times \sqrt{\frac{m_o}{333000M⊙}}
\end{gather}
\] Both methods are algebraically equivalent: the \(g_{quad}\) form
expands \textbf{\emph{exactly}} into the \(g_w\) expression: \[
g_{quad} = \sqrt{g_i^2 + g_o^2} = \sqrt{\left(a^2 \frac{m_i}{M_*}\right) + \left(a^2 \frac{m_o}{M_*}\right)} = a \times \sqrt{\frac{m_i + m_o}{M_*}} = g_w \; \checkmark
\] \#\# Abstract \textbf{Major Topics:}\\
- Harmonic period as a synchronization interval of two cycles.\\
- Equivalence to the synodic period formula.\\
- Application to Earth's solar day vs.~sidereal day (≈ tropical year).\\
- Application to Earth's solar day vs.~stellar day (≈ sidereal year).

\textbf{Key Terms \& Symbols:}\\
- H = harmonic period\\
- P₁ = solar day\\
- P₂ = sidereal day / stellar day\\
- Synodic period (noted as mathematically identical).

\textbf{Cross-Check Notes:}\\
- Bridges the concepts of daily cycles and year-length periods.\\
- Explicitly links harmonic period to \emph{tropical} and \emph{sidereal
year} definitions.\\
- Potential overlaps with glossary entries on \emph{solar day},
\emph{sidereal day}, \emph{stellar day}, and \emph{synodic period}.\\
- Candidate staging milestone: Glossary v1.215 (Harmonic Period added).

\emph{Harmonic periods} are crucial for understanding how a planet's
rotational and orbital cycles synchronize. The harmonic period \(H\) is
the time interval over which the two independent motions align in their
periodicity. \[
H = \dfrac{P_1 \times P_2}{|P_1 - P_2|}
\] Where: - \(P_1\) = length of the solar day - \(P_2\) = length of the
sidereal day - \(H\) = harmonic period

\begin{quote}
Yes, this is actually the same equation as the synodic period.
\end{quote}

\section{Example}\label{example-6}

Using Earth's solar day and sidereal day: \[
\begin{array}{l l}
P_1 = 86400^s &\text{solar day}\\
P_2 = 86164.090531^s &\text{sidereal day}
\end{array}
\] \[
\begin{equation}
\begin{split}
H &= \dfrac{P_1 \times P_2}{|P_1 - P_2|} \\[0.5em]
&= \dfrac{86400 \times 86164.090531}{|86400 - 86164.090531|} \\[0.5em]
&= \dfrac{7444577422}{235.9094692} \\[0.5em]
H &= 31556924.9854376^s\; \checkmark
\end{split}
\end{equation}
\] \ldots{} we find that the harmonic period between the solar day and
the sidereal day is approximately the length of the \emph{tropical
year}, differing from the official value of \(31556925.216^s\) by an
excess of only \(0.2306^s\).

Similarly, calculating the harmonic period between Earth's solar day and
the \emph{stellar day}:

\[
\begin{array}{l l}
P_1 = 86400^s &\text{solar day}\\
P_2 = 86164.0989^s &\text{stellar day}
\end{array}
\] \[
\begin{equation}
\begin{split}
H &= \dfrac{P_1 \times P_2}{|P_1 - P_2|} \\[0.5em]
&= \dfrac{86400 \times 86164.0989}{|86400 - 86164.0989|} \\[0.5em]
&= \dfrac{7444578145}{235.901096} \\[0.5em]
H &= 31558048.1047344^s\; \checkmark
\end{split}
\end{equation}
\] \ldots{} a difference of only \(101.65737^s\) longer than the
official length of the \emph{sidereal year}: \(31558149.7635456^s\).

\section{Abstract}\label{abstract-27}

\textbf{Major Topics:}\\
- Establishes a \textbf{symbolic system} for generating orbital radii
procedurally via multiplicative steps.\\
- Defines \textbf{intrabasal} (inward from baseline) and
\textbf{extrabasal} (outward from baseline) orbit generation.\\
- Uses \textbf{basal orbital radius (B)} and \textbf{system cutoff (Ω)}
as anchors, with optional use of nucleal orbit (𝒩) as B.\\
- Expresses orbital placement through \textbf{randomized multiplicative
factors} (⟨⟨min ∧ max⟩⟩).\\
- Describes strategies: outward-only, inward-only, or bidirectional
scaffolding from a central anchor.

\textbf{Key Terms \& Symbols:}\\
- \textbf{Intrabasal {[}EXPANDED neo{]}:} Now canonized as any orbit
inside a basal reference (broader than just calculations).\\
- \textbf{Extrabasal {[}EXPANDED neo{]}:} Any orbit outside a basal
reference.\\
- \textbf{B (Basal orbital radius) {[}NEW ins{]}.}\\
- \textbf{Ω (System cutoff) {[}NEW ins{]}.}\\
- Random assignment operator (⟨⟨ ⟩⟩) {[}ins{]}.

\textbf{Cross-Check Notes:}\\
- \textbf{{[}EXPANDED{]}} Intrabasal/Extrabasal promoted from
calculation-only to general relational terms.\\
- \textbf{{[}NEW{]}} Basal radius (B) and cutoff (Ω).\\
- No conflicts with existing canon.\\
- Extends \emph{Range Constraints \& Random Assignment}; best read
together.

\chapter{Orbit Randomization
Notation}\label{orbit-randomization-notation}

This symbolic system defines how to procedurally generate a sequence of
orbital radii using randomized multiplicative (or divisive) steps from a
baseline (\textbf{basal}) value. It distinguishes between
\textbf{intrabasal} (moving inward) and \textbf{extrabasal} (moving
outward) orbit generation.

The notation is fully symbolic and compatible with WCB's randomization
and range assignment grammar.

\textbf{Intrabasal} \[
r_i = B;\; \Omega = \text{«▢»}: \qquad
r_{i-1} = \frac{r_i}{⟨⟨ \text{min} ∧ \text{max} ⟩⟩}
\quad \text{while } r_i \geq \Omega
\]

\textbf{Extrabasal} \[
r_i = B;\; \Omega = \text{«▢»}: \qquad
r_{i+1} = r_i \cdot ⟨⟨ \text{min} ∧ \text{max} ⟩⟩
\quad \text{while } r_i \leq \Omega
\] Where: - B = basal orbital radius (e.g.~the nucleal orbit
\(\mathcal{N}\)) - Ω = orbital distance cuttoff (minimum or maximum
allowed orbit based on the star system constraints)

\section{🔄 Usage Strategy}\label{usage-strategy-1}

The \textbf{intrabasal} and \textbf{extrabasal} forms can be used
independently depending on your desired anchoring strategy:

\begin{itemize}
\tightlist
\item
  🧭 \textbf{Outward-Only Generation}\\
  If you begin at the \textbf{innermost permissible orbit} (e.g.~a
  thermal, Roche, or design constraint), use the \textbf{extrabasal}
  form to expand outward via multiplicative steps: \[
  r_0 = «inner limit»;\; L = «system edge»:
  \quad r_{i+1} = r_i \cdot ⟨⟨min ∧ max⟩⟩, \text{ while } r_{i+1} \leq L
  \]
\item
  🧭 \textbf{Outward-Only Generation}\\
  If you begin at the \textbf{outermost permissible orbit} (e.g.~a
  thermal or design constraint), use the \textbf{extrabasal} form to
  expand outward via multiplicative steps: \[
  r_0 = «inner limit»;\; L = «system edge»:
  \quad r_{i-1} = \dfrac{r_i} {⟨⟨min ∧ max⟩⟩}, \text{ while } r_{i+1} \leq L
  \]
\end{itemize}

\begin{quote}
Either method can fully define a system --- or both can be combined with
a central anchor (e.g.~nucleal orbit) to scaffold a bidirectional
structure. \#\# Abstract \textbf{Major Topics:}\\
- Orbital eccentricity and its impact on planemon--star systems.\\
- Periastron (Rₘᵢₙ) and apastron (Rₘₐₓ) distances.\\
- Fractional Distance Asymmetry (Ḋ) as a measure of orbital skew.\\
- Flux Ratio (Fₘᵢₙ/Fₘₐₓ) and insolation contrast.\\
- Climatic implications of eccentricity-driven flux differences.
\end{quote}

\textbf{Key Terms \& Symbols:}\\
- 𝒜 = average orbital separation (semimajor axis).\\
- e = orbital eccentricity.\\
- Rₘᵢₙ, Rₘₐₓ = periastron and apastron
distances:contentReference{oaicite:0}.\\
- Ḋ = fractional distance asymmetry:contentReference{oaicite:1}.\\
- Fₘᵢₙ/Fₘₐₓ = flux ratio (climatic effect):contentReference{oaicite:2}.

\textbf{Cross-Check Notes:}\\
- Canonical terminology: \emph{periastron/apastron} with Rₘᵢₙ/Rₘₐₓ for
star--planemon systems.\\
- Deprecated symbols: ß, rₚ, rₐ (legacy only).\\
- Ḋ introduced in Glossary v1.211 as preferred WCB metric.\\
- Overlaps conceptually with orbital design/insolation notes; interacts
with climate/habitability discussions.

\chapter{Orbital Eccentricity and Seasonal
Effects}\label{orbital-eccentricity-and-seasonal-effects}

For a planemon orbiting a star (M₂ ⋘ M₁): - \textbf{Periastron
distance}:\\
\[
 R_{min} = \mathcal{A}(1 - e)
\] - \textbf{Apastron distance}:\\
\[
 R_{max} = \mathcal{A}(1 + e)
\] Where \textbf{𝒜} is the \emph{average orbital separation} between the
bodies.\\
When describing a planemon's orbit, 𝒜 corresponds to the
\textbf{semimajor axis} of its elliptical path.\\
\#\# Fractional Distance Asymmetry (Ḋ) The dimensionless measure of how
much closer the planemon is at periastron than at apastron: \[
\dot{D} = \frac{R_{max}}{R_{min}} - 1 = \frac{2e}{1-e}
\] - Earth (\(e = 0.0167\)):\\
\(\dot{D} \approx 0.034\) → \textbf{3.4\% closer at periastron}\\
- Rosetta (\(e = 0.05\)):\\
\(\dot{D} \approx 0.105\) → \textbf{10.5\% closer at periastron}

\section{Flux Ratio}\label{flux-ratio}

Because stellar flux \(F ∝ 1/R^2\), the insolation contrast between
periastron and apastron is: \[
\frac{F_{min}}{F_{max}} = \left(\frac{R_{max}}{R_{min}}\right)^2
\] - Earth (\(e = 0.0167\)):\\
\(\dfrac{F_{min}}{F_{max}} \approx 1.068\) → \textbf{6.8\% stronger
insolation at periastron}\\
- Rosetta (\(e = 0.05\)):\\
\(\dfrac{F_{min}}{F_{max}} \approx 1.23\) → \textbf{23\% stronger
insolation at periastron}

\section{WCB Note}\label{wcb-note}

\begin{itemize}
\tightlist
\item
  \textbf{Distance form (Ḋ)} → intuitive ``how much closer/farther''
  language.\\
\item
  \textbf{Flux form (\(F_{min}/F_{max}\))} → climatic impact (``how much
  hotter/colder'').\\
\item
  \textbf{Canonical terminology}: use \emph{periastron/apastron} with
  \emph{Rₘᵢₙ}/\emph{Rₘₐₓ} in star--planemon systems.\\
\item
  Avoid \(r_p\), \(r_a\), or ß notations; these are
  legacy/derivation-only.\\
\item
  In WCB canon, Ḋ is the preferred symbol for distance asymmetry. \#\#
  Abstract\\
  \textbf{Major Topics:}\\
\item
  Provides a \textbf{qualitative, sky-map based method} for estimating
  when the Sun passes behind planetary rings.\\
\item
  Uses \textbf{ring arcs (fixed for latitude)} and the \textbf{Sun's
  declination path (variable by apical chronum)} to identify shadow
  events.\\
\item
  Outlines shadow scenarios:

  \begin{itemize}
  \tightlist
  \item
    No occlusion.\\
  \item
    Partial occlusion (single crossing).\\
  \item
    Double crossing.\\
  \item
    Special equator--equinox case (Sun aligned with ring plane all day,
    but negligible effect due to Sun's disc width).\\
  \end{itemize}
\item
  Explains how \textbf{timing of shadow events} drifts seasonally:
  stable near solstices, rapid shifts near equinoxes.\\
\item
  Discusses \textbf{climate implications} of ring shading: reduced
  insolation, altered convection, possible cooling effects, and role as
  ``life-saving parasols.''\\
\item
  Stresses limits of rigor: exact modeling requires advanced orbital
  climatology; this guide provides \textbf{worldbuilding-level
  approximations}.
\end{itemize}

\textbf{Key Terms \& Symbols:}\\
- \textbf{Apical chronum {[}neo{]}.}\\
- \textbf{Diurn {[}neo{]}.}\\
- \textbf{Declination (δ) {[}sci{]}.}\\
- \textbf{Latitude (λ) {[}sci{]}.}

\textbf{Cross-Check Notes:}\\
- Builds directly on \textbf{Calculating Stellar Sky Paths}: applies its
solar path framework to ring--Sun interactions.\\
- Reinforces canonical use of \textbf{apical chronum} and
\textbf{diurn}.\\
- Uses standard astronomical terms: \textbf{declination, latitude}.\\
- No new symbolic conventions introduced.\\
- \textbf{Status:} {[}EXPANDED{]} --- broadens existing canon by
applying sky-path methods to planetary ring shading and climate effects.

\subsection{\texorpdfstring{\textbf{Quick Method for Ring Shadows
(Approximation)}}{Quick Method for Ring Shadows (Approximation)}}\label{quick-method-for-ring-shadows-approximation}

\emph{This section presents a simplified, diagram-based way to estimate
when and how the Sun may pass behind a planet's rings. It avoids complex
spherical trigonometry by using sky maps and focuses on qualitative
visualization.}

When considering ring shadows on a terrestrial planet, the exact math
quickly becomes too complex for this guide. However, you can
\textbf{approximate} the effect using a sky-map method:

\begin{enumerate}
\def\labelenumi{\arabic{enumi}.}
\tightlist
\item
  \textbf{Choose observer latitude ((\lambda)):}

  \begin{itemize}
  \tightlist
  \item
    For that latitude, the \textbf{ring arcs (inner and outer edges)}
    are constant in the sky. They do \emph{not} change with season.
  \end{itemize}
\item
  \textbf{Plot the ring:}

  \begin{itemize}
  \tightlist
  \item
    Draw the inner and outer arcs of the ring plane on a flattened sky
    chart (altitude vs.~azimuth).
  \end{itemize}
\item
  \textbf{Plot the Sun's path for the day:}

  \begin{itemize}
  \tightlist
  \item
    The Sun's diurnal arc depends on its \textbf{declination ((\delta))}
    for that day in the apical chronum.\\
  \item
    This sets its rising/setting azimems and maximum noon altitude.
  \end{itemize}
\item
  \textbf{Look for intersections:}

  \begin{itemize}
  \tightlist
  \item
    Where the Sun's path crosses the inner/outer ring arcs = entry/exit
    points into ring shadow.
  \end{itemize}
\item
  \textbf{Estimate time in shadow:}

  \begin{itemize}
  \tightlist
  \item
    The \textbf{horizontal (azimuthal) component} of these intersection
    points tells you what \textbf{fraction of the diurn} the Sun spends
    behind the rings.\\
  \item
    The \textbf{vertical (altitude) component} confirms whether the Sun
    actually overlaps the ring band at all.
  \end{itemize}
\end{enumerate}

\textbf{Key Insight:}\\
- \textbf{Ring placement/width = fixed} for a given latitude.\\
- \textbf{Sun's path = variable} with declination.\\
- \textbf{Shadow events} = when/where those two systems overlap.

This gives a practical, diagram-based way to visualize daily ring shadow
effects \emph{without} diving into spherical trig.

\subsection{\texorpdfstring{\textbf{Ring Shadow
Scenarios}}{Ring Shadow Scenarios}}\label{ring-shadow-scenarios}

\emph{Here we outline the possible daily patterns of ring-shadow events
(none, partial, double crossing, special equator--equinox case). These
scenarios are the observable outcomes of the Quick Method.}

The following scenarios illustrate how the Sun's path and the fixed ring
arcs can interact, producing different shadowing outcomes through the
day:

\begin{enumerate}
\def\labelenumi{\arabic{enumi}.}
\tightlist
\item
  \textbf{No Occlusion}

  \begin{itemize}
  \tightlist
  \item
    The Sun's daily path lies entirely outside the ring band.\\
  \item
    The Sun rises and sets south (or north) of the rings and never
    passes behind them.
  \end{itemize}
\item
  \textbf{Partial Occlusion}

  \begin{itemize}
  \tightlist
  \item
    The Sun's path intersects the ring band once.\\
  \item
    The Sun rises outside the rings, passes behind them for part of the
    day, then emerges before setting.
  \end{itemize}
\item
  \textbf{Double Crossing}

  \begin{itemize}
  \tightlist
  \item
    The Sun's path passes through the ring band twice.\\
  \item
    The Sun rises outside the rings, passes behind them, emerges above
    them near noon, and then passes behind again before setting.
  \end{itemize}
\item
  \textbf{Never Entirely Occluded}

  \begin{itemize}
  \tightlist
  \item
    Because the ring arcs converge to points at east and west horizons,
    the Sun always begins and ends the day outside the rings.\\
  \item
    Thus, the Sun cannot be fully hidden for the entire diurn under
    normal circumstances.
  \end{itemize}
\end{enumerate}

\subsection{\texorpdfstring{\textbf{Special Case: Equator at
Equinox}}{Special Case: Equator at Equinox}}\label{special-case-equator-at-equinox}

\begin{itemize}
\tightlist
\item
  At the planetary equator during equinoxes, the Sun's path coincides
  with the ring plane.\\
\item
  The Sun rises due east, passes overhead at zenith, and sets due west
  along the same great circle as the rings.\\
\item
  In this alignment, the Sun is ``behind'' the ring plane all day.\\
\item
  However, since the rings have a much narrower angular width on the sky
  than the Sun's disc, the effect is negligible: no significant dimming
  or shadowing would be seen by surface observers.
\end{itemize}

\subsection{\texorpdfstring{\textbf{Ring Shadow Climate
Effects}}{Ring Shadow Climate Effects}}\label{ring-shadow-climate-effects}

\emph{Dense or opaque rings can noticeably reduce insolation at the
surface. This section considers how such shading might alter daily
temperatures, weather cycles, and even settlement patterns, without
attempting full climate modeling.}

Where rings are dense enough, these shadow scenarios can affect not just
what the sky looks like, but also how much solar energy reaches the
ground. If a planetary ring is dense or optically thick, its shadow can
reduce the amount of sunlight reaching the surface whenever the Sun
passes behind it. The severity of the effect depends on how much of the
Sun's disc is obscured and for how long.

\subsubsection{\texorpdfstring{\textbf{At the Subsolar
Point}}{At the Subsolar Point}}\label{at-the-subsolar-point}

\begin{itemize}
\tightlist
\item
  Shadowing is most direct when the Sun is highest in the sky.\\
\item
  Significant daily shading here could:

  \begin{itemize}
  \tightlist
  \item
    Lower peak surface temperatures.\\
  \item
    Delay or reduce convection cycles (e.g., thunderstorm formation,
    monsoon timing).\\
  \item
    Alter evaporation rates and precipitation.
  \end{itemize}
\end{itemize}

\subsubsection{\texorpdfstring{\textbf{Seasonal \& Locational
Consequences}}{Seasonal \& Locational Consequences}}\label{seasonal-locational-consequences}

\begin{itemize}
\tightlist
\item
  The Sun's path shifts with the \textbf{apical chronum} (season).\\
\item
  Some regions may experience \textbf{daily shading during summer},
  providing relief from heat.\\
\item
  Other regions may be \textbf{unshaded in winter} when the Sun's path
  lies outside the ring band.\\
\item
  Higher latitudes may see the shadow zone migrate poleward or
  equatorward with the seasons.
\end{itemize}

\subsubsection{\texorpdfstring{\textbf{Life-Saving
Respite}}{Life-Saving Respite}}\label{life-saving-respite}

In especially hot climates, rings could act as natural
\textbf{parasols}.\\
- Example: A city in a desert belt might be survivable only because,
during the hottest months, the Sun passes behind the rings from late
morning until early afternoon each day.\\
- Such shading could literally determine where civilizations thrive.

\subsubsection{\texorpdfstring{\textbf{Complexity
Warning}}{Complexity Warning}}\label{complexity-warning}

Modeling this rigorously requires:\\
- Optical depth of the rings.\\
- Fraction of the Sun's disc blocked during occultation.\\
- Duration of shadow events.\\
- Seasonal variation of these durations.\\
- Climate feedback (albedo, heat storage, cloud dynamics).

This is \textbf{well beyond the scope} of this guide. For worldbuilding
purposes, simply note:

\begin{quote}
Rings can alter daily and seasonal insolation patterns, sometimes
dramatically.\\
Authors may invoke rings as life-saving parasols, or as sources of
seasonal hardship.
\end{quote}

\subsection{\texorpdfstring{\textbf{Timing Behavior of Ring
Shadows}}{Timing Behavior of Ring Shadows}}\label{timing-behavior-of-ring-shadows}

\emph{Ring-shadow events happen at predictable times of day that drift
seasonally. This section explains how those timings remain stable near
solstices and shift quickly near equinoxes, echoing the familiar pattern
of sunrise and sunset changes.}

The qualitative outcomes above happen at specific times of day. These
timings drift in a predictable seasonal pattern as the Sun's declination
changes:

\begin{itemize}
\tightlist
\item
  The \textbf{ring arcs} are fixed in the sky for any given latitude.\\
\item
  The \textbf{Sun's path} shifts gradually with the apical chronum
  (season), so the entry/exit times of shadow events drift over the
  year.
\end{itemize}

\textbf{Seasonal Pattern:}\\
- \textbf{Near solstices:} Declination changes slowly → shadow timings
remain nearly constant, varying by only a minute or two over several
weeks.\\
- \textbf{Near equinoxes:} Declination changes rapidly → shadow timings
shift more noticeably, sometimes by several minutes per day.

This mirrors the familiar seasonal drift of sunrise and sunset times.

\subsection{\texorpdfstring{\textbf{Reality
Check}}{Reality Check}}\label{reality-check}

\emph{Here we step back to emphasize limits. While ring shadows can be
imagined, described, and even mapped in principle, exact calculations
require advanced modeling. This guide provides qualitative tools and
approximations for worldbuilding, not rigorous orbital climatology.}

Ring shadows are:\\
- \textbf{Imaginable} → the geometry can be pictured.\\
- \textbf{Describable} → qualitative scenarios (no shadow, partial
shadow, double crossing, equator--equinox exception) are clear.\\
- \textbf{Theoretically mappable} → with spherical trig and patience,
one could chart exact entry/exit times for any latitude, date, and ring
geometry.

But this is \textbf{not back-of-the-envelope work.}\\
- It is \textbf{not coffee-break math.}\\
- It requires \textbf{dedicated modeling} (great-circle geometry +
spherical trigonometry + seasonal solar motion).

\textbf{Worldbuilding takeaway:}\\
Use the qualitative scenarios and the Quick Method approximation for
storytelling. Exact computations belong to advanced orbital mechanics or
climate modeling, not casual reference.

\chapter{Abstract}\label{abstract-28}

\textbf{Major Topics:}\\
- Methods for estimating seasonal lengths on eccentric orbits.\\
- Obliquity azimuth (φ) as orientation marker for seasons.\\
- Sinusoidal approximation method (fast vs.~slow half of orbit).\\
- Fudge factor (f = 10e) to break paired-season symmetry.\\
- Kepler's exact method (eccentric anomaly → mean anomaly → season
fractions).\\
- Worked examples: Earth and Rosetta.

\textbf{Key Terms \& Symbols:}\\
- C = sidereal chronum (WCB-specific orbital year length).\\
- P = generic orbital period (not used here, but standard astrophysical
symbol).\\
- e = orbital eccentricity.\\
- φ = obliquity azimuth.\\
- ν = true anomaly (season midpoint).\\
- Δt = seasonal length.\\
- f = fudge factor (10e).\\
- E = eccentric anomaly, M = mean anomaly.

\textbf{Cross-Check Notes:}\\
- Canonical obliquity azimuth φ (glossary v0.4+).\\
- Sinusoidal method is \textbf{SANC} (Simple, Approximate, Notationally
Clear).\\
- Fudge factor optional; gives four distinct season lengths when
desired.\\
- Kepler method more exact; captures asymmetry without fudge.\\
- Overlaps with \textbf{Orbital Eccentricity and Seasonal Effects.md}
(flux and climate implications).\\
- Important: WCB distinguishes between generic orbital period (P) and
sidereal chronum (C).\\
-

\chapter{📖 Season-Length Estimation
Methods}\label{season-length-estimation-methods}

This process assumes that you have already determined the duration of
your planet's orbit around its star (its \emph{sidereal chronum},
\(C\)). \#\# Obliquity azimuth (\(\phi\)) The \emph{obliquity azimuth}
of your planet's obliquity is determined by the point in its orbit when
the northern hemisphere is tilted directly away from the star. If your
planet's northern hemisphere is tilted away from the star when it is at
the closest point in its orbit (its \emph{periastron}), then its
obliquity azimuth is \(\phi = 0\).

\begin{longtable}[]{@{}
  >{\raggedleft\arraybackslash}p{(\linewidth - 2\tabcolsep) * \real{0.0638}}
  >{\raggedright\arraybackslash}p{(\linewidth - 2\tabcolsep) * \real{0.9362}}@{}}
\toprule\noalign{}
\begin{minipage}[b]{\linewidth}\raggedleft
\(\phi\)
\end{minipage} & \begin{minipage}[b]{\linewidth}\raggedright
Orientation
\end{minipage} \\
\midrule\noalign{}
\endhead
\bottomrule\noalign{}
\endlastfoot
0 & Northern hemisphere tilted directly away from star at periastron \\
90 & Northern hemisphere tilted directly away one-fourth orbit
\emph{after} periastron \\
180 & Northern hemisphere tilted directly away one-half orbit after
periastron (at \emph{apastron}) \\
270 & Northern hemisphere tilted directly away three-fourths orbit after
periastron \\
\end{longtable}

\section{1. Sinusoidal Approximation (Fast vs.~Slow
Half)}\label{sinusoidal-approximation-fast-vs.-slow-half}

Here is a quick, algebra-only method that captures the \emph{main effect
of eccentricity} on seasons: \[
\Delta t \;\approx\; \dfrac{C}{4} + \left(\dfrac{2e}{\pi} C \times \sin \nu\right)
\] Where:\\
- \(C\) = year length (chronum, in diurns or days)\\
- \(e\) = orbital eccentricity\\
- \(\nu\) = central true anomaly of the season (0° = perihelion, 180° =
aphelion)\\
\#\#\# Step 1 --- Seasonal Baseline Length Equal quarter of the chronum:
\[
\dfrac{C}{4}
\] \#\#\# Step 2 --- Correction Factor This is a dimensionless ratio
determined by eccentricity: \[
\dfrac{2e}{\pi}
\] To get the actual correction in diurns, it is multiplied by the full
length of the chronum (\(C\)) in the main equation. \#\#\# Step 3 ---
Seasonal Adjustment True anomaly of each season midpoint is tied to the
obliquity azimuth \(\phi\): \[
\begin{align}
&\nu = (\phi + 90n) \bmod 360 \\[1em]
&\begin{cases}
n = 0 &\text{spring equinox} \\
n = 1 &\text{summer solstice} \\
n = 2 &\text{autumn equinox} \\
n = 3 &\text{winter solstice}
\end{cases}
\end{align}
\] \#\#\#\# Worked Example: Earth (Sinusoidal Approximation, Sidereal
Chronum) Given:\\
- \(\phi = 0\)\\
- \(C = 365.2564\) d (sidereal year)\\
- \(e = 0.0167\)

\textbf{Step 1 --- Baseline} \[
\frac{C}{4} = \frac{365.2564}{4} = 91.31
\] \textbf{Step 2 --- Correction Factor} \[
\frac{2e}{\pi} C = \frac{2 \times 0.0167}{\pi} \times 365.2564 = 3.88
\] \textbf{Step 3 --- Seasonal Adjustments (Earth)} True anomalies from
\(\phi = 0\): \[
\nu = (\phi + 90n) \bmod 360
\] \[
\begin{cases}
n = 0 & \nu = 270^\circ & \text{spring equinox} \\
n = 1 & \nu = 0^\circ & \text{summer solstice} \\
n = 2 & \nu = 90^\circ   & \text{autumn equinox} \\
n = 3 & \nu = 180^\circ  & \text{winter solstice}
\end{cases}
\] \textbf{Step 4 --- Apply the Formula} \[
\Delta t \;\approx\; \frac{C}{4} + \left(\frac{2e}{\pi} C \times \sin\nu\right)
\] \[
\begin{cases}
n=0;\;\nu=270^\circ;\;\sin\nu=1 & \Delta t_\text{spring} \approx 91.31 + 3.88 = 95.20 \\
n=1;\;\nu=0^\circ;\;\sin\nu=0 & \Delta t_\text{summer} \approx 91.31 \\
n=2;\;\nu=90^\circ;\;\sin\nu=-1 & \Delta t_\text{autumn} \approx 91.31 - 3.88 = 87.43 \\
n=3;\;\nu=180^\circ;\;\sin\nu=0 & \Delta t_\text{winter} \approx 91.31
\end{cases}
\] \textbf{Result:}\\
- Spring ≈ 95.2 d\\
- Summer ≈ 91.3 d\\
- Autumn ≈ 87.4 d\\
- Winter ≈ 91.3 d\\
(Total = 365.2564 d)

\textbf{Conclusion} The observed lengths of Earth's seasons are
approximately: - Spring ≈ 92.8 d\\
- Summer ≈ 93.6 d\\
- Autumn ≈ 89.9 d\\
- Winter ≈ 89.0 d \textbf{Errors:}\\
- Spring: +2.4 d\\
- Summer: --2.3 d\\
- Autumn: --2.5 d\\
- Winter: +2.3 d The method captures the \emph{overall pattern} (a
longer spring, a shorter autumn) but forces \textbf{two seasons to
pair}, which Earth does not actually do because its obliquity azimuth
\(\phi\) is offset from 0° by about 13°. \textbf{Advice}

When using the sinusoidal method, you have two options:

\begin{enumerate}
\def\labelenumi{\arabic{enumi}.}
\tightlist
\item
  \textbf{Accept the values as returned.}

  \begin{itemize}
  \tightlist
  \item
    This keeps the math simple and transparent.\\
  \item
    The approximation will always add up to the correct year length
    (\(C\)).\\
  \item
    Errors are usually small if \(e \leq 0.1\) (a few diurns at most for
    Earth-like orbits).
  \end{itemize}
\item
  \textbf{Apply the generalized fudge factor.}

  \begin{itemize}
  \tightlist
  \item
    To break the ``paired seasons'' pattern and create four distinct
    values, redistribute part of the correction term.\\
  \item
    Use the rule: \[
    f = 10e
    \]
  \end{itemize}
\end{enumerate}

\begin{itemize}
\tightlist
\item
  Subtract \(f \times \left(\tfrac{2e}{\pi}C\right)\) from the long
  seasons (those with \(\sin\nu = +1\)).\\
\item
  Add the same amount to the short seasons (those with
  \(\sin\nu = -1\)).\\
\item
  Baseline seasons (where \(\sin\nu = 0\)) remain unchanged.
\end{itemize}

\textbf{Why this works:}\\
- The fudge stays tied to orbital eccentricity, so it scales correctly
for different worlds.\\
- The method remains \textbf{SANC} --- \emph{Simple, Approximate,
Notationally Clear} --- while avoiding arbitrary ``just make something
up'' adjustments.

📖 \textbf{WCB Note:}\\
The sinusoidal shortcut is SANC --- Simple, Approximate, Notationally
Clear. It always gives you a consistent set of season lengths that add
to the right year. If you're working with a mildly eccentric orbit
(\(e \lesssim 0.03\)), you may want to apply the fudge factor \(f=10e\)
to nudge the values closer to reality and give you four distinct
seasons. If you're working with a strongly eccentric orbit, leave the
fudge aside --- the uneven seasons it predicts are already the truth of
the geometry. Ultimately, whether you `fudge' is up to you as the
worldbuilder: do you want clean numbers, or do you want raw extremes?
Both choices are defensible.

\subsubsection{Worked Example: Rosetta (Sinusoidal Approximation,
Sidereal
Chronum)}\label{worked-example-rosetta-sinusoidal-approximation-sidereal-chronum}

Given:\\
- \(\phi = 180^\circ\)\\
- \(C = 492\) diurns (sidereal chronum)\\
- \(e = 0.05\)\\
\textbf{Step 1 --- Baseline} \[
\frac{C}{4} = \frac{492}{4} = 123.0
\] \textbf{Step 2 --- Correction Factor} \[
\frac{2e}{\pi} C = \frac{2 \times 0.05}{\pi} \times 492 = 15.66
\] \textbf{Step 3 --- Seasonal Adjustments (Rosetta)} True anomalies
from \(\phi = 180^\circ\): \[
\nu = (\phi + 90n) \bmod 360
\] \[
\begin{cases}
n = 0 & \nu = 0^\circ & \text{spring equinox} \\
n = 1 & \nu = 90^\circ & \text{summer solstice} \\
n = 2 & \nu = 180^\circ   & \text{autumn equinox} \\
n = 3 & \nu = 270^\circ  & \text{winter solstice}
\end{cases}
\] \textbf{Step 4 --- Apply the Formula} \[
\Delta t \;\approx\; \frac{C}{4} + \left(\frac{2e}{\pi} C \times \sin\nu\right)
\]

\[
\begin{cases}
n=0;\;\nu=0^\circ;\;\sin\nu=-1 & \Delta t_\text{spring} \approx 123.0 - 15.66 = 107.3 \\
n=1;\;\nu=90^\circ;\;\sin\nu=0 & \Delta t_\text{summer} \approx 123.0 \\
n=2;\;\nu=180^\circ;\;\sin\nu=1 & \Delta t_\text{autumn} \approx 123.0 + 15.66 = 138.7 \\
n=3;\;\nu=270^\circ;\;\sin\nu=0 & \Delta t_\text{winter} \approx 123.0 \\
\end{cases}
\] \textbf{Result:}\\
- Spring ≈ 107.3 diurns\\
- Summer ≈ 123.0 diurns\\
- Autumn ≈ 138.7 diurns\\
- Winter ≈ 123.0 diurns (Total = 492 diurns)

\textbf{Conclusion} The sinusoidal approximation predicts for Rosetta: -
Two baseline seasons (winter, summer ≈ 123 diurns)\\
- One shorter season (spring ≈ 107 diurns)\\
- One longer season (autumn ≈ 139 diurns) - The difference between the
longest and shortest seasons is \textbf{over 31 diurns} --- nearly a
whole ``weke'' longer or shorter than the baseline! - This is much more
dramatic than Earth's ±2--3 day distortions, because Rosetta's orbital
eccentricity (\(e=0.05\)) is about \textbf{3× larger than Earth's}.
\textbf{Advice} - You can \textbf{accept these values directly}, which
already tell the story of a planet with noticeably uneven seasons.\\
- Or you can \textbf{tweak them} slightly (redistributing a fraction of
the correction term) to break the symmetry of the paired seasons and
produce four distinct values, closer to what the Kepler method would
return.\\
- Either way, the results remain \emph{SANC} --- Simple, Approximate,
Notationally Clear --- while giving Rosettan cultures a real seasonal
asymmetry to reckon with.

\section{2. Kepler's Exact Method (Eccentric Anomaly
Conversion)}\label{keplers-exact-method-eccentric-anomaly-conversion}

This more precise method uses full orbital geometry to derive season
lengths. \textbf{Key features:}\\
- Produces \textbf{true season lengths}, including asymmetries between
all four.\\
- If periastron lines up with a solstice or equinox, still produces
pairs (but corrected vs.~sinusoidal).\\
- If periastron is offset, yields \textbf{four distinct season lengths}
(like Earth's 93/94/90/89-day split).\\
- Requires a calculator, but no iteration --- just \texttt{arctan} and
\texttt{sin}. \#\#\# Step 1 --- Choose True Anomalies Pick the true
anomalies (\(\nu\)) for the four seasonal markers, offset by obliquity
azimuth \(\phi\): \[
\begin{align}
&\nu = (\phi + 90n) \bmod 360 \\[1em]
&\begin{cases}
n = 0; \;\text{spring equinox} \\
n = 1; \;\text{summer solstice} \\
n = 2; \;\text{autumn equinox} \\
n = 3; \;\text{winter solstice} \\
\end{cases}
\end{align}
\] \#\#\# Step 2 --- Convert to Eccentric Anomaly For each \(\nu\),
compute the eccentric anomaly \(E\): \[
E = 2 \arctan \!\left( \sqrt{\dfrac{1-e}{1+e}} \;\tan \dfrac{\nu}{2} \right)
\] \#\#\# Step 3 --- Convert to Mean Anomaly Translate \(E\) into mean
anomaly \(M\), which grows linearly with time: \[
M = E - e \sin E
\] \#\#\# Step 4 --- Find Season Fractions

Once you have the mean anomalies for each seasonal marker: \[
\begin{align}
&M_\text{spring} \\
&M_\text{summer} \\
&M_\text{autumn} \\
&M_\text{winter} 
\end{align}
\] subtract them in sequence to get the fractional year lengths: \[
\begin{aligned}
f_\text{spring} &= \frac{M_\text{summer} - M_\text{spring}}{2\pi} \\[6pt]
f_\text{summer} &= \frac{M_\text{autumn} - M_\text{summer}}{2\pi} \\[6pt]
f_\text{autumn} &= \frac{(M_\text{winter}+2\pi) - M_\text{autumn}}{2\pi} \\[6pt]
f_\text{winter} &= \frac{M_\text{spring} - M_\text{winter}}{2\pi} \\[6pt]
\end{aligned}
\]

Where:\\
- Each \(f\) = fraction of the year occupied by that season.\\
- The denominator \(2\pi\) normalizes the full orbit to 1.\\
- The \(+2\pi\) in the autumn term closes the loop back to the next
winter.

\subsection{Step 5 --- Scale to Year
Length}\label{step-5-scale-to-year-length}

Multiply each fraction by the chronum (\(C\)) to convert fractions into
diurns:

\[
\begin{gather}
\Delta t_\text{spring} = f_\text{spring}\,C, \\
\Delta t_\text{summer} = f_\text{summer}\,C, \\
\Delta t_\text{autumn} = f_\text{autumn}\,C, \\
\Delta t_\text{winter} = f_\text{winter}\,C
\end{gather}
\]

\subsubsection{Worked Example: Earth (Kepler's Method, Sidereal
Year)}\label{worked-example-earth-keplers-method-sidereal-year}

Given:\\
- \(\phi = 0^\circ\)\\
- \(C = 365.2564\) days (sidereal year)\\
- \(e = 0.0167\)

\textbf{Step 1 --- True Anomalies} Seasonal markers from
\(\phi = 0^\circ\):\\
\[
\nu = (\phi + 90n) \bmod 360
\] \[
\begin{cases}
n = 0 & \nu = 270^\circ & \text{spring equinox} \\
n = 1 & \nu = 0^\circ & \text{summer solstice} \\
n = 2 & \nu = 90^\circ & \text{autumn equinox} \\
n = 3 & \nu = 180^\circ & \text{winter solstice}
\end{cases}
\] \textbf{Step 2 --- Convert to Eccentric Anomaly} \[
E = 2 \arctan \!\left( \sqrt{\tfrac{1-e}{1+e}} \;\tan \tfrac{\nu}{2} \right)
\] \textbf{Step 3 --- Convert to Mean Anomaly} \[
M = E - e \sin E
\] This gives the ``time angle'' at each seasonal marker.

\textbf{Step 4 --- Find Season Fractions} Subtract successive \(M\)
values and normalize by \(2\pi\):\\
\[
\begin{aligned}
F_\text{spring} &= \frac{M_\text{summer} - M_\text{spring}}{2\pi} \\[6pt]
F_\text{summer} &= \frac{M_\text{autumn} - M_\text{summer}}{2\pi} \\[6pt]
F_\text{autumn} &= \frac{(M_\text{winter}+2\pi) - M_\text{autumn}}{2\pi} \\[6pt]
F_\text{winter} &= \frac{M_\text{spring} - M_\text{winter}}{2\pi}
\end{aligned}
\] Numerical results:\\
\[
\begin{align}
&F_\text{spring} \approx 0.2553, \\[.5em]
&F_\text{summer} \approx 0.2553, \\[.5em]
&F_\text{autumn} \approx 0.2447 \\[0.5em]
&F_\text{winter} \approx 0.2447, \\[.5em]
\end{align}
\] \textbf{Step 5 --- Scale to chronum length} Multiply each fraction by
\(C\) to get season lengths in diurns:\\
\[
\begin{gather}
\Delta t = F \times C \\[1em]
\begin{cases}
\Delta t_\text{spring} &\approx 0.2553 \times 365.2564 = 93.3 \\
\Delta t_\text{summer} &\approx 0.2553 \times 365.2564 = 93.3 \\
\Delta t_\text{autumn} &≈ 0.2447 \times 365.2564 = 89.4 \\
\Delta t_\text{winter} &≈ 0.2447 \times 365.2564 = 89.4
\end{cases}
\end{gather}
\] \textbf{Result:}\\
- Spring ≈ 93.3 d\\
- Summer ≈ 93.3 d\\
- Autumn ≈ 89.4 d\\
- Winter ≈ 89.4 d (Total = 365.26 d)

\textbf{Comparison to Actuals} Observed Earth season lengths (tropical
year):\\
- Spring ≈ 92.8 d\\
- Summer ≈ 93.6 d\\
- Autumn ≈ 89.9 d\\
- Winter ≈ 89.0 d

\textbf{Errors:}\\
- Spring: +0.5 d\\
- Summer: --0.3 d\\
- Autumn: --0.5 d\\
- Winter: +0.4 d

\textbf{Conclusion} - The Kepler method reproduces Earth's observed
seasons within \textbf{half a day per season}.\\
- It captures the correct pattern --- two longer seasons and two shorter
--- without needing any fudge factors.

\subsubsection{Worked Example: Rosetta (Kepler's Method, Sidereal
Chronum)}\label{worked-example-rosetta-keplers-method-sidereal-chronum}

Given:\\
- \(\phi = 180^\circ\)\\
- \(C = 492\) diurns (sidereal chronum)\\
- \(e = 0.05\)

\textbf{Step 1 --- True Anomalies} Seasonal markers from
\(\phi = 180^\circ\):\\
\[
\begin{gather}
&\nu = (\phi + 90n) \bmod 360 \\[1em]
&\begin{cases}
n &= 0 & \nu = 0^\circ & \text{spring equinox} \\
n &= 1 & \nu = 90^\circ & \text{summer solstice} \\
n &= 2 & \nu = 180^\circ & \text{autumn equinox} \\
n &= 3 & \nu = 270^\circ & \text{winter solstice}
\end{cases}
\end{gather}
\] \textbf{Step 2 --- Convert to Eccentric Anomaly} \[
E = 2 \arctan \!\left( \sqrt{\tfrac{1-e}{1+e}} \;\tan \tfrac{\nu}{2} \right)
\] \textbf{Step 3 --- Convert to Mean Anomaly} \[
M = E - e \sin E
\]

\textbf{Step 4 --- Find Season Fractions} Subtract successive \(M\)
values and normalize by \(2\pi\):\\
\[
\begin{aligned}
F_\text{spring} &= \frac{M_\text{summer} - M_\text{spring}}{2\pi} \\[6pt]
F_\text{summer} &= \frac{M_\text{autumn} - M_\text{summer}}{2\pi} \\[6pt]
F_\text{autumn} &= \frac{(M_\text{winter}+2\pi) - M_\text{autumn}}{2\pi} \\[6pt]
F_\text{winter} &= \frac{M_\text{spring} - M_\text{winter}}{2\pi} 
\end{aligned}
\] Numerical results:\\
\[
\begin{align}
&F_\text{spring} \approx 0.2180, \\
&F_\text{summer} \approx 0.2537, \\
&F_\text{autumn} \approx 0.2820, \\
&F_\text{winter} \approx 0.2463
\end{align}
\] \textbf{Step 5 --- Scale to Year Length} Multiply each fraction by
\(C\):\\
\[
\begin{cases}
\Delta t_\text{spring} \approx 107.3 \\
\Delta t_\text{summer} \approx 124.9 \\
\Delta t_\text{autumn} \approx 138.6 \\
\Delta t_\text{winter} \approx 121.2
\end{cases}
\] (Total = 492.0 d)

\textbf{Result (Kepler):}\\
- Spring ≈ 107.3 d\\
- Summer ≈ 124.9 d\\
- Autumn ≈ 138.6 d\\
- Winter ≈ 121.2 d

\textbf{Comparison to Sinusoidal} From the sinusoidal shortcut (no
fudge):\\
- Spring ≈ 107.3 d\\
- Summer ≈ 123.0 d\\
- Autumn ≈ 138.7 d\\
- Winter ≈ 123.0 d

\textbf{Differences:}\\
- Spring: 0.0 d\\
- Summer: +1.9 d\\
- Autumn: --0.1 d\\
- Winter: --1.8 d

\textbf{Conclusion} The Kepler method and sinusoidal shortcut agree
closely for Rosetta, but Kepler adds nuance:\\
- Winter is slightly shorter, summer slightly longer, compared to the
sinusoidal baseline.\\
- Spring and autumn remain nearly identical.

With Rosetta's higher eccentricity (\(e = 0.05\)), the \textbf{seasonal
spread is extreme} --- from 107 d to 139 d --- a difference of over 30
diurns.\\
This is exactly the kind of asymmetry you would expect on a world with
such an orbit, and it needs \textbf{no fudge at all}.

\section{Abstract}\label{abstract-29}

\textbf{Major Topics:}\\
- Methodology for calculating horizon intersections of stellar paths.\\
- Relationship between axial tilt (ε), observer latitude (λ), and time
in the apical chronum (C\_f).\\
- Step-by-step procedure to compute star declination, altitude range,
and azimuthal intersections.\\
- Conditions for continuous daylight or continuous night.\\
- Introduction of geometric constructs: Altitudem, Azimuthem, Obliquem,
Azimems.

\textbf{Key Terms \& Symbols:}\\
- ε = obliquity (axial tilt).\\
- λ = observer latitude.\\
- C\_f = fraction of the apical chronum.\\
- δ = star's declination.\\
- h\_max, h\_min = maximum/minimum altitude.\\
- Altitudem = center altitude of ellipse relative to horizon.\\
- Azimuthem = great-circle east--west line (180°).\\
- Obliquem = half the obliquity (ε/2).\\
- Z = azimuthal angle magnitude.\\
- Z\_e, Z\_w = east and west azimems (horizon intersection points).

\textbf{Cross-Check Notes:}\\
- Connects obliquity (ε) with observable sky geometry.\\
- Azimem-related terminology is WCB-specific --- ensure consistent
glossary entries.\\
- Conditions for polar day/night align with habitability and cultural
calendar considerations.\\
- Closely related to climate/seasonal notes and obliquity orientation
concepts.

\chapter{Workflow for Calculating Azimuthal
Intersections}\label{workflow-for-calculating-azimuthal-intersections}

\subsubsection{\texorpdfstring{\textbf{Step 1:
Input}}{Step 1: Input}}\label{step-1-input}

\begin{itemize}
\tightlist
\item
  \textbf{Obliquity (\(\varepsilon_x\))}: The planet's axial tilt.
\item
  \textbf{Latitude (\(\lambda\))}: Observer's \emph{latitude}, degrees
  north or south of the equator.
\item
  \textbf{Fraction of the apical chronum (\(C_f\))}: Time since the
  summer solstice, as a fraction of the chronum.
\end{itemize}

\subsubsection{\texorpdfstring{\textbf{Step 2: Calculate Star's
Declination}}{Step 2: Calculate Star's Declination}}\label{step-2-calculate-stars-declination}

Over the course of the apical chronum, a star's declination shifts north
and south of the celestial equator in a sinusoidal pattern. This
north--south swing is what causes the east--west drift of the star's
rising and setting azimems along the horizon. \[
\delta = \epsilon \cos(2 \pi C_f)
\]

\subsubsection{\texorpdfstring{\textbf{Step 3: Calculate Maximum and
Minimum
Altitudes}}{Step 3: Calculate Maximum and Minimum Altitudes}}\label{step-3-calculate-maximum-and-minimum-altitudes}

\begin{enumerate}
\def\labelenumi{\arabic{enumi}.}
\item
  \textbf{Maximum Altitude (\(h_\text{max}\)):} \[
  h_\text{max} = 90^\circ - |\lambda - \delta|
  \]
\item
  \textbf{Minimum Altitude (\(h_\text{min}\)):} \[
  h_\text{min} = 90^\circ - |\lambda + \delta|
  \]
\end{enumerate}

\subsubsection{\texorpdfstring{\textbf{Step 4: Condition
Check}}{Step 4: Condition Check}}\label{step-4-condition-check}

\begin{itemize}
\tightlist
\item
  \textbf{If \(h_\text{max} > 0^\circ\) and \(h_\text{min} < 0^\circ\):}

  \begin{itemize}
  \tightlist
  \item
    The star's path intersects the horizon. \textbf{Proceed to Step 5}.
  \end{itemize}
\item
  \textbf{Otherwise Stop:}

  \begin{itemize}
  \tightlist
  \item
    If \(h_\text{min} > 0^\circ\): The star never dips below the horizon
    (continuous daylight).
  \item
    If \(h_\text{max} \leq 0^\circ\): The star never rises above the
    horizon (continuous night).
  \end{itemize}
\end{itemize}

\subsubsection{\texorpdfstring{\textbf{Step 5: Calculate Center Offset
(\(altitudem\))}}{Step 5: Calculate Center Offset (altitudem)}}\label{step-5-calculate-center-offset-altitudem}

If the condition in Step 4 is satisfied: \[
\text{altitudem} = 90^\circ - \frac{|\lambda - \delta| + |\lambda + \delta|}{2}
\]

\subsubsection{\texorpdfstring{\textbf{Step 6: Calculate the Azimuthal
Angle and
Azimems}}{Step 6: Calculate the Azimuthal Angle and Azimems}}\label{step-6-calculate-the-azimuthal-angle-and-azimems}

For the star's path intersections with the horizon (\(y = 0^\circ\)): \[
\begin{gather}
Z = \text{ Azimuthem } \cdot \sqrt{1 - \left(\frac{\text{Altitudem}}{\text{Obliquem}}\right)^2} \\
Z_e = +Z \quad\text{;}\quad Z_w = -Z
\end{gather}
\] \ldots where: - \(Azimuthem\) = the great-circle line that connects
due east to due west through the observer's position (passing through
the zenith and nadir). Its angular measure is always \(180^\circ\). -
\(Altitudem\) = the altitude of the ellipse's center relative to the
horizon. - \(Obliquem\) = half the angular measure of the planet's
obliquity \(\left(\dfrac{\varepsilon_x}{2}\right)\) - \(Z\) = the
azimuthal angle magnitude between due north and the horizon-intersection
of either the rising or setting of a star's path - \(Z_e\) = the east
\emph{Azimem}; the angle from due north along the eastern horizon line
at which the star's path intersects the eastern horizon. - \(Z_w\) = the
west \emph{Azimem}; the angle from due north along the western horizon
line at which the star's path intersects the western horizon.

\subsection{\texorpdfstring{\textbf{Key
Considerations}}{Key Considerations}}\label{key-considerations}

\begin{itemize}
\tightlist
\item
  \textbf{No Intersections (Skip Step 5 and 6)}:

  \begin{itemize}
  \tightlist
  \item
    If \(h_\text{min} > 0^\circ\): The star's path lies entirely above
    the horizon.
  \item
    If \(h_\text{max} \leq 0^\circ\): The star's path lies entirely
    below the horizon.
  \end{itemize}
\end{itemize}

\section{Abstract}\label{abstract-30}

\textbf{Major Topics:}\\
- Defines \textbf{obliquity (ε)} as the axial tilt of a planemon
relative to the perpendicular of its orbital plane.\\
- Describes how obliquity influences \textbf{seasons, climate regimes,
and habitability} by redistributing stellar flux across latitudes.\\
- Introduces a set of \textbf{parameters for describing obliquity
dynamics}:\\
- \textbf{Obliquity Envelope (𝓔ε):} min, mean, and max tilt angles.\\
- \textbf{Obliquity Scope (Δε):} range between maximum and minimum
tilts.\\
- \textbf{Obliquity Cycle (Tε):} timescale of oscillations in tilt.\\
- \textbf{Obliquity Tempo (ẋε):} rate of change per year/kyr.\\
- \textbf{Obliquity Phase (φε):} percentage of maximum tilt, with ↑/↓
trend markers.\\
- \textbf{Obliquity Azimuth (ζn):} orientation of tilt relative to
orbital periastron.\\
- \textbf{Obliquity Azimuth Precession Cycle (χ):} period for ζn to
complete a 360° precession.\\
- Provides notation conventions such as \textbf{trend arrows (↑/↓)} to
indicate whether obliquity is increasing or decreasing.\\
- Highlights the role of obliquity in planetary stability and long-term
climate cycles, framing it as a \textbf{core parameter for worldbuilding
habitability modeling}.

\textbf{Key Terms \& Symbols:}\\
- \textbf{ε (Obliquity):} axial tilt angle.\\
- \textbf{𝓔ε, Δε, Tε, ẋε, φε, ζn, χ:} formalized obliquity
descriptors.\\
- \textbf{Trend arrows (↑/↓):} indicate increasing or decreasing tilt.

\textbf{Cross-Check Notes:}\\
- No existing abstract covers obliquity directly; this file provides the
\textbf{canonical reference}.\\
- Several new glossary entries required for v1.223 (parameters listed
above).\\
- Closely related to WCB notes on \textbf{season length, orbital
eccentricity, and planetary orientation}.

\chapter{Obliquity --- Planetary
Orientation}\label{obliquity-planetary-orientation}

\section{\texorpdfstring{Current Obliquity (Axial tilt)
(\(\varepsilon\))}{Current Obliquity (Axial tilt) (\textbackslash varepsilon)}}\label{current-obliquity-axial-tilt-varepsilon}

Obliquity is the \textbf{instantaneous angle} between a planemon's
rotational axis and the perpendicular to its orbital plane. An up arrow
\(\uparrow\) or down arrow \(\downarrow\) may be appended after the
angular measure to indicate whether the obliquity is increasing or
decreasing.

For Earth:\\
\[
\varepsilon = 23.5^\circ\downarrow
\] \#\# Obliquity Envelope (\(\mathcal{E}_\varepsilon\)) \[
\mathcal{E}_\varepsilon =
\begin{bmatrix}
\varepsilon_{min} \\
\varepsilon_{mean} \\
\varepsilon_{max}
\end{bmatrix}
\] Where:\\
- \(\varepsilon_x\) = current tilt angle\\
- \(\varepsilon_{min}\) = minimum obliquity - \(\varepsilon_{mean}\) =
mean obliquity\\
- \(\varepsilon_{max}\) = maximum obliquity

Example (Earth):\\
\[
\mathcal{E}_\varepsilon = 
\begin{bmatrix}
22.1^\circ \\
23.3^\circ \\
24.5^\circ
\end{bmatrix}
\] \#\# Obliquity Scope (\(\Delta\varepsilon\)) \[
\Delta\varepsilon = \varepsilon_{max} - \varepsilon_{min}
\] For Earth: \[
\Delta\varepsilon = 24.5^\circ - 22.1^\circ = 2.5^\circ
\] \#\# Obliquity Cycle (\(T_\varepsilon\)) The time interval between
two maxima (or minima) of obliquity.

For Earth:\\
\[
T_\varepsilon \approx 41{,}000 \text{ years}
\] \#\# Obliquity Tempo (\(\dot{\varepsilon}\)) Rate of obliquity change
per year:\\
\[
\dot{\varepsilon} = \dfrac{\varepsilon_{max} - \varepsilon_{min}}{T_\varepsilon}
\] For Earth:\\
\[
\dot{\varepsilon} = \frac{24.5 - 22.1}{41000} ≈ 0.0000585^\circ/\text{yr} \;=\; 0.00585^\circ/\text{kyr}
\] \#\# Obliquity Phase (\(\phi_\varepsilon\)) The ratio of the current
obliquity to its maximum value, expressed as a percentage, with an arrow
showing whether the trend is increasing \(\uparrow\) or decreasing
\(\downarrow\): \[
\phi_\varepsilon = \dfrac{\varepsilon}{\varepsilon_{max}}\times 100 \; (\uparrow,\downarrow)
\] For Earth:\\
\[
\phi_\varepsilon = \dfrac{23.5}{24.5}\times 100 \approx 95.9\text{\%}\;\downarrow
\] \#\# Obliquity Azimuth (\(\zeta_{n}\)) The angular orientation of a
planemon's axial tilt relative to its orbit. Defined as the angle
between periapsis (the line of apsides) and the projection of the
planet's north pole on the orbital plane.\\
- \(\zeta_{0}\) = (``periaptic zero''): northern solstice occurs at
periastron.\\
- \(\zeta_{90}\) = northern solstice has precessed 90° forward along the
orbit.\\
- \(\zeta_{180}\) = northern solstice occurs at apastron.\\
- \(\zeta_{270}\)= northern solstice has precessed 270° forward along
the orbit.\\
- \(\zeta_{n}\) advances anti-clockwise with respect to the orbit, so
solstices and equinoxes \textbf{precess relative to
periastron/apastron}.

\begin{itemize}
\tightlist
\item
  The \textbf{obliquity azimuth} precesses around the orbit, shifting
  the alignment of solstices and equinoxes with periastron and apastron.
  The \textbf{calendar seasons} remain fixed relative to
  equinoxes/solstices, but their \textbf{orbital context} changes over
  the precessional cycle.
\item
  This precession occurs for all planets with nonzero obliquity
  (\(\varepsilon \neq 0\)).\\
\item
  The cycle is independent of tilt magnitude: even an
  \(\varepsilon = 90^\circ\) planet precesses in the same way, with
  solstices tied to periastron/apastron alignment.
\end{itemize}

\textbf{Canonical Cases for \(\zeta_n\)\hspace{0pt}} -
\(e \neq 0, \varepsilon = 0\)\\
- The orbit has a defined \textbf{periastron}.\\
- There is no axial tilt, so no solstices or equinoxes exist.\\
- By convention, \(\zeta_0\)\hspace{0pt} may still be defined as the
direction of periastron, but it carries \textbf{no physical seasonal
meaning}. - \(e \neq 0, \varepsilon \neq 0\)\\
- The orbit has a defined periastron.\\
- The planet has tilt, so solstices and equinoxes exist.\\
- \(\zeta_n\)\hspace{0pt} is measured from periastron, describing the
orbital longitude where the \textbf{north pole is tilted directly away
from the star}.\\
- \(\zeta_0\)\hspace{0pt}: that event coincides with periastron.\\
- Other \(\zeta_n\)\hspace{0pt}: the event occurs \(n^\circ\) along the
orbit from periastron. - \(e=0, \varepsilon \neq 0\)\\
- Orbit is a perfect circle: no physical periastron exists.\\
- Solstices/equinoxes still exist because of axial tilt.\\
- In this case, \textbf{a 0° reference direction must be chosen
arbitrarily} (often set by convention, e.g.~``perihelion by
definition''), and \(\zeta_n\)\hspace{0pt} is measured relative to that.
- \(e=0, \varepsilon = 0\)\\
- No tilt, no closest approach.\\
- Neither solstices/equinoxes nor apsidal points exist.\\
- \(\zeta_n\)\hspace{0pt} is undefined, unless an \textbf{arbitrary 0°
longitude} is adopted purely for bookkeeping. \#\# Obliquity Azimuth
Precession Cycle (\(\chi\)) The length of time it takes the obliquity
azimuth (\(\zeta_{n}\)) to precess through \(360^\circ\). - Defined such
that \(\zeta_{0}\) occurs when the planet's north pole is tilted
\textbf{away} from the star at periastron.\\
- At \(\zeta_{180}\), the north pole is tilted \textbf{toward} the star
at periastron. - For Earth: \(\chi ≈ 27000\) y.

Thus, for Earth in \textasciitilde13,000 years, northern summer will
occur near December--February if the current Gregorian framework remains
in use unchanged.

\begin{quote}
\textbf{Note:} \(χ\) is undefined for \(\varepsilon = 0\), since there
is no obliquity to precess. In practice, some frameworks may assign
\(\chi = 0\) for bookkeeping, but this has no physical meaning.
\end{quote}

\section{Earth's Current Seasons}\label{earths-current-seasons}

\begin{longtable}[]{@{}
  >{\raggedright\arraybackslash}p{(\linewidth - 10\tabcolsep) * \real{0.0741}}
  >{\raggedright\arraybackslash}p{(\linewidth - 10\tabcolsep) * \real{0.1481}}
  >{\raggedright\arraybackslash}p{(\linewidth - 10\tabcolsep) * \real{0.1481}}
  >{\raggedright\arraybackslash}p{(\linewidth - 10\tabcolsep) * \real{0.1605}}
  >{\raggedright\arraybackslash}p{(\linewidth - 10\tabcolsep) * \real{0.4321}}
  >{\raggedright\arraybackslash}p{(\linewidth - 10\tabcolsep) * \real{0.0370}}@{}}
\toprule\noalign{}
\begin{minipage}[b]{\linewidth}\raggedright
Season
\end{minipage} & \begin{minipage}[b]{\linewidth}\raggedright
Start
\end{minipage} & \begin{minipage}[b]{\linewidth}\raggedright
End
\end{minipage} & \begin{minipage}[b]{\linewidth}\raggedright
Length (days)
\end{minipage} & \begin{minipage}[b]{\linewidth}\raggedright
Why length differs
\end{minipage} & \begin{minipage}[b]{\linewidth}\raggedright
\end{minipage} \\
\midrule\noalign{}
\endhead
\bottomrule\noalign{}
\endlastfoot
Spring & Mar Equinox & Jun Solstice & 92.75 & Earth slows toward
apastron & \\
Summer & Jun Solstice & Sep Equinox & 93.65 & Earth slowest near
apastron & \\
Autumn & Sep Equinox & Dec Solstice & 89.85 & Earth accelerates toward
periastron & \\
Winter & Dec Solstice & Mar Equinox & 88.99 & Earth fastest near
periastron & \\
\end{longtable}

\section{Abstract}\label{abstract-31}

\textbf{Major Topics:}\\
- Clarifies that \textbf{retrograde motion} can apply to
\textbf{rotation} as well as orbital direction.\\
- Defines rotational sense in terms of \textbf{axial tilt (ε):}\\
- \textbf{ε ∈ ⟨0° ∧ 90°⟩} → \textbf{prograde rotation} (spin direction
matches orbital motion).\\
- \textbf{ε ∈ ⟨90° ∧ 180°⟩} → \textbf{retrograde rotation} (spin
direction opposes orbital motion).\\
- Emphasizes that the designation ``prograde'' or ``retrograde'' is
\textbf{conventional}, depending on how ``north'' is defined for the
system.\\
- If Solar System's poles were flipped, Venus would be the only prograde
planet.\\
- Provides case studies:\\
- \textbf{Earth:} ε ≈ 23.4° → prograde.\\
- \textbf{Venus:} ε ≈ 177.4° → retrograde; sidereal day ≈ 243 Earth
days; solar day ≈ 117 Earth days; Sun rises in the west.\\
- \textbf{Uranus:} ε ≈ 97.8° → technically retrograde; rotates nearly on
its side.\\
- Highlights worldbuilding consequences:\\
- Retrograde spin changes \textbf{day/night orientation} and
\textbf{Sun's apparent motion}.\\
- High obliquity + retrograde can yield alien but plausible
\textbf{seasonal/diurnal patterns}.\\
- Slow retrograde rotation may invert expectations about day length and
climate.

\textbf{Key Terms \& Symbols:}\\
- \textbf{Axial Tilt (ε) {[}sci{]}.}\\
- \textbf{Prograde Rotation {[}sci{]}.}\\
- \textbf{Retrograde Rotation {[}sci{]}.}\\
- \textbf{Solar Reversal {[}NEW{]}:} effect of Sun rising in the west,
setting in the east, due to retrograde spin.

\textbf{Cross-Check Notes:}\\
- \textbf{Axial Tilt (ε)} and prograde/retrograde spin already appear in
canon; this file reinforces them with explicit examples.\\
- \textbf{Solar Reversal {[}NEW{]}} is a newly introduced descriptive
label for inverted sunrise/sunset.\\
- \textbf{Status:} {[}EXPANDED + NEW{]} --- expands on existing axial
tilt and rotation terms, introduces solar reversal as a worldbuilding
descriptor.

\begin{quote}
\textbf{Keppy}: Wait\ldots{} isn't retrograde motion an orbital thing?
\end{quote}

Not always: planemons can also \textbf{rotate} in a retrograde sense ---
spinning ``backwards'' compared to their orbital direction. And remember
what is ``up'' in a star system is \emph{purely a matter of convention};
ignoring direction of orbital motion, we could just as easily say that
Venus is the only prograde planemon in the Solar System.

\subsubsection{🧭 How to Know:}\label{how-to-know}

\begin{itemize}
\tightlist
\item
  Axial tilt (ε) is the angle between the planemon's spin axis and the
  perpendicular to its orbital plane.
\item
  If \textbf{ε ∈ ⟨0° ∧ 90°⟩}, the rotation is \textbf{prograde} --- the
  planemon spins the same direction as it orbits.
\item
  If \textbf{ε ∈ ⟨90° ∧ 180°⟩}, the rotation is \textbf{retrograde} ---
  the planemon spins the \emph{opposite} direction from its orbit.
\end{itemize}

\begin{quote}
And here's the twist:\\
What we call ``prograde'' or ``retrograde'' is just a matter of
\textbf{convention}.

We define ``up'' in the Solar System based on Earth's north pole and
orbital direction.\\
But that's completely arbitrary.\\
If we flipped the system and redefined ``north'' as ``south,''
\textbf{Venus would become the \emph{only} prograde planemon}, and all
the others would be retrograde.
\end{quote}

\subsubsection{🪐 Example: Venus}\label{example-venus}

\begin{itemize}
\tightlist
\item
  Axial tilt: \textbf{177.4°}
\item
  Spins incredibly slowly and \textbf{retrograde}
\item
  A full Venus day (sidereal) is \textasciitilde243 Earth days
\item
  But because of its retrograde spin, a \textbf{solar day} (sunrise to
  sunrise) lasts \textasciitilde117 Earth days --- and the Sun rises in
  the \textbf{west}!
\end{itemize}

\subsubsection{🌀 Why This Matters}\label{why-this-matters}

\begin{itemize}
\tightlist
\item
  ε \textgreater{} 90° radically changes \textbf{day/night direction},
  \textbf{sun path across the sky}, and even \textbf{cultural
  orientation} (``sun rises in the west'').
\item
  High obliquity + retrograde spin = \textbf{wildly different} seasonal
  or diurnal patterns.
\item
  For worldbuilders: this is a prime lever to make a world \emph{feel}
  subtly alien while remaining physically plausible.
\end{itemize}

\subsubsection{\texorpdfstring{🧭 \textbf{How It
Works}}{🧭 How It Works}}\label{how-it-works}

\begin{itemize}
\tightlist
\item
  A planemon's \textbf{axial tilt (ε)} is defined as the angle between
  its rotational axis and the perpendicular to its orbital plane.
\item
  \textbf{ε ∈ ⟨0° ∧ 90°⟩} → \textbf{prograde rotation}\\
  (spin direction aligns with orbital motion)
\item
  \textbf{ε ∈ ⟨90° ∧ 180°⟩} → \textbf{retrograde rotation}\\
  (spin direction opposes orbital motion)
\end{itemize}

So:

\begin{itemize}
\tightlist
\item
  \textbf{Earth}: ε ≈ 23.44° → prograde
\item
  \textbf{Venus}: ε ≈ \textbf{177.4°} → retrograde

  \begin{itemize}
  \tightlist
  \item
    It's tipped almost completely upside down --- a rotation that is
    \emph{both slow} and \emph{backwards}
  \end{itemize}
\item
  \textbf{Uranus}: ε ≈ \textbf{97.8°} → technically retrograde

  \begin{itemize}
  \tightlist
  \item
    Lies nearly on its side; its axial pole dips below the orbital plane
  \end{itemize}
\end{itemize}

\subsubsection{\texorpdfstring{📌 \textbf{Retrograde Rotation at a
Glance}}{📌 Retrograde Rotation at a Glance}}\label{retrograde-rotation-at-a-glance}

\begin{longtable}[]{@{}lll@{}}
\toprule\noalign{}
\textbf{Axial Tilt (ε)} & \textbf{Rotation Type} & \textbf{Example} \\
\midrule\noalign{}
\endhead
\bottomrule\noalign{}
\endlastfoot
0° & Perfectly prograde & Theoretical ideal \\
23.4° & Prograde & Earth \\
90° & Sideways / ambiguous & Theoretical (unstable) \\
97.8° & Retrograde & Uranus \\
177.4° & Retrograde & Venus \\
180° & Perfectly retrograde & Theoretical limit \\
\end{longtable}

\subsubsection{\texorpdfstring{🪐 \textbf{Why It
Matters}}{🪐 Why It Matters}}\label{why-it-matters}

\begin{itemize}
\tightlist
\item
  ε \textgreater{} 90° affects \textbf{day-night patterns},
  \textbf{sunrise/sunset direction}, and even \textbf{seasonal
  progression}.
\item
  Can produce a ``\textbf{solar reversal}'' --- the Sun appears to rise
  in the west and set in the east.
\item
  Combined with slow rotation, it may completely upend expectations
  about \textbf{day length}, \textbf{thermal cycling}, and
  \textbf{climatic intuition}. \#\# Abstract\\
  \textbf{Major Topics:}\\
\item
  Consolidates \textbf{Earth-based units of time} into a single
  precision reference, with SI-second values and synodic
  relationships.\\
\item
  Defines:

  \begin{itemize}
  \tightlist
  \item
    \textbf{Ephemeris Day (mean solar day):} exactly 86400 s.\\
  \item
    \textbf{Apparent Solar Day (tropical day):} variable,
    23ʰ59ᵐ38ˢ--24ʰ30ˢ; offsets accumulate into the equation of time.\\
  \item
    \textbf{Sidereal Day:} 86164.09053083288 s (23ʰ56ᵐ4.0905ˢ).\\
  \item
    \textbf{Stellar Day:} 86164.098903691 s (23ʰ56ᵐ4.0989ˢ).\\
  \end{itemize}
\item
  Relative proportions: sidereal day = 99.999990\% of stellar day.\\
\item
  \textbf{Synodic linkages:}

  \begin{itemize}
  \tightlist
  \item
    Ephemeris vs.~sidereal day ≈ tropical year (error 0.265 s).\\
  \item
    Ephemeris vs.~stellar day ≈ sidereal year (error 101.66 s).\\
  \end{itemize}
\item
  Provides sub-second precision and error margins as a ready reference
  for thesiasts.
\end{itemize}

\textbf{Key Terms \& Symbols:}\\
- \textbf{Ephemeris Day {[}sci{]}.}\\
- \textbf{Apparent Solar Day {[}sci{]}.}\\
- \textbf{Sidereal Day {[}sci{]}.}\\
- \textbf{Stellar Day {[}sci{]}.}\\
- \textbf{Tropical Year {[}sci{]}.}\\
- \textbf{Sidereal Year {[}sci{]}.}

\textbf{Cross-Check Notes:}\\
- All terms already appear in canon; this file adds consolidated
precision and relationships.\\
- Should be considered a \textbf{reference supplement}, not a conceptual
expansion.\\
- \textbf{Status:} {[}EXPANDED: Reference/Tabulation{]} --- reinforces
existing canon with high-precision tabulation and synodic calculations.

\subsection{Types of ``Day''}\label{types-of-day}

An \textbf{ephemeris day} (\textbf{mean solar day}) is a unit of time
used in astronomy and celestial mechanics defined as exactly 24 hours
(86400 SI seconds).

Contrast this with an \textbf{apparent solar day (tropical day)}, is the
time it takes for the Sun to appear in the same position in the sky,
from one noontime to the next, and can be either as short as 23ʰ~59ᵐ~38ˢ
(23.9938889ʰ) or as long as 24ʰ~30ˢ (24.5ʰ).~ Long or short days occur
in succession, so the difference builds up until mean time is ahead of
apparent time by about 14 minutes near February 6, and behind apparent
time by about 16 minutes near November 3.

A \textbf{sidereal day} is a unit of time used in astronomy and
celestial mechanics. It is defined as the length of time it takes for
the Earth to make one complete rotation on its axis with respect to the
\emph{fixed stars}, and is defined as:

·~~~~~~~~ 86164.09053083288 SI seconds

·~~~~~~~~ 23ʰ 56ᵐ 4.09053083288ˢ (23.93447192ʰ)

·~~~~~~~~ 0.99726956632908ᵈ

A \textbf{stellar day} is Earth's rotation period relative to the
International Celestial Reference Frame, defined by the International
Earth Rotation and Reference Systems Service (IERS), as:

·~~~~~~~~ 86164.098903691 SI seconds

·~~~~~~~~ 23ʰ 56ᵐ 4.098903691ˢ (23.93446959ʰ)

·~~~~~~~~ 0.99726966323716ᵈ

A sidereal day is 99.9999902827\% of a stellar day.~ The sidereal day is
slightly shorter than the solar day due to the Earth's orbital motion
around the Sun.~ The synodic period between them is slightly over 28000
calendar years.

The \textbf{synodic period} between the \textbf{\emph{ephemeris day}}
and the \textbf{\emph{sidereal day}} calculates to almost exactly the
\textbf{tropical year}.

\ldots{} which calculates to only 0.265295200 seconds (1.00000000840688
times) longer than the actual value of 31556925.2507328ˢ.

The \textbf{synodic period} between the \textbf{\emph{ephemeris day}}
and the \textbf{\emph{stellar day}} calculates to very nearly the
\textbf{sidereal year}.

\ldots{} which calculates to only 101.65737 seconds (1.000003221 times)
longer than the actual value of 31558149.7635456ˢ. \#\# Abstract\\
\textbf{Major Topics:}\\
- Expands beyond the \textbf{basic Geotic model} to construct
atmospheres that are both breathable and physically plausible.\\
- Defines key atmospheric parameters:\\
- \textbf{Surface pressure (P₀):} recommended range ⟨0.5 ∧ 2.0⟩ atm.\\
- \textbf{Scale height (H):} ⟨6 ∧ 12⟩ km; governs pressure drop with
altitude, mountain height, and flight viability.\\
- \textbf{Composition:}\\
- \textbf{O₂ fraction:} ⟨15\% ∧ 30\%⟩; below 15\% = acclimation needed,
above 30\% = fire risk.\\
- \textbf{Buffer gas (N₂, Ar):} ⟨70\% ∧ 85\%⟩; defines density, sound
speed, and thermal response.\\
- \textbf{Trace gases (\textless1\%):} CO₂, H₂O vapor, O₃, CH₄; regulate
greenhouse, UV shielding, and biosystems.\\
- Introduces the \textbf{scale height equation}:\\
\[
  H = \dfrac{RT}{Mg}
  \] with supporting equations for molar mass of gas mixtures.\\
- Provides a worked Earth example (H ≈ 8.5 km).\\
- Demonstrates the effect of different buffer gases (e.g., N₂ vs.~Ar) on
scale height.\\
- Simplifies pressure--altitude decay with exponential form:\\
\[
  P(h) = P_0 e^{-h/H}
  \] - Includes a \textbf{Pressure Plausibility Chart} linking gravity
(g) and atmospheric density to probable surface pressures.\\
- Emphasizes trade-offs: worldcrafters may \textbf{choose} P₀, but it
should remain consistent with gravity, escape velocity, and atmospheric
composition.

\textbf{Key Terms \& Symbols:}\\
- \textbf{H (Scale Height) {[}sci{]}.}\\
- \textbf{M (Mean molar mass) {[}sci{]}.}\\
- \textbf{P₀ (Surface pressure) {[}sci{]}.}\\
- \textbf{Exponential decay law {[}sci{]}.}\\
- \textbf{Buffer Gas {[}sci{]}.}\\
- \textbf{Pressure Plausibility Chart {[}neo{]}.}

\textbf{Cross-Check Notes:}\\
- All major parameters (P₀, H, composition fractions) are already
canonical.\\
- \textbf{Pressure Plausibility Chart {[}neo{]}} is newly introduced as
a heuristic tool.\\
- \textbf{Status:} {[}EXPANDED + NEW{]} --- expands on existing
atmospheric equations; introduces the new Pressure Plausibility Chart
heuristic.

\chapter{Planning A Detailed
Atmosphere}\label{planning-a-detailed-atmosphere}

\chapter{Why Go Deeper Than The Basic Geotic
Model?}\label{why-go-deeper-than-the-basic-geotic-model}

You don't \emph{have to} go beyond the basic Geotic model --- unless, of
course, your players or readers are the kind who pull out calculators
and and gleefully point out why your planemon's air shouldn't work.

If you're the kind of worldcrafter who likes knowing how things
\emph{actually work} (or just wants to stay one step ahead of the smart
alecks), this Sidebar Module walks you through the fundamentals of
atmospheric plausibility.

We're sticking to \textbf{generally habitable atmospheres} here --- ones
that humans or near-humans could plausibly breathe. But the core
principles apply no matter how exotic you want to get.

\begin{itemize}
\tightlist
\item
  \textbf{Average atmospheric pressure (atm)}

  \begin{itemize}
  \tightlist
  \item
    Range: atm ∈ ⟨0.5 ∧ 2.0⟩
  \item
    Supports oxygen respiration and liquid water without requiring
    pressure suits or extreme acclimatization
  \end{itemize}
\item
  \textbf{Atmospheric Scale Height (H)}

  \begin{itemize}
  \tightlist
  \item
    H ∈ ⟨6 ∧ 12⟩ km
  \item
    Governs pressure drop with altitude
  \item
    Affects breathing, mountain height, and high-altitude flight
  \item
    See below for details
  \end{itemize}
\item
  \textbf{Average atmospheric composition}

  \begin{itemize}
  \tightlist
  \item
    Oxygen

    \begin{itemize}
    \tightlist
    \item
      Range: O₂ ∈ ⟨15\% ∧ 30\%⟩

      \begin{itemize}
      \item
        \begin{quote}
        30\% even damp fuel ignites more easily, and spontaneous
        combustion becomes a risk
        \end{quote}
      \item
        \textless{} 15\% requires acclimation or enhanced lung capacity
        (for any creature not evolved in the environment)
      \end{itemize}
    \item
      Needed for aerobic respiration
    \end{itemize}
  \item
    Buffer Gas

    \begin{itemize}
    \tightlist
    \item
      Range: ∈ ⟨70\% ∧ 85\%⟩
    \item
      The chemically inert or low-reactivity bulk gas that fills out the
      atmosphere around oxygen
    \item
      Defines overall \emph{atmospheric} density, scale height, sound
      propagation, and temperature response
    \item
      Nitrogen (N₂) and Argon (Ar) are your only reliable options

      \begin{itemize}
      \tightlist
      \item
        Other candidates (neon, helium, etc.) are too rare, too light,
        or too toxic
      \end{itemize}
    \end{itemize}
  \item
    Trace Gasses

    \begin{itemize}
    \tightlist
    \item
      \textless{} 1\%
    \item
      H₂O vapor, O₃, CH₄
    \item
      CO₂ ideally should comprise \textless{} 0.04\%
    \item
      Required for climate regulation (greenhouse effect), UV shielding,
      biodynamics
    \end{itemize}
  \end{itemize}
\end{itemize}

\section{More About Scale Height}\label{more-about-scale-height}

As related above, atmospheric scale height (H) - Governs pressure drop
with altitude - Affects breathing, mountain height, and high-altitude
flight

However, the value of H varies with the composition of the atmosphere
because the \textbf{\emph{molecular mass}} is different for each
component gas and is not a constant.

\begin{quote}
\textbf{Hippy}: You need to calculate H based on how much O₂, buffer
gas, etc. your atmosphere has.
\end{quote}

Exactly! So, how do you do that?

\subsection{Calculating Scale Height}\label{calculating-scale-height}

Scale height (H) depends on: - \textbf{T} = average temperature of the
atmosphere (in Kelvin)\\
- \textbf{M} = mean molar mass of the gas mixture (in kg/mol)\\
- \textbf{g} = surface gravity (in m/s²)\\
- \textbf{R} = universal gas constant ≈ 8.314 J/mol·K

\ldots{} related in the equation: \[
H = \dfrac{R T}{M g}
\] \#\#\#\# First

Find the mean molecular (molar) mass M. Each gas in the atmosphere
contributes to the average molar mass based on its \textbf{\emph{mole
fraction}}: \[
M = \sum{x_i M_i}
\] \textgreater{} \textbf{Keppy}: \emph{DON'T PANIC}! That Σ might
\emph{look} like calculus, but it's not\ldots{} all it's saying is that
we sum up the molar masses of all the gasses present

Yes, in the above equation: - \(x_i\) is the \textbf{mole fraction} of
each atmospheric gas (in Earth's case, that's 0.78 for N₂; 0.21 for O₂)
- \(M_i\) is the molar mass of the gas in question.

Let's use Earth's atmosphere as an example: - Atmosphere = 78\% N₂, 21\%
O₂, 1\% Ar (using kg/mol):\\
- \(M_{N_2} = 0.028014\)\\
- \(M_{O_2} = 0.031998\) - \(M_{Ar} = 0.039948\) - We lump argon in with
the trace gasses in Earth's case \[
M = (0.78 \times 0.028014) + (0.21 \times 0.031998) + (0.01 \times 0.039948) = 0.02896 \text{ kg/mol}​
\]

\begin{quote}
\textbf{Keppy}: Where did 0.028014, and the other numbers come from?
\end{quote}

Well spotted!

Each gas has a known molar mass --- the mass of one \textbf{\emph{mole}}
of its molecules --- and it's typically expressed in grams per mole
(g/mol). Since we're working in SI units, we convert those to kilograms
per mole (kg/mol) by dividing by 1000. And we look those numbers up in
an appropriate (and reliable) source. Here are a few for reference:

\begin{longtable}[]{@{}
  >{\raggedright\arraybackslash}p{(\linewidth - 6\tabcolsep) * \real{0.2133}}
  >{\raggedright\arraybackslash}p{(\linewidth - 6\tabcolsep) * \real{0.2400}}
  >{\raggedright\arraybackslash}p{(\linewidth - 6\tabcolsep) * \real{0.2667}}
  >{\raggedright\arraybackslash}p{(\linewidth - 6\tabcolsep) * \real{0.2800}}@{}}
\toprule\noalign{}
\begin{minipage}[b]{\linewidth}\raggedright
Gas
\end{minipage} & \begin{minipage}[b]{\linewidth}\raggedright
Chemical Formula
\end{minipage} & \begin{minipage}[b]{\linewidth}\raggedright
Molar Mass (g/mol)
\end{minipage} & \begin{minipage}[b]{\linewidth}\raggedright
Molar Mass (kg/mol)
\end{minipage} \\
\midrule\noalign{}
\endhead
\bottomrule\noalign{}
\endlastfoot
Nitrogen & N₂ & 28.014 & 0.028014 \\
Oxygen & O₂ & 31.998 & 0.031998 \\
Argon & Ar & 39.948 & 0.039948 \\
Carbon Dioxide & CO₂ & 44.01 & 0.044010 \\
Water vapor & H₂O (gas) & 18.015 & 0.018015 \\
Methane & CH₄ & 16.043 & 0.016043 \\
Helium & He & 4.003 & 0.004003 \\
\end{longtable}

\begin{quote}
\textbf{Hippy}: And you use these values in your atmosphere recipe to
get your average molar mass.
\end{quote}

Exactly! Once you've built your mix --- say, 75\% N₂ and 25\% O₂ ---
just multiply each molar mass by its fraction and sum the results.

\begin{quote}
\textbf{Keppy}: Seems like that equation could get lengthy and complex.
\end{quote}

You are \emph{not} wrong about that. Getting warm and friendly with a
spreadsheet or a programmable calculator is very good advice for the
serious worldcrafter, for sure.

\subsubsection{Second}\label{second}

Now that we have our value for M, we plug it into our other \emph{known}
values for R (8.314), T (288), and g (9.8) and get: \[
H=\dfrac{R T}{M g} = \dfrac{8.314 \times 288}{0.02896 \times 9.8} = \dfrac{2395.6}{0.2838} = 8.44\text{ km},
\]

\ldots{} which is usually rounded up to H = 8.5 km for convenience, but
you can use the more exact value if you prefer.

\begin{quote}
\textbf{Hippy}: T = 288 K is for Earth. How does one determine it for a
'crafted planemon?
\end{quote}

Ah --- excellent question.\\
The \textbf{average surface temperature} T depends on factors like: -
The \textbf{luminosity and spectral class} of the central star(s)\\
- The \textbf{orbital distance} of the planemon\\
- The planemon's \textbf{albedo} (reflectivity)\\
- And any \textbf{greenhouse effects} caused by atmospheric composition

That starts to pull us into stellar physics and orbital modeling, which
is covered in:

\begin{quote}
🔗 \textbf{Module XY.Z --- Building Your Star and Setting Your Orbit}
\end{quote}

There, you'll find methods for estimating a planemon's
\textbf{equilibrium temperature} and adjusting it for greenhouse gases
to get a realistic T for your atmosphere model.

For now, if you're working with a roughly Earthlike setup, using: \[
T ∈ \langle260 \wedge 320\rangle K,
\] or roughly ⟨-13° ∧ 47°⟩C will keep your numbers in a plausible range.

Here's some context for that range:

\begin{quote}
\begin{itemize}
\tightlist
\item
  260K (-13°C; 8.6°F)

  \begin{itemize}
  \tightlist
  \item
    Close to the freezing point of seawater
  \item
    Sustained habitability without deep-cold adaptation is still
    possible
  \end{itemize}
\item
  273K (0°C; 32°F)

  \begin{itemize}
  \tightlist
  \item
    Freezing point of water
  \item
    Important psychological and ecological threshold
  \end{itemize}
\item
  288K (15°C; 59°F)

  \begin{itemize}
  \tightlist
  \item
    Earth's average
  \item
    Serves as benchmark for climate comfort and live-rich biomes
  \end{itemize}
\item
  310K (37°C; 98.6°F)

  \begin{itemize}
  \tightlist
  \item
    Human core temperature
  \item
    Upper limit for comfort under heavy exertion
  \end{itemize}
\item
  320K (47°C; 116.6°F)

  \begin{itemize}
  \tightlist
  \item
    At or above this, heat stress becomes deadly without rapid
    evaporative cooling or climate control.
  \end{itemize}
\end{itemize}
\end{quote}

\begin{quote}
❗️ Important note: IF you \emph{choose} a surface temperature for your
world, be sure to note it down, because it will help \emph{determine}
your star's parameters later!
\end{quote}

\begin{quote}
\textbf{Keppy}: So, back to atmospheric composition: If you used argon
instead of nitrogen as your buffer gas, the air would be heavier and H
would shrink?
\end{quote}

Exactly. More mass = more gravity per mole = thinner vertical spread =
smaller H.

\begin{quote}
\textbf{Keppy}: And in this case, we're just ignoring the trace gasses
altogether?
\end{quote}

Mostly, yes --- and here's why:

Trace gases like CO₂, CH₄, and H₂O vapor are typically present in such
\textbf{small quantities} (fractions of a percent) that they
\textbf{barely shift the weighted average} of molar mass.

Let's say your atmosphere is: - 78\% N₂ (0.028014 kg/mol)\\
- 21\% O₂ (0.031998 kg/mol)\\
- 1\% CO₂ (0.04401 kg/mol)

\[
M = (0.78 \times 0.028014) + (0.21 \times 0.031998) + (0.01 \times 0.04401) = 0.02899 \text{ kg/mol}​
\]

That's a difference of only \textbf{+0.00003} compared to Earth-normal
--- \textbf{Less than a tenth of a percent.} So, unless you're at CO₂
levels high enough to make the air toxic or unbreathable, it's safe to
treat the trace gasses as negligible in the \textbf{H} calculation.

\begin{quote}
\textbf{Hippy}: So just include the big players --- buffer gas and
oxygen --- and don't sweat the tenths of a percent?
\end{quote}

Bingo. You can always do a full weighted sum if you really want to be
\emph{precise}, but for 99\% of cases, O₂ and the buffer gas dominate
the molar mass \emph{for Geotic worlds}.

\subsubsection{Third: Pressure vs.~Altitude
Approximation}\label{third-pressure-vs.-altitude-approximation}

For Earth: \[
P(h) = P_0 \times 0.37^{\frac{h}{H}}
\]

Where: - h = altitude in kilometers - H = 8.5 (or 8.44) km

\begin{quote}
\textbf{Keppy}: What is P₀, and how do we know its value?
\end{quote}

Precisely the question I'd ask at this point!

P₀\hspace{0pt} is the \textbf{surface pressure} of the planemon --- that
is, the pressure at \textbf{zero altitude}. On Earth, that value is
defined as: - 1 atm - 101325 Pa* - 101.325 kPa

* Named for Blaise Pascal; there's not space here to go into this in
detail, but it's a fascinating read if you want to look it up!

For any world you create, P₀\hspace{0pt} is one of your \textbf{starting
parameters}. You either: - \textbf{Choose it} (e.g., 1.2 atm, or 0.65
atm), or\\
- \textbf{Derive it} from known gas composition, temperature, and
gravity (which gets more advanced)

So, let's say you've chosen P₀ = 0.9 atm for your planemon, then at one
scale height above the planemon's surface, the atmospheric pressure
calculates to: \[
P(H) = 0.9 \times 0.37 = 0.333\text{ atm}
\] \textgreater{} \textbf{Hippy}: Which raises the question of where
0.37 comes from?

Excellent question. That 0.37 isn't arbitrary --- it's actually derived
from a \textbf{fundamental mathematical constant}: Euler's number, e (≈
2.71828).

The \textbf{pressure--altitude relationship} is an
\textbf{\emph{exponential decay}} equation. In its most general form: \[
P(h) = P_0 \times e^{\frac{-h}{H}}
\] \ldots{} which means that at an altitude of exactly one scale height
(where \emph{h} = \emph{H}): \[
P(H) = P_0 \times e^{-1} = P_0 \times 0.3679 
\] \textgreater{} \textbf{Keppy}: Ah! I see\ldots{} and P₀ is just
whatever multiple of Earth's surface pressure in atm you've chosen for
your world!

\begin{quote}
\textbf{Hippy}: I always frown at ``chosen''; is there any way to at
least \emph{approximate} an appropriate P₀ for your planemon based on
how you decided to compose its atmosphere?
\end{quote}

Well\ldots. yes\ldots. sort of.

You \emph{can} approximate P0P\_0P0\hspace{0pt} from first principles
--- and it's especially useful if you've already defined your
atmosphere's: - \textbf{Gas composition} (molar mix)\\
- \textbf{Surface gravity} (g)\\
- \textbf{Average temperature} (T)

\ldots{} and rearranging the \textbf{\emph{ideal gas law}} for planemon
atmospheres: \[
P_0 = \dfrac{\rho R T}{M}
\] But this requires knowing ρ, the \textbf{near-surface air density}
--- which in turn depends on pressure and composition, so we go a
different route.

Instead, for an atmosphere in hydrostatic equilibrium, you can
approximate: \[
P_0 ≈ \dfrac{g M N}{A}
\] Where: - g = surface gravity - M = average molar mass (kg/mol) - N =
total moles of atmosphere - A = surface area of the planemon

But this, too, gets messy without knowing how much gas the planemon
\emph{started with}, which is based on accretion, outgassing, escape
velocity, etc.

So while you \emph{can} try to derive it from theory --- and I can help
you do that --- you're usually safe choosing a P0P\_0P0\hspace{0pt}
based on: - Your world's \textbf{gravity} (g) - Its \textbf{escape
velocity} (vₑ) - Its \textbf{molar mass and buffer gas makeup} (M --
which we learned how to calculate above.)

Here's a ``pressure plausibility chart'' based on gravity and atmosphere
type to give you an other ``rule of thumb'' to work from:

\subsection{🌍 Pressure Plausibility
Chart}\label{pressure-plausibility-chart}

\emph{Rule-of-thumb surface pressures based on gravity and buffer gas
makeup (M)}

\begin{longtable}[]{@{}
  >{\centering\arraybackslash}p{(\linewidth - 4\tabcolsep) * \real{0.1745}}
  >{\centering\arraybackslash}p{(\linewidth - 4\tabcolsep) * \real{0.2416}}
  >{\raggedright\arraybackslash}p{(\linewidth - 4\tabcolsep) * \real{0.5839}}@{}}
\toprule\noalign{}
\begin{minipage}[b]{\linewidth}\centering
\textbf{Gravity}(g in g⨁)
\end{minipage} & \begin{minipage}[b]{\linewidth}\centering
\textbf{Likely PressureRange (atm)}
\end{minipage} & \begin{minipage}[b]{\linewidth}\raggedright
\textbf{Notes}
\end{minipage} \\
\midrule\noalign{}
\endhead
\bottomrule\noalign{}
\endlastfoot
0.5 & ⟨0.2 ∧ 0.6⟩ & Light gravity;thinner atmosphere unlessretained via
cold temps or high mass gas \\
0.75 & ⟨0.4 ∧ 0.9⟩ & On the thin side but potentiallybreathable;
requires attention to O₂ \% \\
1.0 & ⟨0.8 ∧ 1.2⟩ & Earth-normal, depending onmix of gases and water
vapor \\
1.25 & ⟨1.0 ∧ 1.5⟩ & Denser air; easier to retainlight gases like H₂O,
CH₄ \\
1.5 & ⟨1.2 ∧ 2.0⟩ & Heavy air; pressure at sea levelcould approach
adaptation limits \\
\end{longtable}

This assumes (!) an Earth-like atmosphere: - Mean molar mass ≈ ⟨0.028 ∧
0.032⟩ kg/mol - Normal temperature (\textasciitilde288\,K) - Terrestrial
radius (\textasciitilde1⨁)

You'll need to shift values worlds with high CO₂, significant greenhouse
buildup, or non-volatile-rich origins, but the above should be well
within bounds for \emph{Geotic worlds}.

\chapter{Abstract}\label{abstract-32}

\textbf{Major Topics:}\\
- Definition of \textbf{mean motion resonance}: two (or more) bodies
orbiting a common primary with orbital periods in an integer ratio
(\emph{a:b}).\\
- \textbf{Notation:} ratio expressed as longer period : shorter
period.\\
- \textbf{Resonance order:}\\
- First-order (difference = 1) → e.g., 2:1, 6:5, 13:12.\\
- Second-order (difference = 2) → e.g., 3:1, 5:3.\\
- Higher orders increasingly unstable.\\
- Multi-body case: Io--Europa--Ganymede in 4:2:1 resonance.\\
- Limitation: while \emph{N}-conjunctions are mathematically possible,
stable mean motion resonances with \textgreater3 orbiters are
essentially impossible.

\textbf{Key Terms \& Symbols:}\\
- \textbf{Mean Motion Resonance (MMR):} orbital commensurability
condition.\\
- \textbf{Order of Resonance:} difference between integer ratio
components (1 = strongest).

\textbf{Cross-Check Notes:}\\
- Complements \textbf{Orbits 1 --- Dynamics} (which develops
tresonance/gresonance classification).\\
- Introduces resonance ``order'' concept, not formalized in canon
elsewhere.\\
- No collisions with existing definitions.\\
- To be grouped into \textbf{Orbits 3 --- Resonances} (with Synodion,
Synodial Epoch).

\chapter{Mean Motion Resonance}\label{mean-motion-resonance}

When two bodies orbit a third body in common, there will be resonances
in their orbits. If one body's orbit is an exact integer value different
from the other's, the two are said to be in \emph{mean motion
resonance}, and the ratio of their orbits is \textbf{a : b}, denoted as
the longer orbital period followed by the shorter orbital period,
separated by a colon (∶).

!{[}{[}MeanMotionResonance2-1.jpg{]}{]}

For example, in the figure above, Body 1 orbits twice for every one
orbit of Body 2. This is a \emph{first order resonance}, because the
difference between the two values is only 1 unit: \[
2 - 1 = 1
\] Other examples of first-order resonances\footnote{While the notation
  ²/₁ is notationally equivalent to 2 : 1, we use the colon format
  exclusively.} would be 6 : 5, 13 : 12, etc.

Examples of a \emph{second-order resonance} would be those in which the
orbital periods were separated by a difference of 2; e.g.~3 : 1 or 5 :
3. The higher the order of the resonance, the less stable the orbits
will be over time.

Mean motion resonances can arise between three or more bodies. Jupiter's
moons Io, Europa, and Ganymede experience a 4 : 2 : 1 mean motion
resonance between them, such that for every orbit Ganymede makes, Europa
completes two, and Io completes four. This also means that Io orbits
twice per each orbit of Europa.

However, an interesting phenomenon also occurs, such that Ganymede,
Europa, and Io \emph{never line up} on the same side of Jupiter. Any two
of them may line up at various points in their orbital dance, but all
three never do.

\begin{quote}
While N-conjunctions are \emph{technically} possible in N-orbiton
systems, they are reliant on very specific conditions of orbital
eccentricity and coplanarity; mean motion resonances between more than
three objects should be treated for all intents and purposes as
\emph{impossible}.
\end{quote}

\section{Abstract}\label{abstract-33}

\textbf{Major Topics:}\\
- Defines the \textbf{synodion (Σ):} the synodic period between two
orbitons whose orbital periods are not in integer ratio.\\
- Related concepts:\\
- \textbf{Synodos:} moment of alignment (closest approach).\\
- \textbf{Synodial fraction (FΣ):} fraction of orbits completed by each
body during one synodion.\\
- \textbf{Synodial lag angle (ΔΘ):} angular displacement by which
successive synodoi shift backward.\\
- Demonstrates calculation with example:\\
- \(P_\alpha = 24.36\) d, \(P_\beta = 11.72\) d.\\
- \(\Sigma = 22.587\) d, \(F_\alpha = 0.9272\), \(F_\beta = 1.9272\).\\
- ΔΘ = 26.203°, so alignments drift backward each synodion.\\
- Provides full set of equivalent formulas for Σ, including reciprocal
and ratio forms.\\
- Emphasizes that synodia recur but not at the original base angle
\(B^\theta\), unless by coincidence after many cycles.

\textbf{Key Terms \& Symbols:}\\
- \textbf{Synodion (Σ) {[}NEW{]}.}\\
- \textbf{Synodos {[}NEW{]}.}\\
- \textbf{Synodial fraction (FΣ) {[}NEW{]}.}\\
- \textbf{Synodial lag angle (ΔΘ) {[}NEW{]}.}

\textbf{Cross-Check Notes:}\\
- Canon has no prior abstract of synodic periods.\\
- Complements Orbits 1 (Dynamics), but distinct: this covers non-integer
commensurabilities.\\
- \textbf{Status:} {[}NEW{]} --- establishes synodion framework as part
of orbital resonance canon.

\section{The Synodian: Non-Integer
Resonances}\label{the-synodian-non-integer-resonances}

When two bodies orbit a central body, but their orbital periods are: 1.
not integer multiples of the system's base time unit \emph{T}; and, 2.
not directly commensurate\footnote{literally, ``having a common
  measure''; the orbit of one can't be expressed by a whole number or
  rational fraction of the other's.} with one another, \ldots{} a more
sophisticated approach is required to calculate periodic alignments of
the system.

The \emph{base time unit} (\(T\)), may be years, months, days, even
hours or minutes.

The interval between such instances of alignment is called the
\emph{synodic period}; here, we'll refer to it as a \emph{synodion}
\(\Sigma\) (pl: \emph{synodia}), for reasons which will become clear
presently. The moment of alignment, when the two orbitons are closest to
one another, we call a \emph{synodos} (pl: \emph{synodoi}).

Thus, the synodion represents the time it takes for the two orbitons to
return to the same relative configuration, such as having their centers
fall on a line that includes the center of the central body. While
synodia are still predictable, their periodicity is less intuitive that
in mean motion resonance systems, requiring more precise calculations.

The best way to grasp these concepts is by working through an example.
By appling the rules of synodic periods to a specific pair of orbiting
bodies, we can illustrate how to calculate a synodion, and explore the
patterns it reveals.

\subsection{A Worked Example}\label{a-worked-example}

Let's imagine a system of three bodies, the central body (\emph{centron}
\(\dot{C}\)) and two less massive bodies orbiting it (the
\emph{orbitons} \(M_\alpha\) and \(M_\beta\)). \(M_\alpha\) has the
longer orbital period (\(P_\alpha\); the time it takes \(M_\alpha\) to
orbit the centron) and \(M_\beta\) has the shorter orbital period
(\(P_\beta\); the time it takes \(M_\beta\) to orbit the centron).

\begin{quote}
While the masses of the bodies have no bearing on the following
equations, they \emph{do} matter from a nomenclature perspective. The
more massive body is always the \emph{primary}, the second-most massive
body is the \emph{secondary}; the third most is the \emph{tertiary},
etc. Their \emph{radii} are not a factor in determining their hierarchy.
For instance, Mercury is more massive than Ganymede, but Ganymede has a
larger radius. If the two of them were to form a two-body system,
\emph{Ganymede would orbit Mercury}, even though it is larger in radius
and volume than Mercury.
\end{quote}

For this example, we will define: \[
\begin{align}
P_\alpha = 24.36\; \text{days} \\
P_\beta = 11.72\; \text{days}
\end{align}
\] !{[}{[}synodion01.jpg\textbar300{]}{]} The above figure illustrates
the starting configuration of our system (\(T_0\)). The starting angle
at which all three bodies are lined up we call the \emph{base angle},
denoted by \(B^\theta\).

We can already make some determinations about our system: \[
\begin{align}
P_\delta = P_\alpha - P_\beta = 24.36 - 11.72 = 12.64\; \text{days} \\
P_Q = \dfrac{P_\beta}{P_\alpha} = \dfrac{11.72}{24.36} = 0.481\; \text{days}
\end{align}
\] The ratio of their orbits (\(P_Q\)) is not an integer, so these
orbitons are not in a mean motion resonance; for every orbit of
\(M_\beta\), \(M_\alpha\) only completes \(0.481\) of its orbit, because
\(M_\alpha\) takes \(12.64\) days longer to orbit the centron than does
\(M_\beta\).

Since a full orbit for \emph{either} orbiton is \(360^\circ\), we can
figure out how many degrees of their orbit each orbiton completes within
the system base time unit, which in this case would be \(1\) day. \[
\begin{align}
\theta_\alpha &= \dfrac{360^\circ}{24.36} = 14.778^\circ \qquad \text{The \emph{minor unit angle}}\\
\theta_\beta &= \dfrac{360^\circ}{11.72} = 30.717^\circ  \qquad \text{The \emph{major unit angle}}\\
\end{align}
\]

This tells us how many degrees \(M_\alpha\) ``lags behind'' \(M_\beta\)
per day: \[
\theta_\Delta = \theta_\beta - \theta_\alpha = 30.717^\circ - 14.778^\circ = 15.939^\circ \qquad \text{The \emph{orbiton difference angle}}
\] So, every day \(M_\beta\) moves almost \(16^\circ\) farther in its
orbit than \(M_\alpha\) does in its orbit.

We can also calculate ho many degrees of its orbit \(M_\alpha\)
completes (subtends) per each full orbit of \(M_\beta\): \[
\alpha^\circ = P_Q \times 360 = 0.481 \times 360 = 173.202^\circ 
\] So, after \emph{two} of \(M_\beta\)'s orbits:
!{[}{[}synodion03.jpg\textbar300{]}{]} \[
2 a^\circ = (2)(173.202) = 346.404^\circ
\] \ldots{} \(M_\alpha\) still hasn't completed one of its orbits, it
has only completed \[
\dfrac{346.404}{360} = 0.962
\] orbits of the centron.

Conversely, by the time \(M_\alpha\) completes one full orbit of the
centron, \(M_\beta\) will have completed \[
P_R = \dfrac{24.36}{11.72} = 2.078
\] orbits of the centron.

!{[}{[}synodion04.jpg\textbar300{]}{]} It is a curious feature of
circular motion that if you get far enough ahead of the other body, you
eventually come up behind it. Think of a 5000-meter Olympic footrace.
All of the competitors start out on the same line and travel in the same
direction. Let's say there are two runners, Number 11 and Number 24.

Number 11 quickly pulls out ahead of Number 24 and for a time it is
quite obvious that she is ``way out ahead''. But if she maintains her
higher running speed, eventually she will be \emph{so far ahead} of
Number 24 that \emph{she will appear to be running behind them}. Anyone
looking in on the race at just that moment could be forgiven for
thinking that Number 11 is losing, perhaps badly.

However, eventually Number 11 will catch up to Number 24 and pass them
again; but for a moment they will be side-by-side, just as they were at
the start of the race \emph{but not back on the starting line}; their
meeting point will be some angular distance around the track from where
they began. Where and when this moment occurs is entirely a matter of
how fast each runner is running.

Returning to our orbitons, it becomes obvious that at some point while
they are orbiting, they will ``meet up'' at their closest approach to
one another, just as our runners did in the Olympic race --- they will
have achieved synodos. The first synodos to occur after \(T_0\) we label
\(T_1\); the second we label \(T_2\), and so on.

Here's the really cool thing: we can calculate exactly \emph{where} and
\emph{when} \(T_1, T_2\), etc. take place! \#\#\#\# The When The synodic
period (mentioned above) is the period of time that transpires between
any two synodoi: \[
\begin{equation}
\begin{split}
\Sigma &= \dfrac{P_\alpha \times P_\beta}{|P_\alpha - P_\beta|} \\
&= \dfrac{24.36 \times 11.72}{|24.26 - 11.72|} \\
&= \dfrac{285.499}{12.64} \\
\Sigma &= 22.587\; \text{days ✓}
\end{split}
\end{equation}
\] So, each synodos occurs just over \(22\frac{1}{2}\) days after the
preceding one.

\begin{quote}
Note that if we divide \(360^\circ\) by the difference angle
\(\theta_\Delta\) \[
\Sigma = \dfrac{360}{\theta_\Delta} = \dfrac{360}{15.939}= 22.587\; \text{days}
\] \ldots{} we also get the synodic period.
\end{quote}

Notice that the synodion, \(22.587\) days is more than two orbits of
\(M_\beta\) and less than one orbit of \(M_\alpha\). If we divide the
lengths of each orbiton's period by the length of the synodion: \[
\begin{align}
F_\alpha &= \dfrac{P_\alpha}{\Sigma} = \dfrac{24.36}{22.587} = 0.9272\; \text{orbit}\\
F_\beta &= \dfrac{P_\beta}{\Sigma} = \dfrac{11.72}{22.587} = 1.9272\; \text{orbits}
\end{align}
\] Do you see it? The decimal fraction of both of these quotients is the
same: \(0.9272\). We designate this as the \emph{synodial fraction}, and
denote it as \(F_\Sigma\), and it helps us to answer the question of
``where'' the synodoi take place.

What the synodial fraction tells us is that synodos \(T_1\) occurs
\(1.9272\) orbits of \(M_\beta\) and \(0.9272\) orbits of \(M_\alpha\)
from their starting time (\(T_0\)) and place (the base angle
\(B^\theta\)). Multiplying the synodial fraction by the number of
degrees in a circle: \[
\Theta = F_\Sigma \times 360^\circ = (0.9272)(333.7975^\circ) \qquad \text{The \emph{synodial angle}}
\] Subtracting the synodial angle from \(360^\circ\): \[
\Delta_\Theta = 360^\circ - \Theta = 360^\circ - 333.7975^\circ = 26.203^\circ \qquad \text{The \emph{synodial lag angle}}
\] \ldots{} we now know that each successive synodos occurs \(22.587\)
days after \emph{when} and \(26.203^\circ\) ``short'' of where the
previous one occurred. Thus the synodoi ``march backward'' around the
centron. We can calculate the precise angle relative to \(B^\theta\) any
given synodos occurs by: \[
\widehat{\Theta} = 360^\circ - MOD((S \times \Delta_\Theta), 360)
\] where S is the number of the synodos after \(T_0\); e.g.~for synodos
\(T_6, S = 6\). \#\# Diving Deeper Above, we learned about the synodic
period equation: \[
\begin{align}
\Sigma &= \dfrac{P_\alpha \times P_\beta}{|P_\alpha - P_\beta|} \\
\end{align}
\] Here is a comprehensive listing of the related equations: \[
\begin{aligned}
&&& \text{Given: } P_\alpha > P_\beta \\[1em]
\Sigma &= \dfrac{P_\alpha P_\beta}{|P_\alpha - P_\beta|}
& P_\alpha &= \dfrac{\Sigma P_\beta}{\Sigma - P_\beta}
& P_\beta &= \dfrac{\Sigma P_\alpha}{\Sigma + P_\alpha} \\[1em]
\dfrac{1}{\Sigma} &= \left|\dfrac{1}{P_\alpha}-\dfrac{1}{P_\beta}\right|
& \dfrac{1}{P_\alpha} &= \dfrac{1}{P_\beta}-\dfrac{1}{\Sigma}
& \dfrac{1}{P_\beta} &= \dfrac{1}{P_\alpha}+\dfrac{1}{\Sigma} \\[1em]
\end{aligned}
\] \[
\begin{array}{rcl}
Q &= \dfrac{P_\beta}{P_\alpha} 
&&&P_\alpha = P_\beta \times R = \dfrac{P_\beta}{Q} 
= \left(\dfrac{\Sigma}{Q} - \Sigma\right) = \Sigma(R - 1) \\[1em]
R &= \dfrac{P_\alpha}{P_\beta} 
&&&P_\beta = P_\alpha \times Q = \dfrac{P_\alpha}{R}
= \left(\Sigma - \dfrac{\Sigma}{R}\right) = \Sigma(1-Q)
\end{array}
\]

Where: - \(\Sigma\) = synodion - \(P_\alpha\) = period of the outer
body, \(M_\alpha\) (longer) - \(P_\beta\) = period of the inner body,
\(M_\beta\) (shorter) - \(Q\) = quotient of the outer body period to the
inner body period - \(R\) = ratio of the inner body period to the outer
body period

\begin{quote}
\emph{Note that the second row of equations perform the same
calculations as the first row by using the reciprocals of the orbital
periods.}
\end{quote}

All of this leads us naturally to the question:

\begin{quote}
Does any future synodos ever fall precisely at the base angle
\(B^\theta\) again? In other words is \[
\widehat{\Theta} = 0^\circ
\] \ldots{} ever true after \(T_0\)?
\end{quote}

The answer is ``yes'' and we can calculate that, too.

\section{Abstract}\label{abstract-34}

\textbf{Major Topics:}\\
- Defines the \textbf{synodial epoch (Y):} interval after which two
orbitons return to their original alignment angle (base angle
\(B^\theta\)).\\
- Related measures:\\
- \textbf{Epochal aggregate (Y₀):} total days in a synodial epoch.\\
- \textbf{Epochal interval (Ψ):} number of synodoi within one epoch.\\
- \textbf{Quarter synodial epoch (Y₀/4):} recurrence at cardinal angles
(90°, 180°, 270°).\\
- \textbf{Quarter epochal interval (Ψ/4):} synodoi per quarter epoch.\\
- \textbf{Epochal complements (Yα, Yβ):} complete orbits of each body
per epoch.\\
- Requires integer normalization of orbital periods for LCM/GCD
computation.\\
- Worked example (\(P_\alpha = 24.36\) d, \(P_\beta = 11.72\) d):\\
- \(Y₀ = 7137.48\) d ≈ 19.54 y.\\
- Ψ = 316 synodoi.\\
- Quarter epoch ≈ 1784 d ≈ 4.89 y.\\
- Yα = 293 orbits, Yβ = 609 orbits.

\textbf{Key Terms \& Symbols:}\\
- \textbf{Synodial Epoch (Y) {[}NEW{]}.}\\
- \textbf{Epochal Aggregate (Y₀) {[}NEW{]}.}\\
- \textbf{Epochal Interval (Ψ) {[}NEW{]}.}\\
- \textbf{Quarter Synodial Epoch (Y₀/4) {[}NEW{]}.}\\
- \textbf{Quarter Epochal Interval (Ψ/4) {[}NEW{]}.}\\
- \textbf{Epochal Complements (Yα, Yβ) {[}NEW{]}.}

\textbf{Cross-Check Notes:}\\
- No prior canon coverage of epochal cycles; this is a new layer
building on the \textbf{Synodion} framework.\\
- Depends on GCD/LCM math tools (already canonized in \textbf{The
Euclidean Algorithm}).\\
- \textbf{Status:} {[}NEW{]} --- introduces full synodial epoch
framework as part of resonance canon.

\section{The Synodial Epoch}\label{the-synodial-epoch}

The \emph{synodial epoch} (\(Y\)) is the time interval after which a
synodion occurs at the base angle \(B^\theta\) agin, completing a full
cycle.

The \emph{epochal aggregate} (\(Y_0\)) is the total number of synodia
that transpire within a synodial epoch.

Whereas we used the synodic formula: \[
\begin{align}
\Sigma &= \dfrac{P_\alpha \times P_\beta}{|P_\alpha - P_\beta|} \\
\end{align}
\] to calculate the synodic period, we need to use a different method to
calculate the Epochal Aggregate; we employ both the \emph{Least Common
Multiple} (LCM) and the \emph{Greatest Common Divisor} (GCD): \[
LCM(a, b) = \dfrac{a \times b}{GCD(a, b)}
\] \textgreater❗️Important ❗️ \textgreater1. Most modern calculators,
spreadsheets, and programming languages have built-in functions for
calculating both LCM and GCD, but be aware that \emph{both of these can
only be correctly calculated for integer inputs}. If something like
LCM(1.5, 3.3) returns a value rather than an error, it will likely be
the LCM of 1 and 3, not of 1.5 and 3.3. \textgreater2. The GCD
\emph{can} be calculated by hand using the \emph{Euclidean Algorithm}
(detailed elsewhere), calculators, spreadsheets, or programmed scripts
are much faster.

We need to calculate the LCM of \(P_\alpha = 24.36\) and
\(P_\beta = 11.72\), but neither of these are integers, so we
\emph{normalize} them by multiplying by the factor of ten which will
render the more precise of the two into an integer value. In this case,
both have two decimal places of precision, so multiplying by
\(10^2 = 100\) will be sufficient: \[
P'_\alpha = 24.36 \times 100 = 2436 \quad \text{and} \quad P'_\beta = 11.72 \times 100 = 1172
\] Now we can compute the LCM by: \[
\begin{equation}
\begin{split}
LCM(a, b) &= \dfrac{a \times b}{GCD(a, b)} \\[0.5em]
&= \dfrac{2436 \times 1172}{GCD(2436, 1172)} \\[0.5em]
&= \dfrac{2854992}{4} \\
&= 713748\; ✓
\end{split}
\end{equation}
\] Now, \emph{since we normalized our inputs} by scaling up by a factor
of \(10^2 = 100\), we need to scale this result \emph{down} by the same
factor: \[
Y_0 = \dfrac{713748}{100} = 7137.48\; \text{days } ✓
\] The \emph{epochal interval} (\(\Psi\)) is calculated by dividing the
Epochal Aggregate (\(Y_0\)) by the synodic period (\(\Sigma\)): \[
\begin{equation}
\begin{split}
\Psi &= \dfrac{Y_0}{\Sigma} \\[0.5em]
&= \dfrac{7137.48}{22.587} \\[0.5em]
&= 316\; \text{synodia}
\end{split}
\end{equation}
\] This reveals that \(T_{316}\) occurs in the same configuration at the
base angle \(B^\theta\). We can double-check by supplying \(S=316\) to
our synodian instance angle (\(\widehat{\Theta}\)) equation: \[
\begin{equation}
\begin{split}
\widehat{\Theta} &= 360^\circ - MOD((S \times \Lambda_\Theta), 360^\circ) \\
&= 360^\circ - MOD((316 \times 26.203^\circ), 360^\circ) \\
&= 360^\circ - 360^\circ \\
&= 0\; ✓
\end{split}
\end{equation}
\] To convert the epochal aggregate \(\Psi = 7137.48\) days into
something more useful, we can divide by the number of days in a sidereal
year (\(≈ 365.2564\)): \[
Y_0^y = \dfrac{7137.48}{365.2564} = 19.54\; \text{years }✓
\] \ldots{} which is about \[
19^y\;197^d\;11^h\;16^m\;20.724^s
\] give-or-take a millisecond. \#\#\# Quarter Synodial Epoch \#\#\#\#
The Where A quarter synodial epoch (\(Y_{/4}\)) is an important
quantity, as well; at these intervals, synodoi occur precisely on a
cardinal angle (\(90^\circ, 180^\circ, 270^\circ\)) in succession. We
call these the \emph{cardinal synodoi}. They occur, of course every one
fourth of the synodion: \[
Y_{/4} = \dfrac{Y_0}{4}
\] In our worked example above, the quarter synodial epoch is: \[
\begin{gather}
Y_{/4} = \dfrac{Y_0}{4} = \dfrac{7137.48}{4} = 1784.37\; \text{days} \\[0.5em]
Y_{/4}^y = \dfrac{Y_{/4}}{365.2564} = \dfrac{1784.37}{365.2564} ≈ 4.89\; \text{years}
\end{gather}
\] \#\#\# Quarter Epochal Interval Similarly, a \emph{quarter epochal
interval} (\(\Psi_{/4}\)) is the number of \emph{common synodoi} that
occur between cardinal synodoi: \[
\Psi_{/4} = \dfrac{\Psi_0}{4} = \dfrac{316}{4} = 79\; \text{synodoi}
\] \#\#\# The Epochal Complements There are two of these, the
\emph{epochal major complement} (\(Y_\alpha\)) and the \emph{epochal
minor complement} (\(Y_\beta\)), which are the number of complete orbits
of orbiton \(M_\alpha\) and \(M_\beta\), respectively, per each synodial
epoch: \[
\begin{align}
Y_\alpha &= \dfrac{Y_0}{P_\alpha} \\[1em]
Y_\beta &= \dfrac{Y_0}{P_\beta}
\end{align}
\] In our worked example above: \[
\begin{align}
Y_\alpha &= \dfrac{Y_0}{P_\alpha} = \dfrac{7137.48}{24.36} = 293\; \text{orbits }✓ \\[1em]
Y_\beta &= \dfrac{Y_0}{P_\beta} = \dfrac{7137.48}{11.72} = 609\; \text{orbits }✓
\end{align}
\]

\part{Mononic Classes}

\chapter{Abstract}\label{abstract-35}

\textbf{Major Topics:}\\
- Core algebraic relationships linking the five fundamental
\textbf{mononic} parameters: \textbf{mass (m), radius (r), density (ρ),
surface gravity (g), and escape velocity (vₑ)}.\\
- Provides multiple equivalent formulations, allowing any parameter to
be derived from any two others.\\
- Functions as the canonical \textbf{equation-of-state matrix} for monon
design.\\
- Establishes normalized Earth-relative scaling (⨁ units).

\textbf{Key Terms \& Symbols:}\\
- \textbf{m} --- monon mass (⨁).\\
- \textbf{r} --- monon radius (⨁).\\
- \textbf{ρ} --- Mean density (⨁).\\
- \textbf{g} --- Surface gravity (⨁).\\
- \textbf{vₑ} --- Escape velocity (⨁).

\textbf{Cross-Check Notes:}\\
- Reinforces \textbf{Core Parameter Precedence} by showing equivalences
and constraints.\\
- Serves as the mathematical backbone for all mononic
monoclassifications (Geotic, Gaean, Rheatic).\\
- Companion to \textbf{Geotic Ground States}, which simplifies these
identities when one or more parameters = 1.

\chapter{Monon Equations of State}\label{monon-equations-of-state}

\begin{longtable}[]{@{}
  >{\centering\arraybackslash}p{(\linewidth - 8\tabcolsep) * \real{0.2041}}
  >{\centering\arraybackslash}p{(\linewidth - 8\tabcolsep) * \real{0.1973}}
  >{\centering\arraybackslash}p{(\linewidth - 8\tabcolsep) * \real{0.2585}}
  >{\centering\arraybackslash}p{(\linewidth - 8\tabcolsep) * \real{0.1497}}
  >{\centering\arraybackslash}p{(\linewidth - 8\tabcolsep) * \real{0.1905}}@{}}
\toprule\noalign{}
\begin{minipage}[b]{\linewidth}\centering
Mass(m)
\end{minipage} & \begin{minipage}[b]{\linewidth}\centering
Radius(r)
\end{minipage} & \begin{minipage}[b]{\linewidth}\centering
Density(ρ)
\end{minipage} & \begin{minipage}[b]{\linewidth}\centering
Gravity(g)
\end{minipage} & \begin{minipage}[b]{\linewidth}\centering
Escape Velocity(vₑ)
\end{minipage} \\
\midrule\noalign{}
\endhead
\bottomrule\noalign{}
\endlastfoot
\(m=gr^2\) & \(r=\dfrac{g}{\rho}\) & \(\rho=\dfrac{m}{r^3}\) &
\(g=\dfrac{m}{r^2}\) & \(v_e=\sqrt{g r}\) \\
\(m=\rho r^3\) & \(r=\sqrt{\dfrac{m}{g}}\) & \(\rho=\dfrac{g}{r}\) &
\(g=r \rho\) & \(v_e=\sqrt{\dfrac{m}{r}}\) \\
\(m=\dfrac{g^3}{\rho^2}\) & \(r=\sqrt[3]{\dfrac{m}{\rho}}\) &
\(\rho=\sqrt{\dfrac{g^3}{m}}\) & \(g=\sqrt[3]{m \rho^2}\) &
\(v_e=\dfrac{g}{\sqrt{\rho}}\) \\
\(m=\dfrac{v_e^3}{\sqrt{\rho}}\) & \(r=\dfrac{v_e}{\sqrt{\rho}}\) &
\(\rho=\left(\dfrac{v_e}{r}\right)^2\) & \(g=v_e \sqrt{\rho}\) &
\(v_e=\sqrt[4]{m g}\) \\
\(m=\dfrac{v_e^4}{g}\) & \(r=\dfrac{v_e^2}{g}\) &
\(\rho=\left(\dfrac{g}{v_e}\right)^2\) & \(g=\dfrac{v_e^2}{r}\) &
\(v_e=r \sqrt{\rho}\) \\
\(m=r v_e^2\) & \(r=\dfrac{m}{v_e^2}\) &
\(\rho=\left(\dfrac{v_e^3}{m}\right)^2\) & \(g=\dfrac{v_e^4}{m}\) &
\(v_e=\sqrt[6]{m^2 \rho}\) \\
\end{longtable}

\section{Abstract}\label{abstract-36}

\textbf{Major Topics:}\\
- Outlines the \textbf{dual-branch classification system} for monons.\\
- \textbf{Conformation branch} (structural/physical): based on
composition and physical form.\\
- Orders: Lithic, Astatic, Aetheric, Ulsic.\\
- Forms: Petriform, Carboform, Pagoform, Fluxiform, Transiform,
Pneumoform, Haliform, Neutraform, Quarkform, etc.\\
- \textbf{Ontosomic branch} (life-relevance): based on relation to
biospheric potential.\\
- Classes: Geotic, Gaean, Rheatic, Xenotic.\\
- Types: Subdivisions (e.g., Marginal Geotic, Dense Hot Rheatic).\\
- States (Climostates): Global climate regimes (Cryostate, Hydrostate,
Pyrostate, Pluristate, etc.).\\
- Conditions: Atmospheric/seasonal descriptors (Tempestal, Zephyral,
Serenal, etc.).\\
- Both branches are independent but complementary lenses for describing
the same planemon.

\textbf{Key Terms \& Symbols:}\\
- \textbf{Planemon} --- planetary monon, trunk-level category.\\
- \textbf{Conformation} --- physical/chemical classification.\\
- \textbf{Ontosomy} --- life-relevance classification.\\
- \textbf{Orders/Forms} --- structural breakdown of planemon types.\\
- \textbf{Classes/Types/States/Conditions} --- biospheric breakdown of
livability.

\textbf{Cross-Check Notes:}\\
- Provides the \textbf{master map} for all other monon-related
classifications.\\
- Ensures that \textbf{physical identity} (Conformation) and
\textbf{biospheric identity} (Ontosomic) can coexist without
collision.\\
- Anchors subordinate classifications (e.g., Gaean, Rheatic, Hydrostate
subtypes).

\chapter{🌍 Planemon Framework (Master
Map)}\label{planemon-framework-master-map}

All \textbf{planemons} (planetary monons) share a \textbf{common trunk}
--- then branch into two complementary classification systems.

\section{🌳 Trunk (Shared Levels)}\label{trunk-shared-levels-1}

\begin{itemize}
\tightlist
\item
  \textbf{Frame} → Monon\\
\item
  \textbf{Group} → Planemon
\end{itemize}

\section{🪨 Conformation Branch (Structural /
Physical)}\label{conformation-branch-structural-physical}

Focus: \textbf{What the planemon is made of} - \textbf{Order} →
Composition families\\
- Lithic (rock/metal)\\
- Astatic (mixed/volatile)\\
- Aetheric (gaseous)\\
- Ulsic (exotic/degenerate) - \textbf{Form} → Subdivisions of each
order\\
- Lithic → Petriform, Carboform\\
- Astatic → Pagoform, Fluxiform\\
- Aetheric → Transiform, Pneumoform, Haliform\\
- Ulsic → Neutraform, Quarkform, etc.

\section{🌱 Ontosomic Branch (Life-Relevance /
Biospheric)}\label{ontosomic-branch-life-relevance-biospheric}

Focus: \textbf{How the planemon relates to life (esp.~Earthlike
envelopes)} - \textbf{Class} → Broad livability envelopes\\
- Geotic (Earth-normal range)\\
- Gaean (Earth-twin slice of Geotic)\\
- Rheatic (super-Earths with biospheric richness)\\
- Xenotic (exotic chemistries / non-carbonic) - \textbf{Type} →
Subdivisions of classes\\
- Marginal Geotic, Dense Hot Rheatic, etc.

\begin{itemize}
\tightlist
\item
  \textbf{State (Climostates)} → Global climatic conditions

  \begin{itemize}
  \tightlist
  \item
    Cryostate (frozen)\\
  \item
    Hydrostate (temperate hydrosphere)\\
  \item
    Pyrostate (runaway greenhouse)\\
  \item
    Pluristate (diverse climates)\\
  \item
    etc.
  \end{itemize}
\item
  \textbf{Condition} → Finer descriptors of atmospheric/seasonal
  character

  \begin{itemize}
  \tightlist
  \item
    Tempestal (storm-dominated)\\
  \item
    Zephyral (mild winds, gentle climate)\\
  \item
    Serenal (stable, calm, serene environment)
  \end{itemize}
\end{itemize}

\section{🧭 Principle}\label{principle-1}

\begin{itemize}
\tightlist
\item
  \textbf{Conformation} = \emph{physical chemistry} of the body.\\
\item
  \textbf{Ontosomy} = \emph{life-relevance} of the body.\\
\item
  Both are \textbf{valid, independent lenses} for describing the same
  planemon.
\item
  Example:

  \begin{itemize}
  \tightlist
  \item
    Earth = Lithic (Order) → Petriform (Form)\\
  \item
    Earth = Geotic (Class) → Gaean (Type) → Pluristate (State) → Serenal
    (Condition)\#\#
  \end{itemize}
\end{itemize}

\section{Abstract}\label{abstract-37}

\textbf{Major Topics:}\\
- Monon composition taxonomy (configuration classes).\\
- Four primary categories: Lithic, Astatic, Aetheric, Ulsic.\\
- Density ranges for Lithic, Astatic, and Aetheric.\\
- Subclasses (conformations) within each category.

\textbf{Key Terms \& Symbols:}\\
- Lithic → Petriform (silicate), Carboform (carbon).\\
- Astatic → Pagoform (ice-rich), Fluxiform (liquid-dominated).\\
- Aetheric → Transiform (gas dwarfs), Pneumoform (ice giants), Haliform
(gas giants).\\
- Ulsic → placeholder for exotic/degenerate matter (neutron stars, quark
stars, black holes).\\
- Densities expressed in g/cm³ and ⨁ (Earth-density units).

\textbf{Cross-Check Notes:}\\
- Canon as of Glossary v1.217.\\
- Complements \textbf{Mononic Mass Classes} (size taxonomy), creating a
two-axis classification system (mass × composition).\\
- ``Haliform'' here = Gas Giants → ensure no confusion with earlier
Haliform (halogen-bearing lithic variant).\\
- Overlaps with terms already staged in Glossary: Lithic, Astatic,
Aetheric, Ulsic.

\section{Monon Morphotypes and
Conformations}\label{monon-morphotypes-and-conformations}

All monons should be thought of as falling roughly into one of four
categories of \textbf{\emph{morphotypes}}:

\begin{enumerate}
\def\labelenumi{\arabic{enumi}.}
\tightlist
\item
  \textbf{\emph{Lithic}}~(from Greek \emph{lithos} ``stone, rock'') --
  Fully solid planemons (rocky, carbon-rich, metallic elements).
\item
  \textbf{\emph{Astatic}}~(from Greek \emph{a} ``not, none'' +
  \emph{status} ``state condition'') -- Mixed-phase planemons (icy,
  oceanic, or chemically altered worlds that have no fixed state).
\item
  \textbf{\emph{Aetheric}}~(from Latin \emph{aer} ``air'') -- Gaseous
  planemons (gas dwarfs, ice giants, and gas giants).
\item
  \textbf{\emph{Ulsic}}~(from Latin \emph{uls} ``beyond'') -- Exotic or
  theoretical matter planemons (placeholder for degenerate matter
  planemons --- neutron stars, black holes, quark stars, etc. --- if
  needed).
\end{enumerate}

The first three compositions span specific \textbf{density} ranges:

\begin{enumerate}
\def\labelenumi{\arabic{enumi}.}
\tightlist
\item
  \textbf{Lithic}: 3.5 g/cm³ -- 8.0 g/cm³ (0.6347⨁ -- 1.451⨁)
\item
  \textbf{Astatic}: 1.0 g/cm³ -- \textless3.5 g/cm³ (0.1814⨁ -- 0.6347⨁)
\item
  \textbf{Aetheric}: 0.1 g/cm³ -- 1.0 g/cm³ (0.0181⨁ -- 0.1814⨁)
\end{enumerate}

And each has subsets, called \textbf{\emph{conformations}}:

\begin{enumerate}
\def\labelenumi{\arabic{enumi}.}
\tightlist
\item
  \textbf{Lithic Morphotype}

  \begin{itemize}
  \tightlist
  \item
    \textbf{\emph{Petriform}}: Rocky, silicate-based planemons.
  \item
    \textbf{\emph{Carboform}}: Carbon-dominated (graphite, diamond)
    planemons
  \end{itemize}
\item
  \textbf{Astatic Morphotype}

  \begin{itemize}
  \tightlist
  \item
    \textbf{\emph{Pagoform}}: Ice-rich planemons (frozen volatiles like
    water, ammonia, methane, etc.)
  \item
    \textbf{\emph{Fluxiform}}: Liquid-dominated planemons (H₂O, but also
    other volatiles)
  \end{itemize}
\item
  **Aetheric Morphotype

  \begin{itemize}
  \tightlist
  \item
    \textbf{\emph{Transiform}}: Gas Dwarfs
  \item
    \textbf{\emph{Pneumoform}}: Ice Giants
  \item
    \textbf{\emph{Haliform}}: Gas Giants
  \end{itemize}
\end{enumerate}

These are intended to categorize monons \emph{in their current
configuration}. A monon may (usually will) assume different qualities
depending on its age and environment.\#\# Abstract

\textbf{Major Topics:}\\
- Provides the \textbf{canonical reference table} for \textbf{monon mass
classes and monoclasses}, establishing the taxonomy framework used
across WCB.\\
- Defines the hierarchical structure of \textbf{--moic categories} by
mass domain:\\
- \textbf{micromon, minimon, midimon, planemon, intermon, stellamon,
supermon, ultramon, hypermon.}\\
- Associates each category with its \textbf{mass range, radius bounds,
and representative examples} (from asteroids to stellar bodies).\\
- Establishes the system of \textbf{terracentric prefixes} (centiterran,
deciterran, kiloterran, megaterran, etc.) for finer-grained scaling
across mass domains.\\
- Connects monoclasses (e.g., \textbf{telluric, rheatic, gaean}) with
their appropriate mass classes, showing where habitability-relevant
subtypes emerge.\\
- Serves as the master framework to which specialized notes (e.g.,
\textbf{Gaean planemons, Rheatic planemons}) attach.

\textbf{Key Terms \& Symbols:}\\
- \textbf{Mass classes:} micromon, minimon, midimon, planemon, intermon,
stellamon, supermon, ultramon, hypermon.\\
- \textbf{Terracentric scaling prefixes:} centiterran, deciterran,
kiloterran, megaterran, etc.\\
- \textbf{Ontotype:} compositional subtypes (telluric, rheatic, gaean,
xenotic).\\
- \textbf{µT (microterran units):} mass measure used for micromons and
smaller classes.

\textbf{Cross-Check Notes:}\\
- This file provides the \textbf{systemic backbone} for all WCB
classification work.\\
- All major terms are already present in the glossary as of v1.222.\\
- Future consideration: whether \textbf{terracentric prefixes} warrant
their own glossary entry or remain embedded in this reference note.

\chapter{Monon Mass Classes and
monoclasses}\label{monon-mass-classes-and-monoclasses}

\section{Listed by Mass Class}\label{listed-by-mass-class}

\begin{longtable}[]{@{}
  >{\raggedright\arraybackslash}p{(\linewidth - 8\tabcolsep) * \real{0.1875}}
  >{\raggedleft\arraybackslash}p{(\linewidth - 8\tabcolsep) * \real{0.1944}}
  >{\raggedleft\arraybackslash}p{(\linewidth - 8\tabcolsep) * \real{0.1944}}
  >{\raggedleft\arraybackslash}p{(\linewidth - 8\tabcolsep) * \real{0.1528}}
  >{\raggedright\arraybackslash}p{(\linewidth - 8\tabcolsep) * \real{0.2708}}@{}}
\toprule\noalign{}
\endhead
\bottomrule\noalign{}
\endlastfoot
deniterran & 0.0000000001 & 0.000000001 & 10⁻¹⁰ & nanomon \\
nanoterran & 0.000000001 & 0.00000001 & 10⁻⁹ & nanomon \\
oktiterran & 0.00000001 & 0.0000001 & 10⁻⁸ & nanomon \\
septiterran & 0.0000001 & 0.000001 & 10⁻⁷ & nanomon \\
microterran & 0.000001 & 0.00001 & 10⁻⁶ & micromon \\
pentiterran & 0.00001 & 0.0001 & 10⁻⁵ & minimon \\
demiterran & 0.0001 & 0.001 & 10⁻⁴ & minimon \\
milliterran & 0.001 & 0.01 & 10⁻³ & midimon \\
centiterran & 0.01 & 0.1 & 10⁻² & planemon \\
deciterran & 0.1 & 1 & 10⁻¹ & planemon \\
terran & 1 & 10 & 10⁰ & planemon \\
dekaterran & 10 & 100 & 10¹ & planemon \\
hectoterran & 100 & 1000 & 10² & planemon \\
kiloterran & 1000 & 10000 & 10³ & planemon to 4131⨁,then intermon \\
myriaterran & 10000 & 100000 & 10⁴ & intermon to 2.664 myt,then
stellamon \\
hexaterran & 100000 & 1000000 & 10⁵ & stellamon \\
megaterran & 1000000 & 10000000 & 10⁶ & supermon \\
heptoterran & 10000000 & 100000000 & 10⁷ & ultramon \\
octoterran & 100000000 & 1000000000 & 10⁸ & ultramon \\
gigaterran & 1000000000 & 10000000000 & 10⁹ & hypermon \\
denoterran & 10000000000 & 100000000000 & 10¹⁰ & beyond -mo
classification \\
ondoterran & 100000000000 & 1000000000000 & 10¹¹ & beyond -mo
classification \\
teraterran & 1000000000000 & 10000000000000 & 10¹² & beyond -mo
classification \\
\end{longtable}

\section{Listed by monoclass}\label{listed-by-monoclass}

\begin{longtable}[]{@{}
  >{\raggedright\arraybackslash}p{(\linewidth - 6\tabcolsep) * \real{0.2708}}
  >{\raggedleft\arraybackslash}p{(\linewidth - 6\tabcolsep) * \real{0.2917}}
  >{\raggedleft\arraybackslash}p{(\linewidth - 6\tabcolsep) * \real{0.2917}}
  >{\raggedleft\arraybackslash}p{(\linewidth - 6\tabcolsep) * \real{0.1458}}@{}}
\toprule\noalign{}
\begin{minipage}[b]{\linewidth}\raggedright
\end{minipage} & \begin{minipage}[b]{\linewidth}\raggedleft
\end{minipage} & \begin{minipage}[b]{\linewidth}\raggedleft
\end{minipage} & \begin{minipage}[b]{\linewidth}\raggedleft
Exponent Range
\end{minipage} \\
\midrule\noalign{}
\endhead
\bottomrule\noalign{}
\endlastfoot
nanomon & 0.0000000001 & 0.000001 & 10⁻¹⁰ --- 10⁻⁶ \\
micromon & 0.000001 & 0.00001 & 10⁻⁶ --- 10⁻⁵ \\
minimon & 0.00001 & 0.001 & 10⁻⁵ --- 10⁻³ \\
midimon & 0.001 & 0.01 & 10⁻³ --- 10⁻² \\
planemon & 0.01 & 4131 & 10⁻² --- 10³ \\
intermon & 4131 & 266400 & 10³ --- 10⁵ \\
stellamon & 266400 & 1000000 & 10⁵ --- 10⁶ \\
supermon & 1000000 & 10000000 & 10⁶ --- 10⁷ \\
ultramon & 10000000 & 1000000000 & 10⁷ --- 10⁹ \\
hypermon & 1000000000 & 10000000000 & 10⁹ --- 10¹⁰ \\
\end{longtable}

\section{A Monon Survey}\label{a-monon-survey}

\begin{longtable}[]{@{}llll@{}}
\toprule\noalign{}
Object & Mass ⨁ & Mass Class & Monoclass \\
\midrule\noalign{}
\endhead
\bottomrule\noalign{}
\endlastfoot
Doris & 0.0000020 & microterran & micromon \\
Amphitrite & 0.0000025 & microterran & micromon \\
Sylvia & 0.0000025 & microterran & micromon \\
Iris & 0.0000027 & microterran & micromon \\
Egeria & 0.0000027 & microterran & micromon \\
Diotima & 0.0000027 & microterran & micromon \\
Thisbe & 0.0000031 & microterran & micromon \\
Psyche & 0.0000038 & microterran & micromon \\
Europa (52) & 0.0000038 & microterran & micromon \\
Juno & 0.0000048 & microterran & micromon \\
Eunomia & 0.0000053 & microterran & micromon \\
Herculina & 0.0000055 & microterran & micromon \\
Mimas & 0.0000063 & microterran & micromon \\
Davida & 0.0000063 & microterran & micromon \\
Interamnia & 0.0000065 & microterran & micromon \\
Euphrosyne & 0.0000097 & microterran & micromon \\
Miranda & 0.0000110 & pentiterran & minimon \\
Hygiea & 0.0000145 & pentiterran & minimon \\
Enceladus & 0.0000181 & pentiterran & minimon \\
Pallas & 0.0000337 & pentiterran & minimon \\
Vesta & 0.0000434 & pentiterran & minimon \\
Salacia & 0.0000821 & pentiterran & minimon \\
Tethys & 0.0001034 & demiterran & minimon \\
Orcus & 0.0001059 & demiterran & minimon \\
Ceres & 0.0001573 & demiterran & minimon \\
Dione & 0.0001834 & demiterran & minimon \\
Umbriel & 0.0002010 & demiterran & minimon \\
Ariel & 0.0002261 & demiterran & minimon \\
Quaoar & 0.0002362 & demiterran & minimon \\
Charon & 0.0002546 & demiterran & minimon \\
Gonggong & 0.0002931 & demiterran & minimon \\
Iapetus & 0.0003024 & demiterran & minimon \\
Rhea & 0.0003863 & demiterran & minimon \\
Oberon & 0.0005049 & demiterran & minimon \\
Makemake & 0.0005193 & demiterran & minimon \\
Titania & 0.0005863 & demiterran & minimon \\
Haumea & 0.0006717 & demiterran & minimon \\
Pluto & 0.0021776 & milliterran & midimon \\
Eris & 0.0027638 & milliterran & midimon \\
Triton & 0.0035846 & milliterran & midimon \\
Europa & 0.0080402 & milliterran & midimon \\
Luna & 0.0123077 & centiterran & planemon \\
Io & 0.0149749 & centiterran & planemon \\
Callisto & 0.0180201 & centiterran & planemon \\
Titan & 0.0225327 & centiterran & planemon \\
Ganymede & 0.0248224 & centiterran & planemon \\
Mercury & 0.0553099 & centiterran & planemon \\
Kepler-138b & 0.0667000 & centiterran & planemon \\
Mars & 0.108 & deciterran & planemon \\
Kepler-453b & 0.200 & deciterran & planemon \\
TRAPPIST-1d & 0.300 & deciterran & planemon \\
TRAPPIST-1h & 0.326 & deciterran & planemon \\
Kepler-70b & 0.445 & deciterran & planemon \\
Kepler-70c & 0.636 & deciterran & planemon \\
Kepler-26c & 0.650 & deciterran & planemon \\
TRAPPIST-1e & 0.770 & deciterran & planemon \\
Venus & 0.816 & deciterran & planemon \\
HAT-P-32b & 0.860 & deciterran & planemon \\
TRAPPIST-1f & 0.930 & deciterran & planemon \\
Earth & 1.000 & terran & planemon \\
TRAPPIST-1b & 1.017 & terran & planemon \\
Teegarden's Star b & 1.050 & terran & planemon \\
Teegarden's Star c & 1.110 & terran & planemon \\
TRAPPIST-1g & 1.148 & terran & planemon \\
TRAPPIST-1c & 1.156 & terran & planemon \\
Kepler-138d & 1.170 & terran & planemon \\
Proxima Centauri b & 1.270 & terran & planemon \\
Wolf 1061b & 1.360 & terran & planemon \\
Kepler-186f & 1.400 & terran & planemon \\
Gliese 1061 d & 1.640 & terran & planemon \\
TOI 700 d & 1.720 & terran & planemon \\
Gliese 1061 c & 1.740 & terran & planemon \\
Ross 128 b & 1.800 & terran & planemon \\
Kepler-78b & 1.838 & terran & planemon \\
Kepler-177b & 1.907 & terran & planemon \\
Kepler-138c & 1.970 & terran & planemon \\
Kepler-1972b & 2.034 & terran & planemon \\
Kepler-1972c & 2.098 & terran & planemon \\
Kepler-11f & 2.103 & terran & planemon \\
Kepler-69c & 2.140 & terran & planemon \\
K2-72e & 2.210 & terran & planemon \\
Kepler-51b & 2.225 & terran & planemon \\
Kepler-442b & 2.300 & terran & planemon \\
Kepler-11b & 2.343 & terran & planemon \\
Kepler-62f & 2.800 & terran & planemon \\
Kepler-406c & 2.860 & terran & planemon \\
Kepler-114c & 2.860 & terran & planemon \\
Kepler-11c & 2.860 & terran & planemon \\
Gliese 273b & 2.890 & terran & planemon \\
Kepler-101c & 3.178 & terran & planemon \\
Kepler-97b & 3.496 & terran & planemon \\
Kepler-414b & 3.496 & terran & planemon \\
Kepler-98b & 3.496 & terran & planemon \\
HD 85512 b & 3.600 & terran & planemon \\
Kepler-307c & 3.639 & terran & planemon \\
Gliese 667 Cc & 3.709 & terran & planemon \\
Kepler-102d & 3.814 & terran & planemon \\
Kepler-114d & 3.814 & terran & planemon \\
Kepler-60c & 3.849 & terran & planemon \\
Tau Ceti f & 3.930 & terran & planemon \\
Kepler-48b & 3.941 & terran & planemon \\
Kepler-29c & 4.001 & terran & planemon \\
Kepler-93b & 4.020 & terran & planemon \\
Kepler-80e & 4.128 & terran & planemon \\
Kepler-62c & 4.131 & terran & planemon \\
Kepler-289d & 4.131 & terran & planemon \\
Kepler-79e & 4.131 & terran & planemon \\
Kepler-51c & 4.131 & terran & planemon \\
Kepler-60d & 4.160 & terran & planemon \\
Kepler-10b & 4.164 & terran & planemon \\
Kepler-60b & 4.189 & terran & planemon \\
Wolf 1061c & 4.300 & terran & planemon \\
Kepler-36b & 4.449 & terran & planemon \\
Kepler-1659c & 4.449 & terran & planemon \\
Kepler-62e & 4.500 & terran & planemon \\
Kepler-29b & 4.510 & terran & planemon \\
Kepler-105c & 4.599 & terran & planemon \\
HD 219134 b & 4.740 & terran & planemon \\
Kepler-223e & 4.799 & terran & planemon \\
Kepler-452b & 5.000 & terran & planemon \\
Kepler-21b & 5.078 & terran & planemon \\
Kepler-1655b & 5.085 & terran & planemon \\
61 Virginis b & 5.100 & terran & planemon \\
Kepler-223c & 5.101 & terran & planemon \\
Kepler-26b & 5.120 & terran & planemon \\
Gliese 832 c & 5.400 & terran & planemon \\
Kepler-68b & 5.968 & terran & planemon \\
Kepler-99b & 6.038 & terran & planemon \\
Kepler-92c & 6.038 & terran & planemon \\
Kepler-350c & 6.038 & terran & planemon \\
Kepler-305c & 6.038 & terran & planemon \\
Kepler-79c & 6.038 & terran & planemon \\
Kepler-79d & 6.038 & terran & planemon \\
CoRoT-7b & 6.060 & terran & planemon \\
Kepler-406b & 6.356 & terran & planemon \\
Kepler-87c & 6.356 & terran & planemon \\
PSO J318.5−22 & 6.500 & terran & planemon \\
Kepler-11d & 6.705 & terran & planemon \\
Kepler-80c & 6.741 & terran & planemon \\
Kepler-80d & 6.750 & terran & planemon \\
Gliese 876 d & 6.830 & terran & planemon \\
Kepler-454b & 6.839 & terran & planemon \\
Kepler-80b & 6.928 & terran & planemon \\
LHS 1140 b & 6.980 & terran & planemon \\
Gliese 581d & 6.980 & terran & planemon \\
Kepler-18b & 6.992 & terran & planemon \\
Proxima Centauri c & 7.000 & terran & planemon \\
HD 40307 g & 7.090 & terran & planemon \\
Kepler-68c & 7.198 & terran & planemon \\
Kepler-100b & 7.309 & terran & planemon \\
Kepler-289b & 7.309 & terran & planemon \\
Kepler-10c & 7.370 & terran & planemon \\
Kepler-223b & 7.398 & terran & planemon \\
Kepler-307b & 7.440 & terran & planemon \\
Kepler-177c & 7.627 & terran & planemon \\
Kepler-51d & 7.627 & terran & planemon \\
Wolf 1061d & 7.700 & terran & planemon \\
Kepler-48d & 7.945 & terran & planemon \\
Kepler-36c & 7.945 & terran & planemon \\
Kepler-11e & 7.945 & terran & planemon \\
55 Cancri e & 7.990 & terran & planemon \\
Kepler-223d & 7.999 & terran & planemon \\
Gliese 1214 b & 8.170 & terran & planemon \\
Kepler-131c & 8.263 & terran & planemon \\
Kepler-338e & 8.581 & terran & planemon \\
Kepler-96b & 8.581 & terran & planemon \\
Kepler-88b & 8.581 & terran & planemon \\
K2-18b & 8.630 & terran & planemon \\
Kepler-1659b & 8.898 & terran & planemon \\
Kepler-102e & 8.898 & terran & planemon \\
Kepler-62b & 9.534 & terran & planemon \\
Kepler-25b & 9.598 & terran & planemon \\
Kepler-20b & 9.700 & terran & planemon \\
Kepler-103b & 9.852 & terran & planemon \\
Kepler-20d & 10.068 & dekaterran & planemon \\
Kepler-89b & 10.487 & dekaterran & planemon \\
Kepler-106c & 10.487 & dekaterran & planemon \\
Kepler-305b & 10.487 & dekaterran & planemon \\
Kepler-94b & 10.805 & dekaterran & planemon \\
Kepler-79b & 10.901 & dekaterran & planemon \\
Kepler-417b & 11.123 & dekaterran & planemon \\
Kepler-106e & 11.123 & dekaterran & planemon \\
Kepler-30b & 11.441 & dekaterran & planemon \\
Kepler-113b & 11.759 & dekaterran & planemon \\
Kepler-37c & 12.000 & dekaterran & planemon \\
Kepler-37d & 12.200 & dekaterran & planemon \\
Kepler-95b & 13.030 & dekaterran & planemon \\
Kepler-238f & 13.348 & dekaterran & planemon \\
Kepler-62d & 13.983 & dekaterran & planemon \\
Kepler-20c & 14.425 & dekaterran & planemon \\
Uranus & 14.500 & dekaterran & planemon \\
Kepler-48c & 14.609 & dekaterran & planemon \\
Kepler-350d & 14.937 & dekaterran & planemon \\
Kepler-89c & 15.572 & dekaterran & planemon \\
K2-56b & 16.000 & dekaterran & planemon \\
Kepler-131b & 16.208 & dekaterran & planemon \\
Kepler-276d & 16.208 & dekaterran & planemon \\
Kepler-276c & 16.526 & dekaterran & planemon \\
Kepler-18d & 16.526 & dekaterran & planemon \\
Kepler-1661b & 16.843 & dekaterran & planemon \\
Neptune & 17.150 & dekaterran & planemon \\
Kepler-18c & 17.161 & dekaterran & planemon \\
Kepler-396c & 17.797 & dekaterran & planemon \\
K2-66b & 21.300 & dekaterran & planemon \\
Gliese 436 b & 21.360 & dekaterran & planemon \\
Kepler-56b & 22.246 & dekaterran & planemon \\
Kepler-30d & 23.199 & dekaterran & planemon \\
Kepler-4b & 24.471 & dekaterran & planemon \\
Kepler-25c & 24.598 & dekaterran & planemon \\
Kepler-11g & 25.106 & dekaterran & planemon \\
HAT-P-11b (Kepler-3b) & 25.742 & dekaterran & planemon \\
Kepler-328b & 28.602 & dekaterran & planemon \\
Kepler-414c & 29.873 & dekaterran & planemon \\
Kepler-117b & 29.873 & dekaterran & planemon \\
Kepler-128b & 30.827 & dekaterran & planemon \\
Kepler-89e & 31.780 & dekaterran & planemon \\
Kepler-128c & 33.369 & dekaterran & planemon \\
Kepler-122f & 35.911 & dekaterran & planemon \\
Kepler-22b & 35.911 & dekaterran & planemon \\
Kepler-103c & 36.229 & dekaterran & planemon \\
Kepler-145b & 37.141 & dekaterran & planemon \\
Kepler-279d & 37.500 & dekaterran & planemon \\
Kepler-328c & 39.407 & dekaterran & planemon \\
Kepler-35b & 40.361 & dekaterran & planemon \\
Kepler-1656b & 48.623 & dekaterran & planemon \\
Kepler-279c & 49.259 & dekaterran & planemon \\
Kepler-101b & 50.848 & dekaterran & planemon \\
Kepler-9c & 54.344 & dekaterran & planemon \\
Kepler-282e & 56.251 & dekaterran & planemon \\
Kepler-416b & 58.157 & dekaterran & planemon \\
Kepler-282d & 61.018 & dekaterran & planemon \\
Kepler-9b & 61.793 & dekaterran & planemon \\
Kepler-92b & 64.196 & dekaterran & planemon \\
Kepler-277c & 64.198 & dekaterran & planemon \\
Kepler-413b & 67.056 & dekaterran & planemon \\
Kepler-34b & 69.916 & dekaterran & planemon \\
Kepler-396b & 75.636 & dekaterran & planemon \\
Kepler-145c & 79.450 & dekaterran & planemon \\
Kepler-425b & 79.450 & dekaterran & planemon \\
Kepler-277b & 87.398 & dekaterran & planemon \\
Kepler-427b & 92.162 & dekaterran & planemon \\
Saturn & 95.200 & dekaterran & planemon \\
Kepler-16b & 105.827 & hectoterran & planemon \\
Kepler-89d & 106.145 & hectoterran & planemon \\
Kepler-426b & 108.052 & hectoterran & planemon \\
Kepler-54c & 117.586 & hectoterran & planemon \\
Kepler-415b & 119.811 & hectoterran & planemon \\
Kepler-63b & 120.128 & hectoterran & planemon \\
Kepler-38b & 122.035 & hectoterran & planemon \\
Kepler-289c & 133.476 & hectoterran & planemon \\
Kepler-77b & 136.654 & hectoterran & planemon \\
Kepler-422b & 136.654 & hectoterran & planemon \\
Kepler-12b & 136.970 & hectoterran & planemon \\
Kepler-7b & 137.610 & hectoterran & planemon \\
WASP-17b & 154.451 & hectoterran & planemon \\
Kepler-32c & 158.900 & hectoterran & planemon \\
Kepler-1654b & 158.900 & hectoterran & planemon \\
Kepler-45b & 160.489 & hectoterran & planemon \\
Kepler-64b (PH1 b) & 168.752 & hectoterran & planemon \\
Kepler-238e & 169.705 & hectoterran & planemon \\
Kepler-41b & 177.968 & hectoterran & planemon \\
Kepler-56c & 181.146 & hectoterran & planemon \\
Kepler-8b & 187.502 & hectoterran & planemon \\
Kepler-423b & 189.091 & hectoterran & planemon \\
Kepler-74b & 200.214 & hectoterran & planemon \\
Kepler-15b & 209.748 & hectoterran & planemon \\
Kepler-6b & 212.290 & hectoterran & planemon \\
HD209458b & 225.638 & hectoterran & planemon \\
Kepler-49c & 228.816 & hectoterran & planemon \\
HAT-P-33b & 228.819 & hectoterran & planemon \\
Kepler-23b & 254.240 & hectoterran & planemon \\
Kepler-91b & 257.418 & hectoterran & planemon \\
Kepler-435b & 266.952 & hectoterran & planemon \\
Kepler-54b & 292.376 & hectoterran & planemon \\
Kepler-412b & 299.050 & hectoterran & planemon \\
Kepler-539b & 308.266 & hectoterran & planemon \\
Kepler-49b & 311.444 & hectoterran & planemon \\
Kepler-44b & 317.800 & hectoterran & planemon \\
Jupiter & 317.800 & hectoterran & planemon \\
Kepler-87b & 324.156 & hectoterran & planemon \\
Kepler-424b & 327.334 & hectoterran & planemon \\
Kepler-418b & 349.580 & hectoterran & planemon \\
Kepler-55c & 352.758 & hectoterran & planemon \\
TrES-2b (Kepler-1b) & 380.407 & hectoterran & planemon \\
Kepler-428b & 403.606 & hectoterran & planemon \\
Kepler-28c & 432.208 & hectoterran & planemon \\
Kepler-59c & 435.386 & hectoterran & planemon \\
Kepler-447b & 435.386 & hectoterran & planemon \\
Kepler-58b & 441.742 & hectoterran & planemon \\
Kepler-55b & 473.522 & hectoterran & planemon \\
Kepler-28b & 479.878 & hectoterran & planemon \\
Kepler-1647b & 482.954 & hectoterran & planemon \\
Kepler-24b & 508.480 & hectoterran & planemon \\
Kepler-24c & 508.480 & hectoterran & planemon \\
HAT-P-7b (Kepler-2b) & 566.002 & hectoterran & planemon \\
Kepler-117c & 584.752 & hectoterran & planemon \\
Kepler-1657b & 613.354 & hectoterran & planemon \\
Kepler-47b & 635.600 & hectoterran & planemon \\
Kepler-30c & 638.778 & hectoterran & planemon \\
Kepler-76b & 638.778 & hectoterran & planemon \\
Kepler-59b & 651.490 & hectoterran & planemon \\
Kepler-5b & 670.876 & hectoterran & planemon \\
Kepler-58c & 695.982 & hectoterran & planemon \\
Kepler-40b & 699.160 & hectoterran & planemon \\
Kepler-17b & 778.610 & hectoterran & planemon \\
Kepler-419b & 794.500 & hectoterran & planemon \\
Kepler-23c & 858.060 & hectoterran & planemon \\
Kepler-433b & 896.196 & hectoterran & planemon \\
Kepler-434b & 908.908 & hectoterran & planemon \\
Kepler-43b & 1026.494 & kiloterran & planemon \\
Kepler-32b & 1302.980 & kiloterran & planemon \\
Kepler-1708b & 1461.880 & kiloterran & planemon \\
Kepler-31c & 1493.660 & kiloterran & planemon \\
Kepler-432b & 1719.298 & kiloterran & planemon \\
Kepler-46b & 1906.800 & kiloterran & planemon \\
Kepler-31d & 2161.040 & kiloterran & planemon \\
Kepler-57c & 2208.710 & kiloterran & planemon \\
Kepler-14b & 2669.520 & kiloterran & planemon \\
Kepler-52b & 2764.860 & kiloterran & planemon \\
Kepler-27b & 2895.158 & kiloterran & planemon \\
Kepler-75b & 3209.780 & kiloterran & planemon \\
Kepler-52c & 3308.298 & kiloterran & planemon \\
Kepler-27c & 4385.640 & kiloterran & intermon \\
Kepler-53c & 5002.172 & kiloterran & intermon \\
Kepler-53b & 5850.698 & kiloterran & intermon \\
Kepler-57b & 5993.708 & kiloterran & intermon \\
Kepler-39b & 6387.780 & kiloterran & intermon \\
Kepler-47c & 8898.400 & kiloterran & intermon \\
\end{longtable}

\section{Abstract}\label{abstract-38}

\textbf{Major Topics:}\\
- Definition of the \textbf{mononic condition}: all WCB monoclasses
(planemon, intermon, stellamon, etc.) apply only to single,
gravitationally coherent bodies.\\
- Scope clarification: monoclasses exclude multi-body systems,
aggregates, and regions.\\
- Distinction between \emph{ontological object} vs.~\emph{mathematical
abstraction}.\\
- Ontological boundary rule: barycentered approximations ≠ monoclasses.

\textbf{Key Terms \& Symbols:}\\
- --mon suffix = monoclass indicator (e.g., planemon, stellamon,
intermon).\\
- Center of mass (singular) = necessary condition for --mon
classification.\\
- Cryptomon, supermon, ultramon, hypermon = monoclass variants.\\
- Barycenter = mathematical construct; not a physical monon.

\textbf{Cross-Check Notes:}\\
- Canonical clarification: multi-body systems (binaries, clusters,
galaxies, nebulae) are \textbf{not mononic}, even if composed of mononic
members.\\
- Rule resolves ambiguity: a system of planemons is not itself a
planemon.\\
- Complements \textbf{Mononic Mass Classes} and \textbf{Mononic
Compositions} (\textbf{morphotypes}) as monoclass framework
foundations.\\
- Ensure consistent usage: ``--mon'' suffix always denotes singular
gravitationally self-bound entities, never systems.

\chapter{Classification Rule --- The Monon
Condition}\label{classification-rule-the-monon-condition}

\begin{quote}
\textbf{All WCB monoclasses (planemon, intermon, stellamon)} refer
exclusively to \textbf{individual, gravitationally coherent bodies} ---
not to multi-body \textbf{systems}, \textbf{collections}, or
\textbf{regions}.
\end{quote}

\subsection{\texorpdfstring{🔒 What \emph{Is} a
monon}{🔒 What Is a monon}}\label{what-is-a-monon}

A \textbf{monon} is defined by: - A single \textbf{center of mass}\\
- Internally unified \textbf{gravitational structure}\\
- Physically continuous \textbf{volume} (or event horizon, in the case
of black holes)\\
- E.g., \textbf{planemons, intermons, stellamons, supermons, ultramons,
hypermons}

\subsection{\texorpdfstring{🚫 What \emph{Is Not} a
monon:}{🚫 What Is Not a monon:}}\label{what-is-not-a-monon}

The following are not \textbf{monon}, regardless of total mass:

\begin{longtable}[]{@{}
  >{\raggedright\arraybackslash}p{(\linewidth - 4\tabcolsep) * \real{0.2644}}
  >{\raggedright\arraybackslash}p{(\linewidth - 4\tabcolsep) * \real{0.3678}}
  >{\raggedright\arraybackslash}p{(\linewidth - 4\tabcolsep) * \real{0.3678}}@{}}
\toprule\noalign{}
\begin{minipage}[b]{\linewidth}\raggedright
Category
\end{minipage} & \begin{minipage}[b]{\linewidth}\raggedright
Example
\end{minipage} & \begin{minipage}[b]{\linewidth}\raggedright
Why Excluded
\end{minipage} \\
\midrule\noalign{}
\endhead
\bottomrule\noalign{}
\endlastfoot
\textbf{Star Clusters} & Globular clusters, open clusters & Multi-body
systems \\
\textbf{Nebulae / Clouds} & Giant Molecular Clouds (GMCs) & Diffuse,
non-contiguous \\
\textbf{Galaxies} & Milky Way, Andromeda & Gravitationally bound
aggregates \\
\textbf{Binary Systems} & Sirius A+B, Pluto--Charon & Contain ≥2 centers
of mass \\
\textbf{Black Hole Binaries} & LIGO-detected pairs & Still
system-level \\
\end{longtable}

\section{\texorpdfstring{\textbf{WCB Monon Rule (Monoclass Scope
Clarification)}}{WCB Monon Rule (Monoclass Scope Clarification)}}\label{wcb-monon-rule-monoclass-scope-clarification}

\begin{quote}
\textbf{All mononic classifications (planemon, intermon, stellamon,
cryptomon)} refer to \textbf{individual, contiguous, gravitationally
self-bound objects}.

Multi-object systems --- whether binary, ternary, or aggregate --- are
\textbf{not themselves monons, even if composed entirely of monons.
\#\#\# 🧠 Key Principles: 1. Monoclass is not emergent}:\\
A \textbf{system of planemons} is not a larger planemon.\\
A \textbf{binary of stellamons} is not a single, larger stellamon.\\
2. \textbf{Modeling convenience ≠ ontological unity}:\\
- For orbital mechanics, it's \textbf{useful} to model a binary pair as
a single object of combined mass.\\
- But \textbf{ontotypically}, the barycentered abstraction is
\textbf{not} a monon\\
3. \textbf{Distinct gravitational fields}:\\
- A monon has \textbf{one center of mass}, even if dynamic (e.g.,
fast-rotating planemon).\\
- A system has \textbf{multiple discrete gravitational centers}, even if
mathematically reducible. For instance in a double-star stellmonic
system, it is convenient to treat the two stellamons as \emph{a single
stellamon of their combined mass}, centered on the barycenter of their
orbits, but this is a mathematical construct based on a property of the
\emph{system} they comprise, not an object-in-itself.
\end{quote}

\chapter{Abstract}\label{abstract-39}

\textbf{Major Topics:}\\
- Defines \textbf{five fundamental physical properties} of monons:\\
- \textbf{Mass (m):} total matter present (distinct from weight).\\
- \textbf{Density (ρ):} matter per unit volume, affected by composition
and gravitational compression.\\
- \textbf{Radius (r):} center-to-surface distance, treated as emergent
from mass and density.\\
- \textbf{Surface Gravity (g):} acceleration at the surface, dependent
on mass and radius.\\
- \textbf{Escape Velocity (v):} minimum speed to overcome gravity, also
dependent on mass and radius.\\
- Explains \textbf{interdependence}: radius emergent from mass and
density, density altered by gravitational compression, gravity and
escape velocity derived from the mass--radius relationship.\\
- Notes \textbf{inverse-square law} as foundation for gravitational
effects.\\
- Invokes \textbf{sed ego dico} as an editorial simplification,
prioritizing usability over perfect physical rigor.\\
- Provides real-world analogies (e.g., water vs.~iron density) to
clarify concepts.

\textbf{Key Terms \& Symbols:}\\
- \textbf{m, ρ, r, g, v} --- monon equations of state variables.\\
- \textbf{Gravitational Compression {[}neo{]}.}\\
- \textbf{Inverse-square law {[}sci{]}.}\\
- \textbf{Sed ego dico {[}meta{]}.}

\textbf{Cross-Check Notes:}\\
- All five properties are already defined in canon and equations of
state.\\
- \textbf{Gravitational Compression {[}neo{]}} reinforced.\\
- \textbf{Inverse-square law {[}sci{]}} already established in
stellar/orbital contexts.\\
- \textbf{Sed ego dico {[}meta{]}} consistent with existing editorial
conventions.\\
- \textbf{Status:} {[}EXPANDED{]} --- reinforces existing canon with
emphasis on emergent and interdependent properties.

\section{Physical Properties of
Monons}\label{physical-properties-of-monons}

There are five properties of monons that can be thought of as
\emph{physical properties} --- insofar as they describe the monon
physically or a property of the monon that emerges from the other
physical properties. These are:

\begin{enumerate}
\def\labelenumi{\arabic{enumi}.}
\tightlist
\item
  \textbf{\emph{Mass}} (m): The total amount of matter present.
  (\emph{Mass and weight are not the same --- weight depends on gravity;
  mass does not}.)
\item
  \textbf{\emph{Density}} (ρ): The average amount of matter per unit
  volume --- essentially, how tightly packed the monon's materials are.
  This depends on both composition (e.g.~rock, ice, metal) and, for
  larger bodies, gravitational self-compression.
\item
  \textbf{\emph{Radius}} (r): The distance from the monon's center to
  its surface. Technically, this is derived from the monon's mass and
  average density --- but more on this in a moment.
\item
  \textbf{\emph{Surface Gravity}} (g): The strength of gravitational
  acceleration at the monon's surface --- how strongly it attracts
  objects located one radius away from its center.
\item
  \textbf{\emph{Escape Velocity}} (v): The minimum speed needed to
  completely escape the monon's gravity when starting from the surface
  --- how much velocity is required to leave the monon entirely.
\end{enumerate}

These are all intricately interrelated in ways that are much more
complicated than we need to account for here. For instance, while
\textbf{radius} \emph{is} derived from \textbf{mass} and
\textbf{density}, it also matters how strongly the mass of the monon is
gravitationally acting on itself. This causes
\textbf{\emph{gravitational compression}}, which means that the matter
is forced into a denser configuration than it would otherwise exhibit
\emph{if it were not under such intense gravitational pull}. So, it is
also true that \textbf{density} in some sense is derived from
\textbf{mass} and \textbf{radius}.

Surface gravity and escape velocity are also very dependent on both mass
and radius. The farther a point on the surface of the monon is from the
monon's center, the less gravitational attraction from the monon it
experiences, decreasing according to the \textbf{\emph{inverse-square
law}} (more on this later). And the weaker the surface gravity, the less
energy it takes to overcome it, so the lower the escape velocity.

For our purposes, it is reasonable to simplify things and say:

\begin{quote}
Radius is an \emph{emergent property} of the interaction of mass and
density.
\end{quote}

\ldots{} and not be too overly concerned about the subtleties.

\begin{quote}
\textbf{Keppy}: \emph{Sed ego dico}, right?
\end{quote}

Yes; this is our first official invocation of \emph{sed ego dico},
``because I say''. No, it's not \emph{entirely physically accurate}, but
it's good enough for what we need to accomplish.

Mass and density are also dependent upon what materials the monon is
made of: most monons are composed of several materials (rock, ice,
liquid water, etc.). Rock, itself, can composed of any number of more
basic elements and compounds.

And all of these materials have their own inherent densities. Water, for
instance, has a density of about 1.0 g/cm³ (and is, in fact the
\emph{standard} for material density in the metric system we're using.)
Iron, on the other hand, has a density of about 7.7 g/cm³ --- almost
eight times that of water. So 1 kilogram of water takes up more volume
than one kilogram of iron.

\begin{quote}
There is an old riddle, which was a favorite of my maternal grandfather:
``Which is heavier --- a pound of iron or a pound of feathers?'' The
trick to the question, of course, is that both \emph{weigh} a pound, but
because feathers are less dense than iron, that pound (in its natural
state) takes up far more volume than does the iron.
\end{quote}

\section{Abstract}\label{abstract-40}

\textbf{Major Topics:}\\
- Establishes that \textbf{monon masses are numerically defined} and
serve only as measurements --- not as categorical labels for world
type.\\
- Defines the \textbf{WCB symbolic mass intervals}, logarithmically
divided by powers of ten relative to Earth mass (⨁).\\
- Introduces symbolic naming scheme:\\
- \textbf{Below 1 ⨁:} intervals use ``-i'' endings (e.g., microterran,
centiterran).\\
- \textbf{Above 1 ⨁:} intervals use ``-o'' endings (e.g., kiloterran,
gigaterran).\\
- Where SI prefixes exist (e.g., kilo-, mega-), WCB adopts them
directly.\\
- Provides explicit table of intervals from \textbf{deniterran (10⁻¹⁰
⨁)} through \textbf{teraterran (10¹² ⨁)}.\\
- Clarifies rounding rules: each interval spans up to one demiterran
(10⁻⁴ ⨁) less than the next power of ten; bodies at the threshold round
upward.\\
- Notes practical importance of \textbf{negative intervals} for moons,
asteroids, and micro-monons.\\
- Introduces the \textbf{Microterran Scale (μt)}:\\
- A shorthand symbolic scale for masses between 10⁻⁶ ⨁ and 10⁻¹ ⨁.\\
- Improves readability and narrative clarity for small bodies (e.g.,
Ceres, Miranda, Vesta).\\
- Allows statements like ``Ceres is 157 μt'' instead of 0.000157 ⨁.

\textbf{Key Terms \& Symbols:}\\
- \textbf{Symbolic Mass Intervals {[}NEW{]}:} deniterran, nanoterran,
oktiterran, septiterran, microterran, pentiterran, demiterran,
milliterran, centiterran, deciterran, terran, dekaterran, hectoterran,
kiloterran, myriaterran, hexaterran, megaterran, heptoterran,
octoterran, gigaterran, denoterran, ondoterran, teraterran.\\
- \textbf{Microterran Scale (μt) {[}NEW{]}.}\\
- \textbf{⨁ (Terran mass unit) {[}ins{]}.}

\textbf{Cross-Check Notes:}\\
- Neither symbolic intervals nor the μ-terran scale appeared in prior
canon abstracts.\\
- This is the \textbf{first introduction} of a systematic symbolic
framework for monon mass measurement.\\
- \textbf{Status:} {[}NEW{]} --- establishes an entirely new symbolic
measurement system for monon masses.

\chapter{Principle of Mass
Measurement}\label{principle-of-mass-measurement}

\begin{quote}
\textbf{Monon masses are purely numerically defined.}\\
They \textbf{measure} a body's physical magnitude; they do \textbf{not
categorize} or \textbf{qualify} its nature or type.
\end{quote}

The following table defines the \textbf{official WCB symbolic mass
intervals} --- logarithmic divisions of monon mass based on powers of
ten relative to Earth (⨁). These are used for measurement and comparison
only. They are \textbf{not categorical classes}, and do \textbf{not
imply world type}.

\begin{longtable}[]{@{}
  >{\raggedright\arraybackslash}p{(\linewidth - 10\tabcolsep) * \real{0.1622}}
  >{\raggedright\arraybackslash}p{(\linewidth - 10\tabcolsep) * \real{0.1622}}
  >{\raggedright\arraybackslash}p{(\linewidth - 10\tabcolsep) * \real{0.2568}}
  >{\raggedright\arraybackslash}p{(\linewidth - 10\tabcolsep) * \real{0.2703}}
  >{\raggedright\arraybackslash}p{(\linewidth - 10\tabcolsep) * \real{0.0676}}
  >{\raggedright\arraybackslash}p{(\linewidth - 10\tabcolsep) * \real{0.0811}}@{}}
\toprule\noalign{}
\begin{minipage}[b]{\linewidth}\raggedright
Intervals
\end{minipage} & \begin{minipage}[b]{\linewidth}\raggedright
Abbreviation
\end{minipage} & \begin{minipage}[b]{\linewidth}\raggedright
Min. Mass ≥
\end{minipage} & \begin{minipage}[b]{\linewidth}\raggedright
Max. Mass \textless{}
\end{minipage} & \begin{minipage}[b]{\linewidth}\raggedright
Power
\end{minipage} & \begin{minipage}[b]{\linewidth}\raggedright
Notes
\end{minipage} \\
\midrule\noalign{}
\endhead
\bottomrule\noalign{}
\endlastfoot
deniterran & dnt & 0.0000000001 & 0.000000001 & 10⁻¹⁰ & WCB \\
nanoterran & nnt & 0.000000001 & 0.00000001 & 10⁻⁹ & WCB \\
oktiterran & okt & 0.00000001 & 0.0000001 & 10⁻⁸ & WCB \\
septiterran & spt & 0.0000001 & 0.000001 & 10⁻⁷ & WCB \\
microterran & μt & 0.000001 & 0.00001 & 10⁻⁶ & SI \\
pentiterran & pnt & 0.00001 & 0.0001 & 10⁻⁵ & WCB \\
demiterran & dmt & 0.0001 & 0.001 & 10⁻⁴ & WCB \\
milliterran & mmt & 0.001 & 0.01 & 10⁻³ & WCB/SI \\
centiterran & ctt & 0.01 & 0.1 & 10⁻² & WCB \\
deciterran & dct & 0.1 & 1 & 10⁻¹ & WCB \\
terran & t & 1 & 10 & 10⁰ & WCB \\
dekaterran & dkt & 10 & 100 & 10¹ & WCB \\
hectoterran & hct & 100 & 1000 & 10² & WCB \\
kiloterran & kt & 1000 & 10000 & 10³ & SI \\
myriaterran & myt & 10000 & 100000 & 10⁴ & WCB \\
hexaterran & hxt & 100000 & 1000000 & 10⁵ & WCB \\
megaterran & Mt & 1000000 & 10000000 & 10⁶ & SI \\
heptoterran & hpt & 10000000 & 100000000 & 10⁷ & WCB \\
octoterran & ott & 100000000 & 1000000000 & 10⁸ & WCB \\
gigaterran & Gt & 1000000000 & 10000000000 & 10⁹ & SI \\
denoterran & ddt & 10000000000 & 100000000000 & 10¹⁰ & WCB \\
ondoterran & ont & 100000000000 & 1000000000000 & 10¹¹ & WCB \\
teraterran & trt & 1000000000000 & 10000000000000 & 10¹² & WCB \\
triskaterran & tkt & 10000000000000 & 100000000000000 & 10¹³ & WCB \\
quadraterran & qdt & 100000000000000 & 1000000000000000 & 10¹⁴ & WCB \\
petaterran & Pt & 1000000000000000 & 10000000000000000 & 10¹⁵ & SI \\
sexaterran & sxt & 10000000000000000 & 100000000000000000 & 10¹⁶ &
WCB \\
septaterran & spt & 100000000000000000 & 1000000000000000000 & 10¹⁷ &
WCB \\
exaterran & Et & 1000000000000000000 & 10000000000000000000 & 10¹⁸ &
SI \\
\end{longtable}

\begin{quote}
Notes: 1. Where official SI prefixes exist (e.g.~micro-, milli-, deci-,
centi-, kilo-, mega-, giga-, tera-, peta-, exa-), WCB uses them
directly. 2. For intermediate powers without SI prefixes, WCB adopts
consistent \textbf{neologisms} based on the exponent's magnitude:. •
Exponents \textbf{below zero} use an \textbf{-i} ending (unless
overridden by SI, e.g.~\emph{micro}, not \emph{micri}). • Exponents
\textbf{above zero} use an \textbf{-o} ending (unless overridden by SI,
e.g.~\emph{mega}, not \emph{mego}). 3. While SI defines prefixes beyond
\textbf{tera-} (e.g.~\emph{peta-}, \emph{exa-}), WCB does not typically
use them in monon contexts. Even the observable universe spans only
\textasciitilde46.5 \textbf{giga}lightyears (Gly) in radius --- three
orders of magnitude short of \textbf{teraterran} mass or volume
relevance. 4. Conversely, the \textbf{negative intervals are
symbolically essential}, as many moons, asteroids, and micro-monons fall
well below 1\,⨁ in mass. It is often more meaningful to say • ``The body
is \textbf{1 microterran} (μ⨁)'' \(\qquad\)than • ``The body is
\textbf{0.000001\,⨁}''. 5. For \textbf{practical purposes}, WCB defines
the upper bound of each mass interval as \textbf{one demiterran
(0.00001\,⨁)} below the next power of ten. • In other words, the
\emph{terran} interval spans from \textbf{1.0\,⨁ to 9.99999\,⨁}. • Any
value exceeding that threshold --- even slightly --- is considered to
\textbf{round up} to the next symbolic interval. • \textbf{Example:}
\(\qquad\) 9.9999901\,⨁ → \textbf{10.0\,⨁}, placing the body in the
\textbf{dekaterran} interval. 6. The \textbf{deniterran} is the lowest
mass on this scale because objects below this mass are not capable of
achieving hydrostatic equilibrium under their own gravity. 7. The
exaterran is the largest mass on this scale, corresponding to the
realistic upper limit for Ultra-massive Black Holes (UMBHs), about
\(10^{12}\)⊙ masses.
\end{quote}

\section{The Microterran Scale}\label{the-microterran-scale}

Because so many significant bodies in a stellamon system have masses
best expressed as a \textbf{millionth of a terran} (μ⨁), WCB defines an
optional \textbf{\emph{microterran scale}} for intuitive symbolic
modeling. This scale maps cleanly to the standard symbolic intervals and
absolute powers of ten.

\begin{longtable}[]{@{}
  >{\raggedright\arraybackslash}p{(\linewidth - 12\tabcolsep) * \real{0.1148}}
  >{\raggedright\arraybackslash}p{(\linewidth - 12\tabcolsep) * \real{0.1393}}
  >{\raggedright\arraybackslash}p{(\linewidth - 12\tabcolsep) * \real{0.1393}}
  >{\raggedright\arraybackslash}p{(\linewidth - 12\tabcolsep) * \real{0.1557}}
  >{\raggedright\arraybackslash}p{(\linewidth - 12\tabcolsep) * \real{0.1475}}
  >{\raggedright\arraybackslash}p{(\linewidth - 12\tabcolsep) * \real{0.1475}}
  >{\raggedright\arraybackslash}p{(\linewidth - 12\tabcolsep) * \real{0.1557}}@{}}
\toprule\noalign{}
\begin{minipage}[b]{\linewidth}\raggedright
µ-terran Scale
\end{minipage} & \begin{minipage}[b]{\linewidth}\raggedright
Min. Mass≥ µt
\end{minipage} & \begin{minipage}[b]{\linewidth}\raggedright
Max. Mass\textless{} µt
\end{minipage} & \begin{minipage}[b]{\linewidth}\raggedright
\emph{StandardScale}
\end{minipage} & \begin{minipage}[b]{\linewidth}\raggedright
\emph{Min. Mass≥} ⨁
\end{minipage} & \begin{minipage}[b]{\linewidth}\raggedright
\emph{Max. Mass\textless{}} ⨁
\end{minipage} & \begin{minipage}[b]{\linewidth}\raggedright
\emph{AbsoluteScale}
\end{minipage} \\
\midrule\noalign{}
\endhead
\bottomrule\noalign{}
\endlastfoot
demimicro & 0.0001 & 0.001 & \emph{deniterran} & \emph{0.0000000001} &
\emph{0.000000001} & \emph{10⁻¹⁰} \\
millimicro & 0.001 & 0.01 & \emph{nanoterran} & \emph{0.000000001} &
\emph{0.00000001} & \emph{10⁻⁹} \\
centimicro & 0.01 & 0.1 & \emph{oktiterran} & \emph{0.00000001} &
\emph{0.0000001} & \emph{10⁻⁸} \\
decimicro & 0.1 & 1 & \emph{septiterran} & \emph{0.0000001} &
\emph{0.000001} & \emph{10⁻⁷} \\
microterran & 1 & 10 & \emph{microterran} & \emph{0.000001} &
\emph{0.00001} & \emph{10⁻⁶} \\
dekamicro & 10 & 100 & \emph{pentiterran} & \emph{0.00001} &
\emph{0.0001} & \emph{10⁻⁵} \\
hectomicro & 100 & 1000 & \emph{demiterran} & \emph{0.0001} &
\emph{0.001} & \emph{10⁻⁴} \\
kilomicro & 1000 & 10000 & \emph{milliterran} & \emph{0.001} &
\emph{0.01} & \emph{10⁻³} \\
myriamicro & 10000 & 100000 & \emph{centiterran} & \emph{0.01} &
\emph{0.1} & \emph{10⁻²} \\
\end{longtable}

\subsection{Interpretive Guidance}\label{interpretive-guidance}

\begin{itemize}
\tightlist
\item
  The \textbf{μ-terran scale} improves readability and narrative clarity
  for small-mass bodies such as \textbf{Vesta}, \textbf{Miranda}, and
  \textbf{Ceres}.\\
\item
  It is especially useful for monons with masses between \textbf{10⁻⁶\,⨁
  and 10⁻¹\,⨁}, where decimal ⨁ values become visually dense or
  cognitively opaque.\\
\item
  This symbolic shorthand allows you to say:\\
  \textgreater{} ``Ceres is about \textbf{157 microterrans}
  (157\,μt)''\\
  \textgreater{} instead of\\
  \textgreater{} ``Ceres is 0.000157\,⨁''\\
\item
  However, even with this scaling, \textbf{fractional expressions remain
  common} at the lowest mass levels:\\
  \textgreater{} ``Miranda is approximately \textbf{0.001\,μt} in
  mass''\\
  \textgreater{} is still \textbf{easier to read and parse} than:\\
  \textgreater{} ``Miranda is \textbf{1 demimicroterran} in mass.''
  \textgreater{} \textbf{In short:} The μt unit supports clarity without
  abandoning scalar precision --- a vital aid to both symbolic modeling
  and thesiastic storytelling.
\end{itemize}

\section{Abstract}\label{abstract-41}

\textbf{Major Topics:}\\
- Introduces methods for \textbf{estimating the strength and extent of
planetary magnetospheres} using simplified dynamo scaling laws.\\
- Defines the planetary \textbf{magnetic moment (M)} as the primary
parameter, derived from core radius (rc), core density (ρc), and
rotation period (d).\\
- Relates magnetic moment to \textbf{surface magnetic field strength
(Bsurf)} through empirical scaling relations.\\
- Discusses how \textbf{exponents (p, q, r)} in these formulas vary
across different dynamo models (e.g., Stevenson, Driscoll \& Olson).\\
- Emphasizes the protective role of magnetospheres in shielding
planetary atmospheres from stellar wind and cosmic radiation, with
implications for long-term \textbf{habitability}.\\
- Provides normalized formulas for use in worldbuilding, with Terran
values as benchmarks.

\textbf{Key Terms \& Symbols:}\\
- \textbf{M (Magnetic Moment):} Strength of a planet's magnetic field.\\
- \textbf{Bsurf (Surface Magnetic Field Strength):} Field strength at
planetary surface.\\
- \textbf{rc (Core Radius):} Planetary core radius (relative to
Earth).\\
- \textbf{ρc (Core Density):} Planetary core density (relative to
Earth).\\
- \textbf{d (Day Length):} Rotation period of the planet.\\
- \textbf{p, q, r:} Exponents in dynamo scaling laws.

\textbf{Cross-Check Notes:}\\
- No duplicate abstract exists; this is the \textbf{canonical
magnetosphere reference}.\\
- Glossary updates required for \textbf{M, Bsurf, and dynamo scaling
exponents (p, q, r)}.\\
- Closely related to notes on \textbf{habitability envelopes,
atmospheric retention, and stellar wind interaction}.

\chapter{Estimating Planetary
Magnetospheres}\label{estimating-planetary-magnetospheres}

\begin{itemize}
\tightlist
\item
  \textbf{Magnetic moment (M)} and how it relates to \(B_{surf}\)\\
\item
  Empirical scaling relationships (e.g., from Stevenson or Driscoll \&
  Olson models)\\
\item
  A simplified proportional form like:
\end{itemize}

\[
B_{surf} \propto \left(\dfrac{\rho_c}{\rho_\oplus}\right)^p \times \left(\dfrac{r_c}{r_\oplus}\right)^q \times \left(\dfrac{1}{d}\right)^r
\]

Where: - \(r_c\) = core radius - \(\rho_c\) = core density\\
- \(d\) = day length (rotation period)\\
- \(p, q, r\) are empirical exponents from dynamo scaling studies

This is a \textbf{scaling law approximation} from monon dynamo theory
--- useful, but absolutely needing context.

🧪 \textbf{Where This Comes From (Briefly)}

This form is adapted from scaling relations used by geophysicists and
monon scientists (e.g., Olson \& Christensen, Stevenson, Driscoll \&
Bercovici), who try to predict magnetic field strength based on
\textbf{internal structure and heat flow}.

Values of \(p\), \(q\), and \(r\) vary depending on assumptions about: -
Core thermal convection mode (thermal vs.~compositional)\\
- Rotational regime (rapid vs.~slow)\\
- Efficiency of magnetic field generation

🧭 \textbf{Rule-of-Thumb Exponent Ranges}

\begin{longtable}[]{@{}
  >{\raggedright\arraybackslash}p{(\linewidth - 4\tabcolsep) * \real{0.0792}}
  >{\raggedright\arraybackslash}p{(\linewidth - 4\tabcolsep) * \real{0.2277}}
  >{\raggedright\arraybackslash}p{(\linewidth - 4\tabcolsep) * \real{0.6931}}@{}}
\toprule\noalign{}
\begin{minipage}[b]{\linewidth}\raggedright
Exponent
\end{minipage} & \begin{minipage}[b]{\linewidth}\raggedright
Typical Value (approx.)
\end{minipage} & \begin{minipage}[b]{\linewidth}\raggedright
Notes
\end{minipage} \\
\midrule\noalign{}
\endhead
\bottomrule\noalign{}
\endlastfoot
\(p\) & ⟨1.0 ∧ 1.3⟩ & Reflects that higher-density cores produce
stronger magnetic energy \\
\(q\) & ⟨2.0 ∧ 2.5⟩ & Core radius has a strong effect --- bigger cores
mean more dynamo volume \\
\(r\) & ⟨1.0 ∧ 1.5⟩ & Faster spin increases field strength, up to
saturation \\
\end{longtable}

These aren't exact, but they give a ballpark for building
\textbf{comparative models} --- e.g.:

\begin{quote}
``A world with a core 1.2× Earth's radius, 1.1× Earth's core density,
and a 16-hour day could have a magnetic field \textbf{2--3× stronger}
than Earth's, all else equal.''
\end{quote}

Or simply provide example profiles:

\begin{itemize}
\tightlist
\item
  Earth: ρ = 1.0⨁, d = 24h → \(B_{surf}\) ∈ ⟨25 ∧ 65\textgreater{} μT\\
\item
  Super-Earth: ρ = 1.3⨁, d = 16h → \(B_{surf}\) ∈ ⟨80 ..120⟩ μT\\
\item
  Mars: ρ = 0.71⨁, d = 24.6h → \(B_{surf}\) ≈ 0 μT (solid core)\#\#
  Abstract
\end{itemize}

\textbf{Major Topics:}\\
- Extended parameters for defining geotic (human-hospitable)
conditions.\\
- Habitability ranges for rotation period (D), orbital eccentricity (e),
orbital period (C), axial tilt (εₓ), precession cycle (χ), and obliquity
azimuth (ζₙ).\\
- Magnetosphere strength (Bsurf) as radiation shielding criterion.\\
- Atmospheric baseline conditions: pressure, scale height, composition,
ozone presence.\\
- Surface balance of land and water.\\
- Geotic gravity corridor (0.5--1.5 ⨁) as strict human-hospitable bound.

\textbf{Key Terms \& Symbols:}\\
- \textbf{D} --- Rotational period (diurn length).\\
- \textbf{e} --- Orbital eccentricity.\\
- \textbf{C} --- Orbital period (sidereal chronum).\\
- \textbf{εₓ} --- Axial tilt (obliquity).\\
- \textbf{χ} --- Axial precession cycle.\\
- \textbf{ζₙ} --- Obliquity azimuth relative to periapsis.\\
- \textbf{Bsurf} --- Surface magnetic field strength (μT).\\
- \textbf{Tₛ} --- Average surface temperature (K).\\
- \textbf{H} --- Atmospheric scale height (km).\\
- \textbf{g} --- Surface gravity (⨁).\\
- Land--sea distribution (lithosphere--hydrosphere balance).

\textbf{Cross-Check Notes:}\\
- Reinforces prior geotic bounds with expanded atmospheric, rotational,
orbital, and magnetic criteria.\\
- Clarifies \emph{why} gravity corridor (0.5--1.5 ⨁) defines Geotic
worlds: outside this, monons may be Telluric/parahabitable but not
Geotic.\\
- Orbital period C not freely chosen: constrained by Kepler's Third Law,
tying world design to stellar parameters.\\
- Magnetosphere thresholds emphasize that both too weak and too strong
fields can undermine habitability.\\
- Complements and extends core Geotic definitions; functions as a
reference sheet for designers setting secondary parameters.

\chapter{Extended Geotic Habitability
Guidelines}\label{extended-geotic-habitability-guidelines}

\begin{longtable}[]{@{}
  >{\raggedright\arraybackslash}p{(\linewidth - 2\tabcolsep) * \real{0.3761}}
  >{\raggedright\arraybackslash}p{(\linewidth - 2\tabcolsep) * \real{0.6239}}@{}}
\toprule\noalign{}
\begin{minipage}[b]{\linewidth}\raggedright
Parameter
\end{minipage} & \begin{minipage}[b]{\linewidth}\raggedright
Value(s)
\end{minipage} \\
\midrule\noalign{}
\endhead
\bottomrule\noalign{}
\endlastfoot
\textbf{Average surface temperature (Tₛ)} & Tₛ = 288K (standard unit)Tₛ
= 14.85°CTₛ = 58.73°F \\
\textbf{Solar radiance (insolation) (Q)}(annual average) & Q ≈1361 W/m²
(top of atmosphere)Q ≈170-180 W/m² (surface) \\
\textbf{Stellar luminosity (L)} & 3.828 × 10²⁶ Watt1.0 L⊙ \\
\textbf{Average atmospheric pressure (atm)} & 1 atmosphere (atm)1
bar101.3 kPa14.6959 psi 1.0332 kg/m² \\
\textbf{Atmospheric Scale Height (H)}(more on this later.) & H =
8.5km\(P(H) = 0.37^H\)\(P(km) = 0.37^{\frac{km}{8.5}}\) \\
\textbf{Average atmospheric composition}(idealized; varies with geologic
era, biological evolution,and surface temperature) & Nitrogen (N₂)
78\%Oxygen (O₂): 21\%Argon (Ar) \textbar{}Carbon dioxide (CO₂). ⎸Water
(H₂O) vapor \textbar{} ≈ 1\%Ozone (O₃) ⎸Methane (CH₄), etc.
\textbar{} \\
\textbf{Atmospheric Ozone (O₃)} & Present in upper atmosphere \\
\textbf{Axial Tilt/Obliquity (\(\varepsilon_x\))} & \(\varepsilon_x\) ≈
23.44° ↓\(\varepsilon_x\) ∈ ⟨22.1 ∧ 24.5⟩/41 ky10-12 ky to minimum \\
\textbf{Rotational period/Length of day (d)} & 24ʰ (synodic)23ʰ 56ᵐ
4.091ˢ (sidereal) \\
\textbf{Orbital Period (C)}(all the variations explained later) & 365ᵈ
5ʰ 49ᵐ 12ˢ (ephemeris)365ᵈ 5ʰ 48ᵐ 45ˢ (tropical)365ᵈ 6ʰ 9ᵐ 9.764ˢ
(sidereal) \\
\textbf{Orbital Eccentricity (e)} & 0.0167 ↓e ∈ ⟨0.01 ∧ 0.0⟩/413 kye ∈
⟨0.02 ∧ 0.05⟩/100 ky \\
\textbf{Axial precession period (χ)} & 25.772 ky \\
\textbf{Magnetosphere (radiation shielding) (\(B_{surf}\))} & ≈ ⟨25 ∧
65⟩ μT (microtesla) \\
\textbf{Land (lithosphere)-water (hydrosphere) proportion} &
30\%-70\% \\
\textbf{Hydrospheric distribution} & Five major oceansNumerous smaller
seas \\
\end{longtable}

\begin{quote}
\textbf{Keppy}: Uh\ldots{} wow. Not sure what to make of all that,
frankly.
\end{quote}

Yes, that's a lot.

\begin{quote}
\textbf{Keppy}: Can these be calculated for a fictional world?
\end{quote}

Some; most not easily. Others are completely independent of the physical
parameters (and each other, come to that). We can set some rules of
thumb, as it were.

\section{Extended Parameter Details}\label{extended-parameter-details}

\textbf{Rotational period/Length of diurn (D)} - D ∈ ⟨6ʰ ∧ 120ᵈ⟩ ---
General Geotic Range - D ∈ ⟨6ʰ ∧ 48ʰ⟩ --- Human-adaptable; supports
familiar circadian rhythms\\
- D ∈ ⟨2ᵈ ∧ 20ᵈ⟩ --- Mild extremes; thermal contrast can be buffered
with atmosphere or oceans - D ∈ ⟨20ᵈ ∧ 100ᵈ⟩ --- Edge cases; require
mitigation (dense atmosphere, global hydrosphere)\\
- D ∈ ⟨100ᵈ ∧ 120ᵈ⟩ --- Rare survivable zone\\
- Habitability hinges on:\\
- Efficient heat distribution (atmosphere or oceans)\\
- Slow stellar heating (distant or cool star)\\
- Non-volatile surface conditions\\
- Beyond 120ᵈ, the temperature contrast between day and night becomes
too extreme for habitability \textbf{without technological aid} (which
puts in the \emph{parahabitable} spectrum).\\
- Even a perfect ocean--atmosphere system may fail to smooth the thermal
tide.\\
- monons in this range may be Tellurics, but they are \emph{not}
Geotics.

\begin{quote}
\textbf{Keppy}: Do we know what other (if any) of a monon's physical
parameters may affect its rotational period?
\end{quote}

As a matter of fact, astrophysicists are beginning to \textbf{simulate
probable spin states} for Earth-like exoplanets under different
formation conditions. A few recurring themes: - Tidal locking is
expected for monons in the habitable zones of \textbf{M-dwarfs}, due to
proximity.\\
- Initial rotation rates may depend on:\\
- Accretion history\\
- Giant impacts (e.g., the Moon-forming event)\\
- Early tidal evolution - \textbf{Resonant rotations} - Mercury's 3:2
spin-orbit resonance may be \textbf{more common than full tidal locking}
- Close-in monons may be more likely \textbf{spin slowly or be tidally
locked} - Moons (especially those whose mass is a significant fraction
of the monon's) can stabilize or slow a monon's rotation rate - This is
happening with Earth; our day is lengthening by ≈ 17 microseconds per
year, or 1 second every 58800 years - Giant impacts can reset rotation
direction and/or speed

\textbf{Orbital Eccentricity (e)} - e ∈ ⟨0 ∧ 0.25⟩ --- General Geotic
Tolerance - e ∈ ⟨0.01 ∧ 0.10⟩ --- Earth-clone ideal range - In
multi-monon systems, stable configurations usually result in e
\textless{} 0.10 per monon - In single-monon (only-child) or widely
spaced systems, values up to e ≈ 0.25 may remain dynamically stable, but
tend to reduce overall habitability - Risks of e \textgreater{} 0.25 -
Large insolation differentials between periastron and apastron -
Pronounced climatic volatility - Less likelihood of maintaining
\textbf{persistent, stable biospheres} - Increased vulnerability to
\textbf{orbital perturbations} - Possible \textbf{transit through and
out of the system's habitable zone (HZ)} during a single orbit - Note:
On a world with zero axial tilt (ε = 0), high eccentricity might act as
a surrogate for seasonal variation --- producing orbit-phase-based
temperature cycles. This is \emph{technically viable}, but requires very
careful tuning of orbital shape, atmospheric thermal inertia, and
surface conditions to avoid extreme or catastrophic conditions.

\textbf{Orbital Period (C)} - This parameter \textbf{can't be freely
chosen} --- it's governed by \textbf{Kepler's Third Law}, originally
formulated as \[P^2 \propto a^3\]and later regularized by Newton to
account for the total mass of the system - The orbital period depends on
both the distance of the monon's orbit and the \emph{combined masses of
the star(s) and monon}. \[C = \sqrt{\dfrac{a^3}{M + m}}\] where: - C =
the orbital period of the monon in perannum - a = the monon's orbit's
\textbf{semi-major axis} in \textbf{\emph{Astronomical Units}} - M = the
mass of the star in solar units (⊙) - m = the mass of the monon (also in
solar units) - In most systems, m « M and can be neglected for quick
calculations: \[C = \sqrt{\dfrac{a^3}{M}}\] \ldots However, if the
monon's mass exceeds ≈ 10\% of the mass of the star(s), its contribution
to the period of its orbit begins to have noticeable effects. This is
especially relevant for: - Super-Jovian mass monons orbiting red-dwarf
stars (q.v.) - Binary monon systems - See \emph{Sidebar Module --
Two-body Systems} - See also \emph{Sidebar Module -- Double-monon or
monon-moon}?

\begin{itemize}
\tightlist
\item
  I include \emph{C} here not because it's adjustable, but because it's
  \textbf{crucial to the seasonal dynamics} of a world:

  \begin{itemize}
  \tightlist
  \item
    It modulates how \textbf{axial tilt (\(\varepsilon_x\))} and
    \textbf{eccentricity (e)} express over time\\
  \item
    It defines the \textbf{length of seasons}, and whether rotation (D)
    is \textbf{fast or slow relative to the year}\\
  \item
    It affects the \textbf{precession timescale (χ)} through long-term
    resonances
  \end{itemize}
\item
  In short: \emph{IF} you \emph{declare C}, that choice \emph{constrains
  what kind of star your system can have}.

  \begin{itemize}
  \tightlist
  \item
    See \emph{Sidebar Module -- Stars, Planetary Orbits, and Habitable
    Zones} for details
  \end{itemize}
\item
  \emph{Note that the masses of other monons in the system have no
  effect in this equation!}
\end{itemize}

\textbf{Obliquity (Axial tilt) (\(\varepsilon_x\))} - \(\varepsilon_x\)
∈ ⟨0° ∧ 5°⟩ --- Negligible seasonal variation - \(\varepsilon_x\) ∈ ⟨15°
∧ 35°⟩ --- Plausible range for active seasonal variation -
\(\varepsilon_x\) ∈ ⟨20° ∧ 30°⟩ --- Earth-normal like -
\(\varepsilon_x\) ∈ ⟨35° ∧ 45°⟩ --- Extreme seasonal variation unless
mitigated by atmosphere/oceans - \(\varepsilon_x\) \textgreater{} 45°
--- Potentially unstable - Polar and tropical regions actually reverse!

A monon does not \emph{have} to have an axial tilt, but most do, because
their mass isn't evenly distributed throughout their volume, and that
mass is tugged on by the host star(s), companion Moons, other monons in
their star system, etc.

\textbf{Precession Cycle (χ)} - χ ∈ ⟨15 ∧ 60⟩ ky --- General Geotics
range - χ ∈ ⟨20 ∧ 30⟩ ky --- Earth-clone ideal range - Note: If your
monon has an axial tilt (\(\varepsilon_x ≠ 0\)), it \emph{will} have an
axial precession period. - \textless{} 5 ky - Too fast - Rapid climatic
swings - Possibly preventive of long-term stable ecosystems - 5 -- 15 ky
- Can support meaningful precession cycles - More frequent variation -
20 -- 30 ky - Earth-normal range - Ideal for familar Milankovitch-style
climate pacing - 30 -- 60 ky - Acceptible - Modulates long-term climate,
but with slower variability - \textgreater{} 60 ky - May mute
precessional influence - monon becomes dependent on \emph{ε} or \emph{e}
(or a combination) for seasonal variation - Habitability-Relevant
Impacts of Axial Precession - Precession alone doesn't cause climate
variation---but it \emph{modulates} how obliquity and eccentricity
combine with seasons. - Precession rate is influenced by monon tilt,
mass distribution, rotation rate, and gravitational interactions
(especially from moons or nearby monons). - Moderately long precessional
periods (25--30 kyr) help establish stable, regular Milankovitch cycles
conducive to long-term ecosystem resilience. - Affected by: - Axial tilt
(\(\varepsilon_x\)) --- higher tilts tend to precess faster. - Rotation
rate --- faster rotation yields a stronger equatorial bulge (which
enhances precession). - Internal structure --- core-to-mantle mass
distribution changes the monon's moment of inertia. - Gravitational
interactions --- especially from moons or other monons in the system.

\textbf{Obliquity Azimuth (\(\zeta_{n}\))} - \(\zeta_{n}\) ∈ ⟨0° ∧ 359°⟩
--- Measure of the \emph{directionality} of the monons' obliquity
relative to the periastron of its orbit. - \(\zeta_0\) is \emph{defined}
as the orientation when the planemon's northern hemisphere is tilted
precisely away from the star or system \textbf{barycenter (ḅ)} (northern
solstice) at periastron. - This is called \textbf{\emph{periaptic
zero}}. - \(\zeta_{90}\) would indicate that northern solstice has
\emph{precessed} 90° around the monon's orbital path from
\textbf{periaptic zero}. - \(\zeta_{180}\) would indicate that the
northern solstice is occurring at \emph{apastron}, the point in the
monon's orbit \emph{farthest away} from the star/barycenter. -
IMPORTANT: - For \(\varepsilon_0\) monons (they have no axial tilt),
\(\zeta_x\) is \emph{undefined}.

\textbf{Magnetosphere (radiation shielding) (\(B_{surf}\))} -
\(B_{surf}\) refers to the \textbf{surface magnetic field strength},
measured in \textbf{microteslas (μT)}.\\
- For Earth, the typical surface field ranges between \textbf{25--65
μT}, depending on latitude and local crustal anomalies. -
\(B_{surf} \lt 5 \mu T\) --- Unshielded; high cosmic radiation exposure,
especially polar regions- -
\(B_{surf} \in \langle5 \wedge 20\rangle \mu T\) --- Minimal shielding;
monon parahabitable without O₃ layer and/or thick tmosphere -
\(B_{surf} \in \langle20 \wedge 40\rangle \mu T\) --- Weak field; still
protective but slightly more porous -
\(B_{surf} \in \langle40 \wedge 70\rangle \mu T\) --- Comparable to
Earth; effective radiation shielding -
\(B_{surf} \in \langle25 \wedge 65\rangle \mu T\) --- Earth's typical
magnetosphere strength range -
\(B_{surf} \in \langle70, ..,100\rangle \mu T\) --- Significant
shielding, but with growing secondary effects -
\(B_{surf} \gt 100 \mu T\) --- Magnetosphere begins trapping too much
radiation - Where the magnetosphere is concerned, there is such a thing
as ``too much of a good thing''; once the field strength exceeds
\textasciitilde100 μT, the magnetosphere may begin to trap radiation
instead of deflecting it --- causing the very problems it's meant to
prevent. - \(B_{surf} \gt 400 \mu T\) --- Gas giant strength; radiation
belts around monon make space travel exceedingly hazardous. - Why It
Matters - A strong magnetic field deflects charged particles from the
stellar wind, forming a \textbf{magnetosphere}.\\
- Without this protection:\\
- \textbf{Atmospheric erosion} increases (especially from solar UV and
wind) - \textbf{Surface radiation levels} rise, particularly in
equatorial and polar regions - \textbf{Ozone layers} and other
protective atmospheric chemistry can degrade - Mars is the cautionary
tale: once magnetically active, its \textbf{core solidified early}, its
field collapsed, and it lost most of its atmosphere to space. - What
Generates It? - A monon's magnetic field typically arises from a
\textbf{dynamo effect} --- the movement of \textbf{conductive fluid}
(usually liquid iron) in its outer core: - Requires: -
\textbf{Electrically conductive material} - \textbf{Rotation} (faster
helps) - \textbf{Active convection} in the core (driven by heat and
composition gradients) - Influenced by: - \textbf{Core size and
composition} (metallicity) - \textbf{Rotation rate (d)} --- faster
rotation generally strengthens dynamo action - \textbf{Internal heat
flux} --- related to mass, age, and radioactive element content - See
\emph{Sidebar Module -- Estimating Planetary Magnetospheres} for details
on estimating magnetospheres according to monon parameters.

\chapter{Continuation:
Recommendations}\label{continuation-recommendations}

We mentioned above that ``\ldots{} radius is the most flexible of the
parameters\ldots,'' and that's true, since we're treating radius as an
emergent property of the other parameters, specifically mass and
density.

\textbf{Gravity}, on the other hand, is really the \emph{least}
forgiving of the parameters \emph{for Geotic worlds}. But that deserves
a moment's attention. We've specified that:
\[g \in \langle0.5 \wedge 1.5\rangle\oplus\text{,}\] but we didn't
really explain \emph{why} those are our bounds. What \emph{are} the
physical and biological implications of going beyond them? Let's look.

\begin{quote}
\textbf{Keppy}: But some life forms \emph{might} still evolve on worlds
like these\ldots?
\end{quote}

Yes, with the caveats listed above. The point, here, though, is that
while gravities outside the g ∈ ⟨0.5 ∧ 1.5⟩⨁ range are \emph{certainly}
possible --- even \emph{probable} --- and while life \emph{might evolve}
under these conditions, these worlds would be Tellurics in the
\emph{parahabitable} range, outside even \emph{habitable}, and certainly
not \emph{hospitable to humans} which is our core criterion for
\textbf{Geotic worlds}.\#\# Abstract \textbf{Major Topics:}\\
- Five core monon parameters: mass (m), density (ρ), surface gravity
(g), escape velocity (vₑ), and radius (r).\\
- Distinction between \textbf{physical properties} (m, ρ) and
\textbf{emergent properties} (g, vₑ, r).\\
- WCB convention: density (ρ) treated as \textbf{uncompressed density}
to avoid recursive modeling.\\
- Symbolic precedence hierarchy:\\
1. Mass (m), Uncompressed Density (ρ) --- composition-driven.\\
2. Surface Gravity (g), Escape Velocity (vₑ) --- experiential.\\
3. Radius (r) --- emergent.\\
- Validation: computed values for g, vₑ, and r from Geotic-range m and ρ
remain within acceptable Geotic bounds.

\textbf{Key Terms \& Symbols:}\\
- m = mass.\\
- ρ = uncompressed density.\\
- g = surface gravity.\\
- vₑ = escape velocity.\\
- r = radius (emergent).\\
- Geotic range: ⟨0.5 ∧ 1.5⟩⨁.\\
- Order of calculation: begin with m and ρ for valid outputs.

\textbf{Cross-Check Notes:}\\
- Canonical clarification: radius (r) is \emph{not} arbitrarily chosen
--- it emerges from m and ρ.\\
- Replaces recursive compression modeling with a WCB-friendly
simplification.\\
- Geotic range justification links to habitability framework.\\
- Integrates with prior note \textbf{Example: When Good Values Go Bad}
(edge-case validation).\\
- Ensure consistent update: ``WCB'' → ``WCB'' across this file and
related notes.

\chapter{Close-focus on Parameter
Precedence}\label{close-focus-on-parameter-precedence}

Elsewhere, I stated that

\begin{quote}
``Radius is an \_emergent property --- the most flexible of the five
core monon parameters*.''
\end{quote}

Here, we learn \emph{why} that's the case.

WCB defines five core parameters for modeling monons: - Mass (m) -
Density (ρ) - Surface Gravity (g) - Escape Velocity (vₑ) - Radius (r)

\section{Physical vs Emergent
Properties}\label{physical-vs-emergent-properties}

The first two --- \textbf{mass} and \textbf{density} --- are fundamental
\textbf{physical properties}. If an object is composed of a specific
material (e.g., pure iron), then: - Its \textbf{density} reflects the
compactness of that material under standard conditions.\\
- Its \textbf{mass} is the product of that density and volume. These
values are not ``fuzzy'' --- they are \emph{determined} by composition.
But there's a complicating factor:

\subsection{Uncompressed Density in
WCB}\label{uncompressed-density-in-wcb}

In reality, high-mass monons \textbf{compress themselves} under their
own gravity. This self-compression: - Increases core pressure\\
- Reduces actual volume\\
- Nudges the average (measured) density \emph{upward} Unfortunately, the
feedback between mass, gravity, and compression is: - This feedback loop
is \textbf{nonlinear}\\
- The corrections vary by \textbf{material type}\\
- And modeling it accurately requires \textbf{complex equations of
state} WCB intentionally avoids this complexity in favor of
\textbf{practical worldcrafting}.

\begin{quote}
\textbf{WCB Convention:}\\
We treat \emph{ρ} (density) as the \textbf{uncompressed density} of the
monon --- the intrinsic density of its materials assuming no internal
gravitational compression.
\end{quote}

This keeps \emph{ρ} symbolically \textbf{independent} from \emph{m} and
allows us to calculate \emph{r}, \emph{g}, and \emph{vₑ} without
recursive modeling. Compression effects are minimal within the Geotic
range ⟨0.5 ∧ 1.5⟩⨁, so this simplification is physically tolerable and
worldcrafter-friendly.

\subsection{Parameter Precedence: A Symbolic
Hierarchy}\label{parameter-precedence-a-symbolic-hierarchy}

In WCB, we adopt a symbolic precedence system to clarify \textbf{what
depends on what}. \#\#\#\# First Precedence: Composition-Driven
Parameters - \textbf{Mass} (\emph{m})\\
- \textbf{Uncompressed Density} (\emph{ρ}) These are \textbf{primary
constraints}, based on: - Material class\\
- Planetary category\\
- Biospheric plausibility\\
- Narrative or symbolic intention \#\#\#\# Second Precedence:
Experiential Parameters - \textbf{Surface Gravity} (\emph{g})\\
- \textbf{Escape Velocity} (\emph{vₑ}) These depend on \emph{m} and
\emph{ρ}, and describe how a being would \textbf{experience} the world.
They answer: - \emph{``How heavy do things feel?''}\\
- \emph{``How hard is it to launch something into space?''} Because
they're \textbf{experiential}, \emph{g} and \emph{vₑ} serve as
\textbf{thresholds for habitability} and \textbf{filters for narrative
design}.

\begin{quote}
For writers of SF, escape velocity might be of slightly greater import,
as they are often more concerned with getting characters out of the
monon's gravity well. \#\#\#\# Third Precedence: Emergent Parameter -
\textbf{Radius} (\emph{r}) We treat radius as an \textbf{emergent
quantity}, derived from the interplay between \emph{m} and \emph{ρ}.
Once those are specified, \emph{r} is computed using: \[
\begin{align}
r &= \sqrt[3]{\frac{3 m}{4 \pi \rho}} &&\qquad \text{Absolute Calculation} \\
r &= \sqrt[3]{\frac{m}{\rho}} &&\qquad \text{Relative calculation}
\end{align}
\] By defining \emph{ρ} as uncompressed, we sidestep recursive
compression modeling and free \emph{r} to play its proper role as a
derived result.
\end{quote}

\begin{quote}
\textbf{In short:} Radius is not arbitrarily assigned in WCB --- it
emerges from first principles, and that's exactly what gives it
flexibility.
\end{quote}

However, if a monon meets all other Geotic criteria but falls slightly
outside the ⟨0.5 ∧ 1.5⟩⨁ radius guideline, \textbf{there is room for
flex} --- especially when symbolic or experiential factors support it.
\#\#\# Surface Gravity, Escape Velocity, and Radius ``Flexibility'' We
maintain the general \textbf{Geotic envelope} of ⟨0.5 ∧ 1.5⟩⨁ for all
five core parameters. However --- as discussed in \textbf{Example: When
Good Values Go Bad} --- not all combinations of any two parameters will
yield outputs that remain Geotic.

To test this, we evaluated all combinations of \textbf{mass} (\emph{m})
and \textbf{uncompressed density} (\emph{ρ}) within the Geotic range
(⟨0.5 ∧ 1.5⟩⨁), using an increment of 0.05⨁ for each. From these, we
calculated the corresponding: - \textbf{Surface Gravity} (\emph{g})\\
- \textbf{Escape Velocity} (\emph{vₑ})\\
- \textbf{Radius} (\emph{r})

\#\#\#\# Results from Evaluated Combinations: - \textbf{Gravity}
(\emph{g}) --- derived from \emph{m} and \emph{ρ}:\\
- All values fell within the expected ⟨0.5 ∧ 1.5⟩⨁ range.\\
- \textbf{Escape Velocity} (\emph{vₑ}) --- derived from \emph{m} and
\emph{ρ}:\\
- Calculated range was ⟨0.7071 ∧ 1.2247⟩⨁.\\
- \textbf{Radius} (\emph{r}) --- derived from \emph{m} and \emph{ρ}:\\
- Calculated range was ⟨0.6934 ∧ 1.4423⟩⨁. \textbf{Conclusion:}\\
All outputs for \emph{g}, \emph{vₑ}, and \emph{r} computed from
Geotic-range values of \emph{m} and \emph{ρ} \textbf{remain within or
closely near} the ⟨0.5 ∧ 1.5⟩⨁ constraint window.

\begin{quote}
\textbf{Reminder:}\\
While \emph{mass} and \emph{density} serve as \textbf{primary
constraints}, the values they generate for \emph{gravity}, \emph{escape
velocity}, and \emph{radius} are \textbf{emergent} --- and may push
right up to the edges of Geotic acceptability. monons with marginal
values on one or more of these outputs may still qualify symbolically,
depending on biospheric or narrative justification.
\end{quote}

\section{Order of Parameter
Calculation}\label{order-of-parameter-calculation}

WCB allows you to begin with \textbf{any two} of the five core monon
parameters and calculate the remaining three. For example: - You might
specify a desired \textbf{gravity} (\emph{g}) and \textbf{radius}
(\emph{r}) to match narrative or symbolic needs --- and then calculate
\textbf{mass}, \textbf{density}, and \textbf{escape velocity} from
those. However:

\begin{quote}
⚠️ \textbf{Caution:}\\
\textbf{Mass} (\emph{m}) and \textbf{uncompressed density} (\emph{ρ})
are treated as \textbf{inflexible physical constraints} in WCB. They
must fall within the Geotic envelope ⟨0.5 ∧ 1.5⟩⨁ to yield plausibly
habitable worlds.
\end{quote}

\subsection{WCB Recommendation}\label{wcb-recommendation}

\begin{quote}
\textbf{Start with mass and uncompressed density.}\\
These two parameters:\textgreater{} - Reflect the physical composition
of your monon - Are symbolically independent - Guarantee valid outputs
for \emph{g}, \emph{vₑ}, and \emph{r} if kept within the Geotic range
\end{quote}

By anchoring your model in \emph{m} and \emph{ρ}, you ensure that all
derived parameters remain within acceptable bounds --- streamlining your
creative process and avoiding invalid combinations. \#\# Abstract\\
\textbf{Major Topics:}\\
- Defines a \textbf{symbolic grammar for constraints}:\\
- \textbf{Evaluative} (truth tests), \textbf{Comparative}
(descriptions), \textbf{Prescriptive} (soft rules), \textbf{Mandative}
(hard rules).\\
- Introduces \textbf{range connectives} for closed, open, half-open, and
exterior ranges (∧, ∨, ⩜, ⩝, ⊼, ⩟, ⊽, ⩡).\\
- Establishes \textbf{random assignment syntax} with ⟨⟨ ⟩⟩ brackets to
select values from ranges, including mandated vs.~optional
assignments.\\
- Adds the \textbf{precision inference rule}: result precision matches
the most precise endpoint.\\
- Provides \textbf{axiomatic justification} (WCB Axiom 7.1 ---
Symbolcrafter's Creed) for including a wide range of connectives to
preserve semantic completeness.

\textbf{Key Terms \& Symbols:}\\
- Constraint operators: \texttt{≤.}, \texttt{.≤}.\\
- Range connectives: ∧, ∨, ⩜, ⩝, ⊼, ⩟, ⊽, ⩡.\\
- Randomization operator: ⟨⟨ ⟩⟩.\\
- Precision inference rule.

\textbf{Cross-Check Notes:}\\
- \textbf{{[}NEW{]}} All notation is new {[}ins{]} --- no overlaps with
existing canon.\\
- Serves as the \textbf{symbolic foundation} for other randomization
methods (e.g., Orbit Randomization).

\chapter{Range Constraints \& Random
Assignment}\label{range-constraints-random-assignment}

\section{🧱 Core Constraint Classes}\label{core-constraint-classes}

\begin{longtable}[]{@{}
  >{\raggedright\arraybackslash}p{(\linewidth - 6\tabcolsep) * \real{0.1474}}
  >{\raggedright\arraybackslash}p{(\linewidth - 6\tabcolsep) * \real{0.1895}}
  >{\raggedright\arraybackslash}p{(\linewidth - 6\tabcolsep) * \real{0.3579}}
  >{\raggedright\arraybackslash}p{(\linewidth - 6\tabcolsep) * \real{0.3053}}@{}}
\toprule\noalign{}
\begin{minipage}[b]{\linewidth}\raggedright
Type
\end{minipage} & \begin{minipage}[b]{\linewidth}\raggedright
Symbol Form
\end{minipage} & \begin{minipage}[b]{\linewidth}\raggedright
Meaning
\end{minipage} & \begin{minipage}[b]{\linewidth}\raggedright
Example
\end{minipage} \\
\midrule\noalign{}
\endhead
\bottomrule\noalign{}
\endlastfoot
Evaluative & \texttt{x\ \textless{}\ y}, \texttt{x\ !≈\ y} & Truth test
& Does x satisfy a condition? \\
Comparative & \texttt{ΔT\ \textgreater{}\ 0} & Descriptive comparison &
Not used for enforcement \\
Prescriptive & \texttt{x\ ≤.\ y} & Should (soft rule) & Preferred but
not required \\
Mandative & \texttt{x\ .≤\ y} & Must (hard rule) & Required for
validity \\
\end{longtable}

\section{✴️ Modifiers}\label{modifiers}

\begin{itemize}
\tightlist
\item
  \texttt{!} --- logical negation (\texttt{!=}, \texttt{!∈},
  \texttt{!≈})
\item
  \texttt{.} prefix --- \textbf{mandative} (\texttt{.≤}, \texttt{.∈})
\item
  \texttt{.} suffix --- \textbf{prescriptive} (\texttt{≤.}, \texttt{∈.})
\end{itemize}

⚠️ Do not combine \texttt{!} with dot-prefixed/suffixed forms. Use the
logical inverse instead (e.g., \texttt{.\textgreater{}},
\texttt{\textgreater{}.}).

\section{📏 Range Connectives}\label{range-connectives}

\begin{longtable}[]{@{}
  >{\raggedright\arraybackslash}p{(\linewidth - 6\tabcolsep) * \real{0.0882}}
  >{\raggedright\arraybackslash}p{(\linewidth - 6\tabcolsep) * \real{0.3529}}
  >{\raggedright\arraybackslash}p{(\linewidth - 6\tabcolsep) * \real{0.2059}}
  >{\raggedright\arraybackslash}p{(\linewidth - 6\tabcolsep) * \real{0.3529}}@{}}
\toprule\noalign{}
\begin{minipage}[b]{\linewidth}\raggedright
Symbol
\end{minipage} & \begin{minipage}[b]{\linewidth}\raggedright
Meaning
\end{minipage} & \begin{minipage}[b]{\linewidth}\raggedright
Logical Form
\end{minipage} & \begin{minipage}[b]{\linewidth}\raggedright
Range Type
\end{minipage} \\
\midrule\noalign{}
\endhead
\bottomrule\noalign{}
\endlastfoot
∧ & Inclusive interior & a ≤ ▢ ≤ b & Closed range \\
∨ & Exclusive interior & a \textless{} ▢ \textless{} b & Open range \\
⩜ & Inclusive exterior & ▢ ≤ a or ▢ ≥ b & Outside, includes bounds \\
⩝ & Exclusive exterior & ▢ \textless{} a or ▢ \textgreater{} b &
Strictly outside \\
⊼ & Left-exclusive interior & a \textless{} ▢ ≤ b & Half-open \\
⩟ & Right-exclusive interior & a ≤ ▢ \textless{} b & Half-open \\
⊽ & Left-exclusive exterior & ▢ \textless{} a or ▢ ≥ b & Edge-grazing
exterior \\
⩡ & Right-exclusive exterior & ▢ ≤ a or ▢ \textgreater{} b &
Edge-grazing exterior \\
\end{longtable}

\section{🎲 Random Assignment Syntax}\label{random-assignment-syntax}

\subsection{Basic Rule:}\label{basic-rule}

Use \texttt{⟨⟨\ ⟩⟩} to indicate \textbf{random value assignment} from a
specified range.

\begin{longtable}[]{@{}
  >{\raggedright\arraybackslash}p{(\linewidth - 2\tabcolsep) * \real{0.2121}}
  >{\raggedright\arraybackslash}p{(\linewidth - 2\tabcolsep) * \real{0.7879}}@{}}
\toprule\noalign{}
\begin{minipage}[b]{\linewidth}\raggedright
Expression
\end{minipage} & \begin{minipage}[b]{\linewidth}\raggedright
Meaning
\end{minipage} \\
\midrule\noalign{}
\endhead
\bottomrule\noalign{}
\endlastfoot
x = ⟨⟨a ∧ b⟩⟩ & Assign random value from a to b (inclusive) \\
x .= ⟨⟨a ⩝ b⟩⟩ & Must assign value outside strict range \\
x = ⟨⟨a ⩟ b⟩⟩ & Assign value in left-inclusive, right-excluded range \\
\end{longtable}

\begin{itemize}
\tightlist
\item
  = → assignment
\item
  .= → mandated assignment (value must be generated)
\end{itemize}

\section{🎲 Random Assignment Syntax With
Weighting}\label{random-assignment-syntax-with-weighting}

\[
x = a + (b - a)\,⟨⟨0 ∧ 1⟩⟩^{p}
\]

Where: - \(a\) = low end of the random range\\
- \(b\) = high end of the random range\\
- \(p\) = weighting exponent\\
- \(p > 1\): weights the randomization toward \(b\)\\
- \(0 < p < 1\): weights the randomization toward \(a\)\\
- \(p = 1\): produces a uniform (unbiased) distribution\\
- \(p \neq 0\): undefined at zero

\begin{quote}
Because ⟨⟨0 ∧ 1⟩⟩ represents a continuous uniform variable, and because
the exponent \(p \in (0, \infty)\) continuously reshapes that
distribution, the \textbf{biasing space} is symmetric about \(p = 1\):
bias toward \emph{a} for \(0 < p < 1\), bias toward \emph{b} for
\(p > 1\).
\end{quote}

\section{🔬 Precision Inference Rule}\label{precision-inference-rule}

\begin{quote}
The \textbf{decimal precision of a randomized result} is inferred from
the \textbf{most precise} range endpoint.
\end{quote}

\begin{longtable}[]{@{}ll@{}}
\toprule\noalign{}
Syntax & Result Precision \\
\midrule\noalign{}
\endhead
\bottomrule\noalign{}
\endlastfoot
⟨⟨1.4 ∧ 2.2⟩⟩ & 1 decimal place \\
⟨⟨1.40 ∧ 2.2⟩⟩ & 2 decimal places \\
⟨⟨1.400 ∧ 2.200⟩⟩ & 3 decimal places \\
\end{longtable}

This rule applies \textbf{even if the endpoints are excluded} from the
valid output range.

\section{❌ Invalid Forms}\label{invalid-forms}

\begin{longtable}[]{@{}ll@{}}
\toprule\noalign{}
Expression & Reason \\
\midrule\noalign{}
\endhead
\bottomrule\noalign{}
\endlastfoot
x = ⟨⟨1.414⟩⟩ & ❌ Invalid: one-value range \\
x ∈ !⟨a ∧ b⟩ & ❌ Ambiguous: use \texttt{⩜} or \texttt{⩝} instead \\
x !.∈ \ldots{} & ❌ Invalid modifier stacking \\
\end{longtable}

\section{📜 Axioms}\label{axioms}

\subsection{WCB Axiom 7.1 --- The Symbolcrafter's
Creed}\label{wcb-axiom-7.1-the-symbolcrafters-creed}

\begin{quote}
\emph{``Better to have it and not need it than need it and not have
it.''}
\end{quote}

All range connectives, including obscure ones like \texttt{⩡}, are
retained in W101 to ensure semantic closure and support future or
edge-case modeling needs.

\section{🌌 Example Use}\label{example-use}

markdown K₁ .∈ ⟨a ⩜ b⟩ Kirkwood Gap 1 must lie strictly between a and b
(excluding both endpoints)

r .= ⟨⟨a ⩝ b⟩⟩ Assign a randomized orbital radius outside a forbidden
band

{[}{[}Orbit Randomization ✓{]}{]}

\chapter{Abstract}\label{abstract-42}

\textbf{Major Topics:}\\
- Defines \textbf{asteroids} as \textbf{Small Star System Bodies
(S3Bs)}---the smallest, most numerous members of a stellar system's
mononic population.\\
- Describes them as \textbf{lithic or carbonic monons} occupying the
\textbf{micromonon--minimonon} mass range, typically cohesion-bound and
non-hydrostatic.\\
- Establishes the \textbf{Asteroidal Morphotype}, encompassing multiple
\textbf{conformations} that vary by primordial composition and formation
zone.\\
- Identifies principal conformations: \textbf{C-type, S-type, M-type,
D-type, P-type, V-type}, and transitional intermediates.\\
- Explains that these conformations represent gradients of
\textbf{volatility, metallicity, and thermal history}, producing the
compositional zoning seen in mature asteroid belts.\\
- Notes that S³Bs collectively serve as \textbf{chemical and dynamical
fossils} of stellar-system evolution, illustrating how minor bodies
preserve the record of protoplanetary processes.

\textbf{Key Terms \& Symbols:}\\
- \textbf{S3B (Small Star System Body)} --- colloquial collective for
subplanetary monons within a stellar system.\\
- \textbf{Asteroidal Morphotype {[}mono{]}.}\\
- \textbf{C-, S-, M-, D-, P-, V-type Conformations {[}conf{]}.}\\
- \textbf{Micromonon, Minimonon {[}mono{]}.}\\
- \textbf{Volatility Gradient {[}chem{]}.}\\
- \textbf{Thermal History {[}evo{]}.}

\textbf{Cross-Check Notes:}\\
- \textbf{Asteroidal Morphotype {[}mono{]}} formally defined in
\emph{Meta 3 --- Morphotypes}.\\
- \textbf{S3B {[}mono{]}} usage consistent with \emph{planemons 1 ---
Framework and Equations}.\\
- \textbf{C--P-type Conformations {[}conf{]}} align with established
lithic and carbonic forms.\\
- \textbf{Density and composition correlations {[}phys{]}}
cross-referenced with \emph{Monons 2 --- Properties and Parameters}.\\
- \textbf{Status:} {[}EXPANDED{]} --- adds physical detail,
compositional taxonomy, and system-level context for small-body
populations.

\section{🪨 Asteroid Types}\label{asteroid-types}

Asteroids --- or, more broadly, \textbf{Small Star System Bodies (S3Bs)}
--- are not all alike.\\
Their compositions reflect \textbf{where in the protoplanetary disk}
they formed, and they tend to cluster into \textbf{compositional
families} within belts and resonance niches.\\
\#\#\# 🔹 Main Classes (by composition)

\begin{itemize}
\tightlist
\item
  \textbf{C-type (Carbonaceous)}\\
  \emph{Dark, primitive, volatile-rich.}

  \begin{itemize}
  \tightlist
  \item
    Very common in outer belts.\\
  \item
    High water/ice content.\\
  \item
    Analogues: many outer-belt asteroids, comet precursors.
  \end{itemize}
\item
  \textbf{S-type (Silicaceous)}\\
  \emph{Rocky, stony, silicate-rich.}

  \begin{itemize}
  \tightlist
  \item
    Common in inner belts.\\
  \item
    Often parent bodies of ordinary chondrite meteorites.\\
  \item
    Analogues: much of the inner main belt.
  \end{itemize}
\item
  \textbf{M-type (Metallic)}\\
  \emph{Iron--nickel cores of disrupted proto-planets.}

  \begin{itemize}
  \tightlist
  \item
    Often mid-belt objects.\\
  \item
    Dense, high radar reflectivity.\\
  \item
    Analogues: 16 Psyche and smaller metallic fragments.\\
    \#\#\# 🔹 Other Notable Types
  \end{itemize}
\item
  \textbf{D-type / P-type}\\
  \emph{Reddish, organic-rich, volatile-rich.}

  \begin{itemize}
  \tightlist
  \item
    Very dark, low density.\\
  \item
    Found in outer belt, Trojans, and beyond.
  \end{itemize}
\item
  \textbf{K-type / Q-type}\\
  \emph{Intermediate spectra.}

  \begin{itemize}
  \tightlist
  \item
    Less common.\\
  \item
    Transitional between S and C types.
  \end{itemize}
\item
  \textbf{V-type (Vestoids)}\\
  \emph{Basaltic surface fragments.}

  \begin{itemize}
  \tightlist
  \item
    Pieces of differentiated crust.\\
  \item
    Example: fragments of 4 Vesta.\\
    \#\#\# 🔹 Worldbuilder shorthand
  \end{itemize}
\item
  \textbf{Inner belts (hotter):} S-types (rocky).\\
\item
  \textbf{Middle belts (mixed):} M-types (metallic).\\
\item
  \textbf{Outer belts (colder):} C-types (carbonaceous/icy).\\
\item
  \textbf{Beyond snow line:} D/P-types (icy/organic, comet-like).\\
  \#\#\# 📖 Worldbuilder takeaway\\
  When you divide a belt with resonances, don't just think ``empty
  lanes.''\\
  Each sub-belt can be \textbf{compositionally distinct}: rocky inside,
  metallic mid-belt, icy/carbonaceous outside.\\
  This makes your belts feel \emph{alive} with variety, and gives
  explorers something to discover besides ``more rocks.''
\end{itemize}

\part{Binary Systems}

\chapter{Abstract}\label{abstract-43}

\textbf{Major Topics:}\\
- Introduces the \textbf{fundamental geometry and parameters of binary
star systems}.\\
- Defines \textbf{primary} (\(M_P\)) and \textbf{secondary} (\(M_S\))
bodies, their relationship to the \textbf{barycenter} (ḅ), and how mass
ratio determines orbital extent.\\
- Derives nine key \textbf{dimensional parameters} of binary motion:
minimum, average, and maximum separations for the total system (\(T\)),
the primary (\(P\)), and the secondary (\(S\)).\\
- Establishes core formulas linking:\\
- barycentric distances (\(P_\bullet\), \(S_\bullet\)) to total
separation (\(T_\bullet\)),\\
- orbital \textbf{eccentricity} (\(e\)) to maximum/minimum/average
positions, and\\
- constant proportionalities between stellar \textbf{mass ratios} and
\textbf{orbital distances}.\\
- Introduces the \textbf{Crux Metric} (\(\acute{e}\)) --- the
eccentricity at which unequal-mass orbits become tangentially
adjoined.\\
- Explains \textbf{relative-orbit simplification}, where the more
massive body is treated as stationary when \(M_P \gg M_S\).\\
- Defines \textbf{barycentric motion} equivalence: the barycenter's
``orbit'' mirrors that of the primary, scaled by mass ratio.\\
- Summarizes \textbf{constant equalities} that remain invariant across
all binary configurations.\\
- Incorporates \textbf{observational context} for multiplicity among
solar-type stars (Duquennoy \& Mayor 1991; Raghavan et al.~2010).\\
- Develops an \textbf{empirical model of companion-mass trends} (Moe \&
Di Stefano 2017; Li et al.~2022) showing that primary spectral class
constrains secondary mass range.\\
- Provides a \textbf{stochastic formulation} for secondary mass
selection:\\
\[M_2 = M_1 \times ⟨⟨a ∧ b⟩⟩^{k}\]\\
with optional bias exponent \(k\) to emulate observed distributions.\\
- Discusses \textbf{orbital-eccentricity limits} ensuring physical
separation (\(T_{min} ≥ 0.10\) AU).\\
- Outlines the \textbf{period--eccentricity relation} and its physical
origin in \textbf{tidal circularization}, linking orbital period to
eccentricity damping timescales.\\
- Summarizes observed regimes (short-, intermediate-, and long-period
binaries) and anomalies driven by third-body interactions or
post-transfer effects.

\textbf{Key Terms \& Symbols:}\\
- \textbf{\(M_P\), \(M_S\)} --- primary and secondary masses.\\
- \textbf{\(\mathcal{A}\)} --- average separation of binary
components.\\
- \textbf{\(T_{min}\), \(T_{max}\)} --- minimum and maximum stellar
separations.\\
- \textbf{\(P_\bullet\), \(S_\bullet\)} --- barycentric distances of
primary and secondary.\\
- \textbf{\(e\)} --- orbital eccentricity.\\
- \textbf{\(\acute{e}\)} --- \emph{Crux Metric}, eccentricity where
orbits adjoin tangentially.\\
- \textbf{\(\bar{e}\)} --- limiting eccentricity ensuring
\(T_{min} ≥ 0.10\) AU.\\
- \textbf{\(k\)} --- weighting exponent for stochastic mass pairing.\\
- \textbf{\(\mu\)} --- binary mass ratio = \(M_2 / (M_1 + M_2)\).\\
- \textbf{\(P_{circ}\)} --- circularization period threshold
(\textasciitilde10 days for solar-type systems).

\textbf{Cross-Check Notes:}\\
- Complements \textbf{Stars 1 --- Basics} by applying stellar parameters
to multi-body configurations.\\
- Establishes a quantitative basis for subsequent sections on
\textbf{Roche geometry}, \textbf{Hill spheres}, and \textbf{orbital
stability zones}.\\
- Provides canonical formulae and constants for world-building
applications involving binary or multiple stellar systems.

This section focuses on binary systems in general. While
higher-multiplicity arrangements are common and fascinating, they
introduce significant mathematical and physical complexity beyond the
current scope of this guide. \# The Basics Binary systems consist of two
bodies bound in a mutual gravitational relationship, each tracing an
orbital path around a shared center of mass known as the
\textbf{barycenter} (\emph{ḅ}). This point, \textbar\textbar shown as a
black X in the figure\textbar\textbar, \emph{always} lies along the line
connecting the centers of the two bodies and is not a massive object
itself, but a calculated position determined by the masses and
separation of the components.

When the two objects are of unequal masses (\(M_2 < M_1\)), the more
massive object (the \textbf{primary body}) orbits on an elliptical path,
on average closer to the barycenter, while the \textbf{secondary body}
traces a \emph{larger} elliptical path on average farther from the
barycenter. Both orbits share the same eccentricity (\(e\)), and are
synchronized in period, preserving the balance of angular momentum. They
differ only in extent.

A binary system is described by a total of nine dimensions:

\begin{itemize}
\tightlist
\item
  \(T_{min}\) , \(\mathcal{A}\) , \(T_{max}\) : The minimum, average,
  and maximum separations of the two bodies from one another
\item
  \(P_{min}\) , \(P_{avg}\) , \(P_{max}\): The minimum, average, and
  maximum separations of the \textbf{primary} body (\(P\)) from the
  \textbf{barycenter} (\emph{ḅ})
\item
  \(S_{min}\) , \(S_{avg}\) , \(S_{max}\) : The minimum, average, and
  maximum separations of the \textbf{secondary} body (\(S\)) from the
  \textbf{barycenter} (\emph{ḅ})
\end{itemize}

\ldots{} as well as the masses of the two bodies: - \(M_1\): the primary
mass (\(P\)) - \(M_2\): the secondary mass (\(S\))

\ldots{} and the eccentricity of their orbits: - \(e\): a dimensionless
value that describes the deviation of the orbital paths from perfectly
circular; \(e = 0\) means the orbit is a circle

These are related through a series of equations, which may seem daunting
at first, but are quite straightforward once they are understood. \#\#\#
Primary Dimensions \[
\begin{align}
P_{avg} &= \mathcal{A} \times\dfrac{M_S}{M_P+M_S} \qquad &&\text{Primary average distance}\\[1em]
P_{min} &= P_{avg}(1 - e) \qquad &&\text{Primary minimum distance} \\[1em]
P_{max} &= P_{avg}(1 + e) \qquad &&\text{Primary maximum distance} 
\end{align}
\] \#\#\# Secondary Dimensions \[
\begin{align}
S_{avg} &= \mathcal{A} \times\dfrac{M_P}{M_P+M_S} \qquad &&\text{Secondary average distance}\\[1em]
S_{min} &= S_{avg}(1 - e) \qquad &&\text{Secondary minimum distance} \\[1em]
S_{max} &= S_{avg}(1 + e) \qquad &&\text{Secondary maximum distance}
\end{align}
\] \#\#\# Total (Overall) Dimensions \[
\begin{gather}
T_{min} = \mathcal{A}(1 - e)
= P_{min} + S_{min} = T_{max}\left(\dfrac{1 - e}{1 + e}\right) \\
T_{max} = \mathcal{A}(1 + e)
= P_{max} + S_{max} = T_{min}\left(\dfrac{1 + e}{1 - e}\right) \\
\mathcal{A} = \dfrac{T_{min}}{1 - e}
= \dfrac{T_{max}}{1 + e}
= P_{avg} + S_{avg}\\[0.5em]
\end{gather}
\] \#\#\# Eccentricity Relationships \textgreater{} In the equations
below a subscript dot \(_{\bullet}\) means \emph{any two matching
parameters}; e.g.~\(Max_{\bullet} - Min_{\bullet}\) means any maximum
value minus any minimum value of the same parameter, such as
\(P_{max} - P_{min}\). \#\#\#\# System Eccentricity \[
\begin{align}
e &= \dfrac{Max_\bullet - Min_{\bullet}}{Max_\bullet + Min_{\bullet}}
\;\;=\;\; \left[1 - \dfrac{Min_{\bullet}}{Avg_{\bullet}}\right]
\;\;=\;\; \left[\dfrac{Max_{\bullet}}{Avg_{\bullet}} - 1\right] \\[1em]
&= \left(P_{max} \times \dfrac{M_P + M_S}{\mathcal{A} \times M_S}\right) - 1
\quad = \quad 1 - \left(P_{min} \times \dfrac{M_P + M_S}{\mathcal{A} \times M_S}\right) \\[1em]
&= \left(S_{max} \times \dfrac{M_P + M_S}{\mathcal{A} \times M_P}\right) - 1
\quad = \quad 1 - \left(S_{min} \times \dfrac{M_P + M_S}{\mathcal{A} \times M_P}\right) \\[1em]
\end{align}
\] \#\#\#\# The Crux Metric (\(\acute{e}\)) \textgreater{} The circle
\(_{\circ}\) subscript is used to indicate expressions in which all
terms share the \textbf{same positional magnitude} (e.g., max, min, or
average), regardless of parameter type. \[
\begin{align}
\acute{e} = \dfrac{M_P - M_S}{M_P + M_S}
= \dfrac{|S_{\circ} - P_{\circ}|}{S_{\circ} + P_{\circ}} 
= \dfrac{|S_{\circ} - P_{\circ}|}{T_{\circ}}
\end{align}
\] - \(é\) (e-prime) is the system eccentricity value at which the
orbits of the stars become \emph{\emph{adjoined tangential}}
\((e \gt 0; M_P \neq M_S)\) - For a mass ratio of
\(^{M_S}/_{M_P} = 0.8\), the system requires \(\acute{e} \geq 0.8519\)
for the primary and secondary orbits to adjoin tangentially (see below).

\chapter{Relative Orbit of the
Secondary}\label{relative-orbit-of-the-secondary}

There are times (such as when the Secondary is less massive than the
Primary --- typically \(\frac{M_1}{M_2} \gtrsim 80\), the approximate
star/brown dwarf mass dividing line, but it works for sub-stellamon
bodies as well --- it is convenient to treat the Primary as stationary
and the Secondary as following the relative orbit alone. Therefore, \[
\begin{gather}
R_{min} = T_{min} \\[0.5em]
R_{avg} = \mathcal{A} \\[0.5em]
R_{max} = T_{max} \\
\end{gather}
\] For instance, in the case of the Earth-Sun system: \[
\begin{align}
\mathcal{A} &= 1.0 AU \\[0.5em]
P_{avg} &= \mathcal{A} \times\dfrac{M_S}{M_P+M_S} \\[1em]
&= 1.0 \times \frac{1}{333000+1} \\[1em]
&= 1.0 \times \frac{1}{333001} = 3.009299 \times 10^{-6} AU \\[1.5em]
1 \text{ AU } &= 1.496 \times 10^6 \text{ km} \\
\therefore P_{avg} &= 3.009299 \times 10^{-6} \times 1.496 \times 10^8 ≈ 449 \text{ km}  
\end{align}
\]

Considering that the Sun's radius is \(696{,}340\) km, a wobble of only
\(≈ 450\) km (\(\approx 0.065\%\)) is justifiably negligible, so the
math works out well enough by just treating the Sun as stationary and
the Earth as orbiting it. \# Barycentrics Similarly to the Relative
Orbit of the Secondary, when dealing with a system where \(M_S ≈ M_P\),
it becomes more convenient to think of the masses as stationary and
their barycenter ``orbiting'' between them. \textbf{\textbar graphic to
be added\textbar{}} \[
\begin{align}
B_{min} &=   \frac{\mathcal{a}(1 - e)\; M_S}{M_P + M_S} 
= \frac{T_{min}\; M_S}{M_P + M_S} = B_{avg}(1 - e) = \mathbf{P_{min}} \\[2em]
B_{avg} &= \frac{\mathcal{a}\; M_S}{M_P + M_S} = \mathbf{P_{avg}} \\[2em]
B_{max} &= \frac{\mathcal{a}(1 + e)\; M_S}{M_P + M_S} 
= \frac{T_{max}\; M_S}{M_P + M_S} = B_{avg}(1 + e) = \mathbf{P_{max}}
\end{align}
\] Where: - \(\mathcal{a}\) = the average separation between the Primary
in and the Secondary expressed in terms of the radius of the Primary -
\(M_P\) = the mass of the Primary - \(M_S\) = the mass of the Secondary
- \(e\) = the eccentricity of the system

If the mass of the Secondary (\(M_S\)) is expressed in terms of the mass
of the Primary (\(M_P\)) the equations become: \[
\begin{align}
M_0 &= \frac{M_S}{M_P} \\[1em]
B_{min} &= \frac{\mathcal{a}(1 - e)\; M_0}{M_0 + 1} 
= \frac{T_{min}\; M_0}{M_0 + 1} = B_{avg}(1 - e) = \mathbf{P_{min}} \\[2em]
B_{avg} &= \frac{\mathcal{a}\; M_0}{M_0 + 1} = \mathbf{P_{avg}} \\[2em]
B_{max} &= \frac{\mathcal{a}(1 + e)\; M_0}{M_0 + 1} 
= \frac{T_{max}\; M_0}{M_0 + 1} = B_{avg}(1 + e) = \mathbf{P_{max}}
\end{align}
\] Note that in both sets of equations, the math works out such that the
dimensions of the barycenter's ``orbit'' \emph{precisely match} those of
the Primary's orbit; this is incredibly convenient, as it allows us to
directly compare the geometry of the barycenter's orbit to the physical
dimensions of the Primary, in particular the Primary's radius (\(R_P\))
--- more on this below.

\chapter{Constant Equalities}\label{constant-equalities}

Some relationships between the masses of the Primary and Secondary and
their related orbital separations are constant: \[
\begin{align}
&\frac{S_\circ}{P_\circ} = \frac{M_P}{M_S} \qquad
&&\frac{P_\circ}{S_\circ} = \frac{M_S}{M_P} \\[1em]
&\frac{P_\circ}{T_\circ} = \frac{M_S}{M_P + M_S}   \qquad
&&\frac{S_\circ}{T_\circ} = \frac{M_P}{M_P + M_S} \\[1em]
&\frac{T_\circ}{P_\circ} = \frac{M_P}{M_S} + 1   \qquad
&&\frac{T_\circ}{S_\circ} = \frac{M_S}{M_P} + 1 \\[3em]
&\frac{Min_\bullet}{Max_\bullet} = \frac{1 - e}{1 + e} \qquad
&&\frac{Max_\bullet}{Min_\bullet} = \frac{1 + e}{1 - e}
\end{align}
\]

Again, if the mass of the Secondary (\(M_S\)) is expressed in terms of
the mass of the Primary (\(M_P\)), the equations become: \[
\begin{align}
&M_0 = \frac{M_S}{M_P} \\[1em]
&\frac{S_\circ}{P_\circ} = \frac{1}{M_0} \qquad
&&\frac{P_\circ}{S_\circ} = M_0 \\[1em]
&\frac{P_\circ}{T_\circ} = \frac{M_0}{M_0 + 1}   \qquad
&&\frac{S_\circ}{T_\circ} = \frac{1}{M_0 + 1} \\[1em]
&\frac{T_\circ}{P_\circ} = \frac{1}{M_0} + 1   \qquad
&&\frac{T_\circ}{S_\circ} = M_0 + 1 \\[3em]
&\frac{Min_\bullet}{Max_\bullet} = \frac{1 - e}{1 + e} \qquad
&&\frac{Max_\bullet}{Min_\bullet} = \frac{1 + e}{1 - e}
\end{align}
\]

\section{Mass Pairings}\label{mass-pairings}

Solar analog stars are more often than not found in binary or multiple
systems than not, with over half exhibiting multiplicity.

\begin{itemize}
\tightlist
\item
  \textbf{Duquennoy \& Mayor (1991)} originally found that approximately
  \textbf{57\%} of solar-type stars (spectral types F6-K3 ---
  more-or-less what WCB calls \textbf{Solar Cognates}) in the solar
  neighborhood are part of binary or higher-order systems.
\item
  \textbf{Raghavan et al.~(2010)} updated this with modern data for the
  same spectral class range, reporting:

  \begin{itemize}
  \tightlist
  \item
    \textbf{\textasciitilde44\%} as binaries\\
  \item
    \textbf{\textasciitilde11\%} as triples or higher\\
  \item
    → Yielding a total multiplicity fraction of
    \textbf{\textasciitilde55\%} for solar-type stars.
  \end{itemize}
\end{itemize}

This tendency toward multiplicity defines a pattern that profoundly
influences system architecture, orbital stability zones, and the
landscape of potential habitability. \#\#\# Companion Mass Trends by
Primary Type

From the combined statistical analyses of \textbf{Duquennoy \& Mayor
(1991)}, \textbf{Raghavan et al.~(2010)}, \textbf{Moe \& Di Stefano
(2017)}, and \textbf{Li et al.~(2022, LAMOST)}, a clear pattern of
Primary-to-Secondary spectral type pairings emerges: the
\textbf{spectral class of the primary star} strongly constrains the
\textbf{range of companion masses} likely to occur. In other words,
binary pairing is \emph{not} random across the mass spectrum.

\begin{itemize}
\tightlist
\item
  \textbf{Early-type primaries (A--F)} tend to draw secondaries from a
  \textbf{narrow, high-mass corridor}, favoring near-equal
  companions---what observers call the \emph{twin excess}.\\
\item
  \textbf{Solar-type primaries (G-class)} exhibit a \textbf{broader,
  moderately declining distribution}, allowing both near-equal
  companions and somewhat smaller partners (typically late G or K
  types).\\
\item
  \textbf{Cooler primaries (K--M)} show a \textbf{progressively wider,
  low-mass-biased spread}, where secondaries are often much
  smaller---late K, M, or even substellar companions.
\end{itemize}

This trend can be summarized as:

\begin{quote}
Earlier, hotter stars → \textbf{narrow, high-q pairing range} (favoring
twins).\\
Later, cooler stars → \textbf{broad, low-q range} (favoring lighter
companions).
\end{quote}

Calculating a Secondary star's mass (\(M_2\)) based on the mass of the
Primary (\(M_1\)): \[
M_2 = M_1 \times ⟨⟨a ∧ b⟩⟩^{k}
\] Where: - \(M_1\) --- primary mass\\
- \(M_2\) --- secondary mass (assigned)\\
- \(⟨⟨a ∧ b⟩⟩\) --- random draw between \emph{a} and \emph{b} (inclusive
range)\\
- \(k\) --- \textbf{\emph{optional} weighting exponent}, biasing toward
the lower or upper bound\\
- \(k = 1\): uniform distribution\\
- \(0 < k < 1\): bias toward lower-mass companions\\
- \(k > 1\): bias toward higher-mass (twin-like) companions

\begin{longtable}[]{@{}
  >{\centering\arraybackslash}p{(\linewidth - 10\tabcolsep) * \real{0.0302}}
  >{\raggedleft\arraybackslash}p{(\linewidth - 10\tabcolsep) * \real{0.1236}}
  >{\raggedleft\arraybackslash}p{(\linewidth - 10\tabcolsep) * \real{0.2170}}
  >{\raggedleft\arraybackslash}p{(\linewidth - 10\tabcolsep) * \real{0.1896}}
  >{\raggedleft\arraybackslash}p{(\linewidth - 10\tabcolsep) * \real{0.1923}}
  >{\raggedright\arraybackslash}p{(\linewidth - 10\tabcolsep) * \real{0.2473}}@{}}
\toprule\noalign{}
\begin{minipage}[b]{\linewidth}\centering
\textbf{Primary}
\end{minipage} & \begin{minipage}[b]{\linewidth}\raggedleft
\end{minipage} & \begin{minipage}[b]{\linewidth}\raggedleft
\end{minipage} & \begin{minipage}[b]{\linewidth}\raggedleft
\end{minipage} & \begin{minipage}[b]{\linewidth}\raggedleft
\end{minipage} & \begin{minipage}[b]{\linewidth}\raggedright
\end{minipage} \\
\midrule\noalign{}
\endhead
\bottomrule\noalign{}
\endlastfoot
\textbf{A} & ⟨0.9 ∧ 1.0⟩ & ⟨0.8 ∧ 0.9⟩ & ⟨1.2 ∧ 1.5⟩ & ≈ 1.2 & High twin
fraction; modest upward skew reproduces near-equal masses. \\
\textbf{F} & ⟨0.7 ∧ 0.9⟩ & ⟨0.6 ∧ 0.8⟩ & ⟨1.1 ∧ 1.3⟩ & ≈ 1.0 & Slight
bias toward smaller companions, but twins still possible. \\
\textbf{G} & ⟨0.6 ∧ 0.8⟩ & ⟨0.5 ∧ 0.8⟩ & ⟨1.0 ∧ 1.2⟩ & ≈ 0.8 &
Distribution roughly flat; mild downward bias reproduces the DM91 ``flat
or falling'' trend. \\
\textbf{K} & ⟨0.4 ∧ 0.7⟩ & ⟨0.5 ∧ 0.8⟩ & ⟨1.0 ∧ 1.1⟩ & ≈ 0.7 & Clearly
skewed toward smaller secondaries; use \emph{p} ≈ 0.6∧0.8 to get that
fall-off. \\
\textbf{M} & ⟨0.2 ∧ 0.5⟩ & ⟨0.5 ∧ 0.7⟩ & ⟨1.0 ∧ 1.1 & ≈ 0.6 & Strongly
bottom-heavy distribution; small-q companions common. \\
\end{longtable}

\textbf{How to read this table} - The ``lower-bias'' column gives a
reasonable p range if you want your generator to prefer small
companions. - The ``upper-bias'' column gives values that push draws
upward (more ``twin-like'' systems). - The ``typical neutral value'' is
the mid-case consistent with observed distributions. \#\#\#\# Limiting
Eccentricity (\(\bar{e}\)) For close-binaries, the two stars should
never approach closer than \(T_{min} = 0.10\) AU (ideally \(0.15\) AU).
The eccentricity of the system which forces this limit can be calculated
by: \[
\begin{align}
\bar{e} = \dfrac{T_{max} - 0.100}{T_{max} + 0.100}
\end{align}
\] - \(ē\) (e-bar) is the largest system eccentricity that can be used
with a given \(T_{max}\) , while ensuring that \(T_{min} ≥ 0.100\).

If you know what eccentricity you want the system to have and need to
figure out the minimum maximum separation \(T_{max}\) that will maintain
\(T_{min} ≥ 0.100\) AU, you can calculate it by: \[
\begin{align}
T_{max} \geq 0.100\left(\dfrac{1 + \bar{e}}{1 - \bar{e}}\right)
\end{align}
\] \#\#\# The Period--Eccentricity Relation

\subsubsection{\texorpdfstring{1. \textbf{Observed
Pattern}}{1. Observed Pattern}}\label{observed-pattern}

Empirically, binary systems trace a clear trend:

\begin{longtable}[]{@{}
  >{\raggedright\arraybackslash}p{(\linewidth - 6\tabcolsep) * \real{0.1798}}
  >{\raggedright\arraybackslash}p{(\linewidth - 6\tabcolsep) * \real{0.1798}}
  >{\raggedright\arraybackslash}p{(\linewidth - 6\tabcolsep) * \real{0.3371}}
  >{\raggedright\arraybackslash}p{(\linewidth - 6\tabcolsep) * \real{0.3034}}@{}}
\toprule\noalign{}
\begin{minipage}[b]{\linewidth}\raggedright
Regime
\end{minipage} & \begin{minipage}[b]{\linewidth}\raggedright
Typical Period P
\end{minipage} & \begin{minipage}[b]{\linewidth}\raggedright
Typical Eccentricity e
\end{minipage} & \begin{minipage}[b]{\linewidth}\raggedright
Behavior
\end{minipage} \\
\midrule\noalign{}
\endhead
\bottomrule\noalign{}
\endlastfoot
\textbf{Short-period} & P ≲ 10 days & e ≈ 0 & Nearly circular \\
\textbf{Intermediate} & 10 -- 10³ days & e rises from 0 → 0.6 & Mixed
circular / elliptical \\
\textbf{Wide} & P ≳ 10³ & Broad scatter, e often 0.6--0.9 & Dynamically
``hot'' \\
\end{longtable}

This relation has been confirmed repeatedly---from \textbf{Duquennoy \&
Mayor (1991)} and \textbf{Raghavan et al.~(2010)} to \textbf{Moe \& Di
Stefano (2017)} and modern \textbf{Gaia DR3} samples.

\subsubsection{\texorpdfstring{2. \textbf{Physical Cause --- Tidal
Circularization}}{2. Physical Cause --- Tidal Circularization}}\label{physical-cause-tidal-circularization}

Close binaries experience \textbf{tidal friction}: each star raises
bulges on its companion.\\
Friction within those bulges converts orbital energy into heat, draining
the system's eccentricity.

Circularization timescale roughly scales as \[
t_{circ} \propto \left(\frac{\mathcal{a}}{R}\right)^8 \propto P^{\frac{16}{3}}
\]

so even a modest change in separation produces a huge difference in
damping time.\\
Systems that are young and close become circular long before wide pairs
even notice tides.

\subsubsection{\texorpdfstring{3. \textbf{Residual Eccentricities and
Anomalies}}{3. Residual Eccentricities and Anomalies}}\label{residual-eccentricities-and-anomalies}

\begin{itemize}
\tightlist
\item
  \textbf{``Twin'' binaries} (nearly equal-mass, short-period) are most
  circularized.\\
\item
  \textbf{Eccentric short-period outliers} often show evidence of a
  \textbf{third companion} pumping eee through Kozai--Lidov cycles.\\
\item
  \textbf{Post-mass-transfer pairs} can re-acquire modest eccentricity
  as mass loss changes the potential.
\end{itemize}

\subsubsection{\texorpdfstring{4. \textbf{Approximate Empirical
Envelope}}{4. Approximate Empirical Envelope}}\label{approximate-empirical-envelope}

For solar-type primaries, the upper bound of observed eccentricities can
be sketched as \[
e_{max} \simeq 1 - \left(\frac{P_{circ}}{P}\right)^\frac{2}{3}
\] with \(P_{circ} \sim 10\) days days for main-sequence systems in the
solar neighborhood.\\
Beyond \(P \approx 1000\) days, the envelope flattens near
\(e_{max} \approx 0.9\).

\subsection{🧭 Summary Insight}\label{summary-insight}

\begin{quote}
\textbf{Short orbits → tidal geometry dominates → circular.}\\
\textbf{Long orbits → gravitational memory dominates → eccentric.}

The period--eccentricity curve is the fossilized record of each system's
early interactions and subsequent tidal sculpting.
\end{quote}

\chapter{Abstract}\label{abstract-44}

\textbf{Major Topics:}\\
- Defines the \textbf{Chaos Zone} --- the dynamically unstable region
between circumstellar (S-type) and circumbinary (P-type) orbital regimes
in binary systems.\\
- Establishes the empirical stability range:\\
\[0.3\,\mathcal{A} \lesssim \alpha \lesssim 3.0\,\mathcal{A}\]\\
within which long-term planetary orbits are generally impossible.\\
- Distinguishes between:\\
- \textbf{S-type orbits} --- planets bound to one star of a wide
binary.\\
- \textbf{P-type orbits} --- planets orbiting both stars of a close
binary.\\
- Explains that orbital stability depends primarily on the \textbf{ratio
of the planetary semi-major axis} (\(\alpha\)) to the \textbf{average
stellar separation} (\(\mathcal{A}\)).\\
- Summarizes \textbf{Holman \& Wiegert (1999)} numerical results for
circumbinary (P-type) stability, introducing the full empirical
relation:\\
\[\frac{a_{crit}}{\mathcal{A}} = 1.60 + 5.10e - 2.22e^2 + 4.12\mu - 4.27e\mu - 5.09\mu^2 + 4.61e^2\mu^2\]\\
and the simplified canonical form:\\
\[\alpha \ge \mathcal{F}(e)\,\mathcal{A}, \qquad \mathcal{F}(e) = 4.1e^2 + 2.0e + 3.5\]\\
- Provides tabulated \(\mathcal{F}(e)\) values for typical binary
eccentricities, showing that stability boundaries rise roughly
quadratically with \(e\).\\
- Notes a simplified linear approximation
(\(\alpha \gtrsim \mathcal{A}(3.5 + 4.0e)\)) for rapid estimation.\\
- Incorporates \textbf{Quarles et al.~(2018, 2020)} results defining the
\textbf{S-type stability limit} for wide binaries:\\
\[\mathcal{Q}_L = 0.08\,\mathcal{A} = 0.08\left(\frac{T_{\min}}{1 - e}\right)\]\\
linking circumstellar orbital stability directly to the stars'
\emph{closest approach} (\(T_{\min}\)).\\
- Introduces the \textbf{Quarles Eccentricity Limit}
(\(\mathcal{Q}_e = 0.92\)), the maximum binary eccentricity at which
\emph{any} stable circumstellar orbit can exist.\\
- Emphasizes that the \textbf{inner and outer stability boundaries}
(Holman--Wiegert vs.~Quarles) form the \textbf{complete architecture of
binary orbital habitability}.

\textbf{Key Terms \& Symbols:}\\
- \textbf{\(\mathcal{A}\)} --- average stellar separation (AU).\\
- \textbf{\(\alpha\)} --- planetary orbital semi-major axis (AU).\\
- \textbf{\(a_{crit}\)} --- critical semi-major axis for P-type
stability.\\
- \textbf{\(T_{\min}\)} --- minimum stellar separation at periastron.\\
- \textbf{\(e\)} --- binary eccentricity.\\
- \textbf{\(\mu\)} --- binary mass ratio (\(M_2 / (M_1 + M_2)\)).\\
- \textbf{\(\mathcal{F}(e)\)} --- stability function for circumbinary
(P-type) orbits.\\
- \textbf{\(\mathcal{Q}_L\)} --- Quarles Stability Limit (outer boundary
for S-type).\\
- \textbf{\(\mathcal{Q}_e\)} --- Quarles Eccentricity Limit (max \(e\)
for circumstellar stability).

\textbf{Cross-Check Notes:}\\
- Complements \textbf{Binaries 2 --- Star Systems, General} by extending
from stellar--stellar dynamics to \textbf{planetary--binary
interactions}.\\
- Provides quantitative criteria for \textbf{stable orbital zones}
around and within binary systems.\\
- Serves as the foundation for later sections on \textbf{Roche
geometry}, \textbf{Hill spheres}, and \textbf{habitable zone modeling}
in multi-star environments.

\section{Chaos Zone}\label{chaos-zone}

\[
0.3\,\mathcal{A} \lesssim \alpha \lesssim 3.0\,\mathcal{A}
\] Where: - \(\alpha\) = semi-major axis of the planemon's orbit (AU) -
\(\mathcal{A}\) = average separation between the two stars (AU)

For clarity: - If the ratio \(\alpha/\mathcal{A}\) for a
\textbf{close-binary} is less than 0.3, a circumstellar (S-type) orbit
cannot remain stable. - If the ratio \(\alpha/\mathcal{A}\) for a
\textbf{wide-binary} is greater than 3.0, a circumbinary (P-type) orbit
cannot remain stable.

\section{P-type Orbit (around both stars in a
close-binary)}\label{p-type-orbit-around-both-stars-in-a-close-binary}

\[
\alpha \gtrsim 3.0\,\mathcal{A}
\] \#\#\# Deep Dive: The Empirical Stability Relation Numerical
simulations (Holman \& Wiegert 1999, \emph{AJ} 117:621) yield a widely
used fit for \textbf{circumbinary (P-type)} stability: \[
\frac{a_{\text{crit}}}{\mathcal{A}}
  = 1.60 + 5.10e - 2.22e^2
    + 4.12\mu - 4.27e\mu
    - 5.09\mu^2 + 4.61e^2\mu^2
\] Where: - \(a_{\text{crit}}\) = inner edge of stable circumbinary
orbits\\
- \(\mathcal{A}\) = binary semi-major axis\\
- \(e\) = binary orbital eccentricity\\
- \(\mu = M_2/(M_1 + M_2)\) = binary mass ratio

This (mercifully) can be simplified to: \[
\alpha \ge \mathcal{F}(e)\,\mathcal{A},
\qquad
\mathcal{F}(e) = 4.1e^2 + 2.0e + 3.5
\]

\begin{longtable}[]{@{}ccc@{}}
\toprule\noalign{}
\(e\) & \(\mathcal{F}(e)\) & Stable Ratio \(\alpha/\mathcal{A}\) \\
\midrule\noalign{}
\endhead
\bottomrule\noalign{}
\endlastfoot
0.0 & 3.5 & 3.5 × \\
0.5 & 4.7 & 4--5 × \\
0.7 & 6.2 & 6 × \\
\end{longtable}

\begin{quote}
\textbf{Rule of thumb:} circumbinary (P-type) planets remain stable when
\(\displaystyle\alpha \gtrsim \mathcal{F}(e)\;\mathcal{A}\) , rising
roughly quadratically with binary eccentricity.
\end{quote}

A quicker (but less precise) linear form is: \[
\alpha \gtrsim \mathcal{A}\,(3.5 + 4.0e)
\]

\section{S-type Orbit (around one star in a wide
binary)}\label{s-type-orbit-around-one-star-in-a-wide-binary}

\[
\alpha \lesssim 0.3\,\mathcal{A}
\]

\subsubsection{\texorpdfstring{Quarles Stability Limit
(\(\mathcal{Q}_L\))}{Quarles Stability Limit (\textbackslash mathcal\{Q\}\_L)}}\label{quarles-stability-limit-mathcalq_l}

\[
\mathcal{Q}_L = 0.08\,\mathcal{A}
             = 0.08\!\left(\frac{T_{\min}}{1 - e}\right)
\] Where: - \(\mathcal{Q}_L\) = maximum stable circumstellar orbital
radius (S-type)\\
- \(\mathcal{A}\) = average binary separation (defined from
periastron)\\
- \(T_{\min}\) = minimum stellar separation at periastron\\
- \(e\) = binary eccentricity

\begin{quote}
This expression emphasizes that stability depends primarily on the
\emph{closest approach} between the stars, which defines the maximum
perturbation on circumstellar orbits.
\end{quote}

If \(T_{\min}\) and \(\mathcal{Q}_L\) are known, the system eccentricity
follows from: \[
e = 1 - 0.08\!\left(\frac{T_{\min}}{\mathcal{Q}_L}\right)
\]

\subsubsection{\texorpdfstring{Quarles Eccentricity Limit
(\(\mathcal{Q}_e\))}{Quarles Eccentricity Limit (\textbackslash mathcal\{Q\}\_e)}}\label{quarles-eccentricity-limit-mathcalq_e}

\[
e \le \mathcal{Q}_e = 0.92
\] - \(\mathcal{Q}_e\) is the maximum binary eccentricity that still
allows any stable circumstellar orbit, such that
\(T_{\min} \geq \mathcal{Q}_L\).

\chapter{Abstract}\label{abstract-45}

\textbf{Major Topics:}\\
- Introduces the concept of the \textbf{Roche lobe} --- the
gravitational domain surrounding each star in a binary system within
which orbiting material remains bound to that star.\\
- Defines the \textbf{inner Lagrange point} (\(L_1\)) as the contact
point between the two lobes where mass exchange can occur.\\
- Explains how material crossing \(L_1\) initiates \textbf{mass
transfer}, shaping the evolution of close-binary systems.\\
- Classifies binary configurations by Roche-lobe occupancy:\\
- \textbf{Detached binaries} --- both stars remain within their lobes.\\
- \textbf{Semi-detached binaries} --- one star fills its lobe and
transfers mass through \(L_1\).\\
- \textbf{Contact binaries} --- both stars fill or overflow their lobes,
sharing a common envelope.\\
- Provides a \textbf{summary table} linking each configuration type to
Roche-lobe status, mass-transfer behavior, and representative system
examples.\\
- Presents \textbf{Eggleton's (1983)} empirical approximation for the
Roche-lobe radius (\(R_L\)), accurate to within 1\% for most mass
ratios:\\
\[
  f(x) =
  \frac{0.49\,x^{\tfrac{2}{3}}}
       {0.6\,x^{\tfrac{2}{3}} + \ln\!\left(1 + x^{\tfrac{1}{3}}\right)}
  \]\\
and defines the individual lobe radii as:\\
\[
  R_{L,1} = \mathcal{a}\,f\!\left(\frac{M_1}{M_2}\right)
  \quad\text{and}\quad
  R_{L,2} = \mathcal{a}\,f\!\left(\frac{M_2}{M_1}\right)
  \] - Clarifies that \textbf{\(f(x)\) is asymmetric}, causing unequal
lobe volumes for unequal-mass pairs.\\
- Emphasizes that equal-mass systems (\(M_1 = M_2\)) have
mirror-symmetric lobes sharing a single interface at \(L_1\).\\
- Highlights that lower-mass stars possess \textbf{larger fractional
Roche lobes}, explaining why accretion from the smaller to the larger
star is common in semi-detached systems.

\textbf{Key Terms \& Symbols:}\\
- \textbf{\(R_L\)} --- Roche-lobe radius (effective gravitational
boundary).\\
- \textbf{\(L_1\)} --- inner Lagrange point (mass-transfer interface).\\
- \textbf{\(\mathcal{a}\)} --- average separation of the binary stars
(AU).\\
- \textbf{\(M_1\), \(M_2\)} --- primary and secondary stellar masses
(solar units).\\
- \textbf{\(f(x)\)} --- dimensionless Roche-lobe scaling function.\\
- \textbf{Detached / Semi-Detached / Contact} --- binary configurations
classified by Roche-lobe occupancy.

\textbf{Cross-Check Notes:}\\
- Extends \textbf{Binaries 2 --- Star Systems, General} by introducing
the first \textbf{gravitational-boundary function} used in close-binary
modeling.\\
- Provides the theoretical foundation for subsequent sections on
\textbf{Roche limits}, \textbf{mass transfer}, and \textbf{accretion
physics}.\\
- Connects geometric configuration to \textbf{observational categories}
(Algol-type, W Ursae Majoris-type).

\section{Roche Lobe Geometry}\label{roche-lobe-geometry}

When two stellar bodies share a gravitational system, each defines a
\textbf{Roche lobe}---the three-dimensional region around it within
which orbiting material remains gravitationally bound to that star.
Beyond this boundary, the gravitational influence of the companion star
dominates.

Material that crosses the inner point of contact between the lobes (the
\textbf{inner Lagrange point}, \(L_1\)) can transfer from one star to
the other, a process central to the dynamics of close-binaries and
mass-transfer systems. \#\#\# Semi-Detached and Contact Binaries When
the Roche lobes \textbf{touch} (or one star even slightly
\emph{overflows} its own lobe and spills material through
L1L\_1L1\hspace{0pt}), the system becomes what's called a
\textbf{contact binary} --- or, if only one star fills its lobe, a
\textbf{semi-detached binary}.

\begin{longtable}[]{@{}
  >{\raggedright\arraybackslash}p{(\linewidth - 6\tabcolsep) * \real{0.0988}}
  >{\raggedright\arraybackslash}p{(\linewidth - 6\tabcolsep) * \real{0.3374}}
  >{\raggedright\arraybackslash}p{(\linewidth - 6\tabcolsep) * \real{0.4444}}
  >{\raggedright\arraybackslash}p{(\linewidth - 6\tabcolsep) * \real{0.1193}}@{}}
\toprule\noalign{}
\begin{minipage}[b]{\linewidth}\raggedright
Configuration
\end{minipage} & \begin{minipage}[b]{\linewidth}\raggedright
Roche-Lobe Status
\end{minipage} & \begin{minipage}[b]{\linewidth}\raggedright
Description
\end{minipage} & \begin{minipage}[b]{\linewidth}\raggedright
Example
\end{minipage} \\
\midrule\noalign{}
\endhead
\bottomrule\noalign{}
\endlastfoot
\textbf{Detached Binary} & Both stars are well within their Roche lobes.
& No mass transfer occurs; each star evolves independently under its own
gravity. & Most wide binary systems. \\
\textbf{Semi-Detached Binary} & One star fills (or nearly fills) its
Roche lobe; the other remains detached. & The donor star transfers mass
through \(L_1\) onto its companion, often forming an accretion disk. &
Algol-type systems. \\
\textbf{Contact Binary} & Both stars fill or slightly overflow their
Roche lobes, sharing a common envelope. & The stars exchange mass and
energy directly through \(L_1\); the system behaves like a single,
distorted body. & W Ursae Majoris-type systems. \\
\end{longtable}

\section{Eggleton's Approximation of Roche Lobe
Radii}\label{eggletons-approximation-of-roche-lobe-radii}

For practical modeling, the Roche-lobe radius (\(R_L\)) can be
approximated by the empirical fit derived by \textbf{Eggleton (1983)}.
Expressed in a general form: \[
\begin{align}
\text{Define: }&\quad
f(x) =
\frac{0.49\,x^{\tfrac{2}{3}}}
     {0.6\,x^{\tfrac{2}{3}} +
         \ln\!\left(1 + x^{\tfrac{1}{3}}\right)
     } \\[1em]
\text{Roche Lobe $M_1$:}&\quad R_{L,1} = \mathcal{a}\,
    f\!\left(\frac{M_1}{M_2}\right) \\[1em]
\text{Roche Lobe $M_2$:}&\quad R_{L,2} = \mathcal{a}\,
    f\!\left(\frac{M_2}{M_1}\right)
\end{align}
\] Where: - \(\mathcal{a}\) = average separation of the binary stars (in
AU)\\
- \(M_1,\,M_2\) = stellar masses (in solar units)\\
- \(f(x)\) = dimensionless scaling function describing the fractional
Roche-lobe radius

\subsubsection{Interpretation}\label{interpretation}

\begin{itemize}
\tightlist
\item
  \(R_{L,1}\) and \(R_{L,2}\) give the \textbf{effective radii} of each
  star's gravitational domain within the binary system.\\
\item
  Because \(f(x)\) is not symmetric, the two lobes differ in size unless
  \(M_1 = M_2\).\\
\item
  When \(M_1 = M_2\), both lobes have equal volumes and share a common
  boundary at \(L_1\).\\
\item
  For unequal masses, the lower-mass star has the \emph{larger
  fractional lobe} (a smaller star but a larger gravitational domain
  relative to its own mass).
\end{itemize}

\subsubsection{Practical Rule of Thumb}\label{practical-rule-of-thumb}

For typical mass ratios in close binaries
(\(0.2 \le \frac{M_2}{M_1} \le 5\)), Eggleton's formula is accurate to
within 1\% --- far more than sufficient for system-design or
orbit-stability work in WCB.

\chapter{Abstract}\label{abstract-46}

\textbf{Major Topics:}\\
- Defines the \textbf{Hill sphere} as the spatial region surrounding a
smaller body where its self-gravity dominates over the tidal forces of a
larger primary.\\
- Establishes that within the Hill sphere, satellites, rings, and other
debris can maintain \textbf{stable orbits} about the secondary body.\\
- Describes the \textbf{Hill sphere} as the \textbf{circum-orbital
analog of the Roche lobe}:\\
- The Roche lobe governs gravitational domains between \emph{comparable
masses} (binary stars).\\
- The Hill sphere applies when the \textbf{mass ratio is extreme} (e.g.,
planet--star or moon--planet).\\
- Presents the \textbf{classical Hill sphere equation}:\\
\[
  H_r = \alpha\!\left(\frac{M_2}{3M_1}\right)^{\!\tfrac{1}{3}}
  \]\\
defining the limiting radius of stable orbital influence for the smaller
body.\\
- Interprets \(H_r\) as the \textbf{outer stability boundary} for
circumsecondary orbits --- the point beyond which tidal shear from the
primary dominates.\\
- Introduces a practical constraint for \textbf{long-term stability} of
satellites:\\
\[
  r_{\text{sat}} \lesssim 0.5\,H_r
  \] indicating that stable moons typically orbit well within half the
Hill radius.\\
- Notes that for \textbf{eccentric orbits}, the Hill radius should be
evaluated at \textbf{periapsis} (\(\alpha(1 - e)\)), where tidal stress
is greatest.\\
- Compares the \textbf{Hill sphere} and \textbf{Roche lobe} in both
physical regime and geometric form:\\
- Roche lobes → teardrop-shaped equipotential regions between comparable
masses.\\
- Hill spheres → near-spherical zones for extreme mass ratios
(\(M_2 \ll M_1\)).\\
- Provides a worked example using \textbf{Earth's Hill sphere}, showing
that \(H_r \approx 0.01\) AU (≈ \(1.5×10^6\) km), comfortably
encompassing the Moon's orbit.

\textbf{Key Terms \& Symbols:}\\
- \textbf{\(H_r\)} --- Hill-sphere radius (outer limit of stable
circumsecondary orbits).\\
- \textbf{\(\alpha\)} --- orbital semi-major axis of the secondary
(AU).\\
- \textbf{\(M_1\), \(M_2\)} --- primary and secondary masses.\\
- \textbf{\(e\)} --- orbital eccentricity.\\
- \textbf{\(r_{\text{sat}}\)} --- orbital radius of a satellite.\\
- \textbf{Hill Sphere} --- stability domain for a low-mass secondary.\\
- \textbf{Roche Lobe} --- mutual gravitational domain in a binary pair.

\textbf{Cross-Check Notes:}\\
- Complements \textbf{Roche Lobe Geometry} by describing the analogous
stability boundary in \textbf{extreme mass-ratio systems}.\\
- Forms the theoretical foundation for satellite-system design, ring
stability, and capture mechanics in both planetary and stellar
contexts.\\
- Serves as a bridge to later sections on \textbf{Roche Limits} and
\textbf{gravitational capture thresholds}.

\section{Hill Sphere}\label{hill-sphere}

In any multi-body system, the \textbf{Hill sphere} defines the region
around a smaller body within which its own gravity dominates over the
tidal influence of a larger primary. Within this sphere, satellites,
rings, or retained debris can maintain long-term stable orbits around
the smaller body.

Conceptually, the Hill sphere is the \textbf{circum-orbital analog} of
the Roche lobe: where the Roche lobe marks the boundary between
\emph{two comparable masses}, the Hill sphere marks the boundary between
a \textbf{primary} and a \textbf{much smaller secondary}.

\subsection{Classical Formulation}\label{classical-formulation}

For a secondary body of mass \(M_2\) orbiting a much larger primary of
mass \(M_1\) at semi-major axis \(\alpha\), the radius of its Hill
sphere is approximated by: \[
H_r = \alpha
       \left(
         \frac{M_2}{3M_1}
       \right)^{\!\tfrac{1}{3}}
\] Where: - \(H_r\) = Hill-sphere radius (the limit of stable satellite
orbits)\\
- \(\alpha\) = orbital semi-major axis of the secondary about the
primary\\
- \(M_2\) = mass of the secondary (planet, moon, etc.)\\
- \(M_1\) = mass of the primary (star, planet, etc.)

\subsection{Interpretation}\label{interpretation-1}

\begin{itemize}
\tightlist
\item
  Inside \(H_r\), the secondary's gravity dominates; small bodies can
  orbit it stably.\\
\item
  Near or beyond \(H_r\), tidal forces from the primary destabilize
  those orbits.\\
\item
  The Hill sphere thus defines the \textbf{outer boundary of a planet's
  satellite system} or, in stellar terms, the \textbf{limit of
  gravitational capture}.
\end{itemize}

\subsection{Rule of Thumb}\label{rule-of-thumb}

\begin{itemize}
\item
  Long-term stable satellite orbits typically require\\
  \[
  r_{\text{sat}} \lesssim 0.5\,H_r
  \] --- i.e., within roughly half the Hill radius.
\item
  For planets in circular orbits, the expression is accurate to within a
  few percent.
\item
  For eccentric orbits, replace \(\alpha\) with the \textbf{periapsis
  distance}, \(\alpha(1 - e)\), to account for stronger tidal effects at
  closest approach.
\end{itemize}

\subsection{Comparison with Roche
Lobe}\label{comparison-with-roche-lobe}

\begin{longtable}[]{@{}
  >{\raggedright\arraybackslash}p{(\linewidth - 6\tabcolsep) * \real{0.2500}}
  >{\raggedright\arraybackslash}p{(\linewidth - 6\tabcolsep) * \real{0.2500}}
  >{\raggedright\arraybackslash}p{(\linewidth - 6\tabcolsep) * \real{0.2500}}
  >{\raggedright\arraybackslash}p{(\linewidth - 6\tabcolsep) * \real{0.2500}}@{}}
\toprule\noalign{}
\begin{minipage}[b]{\linewidth}\raggedright
Concept
\end{minipage} & \begin{minipage}[b]{\linewidth}\raggedright
Regime
\end{minipage} & \begin{minipage}[b]{\linewidth}\raggedright
Definition
\end{minipage} & \begin{minipage}[b]{\linewidth}\raggedright
Shape
\end{minipage} \\
\midrule\noalign{}
\endhead
\bottomrule\noalign{}
\endlastfoot
\textbf{Roche Lobe} & Two bodies of comparable mass (binaries). &
Equipotential surface where gravitational and centrifugal forces
balance. & Teardrop-shaped, asymmetric about \(L_1\). \\
\textbf{Hill Sphere} & Large primary with small secondary (planet--moon,
star--planet). & Distance from secondary where its gravitational
influence equals the primary's tidal force. & Roughly spherical for
\(m \ll M\). \\
\end{longtable}

\begin{quote}
\textbf{Analogy:} The Roche lobe is the \emph{shared frontier} between
equals;\\
the Hill sphere is the \emph{personal domain} of a subordinate.
\end{quote}

\subsection{Example --- Earth's Hill
Sphere}\label{example-earths-hill-sphere}

For Earth (\(M_2 = 1\,M_\oplus\)) orbiting the Sun
(\(M_1 = 1\,M_\odot\)) at \(\alpha = 1\,\text{AU}\): \[
H_r = 1\,\text{AU}
       \left(
         \frac{1/333{,}000}{3}
       \right)^{\!\tfrac{1}{3}}
     \approx 0.01\,\text{AU} \approx 1.5\times10^6\,\text{km}
\]

That comfortably encloses the Moon's orbit (\(≈ 384{,}000\) km),
confirming its long-term stability.

\chapter{Abstract}\label{abstract-47}

\textbf{Major Topics:}\\
- Defines the \textbf{Roche Limit} as the critical distance from a
massive primary where tidal forces exceed a secondary body's
self-gravitational cohesion.\\
- Explains that within this radius, a self-gravitating body cannot
remain intact and will be \textbf{tidally disrupted} into fragments or
rings.\\
- Provides the classical expression for the Roche Limit:\\
\[
  R_r \approx 2.44\,R_p
  \left(
    \frac{\rho_p}{\rho_s}
  \right)^{\!\tfrac{1}{3}}
  \]\\
where the ratio of the primary's to the satellite's mean density
determines the disruption distance.\\
- Distinguishes between \textbf{rigid} and \textbf{fluid} satellites:\\
- Rigid (rocky) bodies can survive closer to the primary (≈1.5--2
\(R_p\)).\\
- Fluid (icy or deformable) bodies are disrupted farther out (≈2.5
\(R_p\)).\\
- Emphasizes that the Roche Limit describes the \textbf{tidal
destruction boundary}, whereas the \textbf{Roche Lobe} and \textbf{Hill
Sphere} describe \textbf{gravitational domains of stability}.\\
- Summarizes the three related regimes:\\
\textbar{} Regime \textbar{} Controlling Effect \textbar{} Outcome
\textbar{}\\
\textbar:--\textbar:--\textbar:--\textbar{}\\
\textbar{} \textbf{Roche Lobe} \textbar{} gravitational balance between
companions \textbar{} mass exchange \textbar{}\\
\textbar{} \textbf{Roche Limit} \textbar{} tidal shear exceeds
self-gravity \textbar{} body disruption \textbar{}\\
\textbar{} \textbf{Hill Sphere} \textbar{} self-gravity exceeds external
tides \textbar{} stable satellite region \textbar{}\\
- Clarifies that the Roche Limit applies in \textbf{high mass-ratio
systems} (planet--moon, star--planet), while the Roche Lobe governs
\textbf{comparable-mass binaries}.\\
- Notes that tidal disruption at or within the Roche Limit can produce
\textbf{rings}, \textbf{debris disks}, or \textbf{accretion streams},
depending on the system's scale and composition.

\textbf{Key Terms \& Symbols:}\\
- \textbf{\(R_r\)} --- Roche Limit radius (distance from primary
center).\\
- \textbf{\(R_p\)} --- primary's radius.\\
- \textbf{\(\rho_p\)}, \textbf{\(\rho_s\)} --- mean densities of primary
and secondary.\\
- \textbf{Tidal Gradient} --- differential gravitational force across
the secondary.\\
- \textbf{Rigid / Fluid Satellite} --- composition-dependent resistance
to disruption.

\textbf{Cross-Check Notes:}\\
- Complements \textbf{Roche Lobe Geometry} and \textbf{Hill Sphere} as
the third gravitational-boundary construct.\\
- Provides the theoretical framework for \textbf{ring formation},
\textbf{tidal capture}, and \textbf{disruption cascades}.\\
- Serves as a transition from binary gravitational dynamics to
\textbf{planetary ring and debris-disk modeling} in later sections.

\section{Roche Limit}\label{roche-limit}

The \textbf{Roche Limit} marks the distance from a massive primary at
which the tidal gradient across a secondary body equals the secondary's
own gravitational cohesion. Inside this limit, a self-gravitating
satellite can no longer remain intact; it is tidally disrupted into
fragments or rings.

For a rigid satellite of density \(\rho_s\) orbiting a primary of
density \(\rho_p\) and radius \(R_p\):

\[
R_r \approx 2.44\,R_p
\left(
  \frac{\rho_p}{\rho_s}
\right)^{\!\tfrac{1}{3}}
\] Where: - \(R_r\) = Roche Limit radius (measured from the primary's
center)\\
- \(R_p\) = primary's radius\\
- \(\rho_p\) = primary's mean density\\
- \(\rho_s\) = satellite's mean density

\begin{quote}
\textbf{Rule of thumb:} Fluid bodies (like icy moons) are disrupted near
\(\sim 2.5\,R_p\), while rigid, rocky bodies survive slightly closer
(\textasciitilde1.5 -- 2 \(R_p\)).
\end{quote}

\subsection{Relation to Roche Lobe and Hill
Sphere}\label{relation-to-roche-lobe-and-hill-sphere}

\begin{longtable}[]{@{}
  >{\raggedright\arraybackslash}p{(\linewidth - 4\tabcolsep) * \real{0.1923}}
  >{\raggedright\arraybackslash}p{(\linewidth - 4\tabcolsep) * \real{0.5128}}
  >{\raggedright\arraybackslash}p{(\linewidth - 4\tabcolsep) * \real{0.2949}}@{}}
\toprule\noalign{}
\begin{minipage}[b]{\linewidth}\raggedright
Regime
\end{minipage} & \begin{minipage}[b]{\linewidth}\raggedright
Controlling Effect
\end{minipage} & \begin{minipage}[b]{\linewidth}\raggedright
Outcome
\end{minipage} \\
\midrule\noalign{}
\endhead
\bottomrule\noalign{}
\endlastfoot
\textbf{Roche Lobe} & gravitational balance between companions & mass
exchange \\
\textbf{Roche Limit} & tidal shear exceeds self-gravity & body
disruption \\
\textbf{Hill Sphere} & self-gravity exceeds external tides & stable
satellite region \\
\end{longtable}

\bookmarksetup{startatroot}

\chapter{Abstract WCB Pronunciation
Guide}\label{abstract-wcb-pronunciation-guide}

\begin{longtable}[]{@{}lll@{}}
\toprule\noalign{}
Symbol & Sound & Example \\
\midrule\noalign{}
\endhead
\bottomrule\noalign{}
\endlastfoot
ae & cat, back & /kaet/, /baek/ \\
aw & law, on & /law/, /awn/ \\
ai & time, height & /taim/, /hait/ \\
au & how, brown & /hau/, /braun/ \\
b & but & /buht/ \\
ch & church & /chuwrch/ \\
d & do & /doo/ \\
eh & get, beg & /geht/, /behg/ \\
ee & see, ski & /see/, /skee/ \\
ey & hey, day & /hey/, /dey/ \\
f & fun, puff & /fuhn/, /puhf/ \\
g & go, dog & /go/, /dawg/ \\
h & hold, hear & /hohld/, /heer/ \\
j & join, badge & /join/, /baej/ \\
k & kiss, pick & /kys/, /pyk/ \\
kw & quick & /kwyk/ \\
l & leak, fool & /leek/, /fool/ \\
m & man, dam & /maen/, /daem/ \\
n & no, on & /noh/, /awn/ \\
ng & singer, finger & /Syng-uwr/, /Fyng-guwr/ \\
o & open, flow & /Oh-pen/, /floh/ \\
oi & toy, boil & /toi/, /boil/ \\
oo & rune, moon & /roon/, /moon/ \\
p & top, pot & /tawp/, /pawt/ \\
q & \emph{ich}, human & /yq/, /Qyoo-muhn/ \\
qh & \emph{Buch}, loch & /booqh/, /lawqh/ \\
r & red, deer & /red/, /deer/ \\
s & sit, hiss & /syt/, /hys/ \\
sh & show, bash & /shoh/, /baesh/ \\
t & ton, not & /tuhn/, /nawt/ \\
þ (th) & thin, breath & /þyn/, /behþ/ (/thyn/, /brehth/) \\
ð (dh) & then, breathe & /ðen/, /breeð/ (/dhen/, /breedh) \\
uh & but, tub & /buht/, /tuhb/ \\
yoo & music, fuel & /Myoo-zyk/, /fyool/ \\
uw & put, fur, fern & /puwt/, /fuwr/, /fuwrn/ \\
v & very, have & /veh-ree/, /haev/ \\
w & wet, weather & /weht/, /Weh-ðuwr/ \\
wh & whet, whether & /wheht/, /Wheh-ðuwr/ \\
x & fox, picks & /fawx/, /pyx/ \\
y & myth, with & /myþ/, /wyþ/ \\
ỹ (yh) & yell, young & /ỹehl/, /ỹuhng/ (/yhehl/, /yhuhng/) \\
z & zoo, doze & /zoo/, /dohz/ \\
zh & azure, measure & /ae-Zhuwr/, /Meh-zhuwr/ \\
\end{longtable}


\backmatter


\end{document}
